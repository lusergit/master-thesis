\phantomsection{}
\pdfbookmark{Abstract}{Abstract}
\begingroup

\chapter*{Abstract}
This thesis explores program verification trough abstract
interpretation in the context of computability theory. Abstract
Interpretation is a program analysis technique, based on approximating
the semantics of programs over so-called \emph{abstract domains},
usually represented as complete lattices, whose elements represent
program properties. These approximations rely on some abstract
operators, which usually include fixpoint iterations. Traditionally,
to ensure convergence of such iterations, and therefore ensuring the
termination of the analyzer, the literature relied on two important
operators: the \emph{widening} and the \emph{narrowing} operators,
first defined in~\cite{patrickradhia:one}: the first one to compute an
upper bound on some chain in the complete lattice, and the second one
to recover some additional information from the program and refine the
upper bound provided by the widening. Such technique has however some
limitations: nemely it cannot infer the most precise invariant in the
considered domain. The thesis argues that for a special domain -- the
\emph{interval} domain -- such invariant can be computed in finite
time, while in another domain -- the \emph{non-relational collecting}
domain -- the most precise invariant cannot be inferred in finite
time, but rather it is possible to decide weather the analysis will
halt or not.

\endgroup

\vfill
