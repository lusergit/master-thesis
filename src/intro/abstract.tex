\phantomsection
\pdfbookmark{Abstract}{Abstract}
\begingroup

\chapter*{Abstract}
This thesis explores program verification trough abstract
interpretation in the context of computability theory. Abstract
Interpretation is a program analysis technique, based on approximating
the semantics of programs over so-called \emph{abstract domains},
usually represented as complete lattices, whose elements represent
program properties. These approximations rely on some abstract
operators, which usually include fixpoint iterations. Traditionally,
to ensure convergence of such iterations, and therefore ensuring the
termination of the analyzer, the literature relied on two important
operators: the \emph{widening} and the \emph{narrowing} operators,
first defined in \cite{patrickradiha:one}: the first one to compute an
upper bound on some chain in the complete lattice, and the second one
to recover some additional information from the program and refine the
upper bound provided by the widening. This thesis focuses on a special
abstract domain, called the \emph{intervals} domain, where each
variable of program is assigned to an interval over the integer
numbers. The thesis argues that in such a context widening and
narrowing operators can be replaced by another method, that relies on
deciding program divergence by looking at the behavior of variables in
the context of the program.

\endgroup

\vfill
