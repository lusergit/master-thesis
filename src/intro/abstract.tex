\phantomsection{}
\pdfbookmark{Abstract}{Abstract}
\begingroup

\chapter*{Abstract}
This thesis explores program verification trough abstract
interpretation in the context of computability theory. Abstract
Interpretation is a program analysis technique, based on approximating
the semantics of programs over so-called \emph{abstract domains},
usually represented as complete lattices, whose elements represent
program properties. These approximations rely on some abstract
operators, which usually include fixpoint iterations. Traditionally,
to ensure convergence of such iterations, and therefore the
termination of the analyzer, the literature relied on two important
operators: the \emph{widening} and the \emph{narrowing} operators,
first defined in~\cite{patrickradhia:one}: the first one to compute an
upper bound on some chain in the complete lattice, and the second one
to recover some additional information from the program and refine the
upper bound provided by the widening. This approach brings however a
loss of precision: the inferred invariant is not the one provided by
the abstract semantics, but an overapproximation of it. The thesis
argues that for non relational abstract domains better results can be
obtained. In particular, for the \emph{interval} domain, where each
variable is associated with an interval of integers the abstract
semantics is computed in an exact way. Instead, for the
\emph{non-relational collecting} domain where each variable is mapped
to a (possibly non-convex) subset of \(\z\) over which it can vary, we
just show that it is possible to decide the termination of the
abstract interpreter and, when this is the case, exactly compute the
abstract semantics.

\endgroup

\vfill
