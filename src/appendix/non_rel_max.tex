\section{Lemma~\ref{le:incnr} proof}\label{ap:proofleincrnr}
\begin{lemma}\label{lea:incnr}
  Let \(\com\in \imp\).
  % Then consider a function \({\max_{\Int} : \Int \to \z}\) s.t. for
  % all \({a\in\Int}\) it holds that \(x \leq \max_\Int(a)\) for all
  % \({x \in \concr[\Int](a)}\).
  For all \(\eta \in \bCnr\) and \(\var[y] \in \Var\), if
  \(\max(\semnr{\com} \eta \var[y]) \neq +\infty\) and
  \(\max(\semnr{\com}\eta \var[y]) > \bound{\com}\) then there exist a
  variable \(\var[z] \in \Var\) and an integer \(h \in \mathbb{Z}\)
  such that \(|h| \leq \bound{\com}\) and the following two properties
  hold:
  \begin{enumerate}[label=(\roman*)]
  \item\label{point1nr} \(\max(\semnr{\com}\eta \var[y]) = \max(\eta \var[z]) + h\); 
  \item\label{point2nr} for all \(\eta' \in \bCnr\), if \(\eta' \ovdot\supseteq \eta\)
    % and \(\semnr{\com}\eta' \neq \top\)
    then
    \(\max(\semnr{\com}\eta' \var[y]) \geq \max(\eta' \var[z]) + h\).
  \end{enumerate}
  % dually, if \(\min(\semnr{\com}\eta\var[y]) \neq - \infty\) and
  % \(\min(\semnr{\com}\eta\var[y]) < \lbound{\com}\) then there exists a
  % variable \(\var[w] \in \Var\) and an integer \(i \in \z\) s.t.
  % \(|i| \leq \lbound{\com}\) and the following two properties hold:
  % \begin{enumerate}[label=(\roman*)]
  % \item \(\min(\semnr{\com}\eta \var[y]) = \min(\eta \var[w]) + i\); \label{point1min}
  % \item  for all \(\eta' \in \Int\), if \(\eta' \ovdot\supseteq \eta\)
  %   %   and \(\semnr{\com}\eta' \neq \top\)
  %   then
  %   \(\min(\semnr{\com}\eta' \var[y]) \geq \max(\eta' \var[w]) + i\). \label{point2min}
  % \end{enumerate}
\end{lemma}

% Increment lemma proof
\begin{proof}
  The proof is by structural induction on the command
  \(\com\in \imp\).
  % 
  We preliminarly observe that we can safely assume
  \(\eta \neq \bot\).
  % In fact, if \(\eta = \top\) then \(\semnr{\com} \eta = \top\), so that the
  % hypothesis \(\semnr{\com} \eta \neq \top\) would not
  % be satisfied.
  In fact, if \(\eta = \bot\) then \(\semnr{\com} \bot = \bot\) and
  thus \(\max(\semnr{\com} \eta \var[y]) = -\infty \leq \bound{\com}\),
  against the hypothesis
  \(\max(\semnr{\com}\eta \var[y]) > \bound{\com}\). Moreover, when
  quantifying over \(\eta'\) such that \(\eta' \ovdot\supseteq \eta\)
  in~\ref{point2nr}, if \(\max(\semnr{\com}\eta' \var[y]) = + \infty\)
  holds, then
  \(\max(\semnr{\com}\eta' \var[y]) \geq \max(\eta' \var[z]) + h\)
  trivially holds, hence we will sometimes silently omit to consider
  this case.
  \begin{inductive}
    \case{\(\var \in I\)}
    % 
    Take \(\eta \in \bCnr\) and assume
    \(+\infty\neq\max(\semnr{\var \in I}\eta \var[y]) > \bound{\var \in I}\).
    % 
    Recall that \(I \in \Int\). Clearly
    \({\semnr{\var \in I}\eta \neq \bot}\), otherwise we would get the
    contradiction
    \(\max(\semnr{\var \in I}\eta \var[y]) = -\infty \leq \bound{\var
      \in I}\).  We distinguish two cases:
    \begin{itemize}
      
    \item If \(\var[y] \neq \var\), then for all \(\eta' \in \bCnr\) such
      that \(\eta \ovdot\subseteq \eta'\) it holds
      \[\bot \neq \semnr{\var \in I}\eta' = \eta'[\var \mapsto
        {\eta'\var \cap \concr[\Int](I)}]\]
      and thus
      \begin{equation*}
        \max(\semnr{\var \in I}\eta' \var[y]) = \max(\eta' \var[y]) = \max(\eta' \var[y]) + 0
      \end{equation*}
      hence the thesis follows with \(\var[z]=\var[y]\) and \(h = 0\).

    \item If \(\var[y] = \var[x]\) then
      \begin{equation*}
        \max(\semnr{\var \in I}\eta \var[y]) = \max(\eta\var \cap \concr[\Int](I))
      \end{equation*}
      Note that it cannot be \(\max(I)\in \z\). Otherwise, by
      Definition~\ref{def:bound},
      \(\max ({\eta\var \cap \concr[\Int](I)})
      \leq \max(I)= \bound{\var \in I}\), violating the assumption
      \(\max(\semnr{\var\in I}\eta \var[y]) > \bound{\var \in I}\).
      Hence, \(\max(I) = +\infty\) must hold and therefore % by
      % \ref{inc:hp1}
      \(\max ({\eta\var \cap \concr[\Int](I)}) = \max(\eta(\var)) =
      \max(\eta(\var)) + 0\). It is immediate to check that the same
      holds for all \(\eta' \ovdot\supseteq \eta\), i.e.,
      \begin{equation*}
        \max(\eta'\var \cap \concr[\Int](I)) = \max(\eta'\var) + 0
      \end{equation*}
      % \[
      %   \max (\eta'(\var) \ovdot\cap \abstr(I)) = \max(\eta'(\var))=
      %   \max(\eta'(\var))+0
      % \]
      and thus the thesis follows with  \(\var[z]=\var[y]=\var[x]\) and \(h=0\).
    \end{itemize}  
    
    \case{\(\var := k\)}
    % 
    Take \(\eta \in \bCnr\) and assume
    \(\max(\semnr{\var := k}\eta \var[y]) > \bound{\var := k} = |k|\).

    Observe that it cannot be \(\var = \var[y]\). In fact, since
    \(\semnr{\var := k}\eta = \eta[\var \mapsto \{k\}]\), we would
    have \(\semnr{\var := k}\eta \var[y] = \{k\}\) and
    thus % \todo[inline]{Forse non ovvio perchè
    % \(\abstr[\Int](\{k\})\) può non rappresentare esattamente
    % \(\{k\}\).}
    \begin{equation*}
      \max(\semnr{\var := k}\eta \var[y]) = k  \leq \bound{\var := k}
    \end{equation*}
    violating the assumption.
    % 
    Therefore, it must be \(\var[y] \neq \var\). Now, for all
    \(\eta' \ovdot\supseteq \eta\), we have
    \({\semnr{\var := k}\eta' \var[y] = \eta' \var[y]}\) and thus
    \begin{center}
      \(\max(\semnr{\var := k}\eta' \var[y]) = \max(\eta' \var[y]) =
      \max(\eta' \var[y]) + 0\),
    \end{center}
    hence the thesis holds with \(h =0 \leq \bound{\var :=k}\) and \(\var[z] = \var[y]\).
    
    % ----------------------------------------------
    
    \case{\(\var := \var[w] + k\)}
    % 
    Take \(\eta \in \bCnr\) and assume
    \(\max(\semnr{\var := \var[w] + k}\eta \var[y]) > \bound{\var :=
      \var[w] + k} = |k|\).
    % 
    Recall that
    \(\semnr{\var := \var[w] + k}\eta = \eta[\var \mapsto \eta \var[w] + k]\).
    
    We distinguish two cases:
    \begin{itemize}
      
    \item If \(\var[y] \neq \var\), then for all \(\eta' \ovdot\supseteq \eta\), we have
      \(\semnr{\var := \var[w] + k}\eta' \var[y] = \eta' \var[y]\) and thus
      \begin{equation*}
        \max(\semnr{\var := \var[w] + k}\eta' \var[y]) = \max(\eta' \var[y])
      \end{equation*}
      hence the thesis follows with
      \(h =0 \leq \bound{\var := \var[w] + k}\) and \(\var[z] = \var[y]\).
      
    \item 
      If \(\var = \var[y]\) then  for all \(\eta' \ovdot\supseteq \eta\), we have
      \(\semnr{\var := \var[w] + k}\eta' \var[y] = \eta' \var[w] +
      k\) and thus
      \begin{equation*}
        \max(\semnr{\var := \var[w] + k}\eta' \var[y]) = \max(\eta' \var[w]) +
        k
      \end{equation*}
      hence, the thesis follows with \(h = k\) (recall that
      \(k \leq |k| = \bound{\var := \var[w] + k}\)) and
      \(\var[z] = \var[w]\).
    \end{itemize}

    % ----------------------------------------------
    
    % \case{\(\var := \var[w] - k\)}
    % % 
    % Take \(\eta \in \Int\) and assume
    % \(\max(\semnr{\var := \var[w] - k}\eta \var[y]) > \bound{\var := \var[w] - k} = k\).
    % % 
    % Recall that
    % \(\semnr{\var := \var[w] - k}\eta = \eta[\var \mapsto \eta \var[w] - k]\).
    
    % We distinguish two cases:
    % \begin{itemize}
    
    % \item If \(\var[y] \neq \var\), then for all
    %   \(\eta' \in \Int\) such that
    %   \(\eta \ovdot\subseteq \eta'\), we have
    %   \(\semnr{\var := \var[w] - k}\eta' \var[y] = \eta' \var[y]\) and thus
    %   \begin{center}
    %     \(\max(\semnr{\var := \var[w] - k}\eta' \var[y]) = \max(\eta' \var[y])\).
    %   \end{center}
    %   hence the thesis holds, with
    %   \(h =0 \leq \bound{\var := \var[w] - k}\) and \(\var[z] = \var[y]\).
    
    % \item If \(\var = \var[y]\) then for all \(\eta' \in \Int\) such
    %   that \(\eta \ovdot\subseteq \eta'\), we have
    %   \(\semnr{\var := \var[w] - k}\eta' \var[y] = \eta' \var[w] - k\) and
    %   thus
    %   \begin{center}
    %     \(\max(\semnr{\var := \var[w] - k}\eta' \var[y]) = \max(\eta' \var[w]) -
    %     k\).
    %   \end{center}
    %   Note that the assumption \(\max(\semnr{\var := \var[w] - k}\eta \var[y]) > k\) and thus
    %   \(\max(\semnr{\var := \var[w] - k}\eta' \var[y]) > k\) ensures that subtraction is not
    %   truncated on the maximum.
    
    %   Hence the thesis holds, with \(h = -k\), hence \(|h| = \bound{\var := \var[w] - k}\),
    %   and \(\var[z] = \var[w]\).
    % \end{itemize}
    
    % ----------------------------------------------
    
    \case{\(\com_1 \ndet \com_2\)}
    % 
    Take \(\eta \in \bCnr\) and assume
    \(\max(\semnr{\com_1 \ndet \com_2}\eta) > \bound{\com_1 \ndet \com_2}
    = \bound{\com_1} + \bound{\com_2}\).  Recall that
    \(\semnr{\com_1 \ndet \com_2} \eta = \semnr{\com_1}\eta \ovdot\cup
    \semnr{\com_2}\eta\).
    % 
    Hence, since
    \(\max(\semnr{\com_1 \ndet \com_2} \eta \var[y]) \neq +\infty\), we
    have that
    \(\max(\semnr{\com_1} \eta \var[y]) \neq +\infty \neq
    \max(\semnr{\com_2}\eta \var[y])\).  Moreover
    \begin{align*}
      \max(\semnr{\com_1 \ndet \com_2}\eta \var[y])& =  \max(\semnr{\com_1}\eta \var[y] \cup \semnr{\com_2}\eta \var[y]) \\ 
                                                  & = \max \{ \max(\semnr{\com_1}\eta \var[y]), \max(\semnr{\com_2}\eta \var[y])\} % & \text{by \ref{inc:hp2}} 
    \end{align*}

    Thus
    \(\max(\semnr{\com_1 \ndet \com_2}\eta \var[y]) =
    \max(\semnr{\com_i}\eta \var[y])\) for some \(i \in \{1,2\}\). We can
    assume, without loss of generality, that the maximum is realized by
    the first component, i.e.,
    \(\max(\semnr{\com_1 \ndet \com_2}\eta \var[y]) =
    \max(\semnr{\com_1}\eta \var[y]) > \bound{\com_1 \ndet
      \com_2}\). Hence we can use the inductive hypothesis on \(\com_1\)
    and state that there exists \(h \in \mathbb{Z}\) with
    \(|h| \leq \bound{\com_1}\) and \(\var[z] \in \Var\) such that
    \(\max(\semnr{\com_1}\eta \var[y]) = \max(\eta \var[z]) + h\) and for
    all \(\eta' \in \bCnr\), \(\eta \ovdot\subseteq \eta'\),
    \[
      \max(\semnr{\com_1}\eta' \var[y]) \geq \max(\eta' \var[z]) + h
    \]

    Therefore 
    \[
      \max(\semnr{\com_1 \ndet \com_2}\eta \var[y])
      = \max(\semnr{\com_1}\eta \var[y]) = \max(\eta \var[z]) + h
    \]
    % 
    and and for
    all \(\eta' \in \bCnr\), \(\eta \ovdot\subseteq \eta'\),
    \begin{align*}
      \max(\semnr{\com_1 \ndet \com_2}\eta' \var[y])
      &= \max\{ \max(\semnr{\com_1}\eta' \var[y]),  \max(\semnr{\com_2}\eta' \var[y])\}\\
      % 
      & \geq \max(\semnr{\com_1}\eta' \var[y])\\
      % 
      & \geq \max(\eta' \var[z]) + h
    \end{align*}
    with \(|h| \leq \bound{\com_1} \leq \bound{\com_1 \ndet \com_2}\), as desired.

    % ----------------------------------------------
    
    \case{\(\com_1 \seq \com_2\)}
    % 
    Take \(\eta \in \bCnr\) and assume
    \(\max(\semnr{\com_1 \seq \com_2}\eta\var[y]) > \bound{\com_1 \seq
      \com_2} = \bound{\com_1} + \bound{\com_2}\).  Recall that
    \(\semnr{\com_1 \seq \com_2} \eta =
    \semnr{\com_2}(\semnr{\com_1}\eta)\).
    % 
    If we define
    \begin{center}
      \(\semnr{\com_1} \eta = \eta_1\)
    \end{center}
    since \(\max(\semnr{\com_2}\eta_1 \var[y]) \neq +\infty\) and
    \(\max(\semnr{\com_2} \eta_1 \var[y]) > \bound{\com_1 \seq \com_2}
    \geq \bound{\com_2}\), by inductive hypothesis on \(\com_2\), there
    are \(|h_2| \leq \bound{\com_2}\) and \(\var[w] \in \Var\) such that
    \(\max(\semnr{\com_2} \eta_1 \var[y]) = \max(\eta_1 \var[w]) + h_2\)
    and for all \(\eta_1' \in \bCnr\) with
    \(\eta_1 \ovdot\subseteq \eta_1'\)
    \begin{equation}
      \label{eq:seq2nr}
      \max(\semnr{\com_2} \eta_1' \var[y]) \geq \max(\eta_1' \var[w]) + h_2
      \tag{\dag}
    \end{equation}
    
    Now observe that
    \(\max(\semnr{\com_1}\eta \var[w]) = \max(\eta_1 \var[w]) >
    \bound{\com_1}\). Otherwise, if it were \(\max(\eta_1 \var[w]) \leq
    \bound{\com_1}\) we would have
    \begin{center}
      \(\max(\semnr{\com_2}\eta_1 \var[y]) = \max(\eta_1 \var[w]) + h_2 \leq
      \bound{\com_1} + \bound{\com_2} = \bound{\com_1 \seq \com_2}\),
    \end{center}
    violating the hypotheses. Moreover,
    \(\max(\semnr{\com_1}\eta \var[w]) \neq +\infty\), otherwise we
    would have
    \({\max(\semnr{\com_2} \eta_1 \var[y])} = {\max(\eta_1 \var[w]) +
      h_2} = + \infty\), contradicting the hypotheses.  Therefore we
    can apply the inductive hypothesis also to \(\com_1\) and deduce
    that there are \(|h_1| \leq \bound{\com_1}\) and
    \(\var[w]' \in \Var\) such that
    \(\max(\semnr{\com_1} \eta \var[w]) = \max(\eta \var[w]') + h_1\)
    and for all \(\eta' \in \bCnr\) with \(\eta \ovdot\subseteq \eta'\)
    \begin{equation}
      \label{eq:seq1nr}
      \max(\semnr{\com_1} \eta' \var[w]) \geq \max(\eta' \var[w]') + h_1
      \tag{\ddag}
    \end{equation}

    Summing up:
    \begin{align*}
      \max(\semnr{\com_1 \seq \com_2}\eta \var[y])
      &= \max(\semnr{\com_2}(\semnr{\com_1}\eta) \var[y])\\
      % 
      & = \max(\semnr{\com_2}\eta_1 \var[y])\\
      % 
      & = \max(\eta_1 \var[w]) + h_2\\
      % 
      & = \max(\semnr{\com_1}\eta \var[w]) + h_2\\
      % 
      & = \max(\eta \var[w]')+h_1 + h_2.
    \end{align*}
    Now, for all \(\eta' \in \bCnr\) with \(\eta \ovdot\subseteq \eta'\) we have that:
    \begin{align*}
      \max(\semnr{\com_1 \seq \com_2}\eta' \var[y]) & = & \\ 
      \max(\semnr{\com_2}(\semnr{\com_1}\eta') \var[y]) & \geq & \\ 
      \max(\semnr{\com_1}\eta' \var[w]) + h_2 & \geq & 
                                                      \text{by\ \eqref{eq:seq2nr}, since } \eta_1 = \semnr{\com_1}\eta \ovdot\subseteq \semnr{\com_1}\eta' \text{ and monotonicity} \\
      (\max(\eta' \var[w]') +h_1) + h_2 & & \text{by\ \eqref{eq:seq1nr}}\
    \end{align*}

    Thus, the thesis holds with \(h= h_1+h_2\), as
    \(|h| =|h_1 +h_2| \leq |h_1| + |h_2| \leq \bound{\com_1} +
    \bound{\com_2} = \bound{\com_1 \seq \com_2}\), as needed.

    % ----------------------------------------------------
    
    \case{\(\fix{\com}\)}
    % 
    Let \(\eta \in \bCnr\) such that
    \(\max(\semnr{\fix{\com}} \eta \var[y]) \neq + \infty\). Recall that
    \(\semnr{\fix{\com}} \eta = \lfp\left({\lambda \mu. (\semnr{\com} \mu
        \ovdot\cup\eta)}\right)\). Observe that the least fixpoint of
    \(\lambda \mu. (\semnr{\com} \mu \ovdot\cup\eta)\) coincides with the
    least fixpoint of
    \(\lambda \mu. (\semnr{\com} \mu \ovdot\cup \mu) = \lambda \mu. \semnr{
      \com\ndet \tru} \mu\) above \(\eta\). Hence, if
    \begin{align*}
    \eta_0 & \veq \eta \\
    \text{for all } i\in \nat \quad \eta_{i+1} & \veq \semnr{\com} \eta_i \ovdot\cup \eta_i= \semnr{\com \ndet \tru} \eta_i \ovdot\supseteq \eta_i
    \end{align*}
    then we define an increasing chain
    \({\{\eta_i\}}_{i\in \nat}\subseteq \bCnr\) such that
    \[ 
      \semnr{\fix{\com}} \eta = \textstyle\ovdot\bigcup_{i \in \nat} \eta_i.
    \]
    Since \(\max(\semnr{\fix{\com}}) \eta \var[y] \neq +\infty\), we have
    that for all \(i\in \nat\), \(\max(\eta_i \var[y]) \neq
    +\infty\). Moreover, \(\textstyle\ovdot\bigcup_{i \in \nat} \eta_i\) on
    \(\var[y]\) is finitely reached in the chain
    \({\{\eta_i\}}_{i\in \nat}\), i.e., there exists
    \(m \in \mathbb{N}\) such that for all \(i \geq m+1\)
    \[
      \max(\semnr{\fix{\com}} \eta \var[y]) = \max(\eta_{i} \var[y]).
    \]
    % \(\eta_m \neq \eta_{m+1}\) and 

    The inductive hypothesis holds for \(\com\) and \(\tru\), hence for
    \(\com \ndet \tru\), therefore for all \(\var \in \Var\) and
    \(j \in \{0,1, \ldots, m\}\), if \(\max(\eta_{j+1} \var) >\bound{\com\ndet \tru} =\bound{\com}\) then
    there exist \(\var[z] \in \Var\) and \(h \in \mathbb{Z}\) such that \(|h| \leq \bound{\com}\) and 
    \begin{enumerate}[label=(\alph*)]
    \item\label{pointanr} \(+\infty \neq \max(\eta_{j+1} \var) = \max(\eta_j \var[z]) + h\),
    \item\label{pointbnr} \(\forall \eta' \ovdot\supseteq \eta_j.
      \max(\semnr{\com \ndet \tru}\eta'\var) \geq \max(\eta' \var[z]) + h\).
    \end{enumerate}
    To shortly denote that the two conditions (a) and (b) hold, we write
    \[
      (\var[z],j) \to_{h} (\var, j+1)
    \]
    % when 
    % \(j\in  \{0, \ldots, m \}\), \(\var\in \Var\) and 
    %  % \(\max(\eta_{j+1} \var) >\bound{\com}\) hold, so that the existence of
    % \(h \in \mathbb{Z}\) and \(\var[z] \in \Var\) such that 
    % \(\max(\eta_{j+1} \var) = \max(\eta_j \var[z]) + h\) follows.
    % 
    % 
    
    Now, assume that for some variable \(\var[y] \in \Var\)
    \[\max(\semnr{\fix{\com}} \eta \var[y]) = \max(\eta_{m+1} \var[y]) >
      \bound{\fix{\com}} = (n+1) \bound{\com}\]
    where \(n=|\mathit{vars}(\com)|\). 
    We want to show that the thesis holds, i.e., that there exist
    \(\var[z] \in \Var\) and \(h \in \mathbb{Z}\) with
    \(|h| \leq \bound{\fix{\com}}\) such that:
    \begin{equation}\label{eq:maxendmaxstartaxionr}
      \max(\semnr{\fix{\com}} \eta \var[y]) = \max(\eta \var[z]) + h
      \tag{i}
    \end{equation}
    and for all \(\eta' \ovdot\supseteq \eta\),
    \begin{equation}\label{eq:maxmiddlemaxendaxionr}
      \max(\semnr{\fix{\com}} \eta' \var[y]) \geq \max(\eta' \var[z]) + h
      \tag{ii}
    \end{equation}
    
    Let us consider\ \eqref{eq:maxendmaxstartaxionr}. We first observe that we
    can define a path
    \begin{equation}\label{eq:sigmapathnr}
      \sigma \veq (\var[y]_0, 0) \to_{h_0} (\var[y]_1, 1) \to_{h_1}
      \ldots \to_{h_m} (\var[y]_{m+1}, m+1)
    \end{equation}
    such that \(\var[y]_{m+1}=\var[y]\) and for all \(j \in \{0,\ldots, m+1\}\), 
    \(\var[y]_j\in \Var\)  and
    \(\max(\eta_{j} \var[y]_{j}) > \bound{\com}\).
    % , so that for \(j\neq m+1\), \(|h_j|\leq \bound{\com}\).
    In fact, if, by contradiction, this were not the case, there would
    exist an index \(i \in \{0,\ldots, m\}\) (as
    \(\max(\eta_{m+1} \var[y]_{m+1}) > \bound{\com}\) already holds)
    such that \(\max(\eta_{i} \var[y]_{i}) \leq \bound{\com}\), while
    for all \(j \in \{i+1,\ldots, m+1\}\),
    \(\max(\eta_{j} \var[y]_j) > \bound{\com}\).  Thus, in such a case,
    we consider the nonempty path:
    \begin{equation}\label{eq:pinr}
      \pi \veq (\var[y]_i, i) \to_{h_i} (\var[y]_{i+1}, i+1) \to_{h_{i+1}} \ldots \to_{h_m} (\var[y]_{m+1}, m+1)
    \end{equation}
    and we have that:
    \begin{align*}
      &\Sigma_{j=i}^m h_j =\\ 
      &\Sigma_{j=i}^m \max(\eta_{j+1} \var[y]_{j+1}) - \max(\eta_j \var[y]_{j}) =\\
      &\max(\eta_{m+1} \var[y]_{m+1}) - \max(\eta_i \var[y]_{i}) =\\
      &\max(\eta_{m+1} \var[y]) - \max(\eta_i \var[y]_{i})>\\
      &  (n+1) \bound{\com} - \bound{\com} = n \bound{\com}
    \end{align*}
    with \(|h_j| \leq \bound{\com}\) for \(j \in \{i,\ldots,
    m\}\). Hence we can apply Lemma~\ref{le:cycles} to the projection
    \(\pi_p\) of the nodes of this path \(\pi\) to the variable
    component to deduce that \(\pi_p\) has a subpath which is a cycle
    with a strictly positive weight.  More precisely, there exist
    \(i \leq k_1 < k_2 \leq m+1\) such that
    \(\var[y]_{k_1} = \var[y]_{k_2}\) and
    \(h = \Sigma_{j=k_1}^{k_2-1} h_j > 0\). If we denote
    \(\var[w] = \var[y]_{k_1} = \var[y]_{k_2}\), then we have that
    \begin{align*}
      \max(\eta_{k_2} \var[w]) & =  h_{k_2-1}  + \max(\eta_{k_2-1} \var[w]) \\
                               & =  h_{k_2-1} + h_{k_2-2} + \max(\eta_{k_2-2} \var[w])  \\
                               & = \Sigma_{j=k_1}^{k_2-1} h_j + \max(\eta_{k_1} \var[w])  \\
                               & = h +  \max(\eta_{k_1} \var[w])  
    \end{align*}
    Thus,
    \[\max(\semnr{\com \ndet \tru}^{k_2-k_1} \eta_{k_1} \var[w]) = \max(\eta_{k_1}
      \var[w]) + h\] 
    Observe that for all \(\eta' \ovdot\supseteq \eta_{k_1}\)
    \begin{equation}\label{prop2nr}
      \max\left(\semnr{\com \ndet \tru}^{k_2-k_1} \eta' \var[w]\right) \geq \max(\eta' \var[w]) + h
    \end{equation}
    Let us show that Equation~\eqref{prop2nr} holds. We do so by
    induction on \(\ell = k_2-k_1 \geq 1\).

    \begin{description}
    \item[Case] (\(\ell = 1\)).
    % 
    Notice that by~\ref{pointbnr} used to build \(\pi\)
    in~\eqref{eq:pinr} it holds that
    \(\forall \eta'\ovdot\supseteq \eta_{k_1}\ovdot\supseteq \eta\)
    \begin{equation*}
      \max\left(\semnr{\com\ndet\tru}\eta'\var[w]\right) \geq \max(\eta'\var[w]) + h
    \end{equation*}
    hence the thesis holds.

    \item[Case] (\(\ell \Rightarrow \ell + 1\)).
    % 
    % In this case we can use the inductive hypothesis and state that
    % \(\forall \eta'\ovdot\supseteq \eta_{k_1} \ovdot\supseteq \eta\)
    % \begin{equation*}
    %   \max\left({\left(\semnr{\com\ndet\tru}\right)}^\ell\eta'\var[w]\right) \geq \max(\eta'\var[w]) + h
    % \end{equation*}
    % where \(h = \Sigma_{j=k_1}^{k_1 -1}h_j\).
    Recall that
    \begin{equation*}
      {\left(\semnr{\com\ndet\tru}\right)}^{\ell + 1}\eta' = {\left(\semnr{\com\ndet\tru}\right)}\left(\left({\left(\semnr{\com\ndet\tru}\right)}^\ell\right)\eta'\right)
    \end{equation*}
    and by inductive hypothesis
    \(\max\left({\left(\semnr{\com\ndet\tru}\right)}^\ell\eta'\var[w]\right)
    \geq \max\left(\eta'\var[w]\right) + h\). Recall that for all
    \(\eta'' \in \bCnr\) we know that
    \(\semnr{\com\ndet \tru}\eta'' = \eta '' \ovdot\cup
    \semnr{\com}\eta''\).  Hence we can notice that
    \(\max(\semnr{\com\ndet\tru}\eta''\var) \geq \max(\eta''\var)\)
    for all \(\var \in \Var\). Therefore
    \begin{equation*}
      \max\left(\semnr{\com\ndet\tru}\left({(\semnr{\com\ndet\tru})}^\ell\eta'\right)\var[w]\right) \geq \max\left({(\semnr{\com\ndet\tru})}^\ell\eta'\var[w]\right) \geq \max(\eta'\var[w]) + h
    \end{equation*}
    which is our thesis for Property~\eqref{prop2nr}.
    \end{description}
    
    \noindent    
    Then, an inductive argument allows us to show that for all \(r \in \mathbb{N}\):
    \begin{equation}\label{noppnr}
      \max(\semnr{\com \ndet \tru}^{r(k_2-k_1)} \eta_{k_1} \var[w]) \geq \max(\eta_{k_1}
      \var[w]) + r h
    \end{equation}  
    In fact, for \(r=0\) the claim trivially holds. Assuming the
    validity for \(r\geq 0\) then we have that
    \begin{align*}    
      & \max(\semnr{\com \ndet \tru}^{(r+1)(k_2-k_1)} \eta_{k_1} \var[w]) =\\
      &\max(\semnr{\com \ndet \tru}^{k_2-k_1} ( \semnr{\com \ndet \tru}^{r(k_2-k_1)} \eta_{k_1}) \var[w]) \geq & \mbox{by~\eqref{prop2nr} as } \eta_{k_1} \ovdot\subseteq \semnr{\com \ndet \tru}^{r(k_2-k_1)} \eta_{k_1}\\
      &\max(\semnr{\com \ndet \tru}^{r(k_2-k_1)} \eta_{k_1} \var[w]) + h \geq & \mbox{by inductive hypothesis}\\
      &  \max(\eta_{k_1}\var[w])  + rh + h
        \geq 
        \max(\eta_{k_1}\var[w])  + (r+1)h
    \end{align*}

    \noindent
    However, this would contradict the hypothesis
    \(\semnr{\fix{\com}} \eta \var[y] \neq \infty\). In fact the
    Inequality~\eqref{noppnr} would imply
    \begin{align*}
      \semnr{\fix{\com}} \eta \var[w]
      & = \ovdot\bigcup_{i \in \mathbb{N}} \semnr{\com
        \ndet \tru}^i \eta \var[w] =\\ 
      & =  \ovdot\bigcup_{i \in \mathbb{N}} \semnr{\com \ndet
        \tru}^i \eta_{k_1} \var[w]\\ 
      & = \ovdot\bigcup_{r \in \mathbb{N}} \semnr{\com \ndet
        \tru}^{r(k_2-k_1)} \eta_{k_1} \var[w]\\
      & = +\infty
    \end{align*}

    Now, from~\eqref{eq:sigmapathnr} we deduce that for all
    \(\eta' \ovdot\supseteq \eta_{k_1}\), for \(j \in \{ k_1, \ldots, m\}\),
    if we let \(\mu_{k_1} = \eta'\) and
    \(\mu_{j+1} = \semnr{\com \ndet \tru} \mu_j\), by the choice of the
    subsequence, since \(k_1 \geq i\), we have that
    \[\max(\mu_{j+1} \var[y]_{j+1}) \geq \max(\mu_{j+1} \var[y]_{j}) +
      h_j\] and thus
    \[
      \semnr{\com \ndet \tru}^{m-k_1+1} \eta' \var[y] = \mu_{m+1}
      \var[y]_{m+1} \geq 
      \max(\var[y]_{k_1}) + \Sigma_{i=k_1}^m h_i = \max(\eta' \var[w]) + \Sigma_{i=k_1}^m h_i
    \]
    Since \(\eta' = \semnr{\fix{\com}} \eta \ovdot\supseteq \eta_{k_1}\) we conclude
    \begin{align*}
      \max\left({\semnr{\fix{\com}} \eta \var[y]}\right) & = \max\left({\semnr{\com \ndet \tru}^{m-k_1+1} \semnr{\fix{\com}} \eta \var[w]}\right)\\
                                                        & = \max\left(\semnr{\fix\com}\eta\var[w]\right) + {\Sigma_{i=k_1}}^m{h_i} \\
                                                        & \geq +\infty + \Sigma_{i=k_1}^m h_i = +\infty
    \end{align*}
    contradicting the assumption.
    
    \noindent
    Therefore, the path \(\sigma\) of~\eqref{eq:sigmapathnr} must exist, and
    consequently
    \[\max(\semnr{\fix{\com}} \eta \var[y]) = \max(\eta_{m+1} \var[y]) =
      \max(\eta \var[y]_0) + \Sigma_{i=0}^m h_i\] and
    \(\Sigma_{i=0}^m h_i \leq \bound{\fix{\com}}=(n+1)\bound{\com}\),
    otherwise we could use the same argument above for inferring the
    contradiction \(\max(\semnr{\fix{\com}} \eta \var[y]) = +\infty\).

    \medskip

    Let us now show\ \eqref{eq:maxmiddlemaxendaxionr}. Given
    \(\eta' \ovdot\supseteq \eta\) from\ \eqref{eq:sigmapathnr} we deduce that for
    all \(j \in \{ 0, \ldots, m\}\), if we let \(\mu_0 = \eta'\) and
    \(\mu_{j+1} = \semnr{\com \ndet \tru} \mu_j\), we have that
    \[
      \max(\mu_{j+1} \var[y]_{j+1}) \geq \max(\mu_{j+1} \var[y]_{j}) +
      h_j. \]
    Therefore, since \(\semnr{\fix{\com}} \eta' \ovdot\supseteq \mu_{m+1}\)
    (observe that the convergence of \(\semnr{\fix{\com}} \eta' \) could
    be at an index greater than \(m+1\)), we conclude that:
    \[\max(\semnr{\fix{\com}} \eta' \var[y]) \geq \max(\mu_{m+1}
      \var[y]) = \max(\mu_{m+1} \var[y]_{m+1}) \geq \max(\eta' \var[y]_0)
      + \Sigma_{i=0}^m h_i\] as desired.
  \end{inductive}
\end{proof}
