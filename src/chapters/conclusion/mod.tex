\chapter{Conclusions}\label{ch:conclusion}

% We looked at the most common framework to build static analysis tools:
% starting from hte \(\imp\) language grammar we defined its concrete
% denotational semantics and its small step semantics, proving that
% semantic properties are in general not decidable. We then abstracted
% the execution of programs with the non-relational collecting semantics
% \(\bCnr\): the most general non-relational domain. We then defined and
% showed the basic properties of the intervals domain \(\inte\), a
% common domain in abstract interpretation literature. We then proved
% that for both these abstract domains variables are bounded to increase
% up to a bound we defined as \(\bound{\com}\), which depends on the
% program \(\com\). With such constraint we tried to restrict the
% abstract domains in such a way that there were no ascending and
% descending chains.
In this study, we examined the most prevalent framework for developing
static analysis tools. Beginning with the \(\imp\) language grammar,
we defined its concrete denotational semantics and small step
semantics, demonstrating that semantic properties are generally
undecidable. We then abstracted program execution using the
non-relational collecting semantics \(\bCnr\), the most general
non-relational domain. Subsequently, we defined and established the
fundamental properties of the intervals domain \(\inte\), a well-known
domain in abstract interpretation literature. We proved that in both
these abstract domains, variables are constrained to increase up to a
limit denoted as \(\bound{\com}\), which is dependent on the program
\(\com\). With this constraint, we aimed to restrict the abstract
domains in such a way that ascending and descending chains were
eliminated.

Furthermore, we proved that in the case of \(\binte{k_1}{k_2}\)
analysis, the bounded domain and the original \(\inte\) reach the same
fixpoints, provided that \(k_1\) and \(k_2\) are selected based on the
considered program \(\com\) and the initial environment
\(\eta \in \binte{k_1}{k_2}\). With these constraints, we aimed to
restrict the abstract domains to eliminate ascending and descending
chains.

In our investigation, we referenced~\cite{Gawlitza2009}
highlighting their achievement of computing precise interval
invariants within finite time. They detailed an algorithm capable of
solving systems of equations describing programs in \(O(n^3)\), where
\(n\) denotes the program's size. However, their approach heavily
relied on the convexity of elements within \(\inte\). In response, we
introduced an alternative method focused on constraining the domain of
\(\inte\) and ensuring termination by eliminating infinite ascending
chains.

Initially, we anticipated that this method would extend to
non-relational collecting abstractions, relying solely on variable
increments rather than the convex properties of \(\inte\). To our
surprise, we discovered that achieving precise invariants posed a
greater challenge than anticipated. Specifically, for non-relational
collecting semantics, we determined that while it was feasible to
infer the analyzer's termination, achieving the most accurate
invariant proved elusive. Despite this, we remain optimistic about the
potential to achieve this goal as similar results have already been
reached by James Worrell et al.\ in~\cite{Lefaucheux2024}.

\begin{figure}
  \centering
  \usetikzlibrary{arrows.meta}
  \begin{tikzpicture}[->, >=stealth]
    % Nodes
    \node (dom) {\(\dom\)};
    \node (bcnr) [below=of dom] {\(\bCnr\)};
    \node (ph1) [below=of bcnr] {};
    \node (inte) [left=of ph1] {\(\inte\)};
    \node (bbcnr) [right=of ph1] {\(\bbCnr{k_1}{k_2}\)};
    \node (binte) [below=of inte] {\(\binte{k_1}{k_2}\)};
    \node (btbcnr) [below=of bbcnr] {\(\btbCnr{k_1}{k_2}\)};
    
    % Arrows
    \path
    (dom) edge[bend right=10] node[left]{$\abstr$} (bcnr)
    (bcnr) edge[bend right=10] node[right]{$\concr$} (dom)
    (bcnr) edge[bend right=10] node[left]{$\abstr$} (inte)
    (inte) edge[bend right=10] node[right]{$\concr$} (bcnr)
    (bcnr) edge[bend right=10] node[left]{$\abstr$} (bbcnr)
    (bbcnr) edge[bend right=10] node[right]{$\concr$} (bcnr)
    (inte) edge[bend right=10] node[left]{$\abstr$} (binte)
    (binte) edge[bend right=10] node[right]{$\concr$} (inte)
    (bbcnr) edge[bend right=10] node[left]{$\abstr$} (btbcnr)
    (btbcnr) edge[bend right=10] node[right]{$\concr$} (bbcnr);
  \end{tikzpicture}
  \caption{Abstractions chain we build throughout
    Chapter~\ref{ch:abstractdomains}}\label{fig:abstrchain}
\end{figure}

We built the abstraction chain in Figure~\ref{fig:abstrchain}, and
defined abstract semantics in an inductive way on such domains, s.t.\
for any two domains \(\bA, \abstract\bA\) s.t.
\(\bA \galois{\abstr}{\concr} \abstract\bA\) implies that
\[\semi[\bA]{\com}\concr(\abstract\eta) \sqsubseteq
\concr(\semi[\abstract\bA]{\com}\abstract\eta)\]i.e., if two domains
are connected trough a Galois connection, their abstract semantics are
sound.  We proved with Theorem~\ref{th:bounded} that by restricting
the interval domain to a precisely crafted subset of itself one can
compute the exact fixpoint only by relying on the Kleene iteration
method to compute least fixpoints
\begin{equation*}
  \lfp(f) = \bigsqcup\left\{f^n(\bot) \mid n \in \n\right\}
\end{equation*}

\paragraph*{Future work.}
Throughout the development of this thesis, we encountered various
obstacles, which we present here as potential extensions and
directions for future research. Addressing all of these issues was
beyond the scope of this master's thesis due to time constraints.

The first and most prominent question is whether it is possible to
precisely compute the non-relational collecting semantics. Our
hypothesis is that it is indeed feasible, as indicated by the work
of~\cite{Lefaucheux2024}. Their research builds on Presburger
arithmetic from~\cite{1571698599431503232}, which, although beyond the
scope of this thesis, presents interesting results, particularly its
decidability.

Finally, future work can be focused on the application to real-world
tools. Aforementioned tools such as
Astreé~\cite{10.1007/978-3-540-31987-0_3} and
MOPSA~\cite{10.1007/978-3-031-30820-8_37} utilize the traditional
widening technique, which, while sound, does not infer the most
precise invariant in the domain of intervals. In contrast, we have
demonstrated that it is possible to achieve more precise invariants
within this domain.
