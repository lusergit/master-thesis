\chapter{Conclusions}\label{ch:conclusion}

In this study, we investigated some decidability questions in the
setting of abstract interpretation, the most prevalent framework for
developing static analysis tools.  The general question concerned the
possibility of effectively computing the abstract semantics. The
question is non trivial when the abstract domain can have infinite
ascending chains and thus the fixpoint computations can diverge.
Beginning with the \(\imp\) language grammar, we defined its concrete
denotational semantics in its concrete domain \(\dom\) and small step
semantics, demonstrating that all non-trivial semantic properties are
undecidable due to Rice's Theorem. We therefore biult the abstraction
chain in Figure~\ref{fig:abstrchain}.
\begin{figure}
  \centering
  \usetikzlibrary{arrows.meta}
  \begin{tikzpicture}[, >=stealth]
    % Nodes
    \node (dom) {\(\dom\)};
    \node (bcnr) [below=of dom] {\(\bCnr\)};
    \node (ph1) [below=of bcnr] {};
    \node (inte) [left=of ph1] {\(\inte\)};
    \node (bbcnr) [right=of ph1] {\(\bbCnr{\low}{\upp}\)};
    \node (binte) [below=of inte] {\(\binte{\low}{\upp}\)};
    \node (btbcnr) [below=of bbcnr] {\(\btbCnr{\low}{\upp}\)};

    \node at (-4,-3) (decidable)   {{\color{codegreen}\(\mathsf{decidable}\)}};
    \node at (3, -2) (unknown)     {{\color{airforceblue}\(\mathsf{unknown}\)}};
    \node at (2, .5) (undecidable) {{\color{red}\(\mathsf{undecidable}\)}};

    % divisor lines
    \draw[red, opacity=.4] (-3,1) edge[out=-40,in=180] (3,-1);
    % \draw (description) edge[out=180,in=0,->] (text);
    \draw[red, opacity=.4] (-3,-1) edge[out=-40,in=180] (3,-3.5);
    
    % Arrows
    \path
    (dom) edge[->, bend right=10] node[left]{$\abstr$} (bcnr)
    (bcnr) edge[->, bend right=10] node[right]{$\concr$} (dom)
    (bcnr) edge[->, bend right=10] node[left]{$\dabstr$} (inte)
    (inte) edge[->, bend right=10] node[right]{$\dconcr$} (bcnr)
    (bcnr) edge[->, bend right=10] node[left]{$\sabstr[\low,\upp]$} (bbcnr)
    (bbcnr) edge[->, bend right=10] node[right]{$\sconcr[\low,\upp]$} (bcnr)
    (inte) edge[->, bend right=10] node[left]{$\dabstr[\low,\upp]$} (binte)
    (binte) edge[->, bend right=10] node[right]{$\dconcr[\low,\upp]$} (inte)
    (bbcnr) edge[->, bend right=10] node[left]{$\tabstr[\low,\upp]$} (btbcnr)
    (btbcnr) edge[->, bend right=10] node[right]{$\tconcr[\low,\upp]$} (bbcnr);

  \end{tikzpicture}
  \caption{Abstractions chain we build throughout
    Chapter~\ref{ch:abstractdomains}}\label{fig:abstrchain}
\end{figure}
First we abstracted program execution using the non-relational
collecting semantics \(\bCnr\), the most general non-relational
domain. Subsequently, we defined and established the fundamental
properties of the intervals domain \(\inte\), a well-known domain in
abstract interpretation literature. We proved that in the
aforementioned abstract domains, variables are constrained to increase
up to a limit denoted as \(\bound{\com}\), which is dependent on the
program \(\com\). We therefore introduced the bounded interval domain
\(\binte{\low}{\upp}\): such domain does not contain infinite
ascending and descending chains, and therefore fixpoints are
computable in finite time. We then proved that given a program and an
initial environment \(\com\in\imp, \eta\in\inte\) and by crafting
appropriate bounds for \(\binte{\low}{\upp}\), namely
\(\low = \min(\eta) - \nlbound{\com}\) and
\(\upp = \max(\eta) + \nbound{\com}\), we not only could compute
\(\semi[\binte{\low}{\upp}]{\com}\eta\) but
also \[\semi[\binte{\low}{\upp}]{\com}\eta = \semi[\inte]{\com}\eta\]
which meant that we were able to compute exact interval analysis in
finite time. Following the same practice we bounded \(\bCnr\) to
\(\bbCnr{\low}{\upp}\), and showed with the example of
Code~\ref{code3} that with a simple \(\top\) element for variables
that exceed the bounds we loose too much information to infer equality
with the non-relational collecting semantics.  Therefore we introduced
\(\btbCnr{\low}{\upp}\), bonuded non-relational collecting with a
smashed top element. With this analysis we could infer equality up to
the top element, meaning that if the bounded analysis with smashed top
resulted in something different than \(\top\) then its result was the
same as the non-relational collecting, i.e.,
\[\btsemnr{\low}{\upp}{\com}\eta \neq \top \implies \semnr{\com}\eta =
  \btsemnr{\low}{\upp}{\com}\eta\] Intuitively, this implies that we
can decide the termination of non-relational collecting analysis. The
bounded version with the smashed \(\top\) element, only infers this
\(\top\) element if, during the analysis, a variable exceeds the
predefined bounds and follows an infinite sequence. Consequently, the
original non-relational collecting analysis does not terminate under
these conditions using the Kleene iteration. While if something other
than \(\top\) is inferred, it means that it is the most precise
invariant.

\paragraph*{Future work.}
Throughout the development of this thesis, we encountered various
obstacles, which we present here as potential extensions and
directions for future research. Addressing all of these issues was
beyond the scope of this thesis due to time constraints.

The first and most prominent question is whether it is possible to
precisely compute the non-relational collecting semantics. Our
hypothesis is that it is indeed feasible, as indicated by the work
of~\cite{Lefaucheux2024}. Their research builds on Presburger
arithmetic from~\cite{1571698599431503232}, which, although beyond the
scope of this thesis, presents interesting results, particularly its
decidability.

A second follow up question might involve the extension of the
language to support non-linear expressions. Our hypothesis in this
case is that invariants might become undecidable, as we are able to
write diofantine equations. Take for example the following code
snippet
\begin{lstlisting}[language=Imp, caption=Program with diofantine equations, label=code5]
  (*\(\vec{x}\)*) := null_vec_size(n); /* where n is the number of
                            variables in the diofrantine
                            equation */
  while (true) do
    ((*\(x_i\)*)++) + ((*\(x_i\)*)--)
    z = P((*\(\vec{x}\)*)) /* Where P((*{\color{codegreen}\(\vec{x}\)}*)) is a diofantine equation */
\end{lstlisting}
If the non-relational collecting semantics were decidable in a
language that permitted non-linear equations, we could determine the
solution of any Diophantine equation by deciding whether
\(0 \in \semnr{{\com[P]}_d}\). However, this is known to be an
undecidable problem, as demonstrated in~\cite{zbMATH03336816}.

Finally, future work can be focused on the application to real-world
scenarios. Aforementioned tools such as
Astreé~\cite{10.1007/978-3-540-31987-0_3} and
MOPSA~\cite{10.1007/978-3-031-30820-8_37} use the traditional widening
technique, which, while sound, does not infer the most precise
invariant in the domain of intervals. In contrast, we have
demonstrated that it is possible to achieve more precise invariants
while ensuring the termination of the analyzer using the domain of
intervals.
