% !TeX root = mod.tex
\section{Computing interval semantics}
\label{sec:computingint}

Lemma~\ref{le:inc} provides an effective algorithm for computing the
abstract semantics of commands provided a domain that respects
properties \ref{inc:hp1} and \ref{inc:hp2}. We can in fact verify that
the intervals respect such claims. First consider \(\tuple{\z, \leq}\)
(\(\z\) with the usual ordering) and the max function
\(\max : \Int \to \z\) defined as follows
\begin{align*}
  \max(\bot) & = -\infty \\
  \max(\interval{a}{b}) & = b
\end{align*}

\begin{observation}[\ref{inc:hp1} holds on intervals]
  Let \(\iota \in \Int, S \in \poset{\z}\) and let
  \({\abstr[\Int] : \poset{\z} \to \Int}\) be the abstraction map for
  the interval domain. 
  \begin{equation*}
    \left(\max(\abstr[\Int](S)) = \infty\right) \land \left(\iota \sqcap \abstr[\Int](S) \neq \bot\right)
    \Rightarrow max(\iota \sqcap \abstr[\Int](S)) = \max(\iota)
  \end{equation*}
  where \(S \in \poset{\Z}\) and \(\iota \in \Int\)
\end{observation}

\begin{proof}
  Let \(\iota = \interval{a}{b}\) and
  \(\abstr[\Int](S) = \interval{c}{\infty}\) and remember that
  \({\iota \sqcap \abstr(S) \neq \bot}\). Then
  \begin{equation*}
    \max([a,b] \sqcap \abstr[\Int](S)) = \max(\interval{\max\{a, c\}}{\min\{b , \infty\}}) = b = \max(\iota) \qedhere
  \end{equation*}
\end{proof}

\begin{observation}[\ref{inc:hp2} holds on intervals]
  Let \(\iota, \kappa \in \Int\), then
  \(\max(\iota \sqcup \kappa) = \max\{\max(\iota), \max(\kappa)\}\)
\end{observation}

\begin{proof}
  Let \(\iota = \interval{a}{b}\) and \(\kappa =
  \interval{c}{d}\). Then by definition
  \(\iota \sqcup \kappa = \interval{\min\{a,c\}}{\max\{b,d\}}\), and
  therefore
  \begin{equation*}
    \max(\iota \sqcup \kappa) = \max\{b,d\} = \max\{\max(\iota), \max(\kappa)\} \qedhere
  \end{equation*}
\end{proof}

Now we can also observe that the hypothesis hold if we consider
\(\tuple{\z,\geq}\), i.e., \(\z\) with the inverse ordering and a
modified max function \(\max : \Int \to \z\) defined as follows

\begin{align*}
  \max(\bot) & = -\infty \\
  \max(\interval{a}{b}) & = a \\
\end{align*}

i.e., the \(\min\) function. In this we also have to consider the
\(-\) operation instead of the \(+\) operation, in order to accomodate
for the reverse ordering.
 
% Non è un corollario, dovrebbe essere integrato sopra
\begin{corollary}\label{co:inc}
  if \(\min(\semi{\com}\eta\var[y]) \neq -\infty\) and
  \(\min(\semi{\com}\eta\var[y]) < \lbound{\com}\) then there exists a
  variable \(\var[z] \in\Var\) and an integer \(h\in\z\) such that
  \(|h| \geq \lbound{\com}\) and the following two properties hold:
  \begin{enumerate}[label=(\roman*)]
  \item \(\min(\semi{\com}\eta\var[y]) = \min(\eta\var[z]) - h\)
  \item for all \(\eta'\in \bA\), if \(\eta' \sqsupseteq \eta\) then
    \(\min(\semi{\com}\eta'\var[y]) \leq \min(\eta'\var[z]) - h\)
  \end{enumerate}
\end{corollary}

\begin{proof}
  \todo[inline]{da finire}
\end{proof}

This means that we can apply Lemma~\ref{le:inc} on the intervals
domain \(\inte\).  First, given a command \(\com\), the corresponding
finite set of variables \(\Var_{\com} \veq \varsof{\com}\), and an
interval environment \(\rho : \Var_{\com} \to \Int\), we define both
the \(\min\) and the \(\max\) value of an interval environment:
\begin{align*}
\max(\rho) & \veq \max \{ \max(\rho(\var)) \mid \var \in \Var_{\com}\} \\
\min(\rho) & \veq \max \{ \min(\rho(\var)) \mid \var \in \Var_{\com}\}
\end{align*}

% 
Then, when computing \(\semi[\inte]{\com} \rho\) on such \(\rho\)
having a finite domain, we can restrict to an interval domaian bounded
by some constant \(k\in\n\) s.t.
\(\binte{k_1}{k_2} \veq (\Var_{\com} \to \bInt{k_1}{k_2}) \cup \{\top, \bot\}\)
where
\begin{align*}
  \bInt{k_1}{k_2} & \veq \{ \interval{a}{b} \mid a, b \in \z \; \land \; k_1 \leq a \leq b \leq k_2\} \\
                  & \phantom{\veq} \cup \{\interval{a}{+\infty} \mid a \in \z \; \land \; a \geq k_1\} \\
                  & \phantom{\veq} \cup \{\interval{-\infty}{b} \mid b \in \z \; \land \; b \leq k_2\}
\end{align*}

We can visualize the hasse diagram of the bounded integer domain in
Figure~\ref{fig:bound} and notice that there are no infinite ascending
chains by definition.
% 
\begin{figure}
  \centering
  \begin{tikzpicture}
    \tikzset{node distance = .5cm}
    \node (top) {\(\top\)};
    \node (p1) [below=of top] {};
    \node (1) [left=of p1]{\(\interval{k_1}{+\infty}\)};
    \node (2) [right=of p1]{\(\interval{-\infty}{k_2}\)};
    \node (4) [below=of p1]{\(\interval{k_1}{k_2}\)};
    \node (p2) [left=of 4]{};
    \node (p3) [right=of 4]{};
    \node (3) [left=of p2]{\(\interval{k_1 + 1}{+\infty}\)};
    \node (5) [right=of p3]{\(\interval{-\infty}{k_2 - 1}\)};
    \node (p4) [below=of 4]{};
    \node (6) [left=of p4]{\(\interval{k_1 + 1}{k_2}\)};
    \node (pp1) [left=of 6]{};
    \node (d1) [left=of pp1]{\(\dots\)};
    \node (7) [right=of p4]{\(\interval{k_1}{k_2 - 1}\)};
    \node (pp2) [right=of 7]{};
    \node (d2) [right=of pp2]{\(\dots\)};
    \node (9) [below=of p4]{\(\interval{k_1 + 1}{k_2 - 1}\)};
    \node (p5) [left=of 9]{};
    \node (8) [left=of p5]{\(\interval{k_1 + 2}{k_2}\)};
    \node (p6) [right=of 9]{};
    \node (10) [right=of p6]{\(\interval{k_1}{k_2 - 2}\)};
    \node (d3) [left=of 8]{\(\dots\)};
    \node (d4) [right=of 10]{\(\dots\)};
    \node (vdot1) [below=of d3]{\(\vdots\)};
    \node (vdot2) [below=of d4]{\(\vdots\)};
    \node (dbot) [below=of 9]{\(\dots\)};
    \node (pbot2) [left=of dbot]{};
    \node (pbot1) [left=of pbot2]{};
    \node (pbot3) [right=of dbot]{};
    \node (pbot4) [right=of pbot3]{};
    \node (bot) [below=of dbot]{\(\bot\)};

    \draw
    (top) edge (1) edge (2)
    (1) edge (3) edge (4)
    (2) edge (4) edge (5)
    (3) edge (6) edge (d1)
    (4) edge (6) edge (7)
    (5) edge (7) edge (d2)
    (d1) edge (d3) edge (8)
    (6) edge (8) edge (9)
    (7) edge (9) edge (10)
    (d2) edge (10) edge (d4)
    (9) edge (dbot)
    (d3) edge (pbot1)
    (8) edge (pbot2)
    (10) edge (pbot3)
    (d4) edge (pbot4)
    (d4) edge (vdot2)
    (d3) edge (vdot1)
    (dbot) edge (bot)
    (vdot1) edge (bot)
    (vdot2) edge (bot);
  \end{tikzpicture}
  \caption{\(\Int_{k_1,k_2}\) Hasse diagram}
  \label{fig:bound}
\end{figure}
%
We need however a special bound, based on Lemma~\ref{le:inc}, based on
the program we are considering and the initial environment.  For
simplicity we'll refer to \(\inte_{\com,\rho}\) as
\({\inte_{\com,\rho} = (\Var_{\com} \to \Int_{\com,\rho}) \cup \{
  \top, \bot \}}\) where
\begin{align*}
  \Int_{\com,\rho} & \veq \{ \interval{a}{b} \mid a, b \in \n \land
                     \min(\rho) - \lbound{\com} \leq a \leq b \leq \max(\rho) + \bound{\com}\} \\
                   & \phantom{\veq} \cup \{\interval{a}{+\infty} \mid a \in\z \; \land \; a \geq \min(\rho) - \lbound{\com}\} \\
                   & \phantom{\veq} \cup \{\interval{-\infty}{b} \mid b \in\z \; \land \; b \leq \max(\rho) + \bound{\com}\} \\
                   & = \bInt{\min(\rho)-\lbound{\com}}{\max(\rho) + \bound{\com}}
\end{align*}

We preliminarly observe that for any given \(k_1,k_2\in\n\) the
lattice \(\binte{k_1}{k_2}\) is a sub-lattice of \(\inte\)
\begin{align*}
  \eta \sqcup \theta & \in \binte{k_1}{k_2} & \forall \eta,\theta \in \binte{k_1}{k_2} \\
  \eta \sqcap \theta & \in \binte{k_1}{k_2} & \forall \eta,\theta \in \binte{k_1}{k_2} \\
\end{align*}

i.e., they are closed under \(\sqcap\) and \(\sqcup\). In fact if we
consider \({\eta = \interval{a}{b} \in \binte{k_1}{k_2}}\) and
\({\theta = \interval{c}{d} \in \binte{k_1}{k_2}}\) by definition both
\(b,d \leq k_2\) and \(a,c \geq k_1\). Now consider
\({\eta \sqcup \theta = \interval{\min\{a,c\}}{\max\{b,d\}}}\), where
we can observe \(\max\{b,d\} \leq k_2\) and \(\min\{a,c\} \geq k_1\)
and therefore \({\eta \sqcup \theta \in \binte{k_1}{k_2}}\) by
definition of \(\binte{k_1}{k_2}\).  Also observe that if
\(\eta \sqcap \theta \neq \bot\) then
\({\eta \sqcap \theta = \interval{\max\{a,c\}}{\min\{b,d\}}}\) and it
holds that \({\min\{b,d\} \leq k_2}\) and \({\max\{a,c\} \geq k_1}\)
and therefore \({\eta \sqcap \theta \in \binte{k_1}{k_2}}\) by definition of
\(\binte{k_1}{k_2}\).

\medskip

\noindent
Special attention must be put in defining what the analysis over a
bounded interval domain is. In particular, let's first define our
abstraction and concretization maps

\begin{definition}\label{def:boundedac}
  Let \(k_1, k_2\in\n\). The abstraction map
  \({\abstr[\bInt{k_1}{k_2}] : \poset{\z} \to \bInt{k_1}{k_2}}\) is
  defined as follows
  \begin{align*}
    \abstr[\bInt{k_1}{k_2}](\emptyset) & \defin \bot \\
    \abstr[\bInt{k_1}{k_2}](S) & \defin \begin{cases}
      \interval{\min(S)}{\max(S)} & \text{if } \max(S) \leq k_2 \text{ and } \min(S) \geq k_1\\
      \interval{\min(S)}{+\infty} & \text{if } \max(S) > k_2 \text{ and } \min(S) \geq k_1 \\
      \interval{-\infty}{\max(S)} & \text{if } \min(S) < k_1 \text{ and } \max(S) \leq k_2 \\
      \interval{-\infty}{+\infty} & \text{otherwise}
    \end{cases}
  \end{align*}
  Where \(S\in\poset{\z}\). While the concretization map
  \({\concr[\bInt{k_1}{k_2}] : \bInt{k_1}{k_2} \to \poset{\z}}\)
  actually coincides with the concretization map \(\concr[\Int]\) as
  \(\bInt{k_1}{k_2}\) is a sublattice of \(\Int\).
\end{definition}

Let's also redefine the \(+\) operation in the \(\bInt{k_1}{k_2}\)
lattice, as adding a constant to an interval with the old definition
might overcome the bound, and therefore diverge

\begin{definition}\label{def:sumbound}
  For a nonempty interval \(\interval{a}{b} \in \bInt{k_1}{k_2}\) and
  \(c \in \n\) define
  \begin{equation*}
    \interval{a}{b} \pm c \veq \begin{cases}
      \interval{a\pm c}{b\pm c} & a \pm c \geq k_1 \land b \pm c \leq k_2 \\
      \interval{a\pm c}{+\infty} & b \pm c \geq k_2 \\
      \interval{-\infty}{b \pm c} & a \pm c \leq k_1
    \end{cases}
  \end{equation*}
  recalling that \(\pm \infty + c = \pm\infty - c = \pm\infty\).
\end{definition}

\begin{lemma}\label{le:leq}
  for all \(k_1, k_2\in\nat\)% s.t. \(k \geq \max(\rho)\)
  \begin{equation*}
    \semi[\inte]{\com}\rho \sqsubseteq \semi[\binte{k_1}{k_2}]{\com}\rho
  \end{equation*}
  i.e., with \(\binte{k_1}{k_2}\) we have an over-approximation of \(\inte\).
\end{lemma}
\begin{proof}
  The proof works by induction on \(\com\). Let's therefore first work
  on the base cases.

  \medskip

  \noindent
  \textbf{Case} (\(\var \in S\)).
  % 
  Recall that
  \(\semi[\inte]{\var\in S}\rho = \rho[\var \mapsto {\rho\var \sqcap
    \abstr[\Int](S)}]\). Hence we have two cases:
  \begin{itemize}
  \item \({\rho\var \sqcap \abstr[\Int](S)} = \bot\). In this case it
    holds that
    \begin{equation*}
      \semi[\inte]{\var\in S}\rho = \bot \sqsubseteq \semi[\binte{k_1}{k_2}]{\var\in S}\rho
    \end{equation*}
    
  \item \({\rho\var \sqcap \abstr[\Int](S)} = \interval{a}{b}\) for
    some \({a \in \n \cup \{-\infty\}}, {b \in
      \n\cup\{+\infty\}}\). In this case notice that because of
    \(\abstr[\bInt{k_1}{k_2}]\) definition, it holds that
    \begin{equation*}
      \abstr[\Int](S) \sqsubseteq \abstr[\bInt{k_1}{k_2}](S)
    \end{equation*}
    and therefore
    \({\rho\var \sqcap \abstr[\Int](S)} \sqsubseteq {\rho\var \sqcap
      \abstr[\bInt{k_1}{k_2}](S)}\). Hence
    \begin{equation*}
      {\semi[\inte]{\var\in S}\rho} \sqsubseteq {\semi[\binte{k_1}{k_2}]{\var\in S}\rho}
    \end{equation*}
    which is our thesis.
  \end{itemize}

  \medskip

  \noindent
  \textbf{Case} (\(\var := k\)).
  % 
  Let's recall that
  \({\semi[\inte]{\var := k}\rho = \rho[\var \mapsto
    \abstr[\Int](\{k\})]}\) and since
  \({\abstr[\Int](\{k\})} \sqsubseteq {\abstr[\bInt{k_1}{k_2}](\{k\})}\) it holds
  that
  \begin{equation*}
    \semi[\inte]{\var := k}\rho \sqsubseteq \semi[\binte{k_1}{k_2}]{\var := k}\rho
  \end{equation*}
  which is our thesis.

  \medskip

  \noindent
  \textbf{Case} (\(\var := \var[y] + k\)).
  % 
  Recall again that
  \({\semi[\inte]{\var := \var[y] + k}\rho} = {\rho[\var \mapsto
    \rho\var[y] + k]}\). Also recall that
  \begin{equation*}
    \iota +_{\Int} j \sqsubseteq \iota +_{\bInt{k_1}{k_2}} j
  \end{equation*}
  where \(\iota \in \bInt{k_1}{k_2} \subseteq \Int\), \(j\in\n\),
  i.e., \(\rho\var[y] + j\) in \(\Int\) is more precise than
  \(\rho\var[y] + j\) in \(\bInt{k_1}{k_2}\). Hence it holds that
  \begin{equation*}
    {\semi[\inte]{\var := \var[y] + k}\rho} \sqsubseteq {\semi[\binte{k_1}{k_2}]{\var := \var[y] + k}\rho}
  \end{equation*}
  which is our thesis.

  \medskip
  \noindent
  Now, we can work on the inductive cases.

  \medskip

  \noindent
  \textbf{Case} (\(\com_1 \ndet \com_2\)).
  % 
  Recall that
  \({\semi[\inte]{\com_1 \ndet \com_2}\rho} =
  {\semi[\inte]{\com_1}\rho} \ndet {\semi[\inte]{\com_2}\rho}\). By
  inductive hypothesis
  \({\semi[\inte]{\com_1}\rho} \sqsubseteq
  {\semi[\binte{k_1}{k_2}]{\com_1}\rho}\) and
  \({\semi[\inte]{\com_2}\rho} \sqsubseteq
  {\semi[\binte{k_1}{k_2}]{\com_2}\rho}\). Hence we can conclude by noticing
  that \(\binte{k_1}{k_2}\) is closed under \(\sqcup\)
  \begin{equation*}
    \semi[\inte]{\com_1 \ndet \com_2}\rho \sqsubseteq
    \semi[\binte{k_1}{k_2}]{\com_1}\rho \sqcup \semi[\binte{k_1}{k_2}]{\com_2}\rho =
    \semi[\binte{k_1}{k_2}]{\com_1 \sqcup \com_2}\rho
  \end{equation*}

  \medskip

  \noindent
  \textbf{Case} (\(\com_1 \seq \com_2\)).
  % 
  Recall that
  \({\semi[\inte]{\com_1 \seq \com_2}\rho} =
  {\semi[\inte]{\com_2}\left(\semi[\inte]{\com_1}\rho\right)}\). By
  inductive hypothesis
  \begin{equation}\label{eq:a1}
    {\semi[\inte]{\com_1}\rho} \sqsubseteq {\semi[\binte{k_1}{k_2}]{\com_1}\rho}
  \end{equation}
  We can call \(\rho' = {\semi[\binte{k_1}{k_2}]{\com_1}\rho}\) and observe
  that by inductive hypothesis
  \begin{equation}\label{eq:a2} {\semi[\inte]{\com_2}\rho'}
    \sqsubseteq {\semi[\binte{k_1'}{k_2'}]{\com_2}\rho'}
  \end{equation}
  for all \(k_1',k_2'\in\n\). We can conclude by
  \begin{align*}
    {\semi[\inte]{\com_2}\left(\semi[\inte]{\com_1}\rho\right)} & \sqsubseteq {\semi[\binte{k_1}{k_2}]{\com_2}\left(\semi[\inte]{\com_1}\rho\right)} & \text{by \eqref{eq:a2}}\\
                                                                & \sqsubseteq {\semi[\binte{k_1}{k_2}]{\com_2}\left(\semi[\binte{k_1}{k_2}]{\com_1}\rho\right)} & \text{by \eqref{eq:a1} and monotonicity}
  \end{align*}
  which is our thesis

  \medskip

  \noindent
  \textbf{Case} (\(\fix{\com}\)).
  % 
  Recall that
  \(\fix{\com} = {\lfp(\lambda \mu . (\rho \sqcup
    \semi[\inte]{\com}\mu))}\), which again coincides with
  \({\lfp(\lambda \mu . (\mu \sqcup \semi[\inte]{\com}\mu))}\) above
  \(\rho\). i.e., we can build the chain of iterands
  \begin{align*}
    \rho_0 & \defin \rho \\
    \rho_{i+1} & \defin \semi[\inte]{\com}\rho_i \sqcup \rho_i = \semi[\inte]{\com \ndet \tru}\rho_i
  \end{align*}
  and the analysis becomes
  \({\semi[\inte]{\fix{\com}}\rho} =
  {\textstyle\bigsqcup_{i\in\n}\rho_i} =
  {\textstyle\bigsqcup_{i\in\n}\left(\semi[\inte]{\com \ndet
        \tru}\right)^i\rho}\).  For each iterand, the inductive
  hypothesis works for \(\com\) and for \(\tru\), and therefore it
  works for \(\com \ndet \tru\). Now we can use an inductive argument
  on \(i\) to state that for all \(i \in\n\)
  \begin{equation*}
    {\left(\semi[\inte]{\com\ndet\tru}\right)^i\rho} \sqsubseteq {\left(\semi[\binte{k_1}{k_2}]{\com\ndet\tru}\right)^i\rho}
  \end{equation*}
  Hence, by closure of \(\binte{k_1}{k_2}\) over \(\sqcup\)
  \begin{equation*}
    {\semi[\inte]{\fix{\com}}} =
    {\textstyle\bigsqcup_{i\in\n}\left(\semi[\inte]{\com\ndet\tru}\right)^i \rho}
    \quad \sqsubseteq \quad
    {\textstyle\bigsqcup_{i\in\n}\left(\semi[\binte{k_1}{k_2}]{\com\ndet\tru}\right)^i \rho} =
    {\semi[\binte{k_1}{k_2}]{\fix{\com}}}
  \end{equation*}

\end{proof}
With this consideration we can now proceed to prove that the analysis
on our bounded lattice \(\inte_{\com,\rho}\) produces the same result
as the analysis on \(\inte\).

% Galois connection??

% We could also operate uniformly on all commands, defining the
% semantics for \(\com\) in a domain with intervals bounded by
% \(\max(\rho) +\bound{\com}\)

\begin{lemma}
  Let \(\com\in \imp\) be a command. Then, for all finitely supported
  \(\rho : \Var \to \Int\)
  \begin{equation*}
    \semi[\inte]{\com}\rho = \semi[\inte_{\com,\rho}]{\com}\rho
  \end{equation*}
  i.e., the abstract semantics \(\semi{\com} \rho\)
  % 
  % \semi{\fix{\com}} \rho & = \lfp{\lambda \rho'. (\semi{\com} \rho')
  % \sqcup \rho}
  computed in \(\inte\) and in \(\inte_{\com,\rho}\) coincide.
\end{lemma}

\begin{proof}

  The proof will proceed by induction on the command \(\com\). We can
  preliminarly observe that in case the analysis results in the
  \(\top\) element (i.e., \({\semi[\inte]{\com}\rho = \top}\)), since
  for all \(k_1,k_2 \in \z\) it holds that
  \(\semi[\inte]{\com}\rho \sqsubseteq
  \semi[\binte{k_1}{k_2}]{\com}\rho\) by Lemma~\ref{le:leq} it
  trivially holds that \(\semi[\inte_{\com,\rho}]{\com}\rho = \top\)
  and therefore the two analysis coincide. We will therefore silently
  omit this case.  Now, let's explore the base cases.

  \medskip
  
  \noindent
  \textbf{Case} (\(\var \in S\)).
  % 
  Recall that
  \begin{equation*}
    \semi[\inte]{\var \in S}\rho = \begin{cases}
      \rho[\var \mapsto \rho\var \sqcap \abstr[\Int](S)] & \text{if } \rho\var \sqcap \abstr[\Int](S) \neq \bot \\
      \bot & \text{otherwise}
    \end{cases}
  \end{equation*}
  and that
  \begin{equation*}
    \semi[\inte_{\com,\rho}]{\var \in S}\rho = \begin{cases}
      \rho[\var \mapsto \rho\var \sqcap \abstr[\Int_{\com,\rho}](S)] & \text{if } \rho\var \sqcap \abstr[\Int_{\com,\rho}](S) \neq \bot \\
      \bot & \text{otherwise}
    \end{cases}
  \end{equation*}
  With \(k \geq \max(\rho) + \bound{\var \in S}\). We have 2 cases: if
  \({\rho\var \sqcap \abstr[\Int](S) = \bot}\) then also for all
  bounded sub-intervals \(\abstr[\bInt{k_1}{k_2}](S)\) with
  \(k_1 \leq \min(\rho) - \lbound{\var\in S}\) and
  \(k_2 \geq \max(\rho) + \bound{\var\in S}\) it holds that
  \({\rho\var \sqcap \abstr[\bInt{k_1}{k_2}](S) = \bot}\) and
  therefore
  \({\semi[\inte]{\var\in S}\rho = \bot = \semi[\inte_{\var\in S,
      \rho}]{\var \in S}}\). Otherwise
  \({\rho\var \sqcap \abstr[\Int](S) \neq \bot}\). In this case
  \({\semi[\inte]{\var \in S}\rho = \rho[\var \mapsto \rho\var \sqcap
    \abstr[\Int](S)]}\) and we can notice that
  \(\max(\rho\var \sqcap \abstr[\Int](S)) \leq \max(\rho\var) \leq
  \max(\rho)\). Therefore
  \begin{equation*}
    \semi[\inte_{\var \in S,\rho}]{\var \in S}\rho = \rho[\var \mapsto \rho\var \sqcap \abstr[\Int_{\var \in S,\rho}](S)]
    = \rho[\var \mapsto \rho\var \sqcap \abstr[\Int](S)] = \semi[\inte]{\var\in S}\rho
  \end{equation*}
  which is our thesis.

  \medskip
  
  \noindent
  \textbf{Case} (\(\var := k\)).
  % 
  Let's recall that
  \({\semi[\inte]{\var := k}\rho = \rho[\var \mapsto \interval{k}{k}
    ]}\). Recall that we are considering \(\inte_{\var := k,\rho}\),
  therefore

  \medskip
  
  \noindent
  \textbf{Case} (\(\var := \var[y] + k\)).
  % 
  Let's recall that
  \({\semi[\inte]{\var := \var[y] + k}\rho = \rho[\var \mapsto
    \rho\var[y] + k]}\) and \(\bound{\var := \var[y] + k} =
  |k|\). Recall that we are considering \(\inte_j\) with
  \(j \geq \max(\rho) + \bound{\var := \var[y] + k}\). Notice that
  what we are interested in is the behaviour of \(\var\). We have 2
  cases
  \begin{itemize}
  \item
    \({\max(\semi[\inte]{\var := \var[y] + k}\rho\var)} =
    +\infty\). This means that \(\rho\var[y] + k = +\infty\), but
    since \(k \in \z\), \(\rho\var[y] = \interval{a}{\infty}\) with
    \(a \in \n\), therefore
    \({\rho\var[y] + k} = {\interval{a+k}{+\infty}}\) which we can
    notice is part of \(\Int_j\) for all \(j \in \n\). Hence
    \begin{equation*}
      \semi[\inte]{\var := \var[y] + k}\rho =
      \rho[\var \mapsto \rho\var[y] + k] =
      \semi[\inte_j]{\var := \var[y] + k}\rho
    \end{equation*}
    for all \(j \geq \max(\rho) + \bound{\var := \var[y] + k}\), which
    is our thesis.
  \item
    \({\max(\semi[\inte]{\var := \var[y] + k}\rho\var)} \neq
    +\infty\). In this case we have 2 more cases:
    \begin{itemize}
    \item
      \({\max(\semi[\inte]{\var := \var[y] + k}\rho\var)} \leq
      \bound{\var := \var[y] + k}\). In this case by definition it
      holds that
      \begin{equation*}
        {\semi[\inte]{\var := \var[y] + k}\rho} = \rho[\var \mapsto \rho\var[y] + k] = {\semi[\inte_j]{\var := \var[y] + k}\rho}
      \end{equation*}
      for all \(j \geq \max(\rho) + \bound{\var := \rho\var[y] + k}\),
      which is our thesis;
    \item
      \({\max(\semi[\inte]{\var := \var[y] + k}\rho\var)} <
      \bound{\var := \var[y] + k}\). In this case by
      Lemma~\ref{le:inc} we know that
      \begin{equation*}
        {\max(\semi[\inte]{\var := \var[y] + k}\rho\var)} = {\max(\rho\var[z]) + h}
      \end{equation*}
      for some \(\var[z] \in \Var\) and \(h \in \z\) s.t.
      \(|h| \leq \bound{\var := \var[y] + k} = |k|\), which means that
      \({\rho[\var \mapsto \rho\var[y] + k]} \in \Int_j\) for all
      \(j \geq \max(\rho) + \bound{\var := \var[y] + k}\).
      Hence
      \begin{equation*}
        \semi[\inte]{\var := \var[y] + k}\rho =
        \rho[\var \mapsto \rho\var[y] + k] =
        \semi[\inte_j]{\var := \var[y] + k}\rho
      \end{equation*}
      for all \(j \geq \max(\rho) + \bound{\var := \var[y] + k}\),
      which is our thesis.
    \end{itemize}
  \end{itemize}
  
  \medskip
  \noindent
  Next, we can move to the inductive cases
  
  \medskip
  
  \noindent
  \textbf{Case} (\(\com_1 \ndet \com_2\)).
  % 
  Recall that
  \({\semi[\inte]{\com_1 \ndet \com_2}\rho} =
  {\semi[\inte]{\com_1}\rho} \sqcup {\semi[\inte]{\com_2}}\). By
  inductive hypothesis it holds that
  \begin{align*}
    {\semi[\inte]{\com_1}\rho} & = \semi[\inte_{k_1}]{\com_1}\rho & \forall k_1 \geq \max(\rho) + \bound{\com_1} \\
    {\semi[\inte]{\com_2}\rho} & = \semi[\inte_{k_2}]{\com_2}\rho & \forall k_2 \geq \max(\rho) + \bound{\com_2}
  \end{align*}
  in particular, it holds when both
  \({k_1, k_2} \geq {\max(\rho) + \bound{\com_1} + \bound{\com_2}} =
  {\max(\rho) + \bound{\com_1 \ndet \com_2}}\). Hence
  \begin{equation*}
    {\semi[\inte]{\com_1 \ndet \com_2}\rho} = {\semi[\inte_{\com_1 \ndet \com_2, \rho}]{\com_1}\rho} \sqcup {\semi[\inte_{\com_1 \ndet \com_2, \rho}]{\com_2}}
  \end{equation*}
  and we can conlcude by recalling that \(\binte{k_1}{k_2}\) is closed under
  \(\sqcup\)
  \begin{equation*} {\semi[\inte_{\com_1 \ndet \com_2,
        \rho}]{\com_1}\rho} \sqcup {\semi[\inte_{\com_1 \ndet \com_2,
        \rho}]{\com_2}\rho} = {\semi[\inte_{\com_1\ndet\com_2}]{\com_1
        \ndet \com_2}\rho}
  \end{equation*}

  \medskip
  
  \noindent
  \textbf{Case} (\(\com_1 \seq \com_2\)).
  % 
  Let's recall that
  \(\semi[\inte]{\com_1 \seq \com_2}\rho =
  \semi[\inte_{k_2}]{\com_2}\left({\semi[\inte_{k_1}]{\com_1}\rho}\right)\)
  where \(k_1 \geq \max(\rho) + \bound{\com_1}\) and if we call
  \(\rho' = {\semi[\inte_{k_1}]{\com_1}\rho}\) then
  \({k_2 \geq \max(\rho') + \bound{\com_2}}\). In particular it holds
  for \(k \geq {\max(\rho) + \bound{\com_1} +
    \bound{\com_2}}\). Therefore
  \begin{equation*}
    \semi[\inte]{\com_1 \seq \com_2}\rho = \semi[\inte]{\com_2}\left(\semi[\inte]{\com_1}\rho\right) =
    \semi[\binte{k_1}{k_2}]{\com_2}\left(\semi[\binte{k_1}{k_2}]{\com_1}\rho\right) = \semi[\binte{k_1}{k_2}]{\com_1 \seq \com_2}\rho
  \end{equation*}
  for all
  \(k \geq \max(\rho) + \bound{\com_1} + \bound{\com_2} = \max(\rho) +
  \bound{\com_1 \seq \com_2}\) which is our thesis.

  \medskip
  
  \noindent
  \textbf{Case} (\(\fix{\com}\)).
  % % per il fix(C) credo che un modo di procedere possa essere il seguente:
  % 
  % - uso come dominio I_k con k >= max(rho) + (n+2) C^b
  Let's recall that as we observed in the \(\fix{\com}\) case in
  Lemma~\ref{le:inc} that
  \({\fix{\com} = \lfp(\lambda \mu . \semi[\inte]{\com \ndet
      \tru}\mu)}\) above \(\rho\). We can therefore build the chain of
  iterands
  \begin{align*}
    \rho_0 & \defin \rho \\
    \rho_{i+1} & \defin \semi[\inte]{\com \ndet \tru}\rho_i
  \end{align*}
  Let's consider \(\binte{k_1}{k_2}\) with
  \(k \geq \max(\rho) + (n+2)\bound{\com}\) where
  \(n = |\varsof{\com}|\). We can make the following two observations
  for each variable \(\var[y]\):
  \begin{enumerate}[label=(\roman*)]
  \item if
    \({\max(\semi[\inte]{\fix{\com}}\rho\var[y])} \neq +\infty\) then,
    because of Lemma~\ref{le:inc} it holds that
    \begin{equation*}
      \max(\semi[\inte]{\fix{\com}}\rho\var[y]) \leq \max(\rho) + \bound{\fix{\com}} = \max(\rho) + (n+1)\bound{\com}
    \end{equation*}
    where \(n = |\Var_{\com}|\) is the number of variables appearing
    in the program \(\com\). For all iterands \(\rho_i\) we can notice
    that
    \begin{equation*}
      \rho_i = \left(\semi[\inte]{\com \ndet \tru}\right)^i\rho \sqsubseteq \semi[\inte]{\fix{\com}}\rho = \textstyle\bigsqcup_{i\in\nat}{\left(\semi[\inte]{\com\ndet\tru}\right)}^i\rho
    \end{equation*}
    now observe that because of our choice of \(k\) it holds that
    \({k \geq \max(\rho_i) + \bound{\com}}\). We can therefore use the
    inductive hypothesis and deduce that
    \begin{equation*}
      \semi[\inte]{\com\ndet\tru}\rho_i = \semi[\binte{k_1}{k_2}]{\com\ndet\tru}\rho_i
    \end{equation*}
    for all \(k \geq \max(\rho_i) + \bound{\com}\). In particular it
    holds for
    \(k \geq \max(\rho) + \bound{\fix{\com}} \geq \max(\rho_i) +
    \bound{\com}\) for all \(i\in\n\). Hence by \(\binte{k_1}{k_2}\) closure
    over \(\sqcup\)
    \begin{equation*}
      \semi[\inte]{\fix{\com}}\rho
      =
      \textstyle\bigsqcup_{i\in\nat}\left(\semi[\inte]{\com\ndet\tru}\right)^i\rho
      =
      \textstyle\bigsqcup_{i\in\nat}\left(\semi[\binte{k_1}{k_2}]{\com\ndet\tru}\right)^i\rho
      =
      \semi[\binte{k_1}{k_2}]{\fix{\com}}\rho
    \end{equation*}
    
  \item instead, if
    \({\max(\semi[\inte]{\fix{\com}}\rho\var[y])} = +\infty\) for some
    \(\var[y]\in \Var_{\com}\) then we can use the observation made
    with Lemma~\ref{le:leq}, i.e., 
    \({\semi[\inte]{\com}\rho} \sqsubseteq
    {\semi[\binte{k_1}{k_2}]{\com}\rho}\) for all \({k \in \nat}\), and
    conclude because it means that
    \({\semi[\inte]{\fix{\com}}\rho} = \top =
    {\semi[\binte{k_1}{k_2}]{\fix{\com}}\rho}\).
  \end{enumerate}
  % 
  % - osservo che
  % 
  % (i) se max([ fix(C) ] rho y) calcolato in I è finito, per il lemma, vale
  % 
  % max([ fix(C) ] rho y) = max(rho) + fix(C)^b = max(rho) + (n+1) C^b
  % 
  % in tutti gli iterati ho che
  % 
  % rho_i = [C+id]^i rho <= [ fix(C) ] rho y 
  % 
  % e quindi per la scelta di k, sono certo che k >= max(rho_i) +
  % C^b. Pertanto posso usare l'ipotesi induttiva su C per concludere
  % che gli iterati coincidono su I e I_k
  % 
  % (ii) se max([ fix(C) ] rho y) in I infinito, dal fatto che in I_k
  % faccio una sovrapprossimazione concludo che anche il calcolo in
  % I_k da' infinito.
  % 
  % (in questo secondo caso non mi è evidente come concludere anche
  % che il lowerbound dell'intervallo è lo stesso, ma suppongo si
  % possa o derivi dalla trattazione duale con intervalli in Z)
\end{proof}
