% !TeX root = mod.tex
\section{Computing interval semantics}
\label{sec:computingint}

Lemma~\ref{le:inc} provides an effective algorithm for computing the abstract
semantics of commands%  provided a domain that respects properties \ref{inc:hp1}
% and \ref{inc:hp2}
. This means that we can apply Lemma~\ref{le:inc} on the intervals
domain \(\inte\).  First, given a command \(\com\), the corresponding
finite set of variables \(\Var_{\com} \veq \varsof{\com}\), and an
interval environment \(\rho : \Var_{\com} \to \Int\), we define both
the \(\min\) and the \(\max\) value of an interval environment:
\begin{align*}
\max(\rho) & \veq \max \{ \max(\rho(\var)) \mid \var \in \Var_{\com}\} \\
\min(\rho) & \veq \max \{ \min(\rho(\var)) \mid \var \in \Var_{\com}\}
\end{align*}

% 
Then, when computing \(\semi[\inte]{\com} \rho\) on such \(\rho\)
having a finite domain, we can restrict to an interval domain bounded
by some constant \(k\in\n\):

\begin{definition}[Bounded interval]\label{def:boundedint}
  We define
  \(\binte{k_1}{k_2} \veq (\Var_{\com} \to \bInt{k_1}{k_2}) \cup
  \{\top, \bot\}\) where
  \begin{align*}
    \bInt{k_1}{k_2} & \veq \{ \interval{a}{b} \mid a, b \in \z \; \land \; k_1 \leq a \leq b \leq k_2\} \\
                    & \phantom{\veq} \cup \{\interval{a}{+\infty} \mid a \in \z \; \land \; a \geq k_1\} \\
                    & \phantom{\veq} \cup \{\interval{-\infty}{b} \mid b \in \z \; \land \; b \leq k_2\}
  \end{align*}
\end{definition}

We can visualize the Hasse diagram of the bounded integer domain in
Figure~\ref{fig:bound} and notice that there are no infinite ascending
chains by definition.
% 
\begin{figure}
  \centering
  \begin{tikzpicture}
    \tikzset{node distance = .5cm}
    \node (top) {\(\top\)};
    \node (p1) [below=of top] {};
    \node (1) [left=of p1]{\(\interval{k_1}{+\infty}\)};
    \node (2) [right=of p1]{\(\interval{-\infty}{k_2}\)};
    \node (4) [below=of p1]{\(\interval{k_1}{k_2}\)};
    \node (p2) [left=of 4]{};
    \node (p3) [right=of 4]{};
    \node (3) [left=of p2]{\(\interval{k_1 + 1}{+\infty}\)};
    \node (5) [right=of p3]{\(\interval{-\infty}{k_2 - 1}\)};
    \node (p4) [below=of 4]{};
    \node (6) [left=of p4]{\(\interval{k_1 + 1}{k_2}\)};
    \node (pp1) [left=of 6]{};
    \node (d1) [left=of pp1]{\(\dots\)};
    \node (7) [right=of p4]{\(\interval{k_1}{k_2 - 1}\)};
    \node (pp2) [right=of 7]{};
    \node (d2) [right=of pp2]{\(\dots\)};
    \node (9) [below=of p4]{\(\interval{k_1 + 1}{k_2 - 1}\)};
    \node (p5) [left=of 9]{};
    \node (8) [left=of p5]{\(\interval{k_1 + 2}{k_2}\)};
    \node (p6) [right=of 9]{};
    \node (10) [right=of p6]{\(\interval{k_1}{k_2 - 2}\)};
    \node (d3) [left=of 8]{\(\dots\)};
    \node (d4) [right=of 10]{\(\dots\)};
    \node (vdot1) [below=of d3]{\(\vdots\)};
    \node (vdot2) [below=of d4]{\(\vdots\)};
    \node (dbot) [below=of 9]{\(\dots\)};
    \node (pbot2) [left=of dbot]{};
    \node (pbot1) [left=of pbot2]{};
    \node (pbot3) [right=of dbot]{};
    \node (pbot4) [right=of pbot3]{};
    \node (bot) [below=of dbot]{\(\bot\)};

    \draw
    (top) edge (1) edge (2)
    (1) edge (3) edge (4)
    (2) edge (4) edge (5)
    (3) edge (6) edge (d1)
    (4) edge (6) edge (7)
    (5) edge (7) edge (d2)
    (d1) edge (d3) edge (8)
    (6) edge (8) edge (9)
    (7) edge (9) edge (10)
    (d2) edge (10) edge (d4)
    (9) edge (dbot)
    (d3) edge (pbot1)
    (8) edge (pbot2)
    (10) edge (pbot3)
    (d4) edge (pbot4)
    (d4) edge (vdot2)
    (d3) edge (vdot1)
    (dbot) edge (bot)
    (vdot1) edge (bot)
    (vdot2) edge (bot);
  \end{tikzpicture}
  \caption{\(\bInt{k_1}{k_2}\) Hasse diagram}\label{fig:bound}
\end{figure}
%
Now we can notice that given \(k_1, k_2 \in \z\) we can build a Galois
Connection (Definition~\ref{def:galoiscon}) between the interval
domain \(\inte\) (the \emph{concrete} domain) and the bounded interval
domain \(\binte{k_1}{k_2}\) (the \emph{abstract} domain). To do so we
first need to define a concretization and abstraction maps.

\begin{definition}\label{def:abstrcons}
  Given \(k_1, k_2\in\z\) we define a concretization map
  \(\concr[k_1,k_2] : \bInt{k_1}{k_2} \to \Int\) as the
  identity function
  \begin{equation*}
    \concr[k_1,k_2] = \id
  \end{equation*}
  while we define an abstraction map
  \(\abstr[k_1, k_2] : \Int \to \bInt{k_1}{k_2}\) in the following way
  \begin{align*}
    \abstr[k_1,k_2](\bot) & = \bot \\
    \abstr[k_1,k_2](\interval{a}{b}) & =
                                       \begin{cases}
                                         \interval{a}{b} & \text{if } a \geq k_1 \wedge b \leq k_2 \\
                                         \interval{-\infty}{b} & \text{if } a < k_1 \wedge b \leq k_2 \\
                                         \interval{a}{+\infty} & \text{if } a \geq k_1 \wedge b > k_2 \\
                                         \interval{-\infty}{+\infty} & \text{otherwise}
                                       \end{cases}
  \end{align*}
\end{definition}

Next, we prove that given \(k_1, k_2\in\z\) we in fact have a Galois
Connection:

\begin{lemma}
  Given \(k_1, k_2 \in \z\) s.t.~\({k_1 \leq k_2}\)
  \begin{equation*}
    \tuple{\Int, \sqsubseteq} \galois{\abstr[k_1, k_2]}{\id} \tuple{\bInt{k_1}{k_2}, \sqsubseteq}
  \end{equation*}
  i.e., \(\tuple{\abstr[k_1, k_2], \Int, \bInt{k_1}{k_2}, \id}\) is
  a Galois Connection.
\end{lemma}

\begin{proof}
  We want to prove that \(\id\) and \(\abstr[k_1,k_2]\) satisfy the
  property of Theorem~\ref{th:galoiscon}, i.e., we want to prove
  the following:
  \begin{enumerate}[label=(\arabic*)]
  \item\label{prop1gal} \(\abstr[k_1,k_2], \id\) are monotonic;
  \item\label{prop2gal} \(\id \circ \abstr[k_1, k_2]\) is
    extensive, i.e., \(\forall \iota \in \Int\) it holds that
    \(\iota \sqsubseteq \id(\abstr[k_1, k_2](\iota))\);
    
  \item\label{prop3gal} \(\abstr[k_1,k_2] \circ \id\) is reductive,
    i.e., \(\forall \iota_b \in \bInt{k_1}{k_2}\) it holds that
    \(\abstr[k_1, k_2](\id(\iota_b)) \sqsubseteq \iota_b\).
  \end{enumerate}

  \medskip

  \noindent
  Let us show~\ref{prop1gal}.\ \(\id\) is monotone since
  \(\forall \iota, \kappa \in \bInt{k_1}{k_2}\) it holds that
  \(\iota \sqsubseteq \kappa \Rightarrow \iota \sqsubseteq
  \kappa\). For \(\abstr[k_1,k_2]\) we have to prove that for all
  \(\iota, \kappa \in \Int\) it holds that
  \(\iota \sqsubseteq \kappa \Rightarrow \abstr[k_1,k_2](\iota)
  \sqsubseteq \abstr[k_1,k_2](\kappa)\). Now notice that
  \(\iota \sqsubseteq \kappa \) means that
  \(\min(\iota) \geq \min(\kappa)\) and
  \(\max(\iota) \leq \max(\kappa)\). Hence, by
  Definition~\ref{def:abstrcons} of \(\abstr[k_1, k_2]\) it holds that
  \(\abstr[k_1, k_2](\iota) \sqsubseteq \abstr[k_1, k_2](\kappa)\),
  which is our thesis.

  \medskip

  \noindent
  Let us now show~\ref{prop2gal}. We have to prove that
  \(\forall \iota \in \Int\) it holds that
  \(\iota \sqsubseteq \concr(\abstr[k_1,k_2](\iota))\). By hypothesis
  \(\concr = \id\), hence we just have to prove that
  \(\iota \sqsubseteq \abstr[k_1, k_2](\iota)\). Based on the
  definition of \(\abstr[k_1, k_2]\) from
  Definition~\ref{def:abstrcons} both the following hold:
  \begin{align*}
    \min(\abstr[k_1,k_2](\iota)) \leq \min(\iota) \\
    \max(\abstr[k_1,k_2](\iota)) \geq \max(\iota)
  \end{align*}
  Hence it holds that
  \begin{equation}
    \iota \sqsubseteq \abstr[k_1, k_2](\iota)
  \end{equation}

  \medskip

  \noindent
  We can finally prove~\ref{prop3gal}. In other words
  \(\abstr[k_1, k_2] \circ \concr\) is reductive, i.e.,
  \(\forall \iota_b \in \bInt{k_1}{k_2},
  \abstr[k_1,k_2](\id(\iota_b)) \sqsubseteq \iota_b\).  Notice that
  \(\forall \iota_b \in \bInt{k_1}{k_2}\) it holds that
  \begin{equation}\label{eq:injection}
    \abstr[k_1,k_2](\iota_b) = \iota_b
  \end{equation}
  hence it holds that \(\abstr[k_1,k_2](\iota_b) \sqsubseteq \iota_b\)
\end{proof}

Notice that because of Equation~\eqref{eq:injection} holds we know
that \(\abstr[k_1,k_2] \circ\id = \id\), hence we not only have a
Galois Connection but a Galois Injection
(Definition~\ref{def:insertion}):
\begin{equation}
  \tuple{\Int, \sqsubseteq} \galoiS{\abstr[k_1, k_2]}{\id} \tuple{\bInt{k_1}{k_2}, \sqsubseteq}
\end{equation}

because of the latter observation and since
\(\inte, \binte{k_1}{k_2}\) are the point(variable)-wise lifting of
\(\Int, \bInt{k_1}{k_2}\) respectively, by Theorem~\ref{th:liftingins}
it holds that
\begin{equation}\label{eq:inteinsert}
  \tuple{\inte, \sqsubseteq} \galoiS{\dabstr[k_1,k_2]}{\id} \tuple{\binte{k_1}{k_2}, \sqsubseteq}
\end{equation}
where
\({\dabstr[k_1,k_2]}(\eta) = \lambda \var
. \abstr[k_1,k_2](\eta\var)\).
% From this we can observe that given a program \(\com\),
% \(\semi[\inte]{\com} : \inte \to \inte\) is a concrete operator on
% \(\inte\) and
% \(\semi[\binte{k_1}{k_2}]{\com} : \binte{k_1}{k_2} \to
% \binte{k_1}{k_2}\) an abstract operator on \(\binte{k_1}{k_2}\)
We can therefore define our analysis in \(\binte{k_1}{k_2}\) by means
of best correct approximations over \(\binte{k_1}{k_2}\):

\begin{definition}[Bounded interval analysis]\label{def:boundedanalysis}
  Let \(\rho \in \binte{k_1}{k_2}\) for some \(k_1, k_2 \in \z\) s.t.\
  \(k_1 \leq k_2\) and \(\com \in \imp\). We define
  \(\semi[\binte{k_1}{k_2}]{\com}\rho\) as follows
  \begin{align*}
    \semi[\binte{k_1}{k_2}]{\com[e]}\rho & \defin \dabstr[k_1,k_2] \left(\semi[\inte]{\com[e]}\rho\right) \\
    \semi[\binte{k_1}{k_2}]{\com_1 \ndet \com_2}\rho & \defin \semi[\binte{k_1}{k_2}]{\com_1}\rho \sqcup \semi[\binte{k_1}{k_2}]{\com_2}\rho \\
    \semi[\binte{k_1}{k_2}]{\com_1 \seq \com_2}\rho & \defin \left(\semi[\binte{k_1}{k_2}]{\com_2} \circ \semi[\binte{k_1}{k_2}]{\com_1}\right)\rho \\
    \semi[\binte{k_1}{k_2}]{\kleene{\com}}\rho & \defin \textstyle\bigsqcup_{i\in\n}{\left({\semi[\binte{k_1}{k_2}]{\com}}\right)}^i\rho \\
    \semi[\binte{k_1}{k_2}]{\fix{\com}}\rho & \defin \lfp\left(\lambda \mu . \rho \sqcup \semi[\binte{k_1}{k_2}]{\com}\mu\right)
  \end{align*}
  where \(\com[e]\in\expr\). 
\end{definition}
\noindent
Notice that for basic expressions we're using the best correct
approximation
\(\dabstr[k_1,k_2] \circ \semi[\inte]{\com[e]} \circ \id\), which
allows us to prove the soundness of the analysis over
\(\binte{k_1}{k_2}\) w.r.t. the analysis over \(\inte\):
\begin{lemma}\label{le:leq}
  for all \(k_1, k_2\in\z\) s.t.\ \(k_1 \leq k_2\),
  \(\rho \in \binte{k_1}{k_2}\)
  \begin{equation*}
    \semi[\inte]{\com}\rho \sqsubseteq \semi[\binte{k_1}{k_2}]{\com}\rho
  \end{equation*}
  i.e., with \(\binte{k_1}{k_2}\) we have an over-approximation of \(\inte\).
\end{lemma}

\begin{proof}
  The theorem follows from the fact that for basic expressions we're
  using best correct approximations (from
  Definition~\ref{def:boundedanalysis}), the Galois connection ensures
  that composition preserves soundness (by Theorem~\ref{th:opcomp})
  and that least upper bounds are sound (by Property~\ref{prop:five}
  of Theorem~\ref{th:galoisprop}).
\end{proof}
With this consideration we can now proceed to prove that the analysis
on our bounded lattice \(\inte_{\com,\rho}\) produces the same result
as the analysis on \(\inte\).

% We could also operate uniformly on all commands, defining the
% semantics for \(\com\) in a domain with intervals bounded by
% \(\max(\rho) +\bound{\com}\)

\begin{theorem}
  Let \(\com\in \imp\) be a command. Then, for all finitely supported
  \(\rho : \Var \to \Int\), and \(k_1, k_2\in \z\) s.t.
  \(\inte_{\com,\rho} \sqsubseteq \binte{k_1}{k_2}\), i.e.,
  \(k_1 \leq \min(\rho) - \lbound{\com}\) and
  \(k_2 \geq \max(\rho) + \bound{\com}\)
  \begin{equation*}
    \semi[\inte]{\com}\rho = \semi[\binte{k_1}{k_2}]{\com}\rho
  \end{equation*}
  i.e., the abstract semantics \(\semi{\com} \rho\)
  % 
  % \semi{\fix{\com}} \rho & = \lfp{\lambda \rho'. (\semi{\com} \rho')
  % \sqcup \rho}
  computed in \(\inte\) and in \(\binte{k_1}{k_2}\) coincide.
\end{theorem}

\begin{proof}

  The proof will proceed by induction on the command \(\com\). We can
  preliminarly observe that in case the analysis results in the
  \(\top\) element (i.e., \({\semi[\inte]{\com}\rho = \top}\)), since
  for all \(k_1,k_2 \in \z\) it holds that
  \(\semi[\inte]{\com}\rho \sqsubseteq
  \semi[\binte{k_1}{k_2}]{\com}\rho\) by Lemma~\ref{le:leq} it
  trivially holds that \(\semi[\inte_{\com,\rho}]{\com}\rho = \top\)
  and therefore the two analysis coincide. We will therefore silently
  omit this case.  Now, let's explore the base cases.

  \medskip
  
  \noindent
  \textbf{Case} (\(\var \in S\)).
  % 
  Recall that
  \begin{equation*}
    \semi[\inte]{\var \in S}\rho = \begin{cases}
      \rho[\var \mapsto \abstr[\Int](\concr[\Int](\rho\var) \cap \concr[\Int](S))] & \text{if } \concr[\Int](\rho\var) \cap \concr[\Int](S) \neq \emptyset \\
      \bot & \text{otherwise}
    \end{cases}
  \end{equation*}
  and that
  \begin{equation*}
    \semi[\binte{k_1}{k_2}]{\var \in S}\rho = \begin{cases}

      \rho[\var \mapsto \abstr[\bInt{k_1}{k_2}](\concr[\Int](\rho\var) \cap \concr[\Int](S))] & \text{if } \concr[\Int](\rho\var) \cap \concr[\Int](S) \neq \emptyset \\
      \bot & \text{otherwise}
    \end{cases}
  \end{equation*}
  With \(k_1 \leq \min(\rho) - \lbound{\var \in S}\) and
  \(k_2 \geq \max(\rho) + \bound{\var \in S}\). We have 2 cases:
  \begin{enumerate}[label=(\arabic*)]
  \item \({\concr[\Int](\rho\var) \cap \concr[\Int](S) = \emptyset}\).

    \noindent
    In this case it holds that
    \begin{equation*}
      \semi[\inte]{\var\in S}\rho = \bot = \semi[\binte{k_1}{k_2}]{\var \in S}\rho.
    \end{equation*}
    
  \item \({\concr[\Int](\rho\var) \cap \concr[\Int](S) \neq
      \emptyset}\).

    \noindent
    In this case
    \({\semi[\inte]{\var \in S}\rho = \rho[\var \mapsto
      \abstr[\Int](\concr[\Int](\rho\var) \cap \concr[\Int](S))]}\) and
    we can notice that
    \(\max(\abstr[\Int](\concr[\Int](\rho\var) \cap \concr[\Int](S)))
    \leq \max(\rho\var) \leq \max(\rho)\). Therefore
    \begin{align*}
      \semi[\binte{k_1}{k_2}]{\var \in S}\rho & = \rho[\var \mapsto \abstr[\bInt{k_1}{k_2}](\concr[\Int](\rho\var) \cap \concr[\Int](S))] \\
                                              & = \rho[\var \mapsto \abstr[\Int](\concr[\Int](\rho\var) \cap \concr[\Int](S))] \\
                                              & = \semi[\inte]{\var\in S}\rho
    \end{align*}
    for all \(k_1 \leq \min(\rho) - \lbound{\var \in S}\) and
    \(k_2 \geq \max(\rho) + \bound{\var\in S}\). Which is our thesis.
  \end{enumerate}

  \medskip
  
  \noindent
  \textbf{Case} (\(\var := k\)).
  % 
  Let's recall that
  \({\semi[\inte]{\var := k}\rho = \rho[\var \mapsto \interval{k}{k}
    ]}\). Recall that we are considering \(\binte{k_1}{k_2}\) with
  \(k_1 \leq \min(\rho) - \lbound{\var := k}\) and
  \(k_2 \geq \max(\rho) + \bound{\var := k}\). Hence we can conclude by
  observing that
  \begin{equation*}
    \min(\rho) - \lbound{\var := k} \leq k \leq k \leq \max(\rho) + \bound{\var := k}
  \end{equation*}
  and therefore for all \(k_1 \leq \min(\rho) - \lbound{\var:=k}\) and
  \(k_2 \geq \max(\rho) + \bound{\var := k}\) it holds that
  \begin{equation*}
    \semi[\inte]{\var := k}\rho = \rho[\var \mapsto \interval{k}{k}] = \semi[\binte{k_1}{k_2}]{\var := k}\rho
  \end{equation*}
  which is our thesis.

  \medskip
  
  \noindent
  \textbf{Case} (\(\var := \var[y] + k\)).
  % 
  Let's recall that
  \({\semi[\inte]{\var := \var[y] + k}\rho = \rho[\var \mapsto
    \rho\var[y] + k]}\) and \(\bound{\var := \var[y] + k} =
  |k|\). Also remember that we are considering \(\binte{k_1}{k_2}\)
  with \(k_1 \leq \min(\rho) - \lbound{\var := \var[y] + k}\) and
  \(k_2 \geq \max(\rho) + \bound{\var := \var[y] + k}\). Notice that
  for all \(\Var \ni \var[w] \neq \var \) it holds that
  \(\rho\var[w] = \semi[\inte]{\var := \var[y] + k}\rho\var[w]\),
  hence we consider \(\var\). We have 2 cases
  \begin{enumerate}[label=(\arabic*)]
  \item
    \({\max(\semi[\inte]{\var := \var[y] + k}\rho\var)} =
    +\infty\). In this case we have 2 more cases.
    \begin{enumerate}[label=(\roman*)]
    \item
      \({\min(\semi[\inte]{\var := \var[y] + k}\rho\var)} =
      -\infty\). In this case
      \({\semi[\inte]{\var := \var[y] + k}\rho} = \top\). Since by
      Lemma~\ref{le:leq}
      \(\semi[\inte]{\com}\rho \sqsubseteq
      \semi[\binte{k_1}{k_2}]{\com}\rho\) for all \(k_1, k_2 \in \z\)
      s.t. \(k_1 \leq k_2\), which means that
      \begin{equation*}
        \semi[\inte]{\var := \var[y] + k}\rho\var = \interval{-\infty}{+\infty} = \semi[\binte{k_1}{k_2}]{\var := \var[y] + k}\rho\var
      \end{equation*}
      in particular for all
      \(k_1 \leq \min(\rho) - \lbound{\var := \var[y] + k}\) and
      \(k_2 \geq \max(\rho) + \bound{\var := \var[y] + k}\), which is
      our thesis.
    \item
      \({\min(\semi[\inte]{\var := \var[y] + k}\rho\var)} \neq -
      \infty\). In this case by Lemma~\ref{co:inc}
      \begin{equation*}
        \min(\semi[\inte]{\var := \var[y] + k}\rho\var) = \min(\rho\var[y]) + k
      \end{equation*}
      and therefore
      \({\semi[\inte]{\var := \var[y] + k}\rho\var} =
      \interval{a}{+\infty} = {\semi[\binte{k_1}{k_2}]{\var := \var[y]
          + k}\rho\var}\) for some
      \(a \geq \min(\rho) - \lbound{\var := \var[y] + k}\), and for
      all \(k_1 \leq \min(\rho)- \lbound{\var := \var[y] + k}\) and
      \(k_2 \geq \max(\rho) + \bound{\var := \var[y] + k}\). Hence our
      thesis holds.

    \end{enumerate}
  \item
    \({\max(\semi[\inte]{\var := \var[y] + k}\rho\var)} \neq
    +\infty\). In this case by Lemma~\ref{le:inc} it holds that
    \begin{equation*}
      \max(\semi[\inte]{\var:= \var[y] + k}\rho\var) = \max(\rho\var[y]) + k
    \end{equation*}
    Here we have 2 more cases depending on the value of
    \(\min(\semi[\inte]{\var:= \var[y] + k}\rho\var)\):
    \begin{enumerate}[label=(\roman*)]
    \item
      \({\min(\semi[\inte]{\var:= \var[y] + k}\rho\var[w])} =
      -\infty\). In this case
      \(\semi[\inte]{\var := \var[y] + k}\rho\var[w] =
      \interval{-\infty}{b}\) with
      \(b \leq \max(\rho) + \bound{\var := \var[y] + k}\), in
      particular, because of the semantics and Lemma~\ref{le:inc} it
      holds that
      \begin{equation*}
        \max(\semi[\inte]{\var := \var[y] + k}\rho\var) = \max(\semi[\binte{k_1}{k_2}]{\var := \var[y] + k}\rho\var)
      \end{equation*}
      for all \(k_1 \leq \min(\rho) - \lbound{\var := \var[y] + k}\)
      and \(k_2 \geq \max(\rho) + \bound{\var := \var[y] + k}\). Hence
      \begin{equation*}
        \semi[\inte]{\var := \var[y] + k}\rho\var[w] = \interval{-\infty}{b} = \semi[\binte{k_1}{k_2}]{\var:=\var[y]+k}\rho\var
      \end{equation*}
      which is our thesis.
      
    \item
      \({\min(\semi[\inte]{\var:= \var[y] + k}\rho\var)} \neq
      -\infty\). In this case by Lemma~\ref{co:inc} it also holds that
      \begin{equation*}
        \min(\semi[\inte]{\var := \var[y] + k}\rho\var) = \min(\rho\var[y]) + k
      \end{equation*}
      hence the thesis follows for all
      \(k_1 \leq \min(\rho) - \lbound{\var := \var[y] + k}\) and
      \(k_2 \geq \max(\rho) + \bound{\var := \var[y] + k}\)
      \begin{equation*}
        \semi[\inte]{\var := \var[y] + k}\rho = \rho[\var \mapsto \rho\var[y] + k] = \semi[\binte{k_1}{k_2}]{\var := \var[y] + k}\rho
      \end{equation*}
    \end{enumerate}
  \end{enumerate}
  
  \medskip
  \noindent
  Next, we can move to the inductive cases
  
  \medskip
  
  \noindent
  \textbf{Case} (\(\com_1 \ndet \com_2\)).
  % 
  Recall that
  \({\semi[\inte]{\com_1 \ndet \com_2}\rho} =
  {\semi[\inte]{\com_1}\rho} \sqcup {\semi[\inte]{\com_2}}\). By
  inductive hypothesis it holds that
  \begin{align*}
    {\semi[\inte]{\com_1}\rho} & = \semi[\binte{k_1}{k_2}]{\com_1}\rho & \forall k_1 \leq \min(\rho) - \lbound{\com_1} \; \land \; k_2 \geq \max(\rho) + \bound{\com_1} \\
    {\semi[\inte]{\com_2}\rho} & = \semi[\binte{k_3}{k_4}]{\com_2}\rho & \forall k_3 \leq \min(\rho) - \lbound{\com_2} \; \land \; k_4 \geq \max(\rho) + \bound{\com_2}
  \end{align*}
  in particular, it holds for all \(n,m\) s.t.
  \begin{align*}
    n & \leq \min(\rho) - \lbound{\com_1} - \lbound{\com_2} & = \min(\rho) - \lbound{\com_1 \ndet \com_2} \\
    m & \geq \max(\rho) + \bound{\com_1} + \bound{\com_2} & = \max(\rho) + \bound{\com_1 \ndet \com_2}
  \end{align*}
  and we can conclude by recalling that \(\binte{n}{m}\) is closed
  under \(\sqcup\)
  \begin{equation*} {\semi[\inte_{\com_1 \ndet \com_2,
        \rho}]{\com_1}\rho} \sqcup {\semi[\inte_{\com_1 \ndet \com_2,
        \rho}]{\com_2}\rho} = {\semi[\inte_{\com_1\ndet\com_2}]{\com_1
        \ndet \com_2}\rho}
  \end{equation*}

  \medskip
  
  \noindent
  \textbf{Case} (\(\com_1 \seq \com_2\)).
  % 
  Let's recall that
  \(\semi[\inte]{\com_1 \seq \com_2}\rho =
  \semi[\inte]{\com}\left(\semi[\inte]{\com_1}\rho\right)\). By
  inductive hypothesis it holds that
  \begin{align}
    \semi[\inte]{\com_1}\rho & = \semi[\binte{k_1}{k_2}]{\com_1}\rho & \forall k_1 \leq \min(\rho) - \lbound{\com_1} \; \land \; k_2 \geq \max(\rho) + \bound{\com_1}\tag{\dag}\label{eq:ind1}\\
    \semi[\inte]{\com_2}\rho' & = \semi[\binte{k_3}{k_4}]{\com_2}\rho' & \forall k_3 \leq \min(\rho') - \lbound{\com_2} \; \land \; k_4 \geq \max(\rho') + \bound{\com_2}\tag{\ddag}\label{eq:ind2}
  \end{align}
  where \(\rho' = \semi[\inte]{\com_1}\rho\). In particular notice
  that both \eqref{eq:ind1} and \eqref{eq:ind2} hold for all \(n,m\)
  s.t.
  \begin{align*}
    m & \leq \min(\rho) - \lbound{\com_1} - \lbound{\com_2} \leq \min(\rho) - \lbound{\com_2} \\
    n & \geq \max(\rho) + \bound{\com_1} + \bound{\com_2} \geq \max(\rho) + \bound{\com_2}.
  \end{align*}
  Hence
  \begin{equation*}
    \semi[\inte]{\com_1 \seq \com_2}\rho =
    \semi[\inte]{\com_2} \left( \semi[\inte]{\com_1}\rho \right) =
    \semi[\binte{m}{n}]{\com_2}\left(
      \semi[\binte{m}{n}]{\com_1}\rho \right) =
    \semi[\binte{m}{n}]{\com_1 \seq \com_2}\rho
  \end{equation*}
  which is our thesis.

  \medskip
  
  \noindent
  \textbf{Case} (\(\fix{\com}\)).
  % 
  Let's recall that as we observed in the \(\fix{\com}\) case in
  Lemma~\ref{le:inc} that
  \[{\fix{\com} = \lfp(\lambda \mu . \semi[\inte]{\com \ndet
        \tru}\mu)}\] above \(\rho\). We can therefore build the chain
  of iterands
  \begin{align*}
    \rho_0 & \defin \rho \\
    \rho_{i+1} & \defin \semi[\inte]{\com \ndet \tru}\rho_i
  \end{align*}
  Let's consider \(\binte{m}{n}\) with
  \(n \leq \min(\rho) - (v+1)\bound{\com}\) and
  \(m \geq \max(\rho) + (v+1)\bound{\com}\) where
  \(v = |\varsof{\com}|\). We can make the following observations for
  each variable \(\var[y]\in \Var_{\com}\):

  \begin{enumerate}[label=(\roman*)]
  \item if either
    \({\max\left(\semi[\inte]{\fix\com}\rho\var[y]\right) = +\infty}\)
    or
    \({\min\left(\semi[\inte]{\fix\com}\rho\var[y]\right) =
      -\infty}\), we can recall Lemma~\ref{le:leq}:
    \begin{equation*}
      \semi[\inte]{\fix\com}\rho \sqsubseteq \semi[\binte{k_1}{k_2}]{\fix\com}\rho
    \end{equation*}
    for all \(k_1,k_2 \in \z \mid k_1 \leq k_2\) and observe that it
    means that
    \begin{align*}
      +\infty = \max\left(\semi[\inte]{\fix\com}\rho\var[y]\right) & \leq \max\left(\semi[\binte{k_1}{k_2}]{\fix\com}\rho\var[y]\right) = +\infty \\
      -\infty = \min\left(\semi[\inte]{\fix\com}\rho\var[y]\right) & \geq \min\left(\semi[\binte{k_1}{k_2}]{\fix\com}\rho\var[y]\right) = -\infty
    \end{align*}
    respectively, again for all \(k_1,k_2 \in \z \mid k_1 \leq
    k_2\).
  \item Otherwise if either
    \({\max\left(\semi[\inte]{\fix\com}\rho\var[y]\right) \neq
      +\infty}\) or
    \({\min\left(\semi[\inte]{\fix\com}\rho\var[y]\right) \neq
      -\infty}\) we can notice that for each iterand
    \(\rho_i\) it holds that
    \begin{equation*}
      \rho_{i+1} = \semi[\inte]{\com \ndet \tru}\rho_i
      \sqsubseteq
      \textstyle\bigsqcup_{i\in\n} {\left(\semi[\inte]{\com \ndet \tru}\right)}^i\rho
      =
      \semi[\inte]{\fix\com}\rho
    \end{equation*}
    Here again we have 2 cases.
    \begin{enumerate}[label=(\alph*)]
    \item In case
      \(\max\left(\semi[\inte]{\fix\com}\rho\var[y]\right) \leq
      \bound{\fix\com}\) (or
      \(\min\left(\semi[\inte]{\fix\com}\rho\var[y]\right) \geq
      -\lbound{\fix\com}\)). In this case it holds that either
      \begin{align*}
        \max(\semi[\inte]{\fix\com}\rho\var[y]) & \leq \bound{\fix\com} \leq \max(\rho) + \bound{\fix\com} \\
        \min(\semi[\inte]{\fix\com}\rho\var[y]) & \geq - \lbound{\fix\com} \geq \min(\rho) - \bound{\fix\com}
      \end{align*}

    \item  Instead, in case
      \(\max\left(\semi[\inte]{\fix\com}\right) > \bound{\fix\com}\) (or
      \(\min\left(\semi[\inte]{\fix\com}\right) < -\lbound{\fix\com}\)),
      then by Lemma~\ref{le:inc} and Lemma~\ref{co:inc} respectively, it
      holds that
      \begin{align*}
        \max\left(\semi[\inte]{\fix\com}\rho\var[y]\right) & \leq \max(\rho) + \bound{\fix\com} = \max(\rho) + (v + 1)\bound{\com} \\
        \min\left(\semi[\inte]{\fix\com}\rho\var[y]\right) & \geq \min(\rho) - \lbound{\fix\com} = \max(\rho) - (v + 1)\bound{\com} 
      \end{align*}
    \end{enumerate}
    Hence in both cases by choice of \(n,m\) we can use the inductive
    hypothesis on the iterands \(\rho_i\). %, by inductive
    % hypothesis it holds that, for each iterand \(\rho_i\)
    % \begin{equation*}
    %   \rho_{i+1} = \semi[\inte]{\com\ndet\tru}\rho_i = \semi[\binte{n}{m}]{\com\ndet\tru}\rho_i
    % \end{equation*}
    and work by induction on \(i\) on the iterands to prove that
    \begin{equation*}
      \max\left(\rho_{i}\var[y]\right) = \max\left({\left(\semi[\binte{n}{m}]{\com\ndet\tru}\right)}^{i}\rho\var[y]\right).
    \end{equation*}
    Here we present the case for \(\max\) and
    \({m \geq \max(\rho) + \bound{\fix\com}}\), but the case where we
    consider the \(\min\) value and
    \({n \leq \min(\rho) - \lbound{\fix\com}}\) is analogous.
    
    \medskip
    
    \noindent
    \textbf{Case} (\(i=0\)).  Then
    \begin{align*}
      \max(\rho_{0+1}\var[y]) & = \max\left(\semi[\inte]{\com\ndet\tru}\rho_0\var[y]\right) & \text{by induction on } \com\ndet\tru \text{ bnd by choice of } m \\
                                  & = \max\left(\semi[\binte{n}{m}]{\com\ndet\tru}\rho_0\var[y]\right) & \text{by definition of } \rho_0 \\
                                  & = \max\left({\left(\semi[\binte{n}{m}]{\com\ndet\tru}\right)}^{0+1}\rho\var[y]\right) 
    \end{align*}

    \medskip

    \noindent
    \textbf{Case} (\(i \Rightarrow i+1\)).  Then
    \begin{align*}
      \max(\rho_{i+1}\var[y]) & = \max\left(\semi[\inte]{\com\ndet\tru}\rho_i\var[y]\right)& \text{by induction on } \com \ndet \tru \text{ and by choice of } m \\ 
                              & = \max\left(\semi[\binte{n}{m}]{\com\ndet\tru}\rho_i\var[y]\right) & \text{by induction on } i \\
                              & = \max\left(\semi[\binte{n}{m}]{\com\ndet\tru}{\left(\semi[\binte{n}{m}]{\com\ndet\tru}\right)}^i\rho\var[y]\right) \\
                 & = \max\left({\left(\semi[\binte{n}{m}]{\com\ndet\tru}\right)}^{i+1}\rho\var[y]\right)
    \end{align*}
    hence by closure over \(\sqcup\)
    \begin{align*}
      \max\left(\semi[\inte]{\fix\com}\rho\var[y]\right)
      & =
        \max\left(\textstyle\bigsqcup_{i\in\n}{\left(\semi[\inte]{\com\ndet\tru}\right)}^i\rho\var[y]\right) \\
      & =
        \max\left(\textstyle\bigsqcup_{i\in\n}{\left(\semi[\binte{n}{m}]{\com\ndet\tru}\right)}^i\rho\var[y]\right) \\
      & =
        \max\left(\semi[\binte{n}{m}]{\fix\com}\rho\var[y]\right)
    \end{align*}
    Notice that the latter observations also holds for the \(\min\)
    value and for all \(n \in \z\) s.t.
    \({n \leq \min(\rho) - \lbound{\fix\com}}\).
  \end{enumerate}
  Since the two cases apply for each variable and every possible
  \(\max\) and \(\min\) value of \(\semi[\inte]{\fix\com}\rho\), we
  can say that
  \begin{equation*}
    \semi[\inte]{\fix\com}\rho = \semi[\binte{n}{m}]{\fix\com}\rho
  \end{equation*}
  for all \(n \leq \min(\rho) - \lbound{\fix\com}\),
  \(m \geq \max(\rho) + \bound{\fix\com}\) (hence
  \(\binte{n}{m} \sqsubseteq \inte\)), which is our thesis.
  % 
  % - osservo che
  % 
  % (i) se max([ fix(C) ] rho y) calcolato in I è finito, per il lemma, vale
  % 
  % max([ fix(C) ] rho y) = max(rho) + fix(C)^b = max(rho) + (n+1) C^b
  % 
  % in tutti gli iterati ho che
  % 
  % rho_i = [C+id]^i rho <= [ fix(C) ] rho y 
  % 
  % e quindi per la scelta di k, sono certo che k >= max(rho_i) +
  % C^b. Pertanto posso usare l'ipotesi induttiva su C per concludere
  % che gli iterati coincidono su I e I_k
  % 
  % (ii) se max([ fix(C) ] rho y) in I infinito, dal fatto che in I_k
  % faccio una sovrapprossimazione concludo che anche il calcolo in
  % I_k da' infinito.
  % 
  % (in questo secondo caso non mi è evidente come concludere anche
  % che il lowerbound dell'intervallo è lo stesso, ma suppongo si
  % possa o derivi dalla trattazione duale con intervalli in Z)
\end{proof}
