% !TeX root = mod.tex
\section{Computing interval semantics}
\label{sec:computingint}

Lemma~\ref{le:inc} provides an effective algorithm for computing the abstract
semantics of commands%  provided a domain that respects properties \ref{inc:hp1}
% and \ref{inc:hp2}
. This means that we can apply Lemma~\ref{le:inc} on the intervals
domain \(\inte\).  First, given a command \(\com\), the corresponding
finite set of variables \(\Var_{\com} \veq \varsof{\com}\), and an
interval environment \(\rho : \Var_{\com} \to \Int\), we define both
the \(\min\) and the \(\max\) value of an interval environment:
\begin{align*}
\max(\rho) & \veq \max \{ \max(\rho(\var)) \mid \var \in \Var_{\com}\} \\
\min(\rho) & \veq \max \{ \min(\rho(\var)) \mid \var \in \Var_{\com}\}
\end{align*}

% 
Then, when computing \(\semi[\inte]{\com} \rho\) on such \(\rho\)
having a finite domain, we can restrict to an interval domain bounded
by some constant \(k\in\n\):

\begin{definition}[Bounded interval]\label{def:boundedint}
  We define
  \(\binte{k_1}{k_2} \veq (\Var_{\com} \to \bInt{k_1}{k_2}) \cup
  \{\top, \bot\}\) where
  \begin{align*}
    \bInt{k_1}{k_2} & \veq \{ \interval{a}{b} \mid a, b \in \z \; \land \; k_1 \leq a \leq b \leq k_2\} \\
                    & \phantom{\veq} \cup \{\interval{a}{+\infty} \mid a \in \z \; \land \; a \geq k_1\} \\
                    & \phantom{\veq} \cup \{\interval{-\infty}{b} \mid b \in \z \; \land \; b \leq k_2\}
  \end{align*}
\end{definition}

We can visualize the Hasse diagram of the bounded integer domain in
Figure~\ref{fig:bound} and notice that there are no infinite ascending
chains by definition.
% 
\begin{figure}
  \centering
  \begin{tikzpicture}
    \tikzset{node distance = .5cm}
    \node (top) {\(\top\)};
    \node (p1) [below=of top] {};
    \node (1) [left=of p1]{\(\interval{k_1}{+\infty}\)};
    \node (2) [right=of p1]{\(\interval{-\infty}{k_2}\)};
    \node (4) [below=of p1]{\(\interval{k_1}{k_2}\)};
    \node (p2) [left=of 4]{};
    \node (p3) [right=of 4]{};
    \node (3) [left=of p2]{\(\interval{k_1 + 1}{+\infty}\)};
    \node (5) [right=of p3]{\(\interval{-\infty}{k_2 - 1}\)};
    \node (p4) [below=of 4]{};
    \node (6) [left=of p4]{\(\interval{k_1 + 1}{k_2}\)};
    \node (pp1) [left=of 6]{};
    \node (d1) [left=of pp1]{\(\dots\)};
    \node (7) [right=of p4]{\(\interval{k_1}{k_2 - 1}\)};
    \node (pp2) [right=of 7]{};
    \node (d2) [right=of pp2]{\(\dots\)};
    \node (9) [below=of p4]{\(\interval{k_1 + 1}{k_2 - 1}\)};
    \node (p5) [left=of 9]{};
    \node (8) [left=of p5]{\(\interval{k_1 + 2}{k_2}\)};
    \node (p6) [right=of 9]{};
    \node (10) [right=of p6]{\(\interval{k_1}{k_2 - 2}\)};
    \node (d3) [left=of 8]{\(\dots\)};
    \node (d4) [right=of 10]{\(\dots\)};
    \node (vdot1) [below=of d3]{\(\vdots\)};
    \node (vdot2) [below=of d4]{\(\vdots\)};
    \node (dbot) [below=of 9]{\(\dots\)};
    \node (pbot2) [left=of dbot]{};
    \node (pbot1) [left=of pbot2]{};
    \node (pbot3) [right=of dbot]{};
    \node (pbot4) [right=of pbot3]{};
    \node (bot) [below=of dbot]{\(\bot\)};

    \draw
    (top) edge (1) edge (2)
    (1) edge (3) edge (4)
    (2) edge (4) edge (5)
    (3) edge (6) edge (d1)
    (4) edge (6) edge (7)
    (5) edge (7) edge (d2)
    (d1) edge (d3) edge (8)
    (6) edge (8) edge (9)
    (7) edge (9) edge (10)
    (d2) edge (10) edge (d4)
    (9) edge (dbot)
    (d3) edge (pbot1)
    (8) edge (pbot2)
    (10) edge (pbot3)
    (d4) edge (pbot4)
    (d4) edge (vdot2)
    (d3) edge (vdot1)
    (dbot) edge (bot)
    (vdot1) edge (bot)
    (vdot2) edge (bot);
  \end{tikzpicture}
  \caption{\(\bInt{k_1}{k_2}\) Hasse diagram}
  \label{fig:bound}
\end{figure}
%
First we can notice that based on the values of \(k_1\) and \(k_2\) we
have a complete lattice of interval subdomains, whose top element is
the \(\inte = \binte{-\infty}{+\infty}\) domain. We can define a
notion of order for the lattices by overloading the \(\sqsubseteq\)
symbol in the following way

\begin{definition}
  For all \(\bInt{k_1}{k_2}, \bInt{k_3}{k_4}\) bounded interval
  domains
  \begin{equation*}
    \bInt{k_1}{k_2} \sqsubseteq \bInt{k_3}{k_4} \iff \interval{k_1}{k_2} \sqsubseteq \interval{k_3}{k_4}
  \end{equation*}
  \noindent
  Dually, for all \(\binte{k_1}{k_2}, \binte{k_3}{k_4}\) bounded
  interval domains
  \begin{equation*}
    \binte{k_1}{k_2} \sqsubseteq \binte{k_3}{k_4} \iff \bInt{k_1}{k_2} \sqsubseteq \bInt{k_3}{k_4}
    \left(\iff \interval{k_1}{k_2} \sqsubseteq \interval{k_3}{k_4} \right)
  \end{equation*}
\end{definition}

Notice that because of the latter definition
\(\binte{k_1}{k_2} \sqsubseteq \inte\) for all \(k_1, k_2 \in \z\)
s.t.\ \(k_1 \leq k_2\). By calling
\(\inte_b = \{\binte{i}{j} \mid i,j \in \z \; \land \; i \leq j\} \cup
\{\inte, \bot\}\) we can notice that \(\tuple{\inte_b, \sqsubseteq}\)
is a complete lattice.

\noindent
We need however special bounds, based on Lemma~\ref{le:inc}, based on
the program we are considering and the initial environment.  Hence we
define \(\inte_{\com,\rho}\) as
\({\inte_{\com,\rho} = (\Var_{\com} \to \Int_{\com,\rho}) \cup \{
  \top, \bot \}}\) where
\begin{align*}
  \Int_{\com,\rho} & = \bInt{\min(\rho)-\lbound{\com}}{\max(\rho) + \bound{\com}} \\
                   & = \{ \interval{a}{b} \mid a, b \in \n \land
                     \min(\rho) - \lbound{\com} \leq a \leq b \leq \max(\rho) + \bound{\com}\} \\
                   & \phantom{\veq} \cup \{\interval{a}{+\infty} \mid a \in\z \; \land \; a \geq \min(\rho) - \lbound{\com}\} \\
                   & \phantom{\veq} \cup \{\interval{-\infty}{b} \mid b \in\z \; \land \; b \leq \max(\rho) + \bound{\com}\} \\
                   & 
\end{align*}

We preliminarly observe that for any given \(k_1,k_2\in\n\) the
lattice \(\binte{k_1}{k_2}\) is a sub-lattice of \(\inte\)
\begin{align*}
  \eta \sqcup \theta & \in \binte{k_1}{k_2} & \forall \eta,\theta \in \binte{k_1}{k_2} \\
  \eta \sqcap \theta & \in \binte{k_1}{k_2} & \forall \eta,\theta \in \binte{k_1}{k_2}
\end{align*}

i.e., they are closed under \(\sqcap\) and \(\sqcup\). In fact if we
consider \({\eta = \interval{a}{b} \in \binte{k_1}{k_2}}\) and
\({\theta = \interval{c}{d} \in \binte{k_1}{k_2}}\) by definition both
\(b,d \leq k_2\) and \(a,c \geq k_1\). Now consider
\({\eta \sqcup \theta = \interval{\min\{a,c\}}{\max\{b,d\}}}\), where
we can observe \(\max\{b,d\} \leq k_2\) and \(\min\{a,c\} \geq k_1\)
and therefore \({\eta \sqcup \theta \in \binte{k_1}{k_2}}\) by
definition of \(\binte{k_1}{k_2}\).  Also observe that if
\(\eta \sqcap \theta \neq \bot\) then
\({\eta \sqcap \theta = \interval{\max\{a,c\}}{\min\{b,d\}}}\) and it
holds that \({\min\{b,d\} \leq k_2}\) and \({\max\{a,c\} \geq k_1}\)
and therefore \({\eta \sqcap \theta \in \binte{k_1}{k_2}}\) by definition of
\(\binte{k_1}{k_2}\).

\medskip

\noindent
Special attention must be put in defining what the analysis over a
bounded interval domain. In particular, let's first define our
abstraction and concretization maps

\begin{definition}\label{def:boundedac}
  Let \(k_1, k_2\in\n\). The abstraction map
  \({\abstr[\bInt{k_1}{k_2}] : \poset{\z} \to \bInt{k_1}{k_2}}\) is
  defined as follows
  \begin{align*}
    \abstr[\bInt{k_1}{k_2}](\emptyset) & \defin \bot \\
    \abstr[\bInt{k_1}{k_2}](P) & \defin \begin{cases}
      \interval{\min(P)}{\max(P)} & \text{if } \max(P) \leq k_2 \text{ and } \min(P) \geq k_1\\
      \interval{\min(P)}{+\infty} & \text{if } \max(P) > k_2 \text{ and } \min(P) \geq k_1 \\
      \interval{-\infty}{\max(P)} & \text{if } \min(P) < k_1 \text{ and } \max(P) \leq k_2 \\
      \interval{-\infty}{+\infty} & \text{otherwise}
    \end{cases}
  \end{align*}
  Where \(P\in\poset{\z}\). While the concretization map
  \({\concr[\bInt{k_1}{k_2}] : \bInt{k_1}{k_2} \to \poset{\z}}\)
  actually coincides with the concretization map \(\concr[\Int]\) as
  \(\bInt{k_1}{k_2}\) is a sub-lattice of \(\Int\).
\end{definition}

Let's also redefine the \(+\) operation in the \(\bInt{k_1}{k_2}\)
lattice, as adding a constant to an interval with the old definition
might overcome the bound, and therefore diverge

\begin{definition}\label{def:sumbound}
  For a nonempty interval \(\interval{a}{b} \in \bInt{k_1}{k_2}\) and
  \(c \in \n\) define
  \begin{equation*}
    \interval{a}{b} \pm c \veq \begin{cases}
      \interval{a\pm c}{b\pm c} & a \pm c \geq k_1 \land b \pm c \leq k_2 \\
      \interval{a\pm c}{+\infty} & b \pm c \geq k_2 \\
      \interval{-\infty}{b \pm c} & a \pm c \leq k_1
    \end{cases}
  \end{equation*}
  recalling that \(\pm \infty + c = \pm\infty - c = \pm\infty\).
\end{definition}

\begin{lemma}\label{le:leq}
  for all \(k_1, k_2\in\z\) s.t. \(k_1 \leq k_2\)
  \begin{equation*}
    \semi[\inte]{\com}\rho \sqsubseteq \semi[\binte{k_1}{k_2}]{\com}\rho
  \end{equation*}
  i.e., with \(\binte{k_1}{k_2}\) we have an over-approximation of \(\inte\).
\end{lemma}

\begin{proof}
  The proof works by induction on \(\com\). Let's therefore first work
  on the base cases.

  \medskip

  \noindent
  \textbf{Case} (\(\var \in S\)).
  % 
  Recall that
  \(\semi[\inte]{\var\in S}\rho = \rho[\var \mapsto
  {\abstr[\Int](\concr[\Int](\rho\var) \cap
    \concr[\Int](S))}]\). Here we have two cases:
  \begin{enumerate}[label=(\arabic*)]
  \item \({\concr[\Int](\rho\var) \cap \concr[\Int](S)} =
    \emptyset\). In this case it holds that
    \begin{equation*}
      \semi[\inte]{\var\in S}\rho = \bot \sqsubseteq \semi[\binte{k_1}{k_2}]{\var\in S}\rho
    \end{equation*}
    
  \item
    \({\concr[\Int](\rho\var) \cap \concr[\Int](S)} \neq \emptyset\).
    Hence
    \begin{equation*}
      \abstr[\Int](\concr[\Int](\rho\var) \cap \concr[\Int](S)) = \interval{a}{b} \quad \text{for some }
      {a \in \n \cup \{-\infty\}}, {b \in \n\cup\{+\infty\}}.
    \end{equation*}
    In
    this case notice that because of \(\abstr[\bInt{k_1}{k_2}]\)
    definition, it holds that
    \begin{equation*}
      \abstr[\Int](P) \sqsubseteq \abstr[\bInt{k_1}{k_2}](P)
    \end{equation*}
    for any \(P \in \poset{\z}\) and
    \(k_1, k_2 \in z \mid k_1 \leq k_2\). Therefore
    \begin{equation*}
      \abstr[\Int](\concr[\Int](\rho\var) \cap \concr[\Int](S)) \sqsubseteq
      \abstr[\bInt{k_1}{k_2}](\concr[\Int](\rho\var) \cap \concr[\Int](S))
    \end{equation*}
    which means
    \begin{equation*}
      {\semi[\inte]{\var\in S}\rho} \sqsubseteq {\semi[\binte{k_1}{k_2}]{\var\in S}\rho}
    \end{equation*}
    which is our thesis.
  \end{enumerate}

  \medskip

  \noindent
  \textbf{Case} (\(\var := k\)).
  % 
  Let's recall that
  \({\semi[\inte]{\var := k}\rho = \rho[\var \mapsto
    \abstr[\Int](\{k\})]}\) and since
  \({\abstr[\Int](\{k\})} \sqsubseteq {\abstr[\bInt{k_1}{k_2}](\{k\})}\) it holds
  that
  \begin{equation*}
    \semi[\inte]{\var := k}\rho \sqsubseteq \semi[\binte{k_1}{k_2}]{\var := k}\rho
  \end{equation*}
  which is our thesis.

  \medskip

  \noindent
  \textbf{Case} (\(\var := \var[y] + k\)).
  % 
  Recall again that
  \({\semi[\inte]{\var := \var[y] + k}\rho} = {\rho[\var \mapsto
    \rho\var[y] + k]}\). Also recall that
  \begin{equation*}
    \iota +_{\Int} j \sqsubseteq \iota +_{\bInt{k_1}{k_2}} j
  \end{equation*}
  where \(\iota \in \bInt{k_1}{k_2} \subseteq \Int\), \(j\in\n\),
  \(+_{\Int}\) is the sum operated in the intervals domain and
  \(+_{\bInt{k_1}{k_2}}\) is the sum operated in the interval domain
  bounded by \(k_1\) and \(k_2\). % That is to say \(\rho\var[y] + j\)
  % in \(\Int\) is more precise than \(\rho\var[y] + j\) in
  % \(\bInt{k_1}{k_2}\).
  Hence it holds that
  \begin{equation*}
    {\semi[\inte]{\var := \var[y] + k}\rho} \sqsubseteq {\semi[\binte{k_1}{k_2}]{\var := \var[y] + k}\rho}
  \end{equation*}
  which is our thesis.

  \medskip
  \noindent
  Now, we can work on the inductive cases.

  \medskip

  \noindent
  \textbf{Case} (\(\com_1 \ndet \com_2\)).
  % 
  Recall that
  \({\semi[\inte]{\com_1 \ndet \com_2}\rho} =
  {\semi[\inte]{\com_1}\rho} \sqcup {\semi[\inte]{\com_2}\rho}\). By
  inductive hypothesis
  \({\semi[\inte]{\com_1}\rho} \sqsubseteq
  {\semi[\binte{k_1}{k_2}]{\com_1}\rho}\) and
  \({\semi[\inte]{\com_2}\rho} \sqsubseteq
  {\semi[\binte{k_1}{k_2}]{\com_2}\rho}\). Hence we can conclude by noticing
  that \(\binte{k_1}{k_2}\) is closed under \(\sqcup\)
  \begin{equation*}
    \semi[\inte]{\com_1 \ndet \com_2}\rho \sqsubseteq
    \semi[\binte{k_1}{k_2}]{\com_1}\rho \sqcup \semi[\binte{k_1}{k_2}]{\com_2}\rho =
    \semi[\binte{k_1}{k_2}]{\com_1 \sqcup \com_2}\rho
  \end{equation*}

  \medskip

  \noindent
  \textbf{Case} (\(\com_1 \seq \com_2\)).
  % 
  Recall that
  \({\semi[\inte]{\com_1 \seq \com_2}\rho} =
  {\semi[\inte]{\com_2}\left(\semi[\inte]{\com_1}\rho\right)}\). By
  inductive hypothesis
  \begin{equation}\label{eq:a1}
    {\semi[\inte]{\com_1}\rho} \sqsubseteq {\semi[\binte{k_1}{k_2}]{\com_1}\rho}
  \end{equation}
  We can call \(\rho' = {\semi[\binte{k_1}{k_2}]{\com_1}\rho}\) and observe
  that by inductive hypothesis
  \begin{equation}\label{eq:a2} {\semi[\inte]{\com_2}\rho'}
    \sqsubseteq {\semi[\binte{k_1'}{k_2'}]{\com_2}\rho'}
  \end{equation}
  for all \(k_1',k_2'\in\n\). We can conclude by
  \begin{align*}
    {\semi[\inte]{\com_2}\left(\semi[\inte]{\com_1}\rho\right)} & \sqsubseteq {\semi[\binte{k_1}{k_2}]{\com_2}\left(\semi[\inte]{\com_1}\rho\right)} & \text{by \eqref{eq:a2}}\\
                                                                & \sqsubseteq {\semi[\binte{k_1}{k_2}]{\com_2}\left(\semi[\binte{k_1}{k_2}]{\com_1}\rho\right)} & \text{by \eqref{eq:a1} and monotonicity}
  \end{align*}
  which is our thesis

  \medskip

  \noindent
  \textbf{Case} (\(\fix{\com}\)).
  % 
  Recall that
  \(\fix{\com} = {\lfp(\lambda \mu . (\rho \sqcup
    \semi[\inte]{\com}\mu))}\), which again coincides with
  \({\lfp(\lambda \mu . (\mu \sqcup \semi[\inte]{\com}\mu))}\) above
  \(\rho\). i.e., we can build the chain of iterands
  \begin{align*}
    \rho_0 & \defin \rho \\
    \rho_{i+1} & \defin \semi[\inte]{\com}\rho_i \sqcup \rho_i = \semi[\inte]{\com \ndet \tru}\rho_i
  \end{align*}
  and the analysis becomes
  \({\semi[\inte]{\fix{\com}}\rho} =
  {\textstyle\bigsqcup_{i\in\n}\rho_i} =
  {\textstyle\bigsqcup_{i\in\n}\left(\semi[\inte]{\com \ndet
        \tru}\right)^i\rho}\).  For each iterand, the inductive
  hypothesis works for \(\com\) and for \(\tru\), and therefore it
  works for \(\com \ndet \tru\). Now we can use an inductive argument
  on \(i\) to state that for all \(i \in\n\)
  \begin{equation*}
    {\left(\semi[\inte]{\com\ndet\tru}\right)^i\rho} \sqsubseteq {\left(\semi[\binte{k_1}{k_2}]{\com\ndet\tru}\right)^i\rho}
  \end{equation*}
  Hence, by closure of \(\binte{k_1}{k_2}\) over \(\sqcup\)
  \begin{equation*}
    {\semi[\inte]{\fix{\com}}} =
    {\textstyle\bigsqcup_{i\in\n}\left(\semi[\inte]{\com\ndet\tru}\right)^i \rho}
    \quad \sqsubseteq \quad
    {\textstyle\bigsqcup_{i\in\n}\left(\semi[\binte{k_1}{k_2}]{\com\ndet\tru}\right)^i \rho} =
    {\semi[\binte{k_1}{k_2}]{\fix{\com}}}
  \end{equation*}

\end{proof}
With this consideration we can now proceed to prove that the analysis
on our bounded lattice \(\inte_{\com,\rho}\) produces the same result
as the analysis on \(\inte\).

% We could also operate uniformly on all commands, defining the
% semantics for \(\com\) in a domain with intervals bounded by
% \(\max(\rho) +\bound{\com}\)

\begin{theorem}
  Let \(\com\in \imp\) be a command. Then, for all finitely supported
  \(\rho : \Var \to \Int\), and \(k_1, k_2\in \z\) s.t.
  \(\inte_{\com,\rho} \sqsubseteq \binte{k_1}{k_2}\), i.e.,
  \(k_1 \leq \min(\rho) - \lbound{\com}\) and
  \(k_2 \geq \max(\rho) + \bound{\com}\)
  \begin{equation*}
    \semi[\inte]{\com}\rho = \semi[\binte{k_1}{k_2}]{\com}\rho
  \end{equation*}
  i.e., the abstract semantics \(\semi{\com} \rho\)
  % 
  % \semi{\fix{\com}} \rho & = \lfp{\lambda \rho'. (\semi{\com} \rho')
  % \sqcup \rho}
  computed in \(\inte\) and in \(\binte{k_1}{k_2}\) coincide.
\end{theorem}

\begin{proof}

  The proof will proceed by induction on the command \(\com\). We can
  preliminarly observe that in case the analysis results in the
  \(\top\) element (i.e., \({\semi[\inte]{\com}\rho = \top}\)), since
  for all \(k_1,k_2 \in \z\) it holds that
  \(\semi[\inte]{\com}\rho \sqsubseteq
  \semi[\binte{k_1}{k_2}]{\com}\rho\) by Lemma~\ref{le:leq} it
  trivially holds that \(\semi[\inte_{\com,\rho}]{\com}\rho = \top\)
  and therefore the two analysis coincide. We will therefore silently
  omit this case.  Now, let's explore the base cases.

  \medskip
  
  \noindent
  \textbf{Case} (\(\var \in S\)).
  % 
  Recall that
  \begin{equation*}
    \semi[\inte]{\var \in S}\rho = \begin{cases}
      \rho[\var \mapsto \abstr[\Int](\concr[\Int](\rho\var) \cap \concr[\Int](S))] & \text{if } \concr[\Int](\rho\var) \cap \concr[\Int](S) \neq \emptyset \\
      \bot & \text{otherwise}
    \end{cases}
  \end{equation*}
  and that
  \begin{equation*}
    \semi[\binte{k_1}{k_2}]{\var \in S}\rho = \begin{cases}

      \rho[\var \mapsto \abstr[\bInt{k_1}{k_2}](\concr[\Int](\rho\var) \cap \concr[\Int](S))] & \text{if } \concr[\Int](\rho\var) \cap \concr[\Int](S) \neq \emptyset \\
      \bot & \text{otherwise}
    \end{cases}
  \end{equation*}
  With \(k_1 \leq \min(\rho) - \lbound{\var \in S}\) and
  \(k_2 \geq \max(\rho) + \bound{\var \in S}\). We have 2 cases:
  \begin{enumerate}[label=(\arabic*)]
  \item \({\concr[\Int](\rho\var) \cap \concr[\Int](S) = \emptyset}\).

    \noindent
    In this case it holds that
    \begin{equation*}
      \semi[\inte]{\var\in S}\rho = \bot = \semi[\binte{k_1}{k_2}]{\var \in S}\rho.
    \end{equation*}
    
  \item \({\concr[\Int](\rho\var) \cap \concr[\Int](S) \neq
      \emptyset}\).

    \noindent
    In this case
    \({\semi[\inte]{\var \in S}\rho = \rho[\var \mapsto
      \abstr[\Int](\concr[\Int](\rho\var) \cap \concr[\Int](S))]}\) and
    we can notice that
    \(\max(\abstr[\Int](\concr[\Int](\rho\var) \cap \concr[\Int](S)))
    \leq \max(\rho\var) \leq \max(\rho)\). Therefore
    \begin{align*}
      \semi[\binte{k_1}{k_2}]{\var \in S}\rho & = \rho[\var \mapsto \abstr[\bInt{k_1}{k_2}](\concr[\Int](\rho\var) \cap \concr[\Int](S))] \\
                                              & = \rho[\var \mapsto \abstr[\Int](\concr[\Int](\rho\var) \cap \concr[\Int](S))] \\
                                              & = \semi[\inte]{\var\in S}\rho
    \end{align*}
    for all \(k_1 \leq \min(\rho) - \lbound{\var \in S}\) and
    \(k_2 \geq \max(\rho) + \bound{\var\in S}\). Which is our thesis.
  \end{enumerate}

  \medskip
  
  \noindent
  \textbf{Case} (\(\var := k\)).
  % 
  Let's recall that
  \({\semi[\inte]{\var := k}\rho = \rho[\var \mapsto \interval{k}{k}
    ]}\). Recall that we are considering \(\binte{k_1}{k_2}\) with
  \(k_1 \leq \min(\rho) - \lbound{\var := k}\) and
  \(k_2 \geq \max(\rho) + \bound{\var := k}\). Hence we can conclude by
  observing that
  \begin{equation*}
    \min(\rho) - \lbound{\var := k} \leq k \leq k \leq \max(\rho) + \bound{\var := k}
  \end{equation*}
  and therefore for all \(k_1 \leq \min(\rho) - \lbound{\var:=k}\) and
  \(k_2 \geq \max(\rho) + \bound{\var := k}\) it holds that
  \begin{equation*}
    \semi[\inte]{\var := k}\rho = \rho[\var \mapsto \interval{k}{k}] = \semi[\binte{k_1}{k_2}]{\var := k}\rho
  \end{equation*}
  which is our thesis.

  \medskip
  
  \noindent
  \textbf{Case} (\(\var := \var[y] + k\)).
  % 
  Let's recall that
  \({\semi[\inte]{\var := \var[y] + k}\rho = \rho[\var \mapsto
    \rho\var[y] + k]}\) and \(\bound{\var := \var[y] + k} =
  |k|\). Also remember that we are considering \(\binte{k_1}{k_2}\)
  with \(k_1 \leq \min(\rho) - \lbound{\var := \var[y] + k}\) and
  \(k_2 \geq \max(\rho) + \bound{\var := \var[y] + k}\). Notice that
  for all \(\Var \ni \var[w] \neq \var \) it holds that
  \(\rho\var[w] = \semi[\inte]{\var := \var[y] + k}\rho\var[w]\),
  hence we consider \(\var\). We have 2 cases
  \begin{enumerate}[label=(\arabic*)]
  \item
    \({\max(\semi[\inte]{\var := \var[y] + k}\rho\var)} =
    +\infty\). In this case we have 2 more cases.
    \begin{enumerate}[label=(\roman*)]
    \item
      \({\min(\semi[\inte]{\var := \var[y] + k}\rho\var)} =
      -\infty\). In this case
      \({\semi[\inte]{\var := \var[y] + k}\rho} = \top\). Since by
      Lemma~\ref{le:leq}
      \(\semi[\inte]{\com}\rho \sqsubseteq
      \semi[\binte{k_1}{k_2}]{\com}\rho\) for all \(k_1, k_2 \in \z\)
      s.t. \(k_1 \leq k_2\), which means that
      \begin{equation*}
        \semi[\inte]{\var := \var[y] + k}\rho\var = \interval{-\infty}{+\infty} = \semi[\binte{k_1}{k_2}]{\var := \var[y] + k}\rho\var
      \end{equation*}
      in particular for all
      \(k_1 \leq \min(\rho) - \lbound{\var := \var[y] + k}\) and
      \(k_2 \geq \max(\rho) + \bound{\var := \var[y] + k}\), which is
      our thesis.
    \item
      \({\min(\semi[\inte]{\var := \var[y] + k}\rho\var)} \neq -
      \infty\). In this case by Lemma~\ref{co:inc}
      \begin{equation*}
        \min(\semi[\inte]{\var := \var[y] + k}\rho\var) = \min(\rho\var[y]) + k
      \end{equation*}
      and therefore
      \({\semi[\inte]{\var := \var[y] + k}\rho\var} =
      \interval{a}{+\infty} = {\semi[\binte{k_1}{k_2}]{\var := \var[y]
          + k}\rho\var}\) for some
      \(a \geq \min(\rho) - \lbound{\var := \var[y] + k}\), and for
      all \(k_1 \leq \min(\rho)- \lbound{\var := \var[y] + k}\) and
      \(k_2 \geq \max(\rho) + \bound{\var := \var[y] + k}\). Hence our
      thesis holds.

    \end{enumerate}
  \item
    \({\max(\semi[\inte]{\var := \var[y] + k}\rho\var)} \neq
    +\infty\). In this case by Lemma~\ref{le:inc} it holds that
    \begin{equation*}
      \max(\semi[\inte]{\var:= \var[y] + k}\rho\var) = \max(\rho\var[y]) + k
    \end{equation*}
    Here we have 2 more cases depending on the value of
    \(\min(\semi[\inte]{\var:= \var[y] + k}\rho\var)\):
    \begin{enumerate}[label=(\roman*)]
    \item
      \({\min(\semi[\inte]{\var:= \var[y] + k}\rho\var[w])} =
      -\infty\). In this case
      \(\semi[\inte]{\var := \var[y] + k}\rho\var[w] =
      \interval{-\infty}{b}\) with
      \(b \leq \max(\rho) + \bound{\var := \var[y] + k}\), in
      particular, because of the semantics and Lemma~\ref{le:inc} it
      holds that
      \begin{equation*}
        \max(\semi[\inte]{\var := \var[y] + k}\rho\var) = \max(\semi[\binte{k_1}{k_2}]{\var := \var[y] + k}\rho\var)
      \end{equation*}
      for all \(k_1 \leq \min(\rho) - \lbound{\var := \var[y] + k}\)
      and \(k_2 \geq \max(\rho) + \bound{\var := \var[y] + k}\). Hence
      \begin{equation*}
        \semi[\inte]{\var := \var[y] + k}\rho\var[w] = \interval{-\infty}{b} = \semi[\binte{k_1}{k_2}]{\var:=\var[y]+k}\rho\var
      \end{equation*}
      which is our thesis.
      
    \item
      \({\min(\semi[\inte]{\var:= \var[y] + k}\rho\var)} \neq
      -\infty\). In this case by Lemma~\ref{co:inc} it also holds that
      \begin{equation*}
        \min(\semi[\inte]{\var := \var[y] + k}\rho\var) = \min(\rho\var[y]) + k
      \end{equation*}
      hence the thesis follows for all
      \(k_1 \leq \min(\rho) - \lbound{\var := \var[y] + k}\) and
      \(k_2 \geq \max(\rho) + \bound{\var := \var[y] + k}\)
      \begin{equation*}
        \semi[\inte]{\var := \var[y] + k}\rho = \rho[\var \mapsto \rho\var[y] + k] = \semi[\binte{k_1}{k_2}]{\var := \var[y] + k}\rho
      \end{equation*}
    \end{enumerate}
  \end{enumerate}
  
  \medskip
  \noindent
  Next, we can move to the inductive cases
  
  \medskip
  
  \noindent
  \textbf{Case} (\(\com_1 \ndet \com_2\)).
  % 
  Recall that
  \({\semi[\inte]{\com_1 \ndet \com_2}\rho} =
  {\semi[\inte]{\com_1}\rho} \sqcup {\semi[\inte]{\com_2}}\). By
  inductive hypothesis it holds that
  \begin{align*}
    {\semi[\inte]{\com_1}\rho} & = \semi[\binte{k_1}{k_2}]{\com_1}\rho & \forall k_1 \leq \min(\rho) - \lbound{\com_1} \; \land \; k_2 \geq \max(\rho) + \bound{\com_1} \\
    {\semi[\inte]{\com_2}\rho} & = \semi[\binte{k_3}{k_4}]{\com_2}\rho & \forall k_3 \leq \min(\rho) - \lbound{\com_2} \; \land \; k_4 \geq \max(\rho) + \bound{\com_2}
  \end{align*}
  in particular, it holds for all \(n,m\) s.t.
  \begin{align*}
    n & \leq \min(\rho) - \lbound{\com_1} - \lbound{\com_2} & = \min(\rho) - \lbound{\com_1 \ndet \com_2} \\
    m & \geq \max(\rho) + \bound{\com_1} + \bound{\com_2} & = \max(\rho) + \bound{\com_1 \ndet \com_2}
  \end{align*}
  and we can conclude by recalling that \(\binte{n}{m}\) is closed
  under \(\sqcup\)
  \begin{equation*} {\semi[\inte_{\com_1 \ndet \com_2,
        \rho}]{\com_1}\rho} \sqcup {\semi[\inte_{\com_1 \ndet \com_2,
        \rho}]{\com_2}\rho} = {\semi[\inte_{\com_1\ndet\com_2}]{\com_1
        \ndet \com_2}\rho}
  \end{equation*}

  \medskip
  
  \noindent
  \textbf{Case} (\(\com_1 \seq \com_2\)).
  % 
  Let's recall that
  \(\semi[\inte]{\com_1 \seq \com_2}\rho =
  \semi[\inte]{\com}\left(\semi[\inte]{\com_1}\rho\right)\). By
  inductive hypothesis it holds that
  \begin{align}
    \semi[\inte]{\com_1}\rho & = \semi[\binte{k_1}{k_2}]{\com_1}\rho & \forall k_1 \leq \min(\rho) - \lbound{\com_1} \; \land \; k_2 \geq \max(\rho) + \bound{\com_1}\tag{\dag}\label{eq:ind1}\\
    \semi[\inte]{\com_2}\rho' & = \semi[\binte{k_3}{k_4}]{\com_2}\rho' & \forall k_3 \leq \min(\rho') - \lbound{\com_2} \; \land \; k_4 \geq \max(\rho') + \bound{\com_2}\tag{\ddag}\label{eq:ind2}
  \end{align}
  where \(\rho' = \semi[\inte]{\com_1}\rho\). In particular notice
  that both \eqref{eq:ind1} and \eqref{eq:ind2} hold for all \(n,m\)
  s.t.
  \begin{align*}
    m & \leq \min(\rho) - \lbound{\com_1} - \lbound{\com_2} \leq \min(\rho) - \lbound{\com_2} \\
    n & \geq \max(\rho) + \bound{\com_1} + \bound{\com_2} \geq \max(\rho) + \bound{\com_2}.
  \end{align*}
  Hence
  \begin{equation*}
    \semi[\inte]{\com_1 \seq \com_2}\rho =
    \semi[\inte]{\com_2} \left( \semi[\inte]{\com_1}\rho \right) =
    \semi[\binte{m}{n}]{\com_2}\left(
      \semi[\binte{m}{n}]{\com_1}\rho \right) =
    \semi[\binte{m}{n}]{\com_1 \seq \com_2}\rho
  \end{equation*}
  which is our thesis.

  \medskip
  
  \noindent
  \textbf{Case} (\(\fix{\com}\)).
  % 
  Let's recall that as we observed in the \(\fix{\com}\) case in
  Lemma~\ref{le:inc} that
  \[{\fix{\com} = \lfp(\lambda \mu . \semi[\inte]{\com \ndet
        \tru}\mu)}\] above \(\rho\). We can therefore build the chain
  of iterands
  \begin{align*}
    \rho_0 & \defin \rho \\
    \rho_{i+1} & \defin \semi[\inte]{\com \ndet \tru}\rho_i
  \end{align*}
  Let's consider \(\binte{m}{n}\) with
  \(n \leq \min(\rho) - (v+1)\bound{\com}\) and
  \(m \geq \max(\rho) + (v+1)\bound{\com}\) where
  \(v = |\varsof{\com}|\). We can make the following observations for
  each variable \(\var[y]\in \Var_{\com}\):

  \begin{enumerate}[label=(\roman*)]
  \item if either
    \({\max\left(\semi[\inte]{\fix\com}\rho\var[y]\right) = +\infty}\)
    or
    \({\min\left(\semi[\inte]{\fix\com}\rho\var[y]\right) =
      -\infty}\), we can recall Lemma~\ref{le:leq}:
    \begin{equation*}
      \semi[\inte]{\fix\com}\rho \sqsubseteq \semi[\binte{k_1}{k_2}]{\fix\com}\rho
    \end{equation*}
    for all \(k_1,k_2 \in \z \mid k_1 \leq k_2\) and observe that it
    means that
    \begin{align*}
      +\infty = \max\left(\semi[\inte]{\fix\com}\rho\var[y]\right) & \leq \max\left(\semi[\binte{k_1}{k_2}]{\fix\com}\rho\var[y]\right) = +\infty \\
      -\infty = \min\left(\semi[\inte]{\fix\com}\rho\var[y]\right) & \geq \min\left(\semi[\binte{k_1}{k_2}]{\fix\com}\rho\var[y]\right) = -\infty
    \end{align*}
    respectively, again for all \(k_1,k_2 \in \z \mid k_1 \leq
    k_2\).
  \item Otherwise if either
    \({\max\left(\semi[\inte]{\fix\com}\rho\var[y]\right) \neq
      +\infty}\) or
    \({\min\left(\semi[\inte]{\fix\com}\rho\var[y]\right) \neq
      -\infty}\) we can notice that for each iterand
    \(\rho_i\) it holds that
    \begin{equation*}
      \rho_{i+1} = \semi[\inte]{\com \ndet \tru}\rho_i
      \sqsubseteq
      \textstyle\bigsqcup_{i\in\n} {\left(\semi[\inte]{\com \ndet \tru}\right)}^i\rho
      =
      \semi[\inte]{\fix\com}\rho
    \end{equation*}
    Here again we have 2 cases.
    \begin{enumerate}[label=(\alph*)]
    \item In case
      \(\max\left(\semi[\inte]{\fix\com}\rho\var[y]\right) \leq
      \bound{\fix\com}\) (or
      \(\min\left(\semi[\inte]{\fix\com}\rho\var[y]\right) \geq
      -\lbound{\fix\com}\)), by choice of \(n,m\) %, by inductive
      % hypothesis it holds that, for each iterand \(\rho_i\)
      % \begin{equation*}
      %   \rho_{i+1} = \semi[\inte]{\com\ndet\tru}\rho_i = \semi[\binte{n}{m}]{\com\ndet\tru}\rho_i
      % \end{equation*}
      we can use the
      inductive hypothesis and say that for each iterand
      \begin{equation*}
        \rho_i = {\left(\semi[\inte]{\com\ndet\tru}\right)}^i\rho = {\left(\semi[\binte{n}{m}]{\com\ndet\tru}\right)}^i\rho
      \end{equation*}
      hence by closure over \(\sqcup\)
      \begin{equation*}
        \semi[\inte]{\fix\com}\rho
        =
        \textstyle\bigsqcup_{i\in\n}\rho_i
        =
        \textstyle\bigsqcup_{i\in\n}{\left(\semi[\binte{n}{m}]{\com\ndet\tru}\right)}^i\rho
        =
        \semi[\binte{n}{m}]{\fix\com}\rho.
      \end{equation*}
      
      
    \item  Instead, in case
      \(\max\left(\semi[\inte]{\fix\com}\right) > \bound{\fix\com}\) (or
      \(\min\left(\semi[\inte]{\fix\com}\right) < -\lbound{\fix\com}\)),
      then by Lemma~\ref{le:inc} and Lemma~\ref{co:inc} respectively, it
      holds that
      \begin{align*}
        \max\left(\semi[\inte]{\fix\com}\rho\var[y]\right) & \leq \max(\rho) + \bound{\fix\com} = \max(\rho) + (v + 1)\bound{\com} \\
        \min\left(\semi[\inte]{\fix\com}\rho\var[y]\right) & \geq \min(\rho) - \lbound{\fix\com} = \max(\rho) - (v + 1)\bound{\com} 
      \end{align*}
      
      hence again, by choice of \(n,m\) we can use the inductive
      hypothesis in both cases and state that
      \begin{equation*}
        \semi[\inte]{\com \ndet \tru}\rho_i
        =
        {\left(\semi[\inte]{\com\ndet\tru}\right)}^i\rho
        =
        {\left(\semi[\binte{n}{m}]{\com\ndet\tru}\right)}^i\rho
      \end{equation*}
      and by closure over \(\sqcup\)
      \begin{equation*}
        \semi[\inte]{\fix\com}\rho
        =
        \textstyle\bigsqcup_{i\in\n}\rho_i
        =
        \textstyle\bigsqcup_{i\in\n}{\left(\semi[\binte{n}{m}]{\com\ndet\tru}\right)}^i\rho
        =
        \semi[\binte{n}{m}]{\fix\com}\rho.
      \end{equation*}
    \end{enumerate}
  \end{enumerate}
  % 
  % - osservo che
  % 
  % (i) se max([ fix(C) ] rho y) calcolato in I è finito, per il lemma, vale
  % 
  % max([ fix(C) ] rho y) = max(rho) + fix(C)^b = max(rho) + (n+1) C^b
  % 
  % in tutti gli iterati ho che
  % 
  % rho_i = [C+id]^i rho <= [ fix(C) ] rho y 
  % 
  % e quindi per la scelta di k, sono certo che k >= max(rho_i) +
  % C^b. Pertanto posso usare l'ipotesi induttiva su C per concludere
  % che gli iterati coincidono su I e I_k
  % 
  % (ii) se max([ fix(C) ] rho y) in I infinito, dal fatto che in I_k
  % faccio una sovrapprossimazione concludo che anche il calcolo in
  % I_k da' infinito.
  % 
  % (in questo secondo caso non mi è evidente come concludere anche
  % che il lowerbound dell'intervallo è lo stesso, ma suppongo si
  % possa o derivi dalla trattazione duale con intervalli in Z)
\end{proof}
