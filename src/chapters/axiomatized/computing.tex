\section{Bounding interval analysis}\label{sec:computing}
The following section aims at proving that by bounding the interval
domain to a subdomain with no infinite ascending chains, i.e., where
every chain converges in finite time, we can still compute the most
precise interval raresentation for each variable in our program.

\todo[inline]{intro} To do so, we first prove an easy graph-theoretic
property which will later be helpful.
% We focus on graphs with edges labelled over integer numbers. 
Consider a finite directed and edge-weighted graph
\(\langle X,\to\rangle\) where
\(\to \mathord{\subseteq} X \times \mathbb{Z} \times X\) and
\(x \to_h x'\) denotes that \((x, h, x') \in \mathbin{\to}\). Consider
a finite path in \(\langle X,\to\rangle\)
\[p= x_0 \to_{h_0} x_1 \to_{h_1} x_2 \to_{h_2} \ldots \to_{h_{\ell-1}}
  x_{\ell}\] where:
\begin{enumerate}[label=(\roman*).]
\item \(\ell\geq 1\)
\item the carrier size of \(p\) is \(s(p) \veq |\{x_0,...,x_\ell\}|\)
\item the weight of \(p\) is \(w(p) \veq \Sigma_{k=0}^{\ell-1} h_k\)
\item the length of \(p\) is \(|p|\veq \ell\)
\item given indices \(0 \leq i < j \leq \ell\), \(p_{i,j}\) denotes
  the subpath of \(p\) given by
  \(x_i \to_{h_i} x_{i+1} \to_{i+1} \cdots \to_{h_{j-1}} x_j\) whose
  length is \(j-i\); \(p_{i,j}\) is a cycle if \(x_i=x_j\).
\end{enumerate}

% Graph lemma
\begin{lemma}[\textbf{Positive cycles in weighted directed graphs}]
  \label{le:cycles}
  Let \(p\) be a finite path
  \[p = x_0 \to_{h_0} x_1 \to_{h_1} x_2 \to_{h_2} \cdots \to_{h_{\ell-1}} x_{\ell}\]
  with
  \(m \veq \max \{ |h_j| \mid j \in \{0, \ldots, \ell-1\}\}\in \nat\) and 
  \(w(p) > (|X|-1)m\). Then, \(p\) has a subpath which is a cycle having a 
  strictly positive weight.
\end{lemma}

% Graph lemma proof 
\begin{proof}
  First note that \(w(p) = \Sigma_{k=0}^{\ell-1} h_k > (|X|-1)m\) implies that \(|p|=\ell \geq |X|\). 
  Then, we show our claim by induction on \(|p|=\ell \geq |X|\).


  \noindent
  \((|p|=|X|)\): Since the path \(p\) includes exactly \(|X|+1=\ell+1\) nodes, there exist
  indices \(0\leq i < j\leq \ell\) such that \(x_i=x_j\), i.e., \(p_{i,j}\) is a subpath of \(p\) which is a
  cycle. Moreover, since this cycle \(p_{i,j}\) includes at least one edge, we have that
  \begin{align*}
    w(p_{i,j}) = w(p) - (\Sigma_{k=0}^{i-1} h_k + \Sigma_{k=j}^{\ell-1}
    h_k) &> & \text{as \(w(p) > (|X|-1)m\)}\\
    \left(|X|-1\right)m - (\Sigma_{k=0}^{i-1} h_k + \Sigma_{k=j}^{\ell-1}
    h_k) &\geq & \text{as \(\Sigma_{k=0}^{i-1} h_k + \Sigma_{k=j}^{\ell-1}
        h_k \leq (\ell-1)m\)}\\
    (|X|-1)m - (\ell-1)m &= & \text{[as \(\ell=|X|\)]}  \\  
    (|X|-1)m - (|X|-1)m &= 0
  \end{align*}
  so that \(w(p_{i,j})>0\) holds. 

  \medskip
  \noindent

  \noindent
  \((|p|>|X|)\): Since the path \(p\) includes at least \(|X|+2\) nodes, as in
  the base case, we have that \(p\) has a subpath which is a cycle. Then, we consider a
  cycle \(p_{i,j}\) in \(p\), for some indices \(0\leq i < j\leq \ell\), which is maximal, i.e.,  such that
  if  \(p_{i',j'}\) is a cycle in \(p\),  for some 
  \(0\leq i' < j'\leq \ell\), then \(p_{i,j}\) is not a proper subpath of \(p_{i',j'}\).

  \noindent
  If \(w(p_{i,j})>0\) then we are done. Otherwise we have that  \(w(p_{i,j})\leq 0\) 
  and we consider the path \(p'\) obtained from \(p\) by stripping off the cycle \(p_{i,j}\), i.e.,
  \begin{center}
    \(p' \equiv \overbrace{x_0 \to_{h_0} x_1 \to_{h_1} \cdots \to_{h_{i-1}} x_i}^{p'_0,i} =
    \overbrace{
      x_j\to_{h_{j+1}} \ldots \to_{h_{\ell-1}} x_{\ell}}^{p'_{j+1,\ell}}\)
  \end{center}
  Since \(|p'| < |p|\) and
  \(w(p')=w(p)-w(p_{i,j})\geq w(p)> (|X|-1)m\), we can apply the
  inductive hypothesis on \(p'\).  We therefore derive that \(p'\) has
  a subpath \(q\) which is a cycle having strictly positive
  weight. This cycle \(q\) is either entirely in \(p'_{0,i}\) or in
  \(p'_{j+1,\ell}\), otherwise \(q\) would include the cycle
  \(p_{i,j}\) thus contradicting the maximality of \(p_{i,j}\). Hence,
  \(q\) is a cycle in the original path \(p\) having a strictly
  positive weight.
\end{proof}
% We can now proceed to prove the fundamental lemma we will later use in
% Theorem~\ref{} to conclude. The following Lemma proves that for
% abstract domains that respect the properties \ref{inc:hp1} and
% \ref{inc:hp2} it holds that the maximum value each value can reach
% without diverging to infinity strictly depends on the constants inside
% of the program itself.
% \medskip
% More in detail, for the interval domain \(\Int\) and its variable-wise
% lifing \(\inte\) the following lemma holds:
We also preliminarly need to define the \(\max\) function for
intervals. More in detail \(\max : \Int \to \z\) is defined as follows
\begin{align*}
  \max(\bot) & \defin - \infty \\
  \max(\interval{a}{b}) & \defin b
\end{align*}

% Increment lemma
\begin{lemma}\label{le:inc}
  Let \(\com\in \imp\).
  % Then consider a function \({\max_{\Int} : \Int \to \z}\) s.t. for
  % all \({a\in\Int}\) it holds that \(x \leq \max_\Int(a)\) for all
  % \({x \in \concr[\Int](a)}\).
  For all \(\eta \in \inte\) and \(\var[y] \in \Var\), if
  \(\max(\semi{\com} \eta \var[y]) \neq +\infty\) and
  \(\max(\semi{\com}\eta \var[y]) > \bound{\com}\) then there exist a
  variable \(\var[z] \in \Var\) and an integer \(h \in \mathbb{Z}\)
  such that \(|h| \leq \bound{\com}\) and the following two properties
  hold:
  \begin{enumerate}[label=(\roman*)]
  \item\label{point1} \(\max(\semi{\com}\eta \var[y]) = \max(\eta \var[z]) + h\); 
  \item\label{point2} for all \(\eta' \in \inte\), if \(\eta' \sqsupseteq \eta\)
    % and \(\semi{\com}\eta' \neq \top\)
    then
    \(\max(\semi{\com}\eta' \var[y]) \geq \max(\eta' \var[z]) + h\).
  \end{enumerate}
  % dually, if \(\min(\semi{\com}\eta\var[y]) \neq - \infty\) and
  % \(\min(\semi{\com}\eta\var[y]) < \lbound{\com}\) then there exists a
  % variable \(\var[w] \in \Var\) and an integer \(i \in \z\) s.t.
  % \(|i| \leq \lbound{\com}\) and the following two properties hold:
  % \begin{enumerate}[label=(\roman*)]
  % \item \(\min(\semi{\com}\eta \var[y]) = \min(\eta \var[w]) + i\); \label{point1min}
  % \item  for all \(\eta' \in \Int\), if \(\eta' \sqsupseteq \eta\)
  %   % and \(\semi{\com}\eta' \neq \top\)
  %   then
  %   \(\min(\semi{\com}\eta' \var[y]) \geq \max(\eta' \var[w]) + i\). \label{point2min}
  % \end{enumerate}
\end{lemma}

% Increment lemma proof
\begin{proof}
   The proof is by
  structural induction on the command \(\com\in \imp\).
  %
  We preliminarly observe that we can safely assume
  \(\eta \neq \bot\).
  % In fact, if \(\eta = \top\) then \(\semi{\com} \eta = \top\), so that the
  % hypothesis \(\semi{\com} \eta \neq \top\) would not
  % be satisfied.
  In fact, if \(\eta = \bot\) then \(\semi{\com} \bot = \bot\) and
  thus \(\max(\semi{\com} \eta \var[y]) = 0 \leq \bound{\com}\),
  against the hypothesis
  \(\max(\semi{\com}\eta \var[y]) > \bound{\com}\). Moreover, when
  quantifying over \(\eta'\) such that \(\eta' \sqsupseteq \eta\) in
  \ref{point1}, if \(\max(\semi{\com}\eta' \var[y]) = + \infty\)
  holds, then
  \(\max(\semi{\com}\eta' \var[y]) \geq \max(\eta' \var[z]) + h\)
  trivially holds, hence we will sometimes silently omit to consider
  this case.
  
  \medskip
  
  \noindent
  \textbf{Case} (\(\var \in S\))\\
  % 
  Take \(\eta \in \inte\) and assume
  \(\infty\neq\max(\semi{\var \in S}\eta \var[y]) > \bound{\var \in S}\).
  %
  Clearly \(\semi{\var \in S}\eta \neq \bot\), otherwise we would get
  the contradiction
  \(\max(\semi{\var \in S}\eta \var[y]) = -\infty \leq \bound{\var \in
    S}\).
  
  We distinguish two cases:
  \begin{itemize}
    
  \item If \(\var[y] \neq \var\), then for all \(\eta' \in \inte\) such
    that \(\eta \sqsubseteq \eta'\) it holds
    \[\bot \neq \semi{\var \in S}\eta' = \eta'[\var \mapsto
      {\abstr[\Int](\concr[\Int](\eta(\var)) \cap \concr[\Int](S))}]\]
    and thus
    \begin{equation*}
      \max(\semi{\var \in S}\eta' \var[y]) = \max(\eta' \var[y]) = \max(\eta' \var[y]) + 0
    \end{equation*}
    hence the thesis follows with \(\var[z]=\var[y]\) and \(h = 0\).

  \item If \(\var[y] = \var[x]\) then \(\eta(\var) \in \Int\) and
    \begin{equation*}
      \max(\semi{\var \in S}\eta \var[y]) = \max(\abstr[\Int](\concr[\Int](\eta\var) \cap \concr[\Int](S)))
    \end{equation*}
    Note that it cannot be \(\max(S)\in \z\). Otherwise, by
    Definition~\ref{de:bound},
    \(\max ({\abstr[\Int](\concr[\Int](\eta\var) \cap \concr[\Int](S))})
    \leq \max(S)= \bound{\var \in S}\), violating the assumption
    \(\max(\semi{\var \in S}\eta \var[y]) > \bound{\var \in S}\).
    Hence, \(\max(S) = +\infty\) must hold and therefore % by
    % \ref{inc:hp1}
    \(\max ({\abstr[\Int](\concr[\Int](\eta\var) \cap
      \concr[\Int](S))}) = \max(\eta(\var)) = \max(\eta(\var)) +
    0\). It is immediate to check that the same holds for all
    \(\eta' \sqsupseteq \eta\), i.e.,
    \begin{equation*}
      \max(\abstr[\Int](\concr[\Int](\eta'\var) \cap \concr[\Int](S))) = \max(\eta'\var) + 0
    \end{equation*}
    % \[
    % \max (\eta'(\var) \sqcap \abstr(S)) = \max(\eta'(\var))=
    % \max(\eta'(\var))+0
    % \]
    and thus the thesis follows with  \(\var[z]=\var[y]=\var[x]\) and \(h=0\).
  \end{itemize}  
  
  \medskip
  
  \noindent
  \textbf{Case} (\(\var := k\))
  % 
  Take \(\eta \in \inte\) and assume
  \(\max(\semi{\var := k}\eta \var[y]) > \bound{\var := k} = |k|\).

  Observe that it cannot be \(\var = \var[y]\). In fact, since
  \(\semi{\var := k}\eta = \eta[\var \mapsto \abstr[\Int](\{k\})]\),
  we would have
  \(\semi{\var := k}\eta \var[y] = \abstr[\Int](\{k\}) =
  \interval{k}{k}\) and thus % \todo[inline]{Forse non ovvio perchè
    % \(\abstr[\Int](\{k\})\) può non rappresentare esattamente
    % \(\{k\}\).}
  \begin{equation*}
    \max(\semi{\var := k}\eta \var[y]) = k  \leq \bound{\var := k}
  \end{equation*}
  violating the assumption.
  %
  Therefore, it must be \(\var[y] \neq \var\). Now, for all
  \(\eta' \sqsupseteq \eta\), we have
  \(\semi{\var := k}\eta' \var[y] = \eta' \var[y]\) and thus
  \begin{center}
    \(\max(\semi{\var := k}\eta' \var[y]) = \max(\eta' \var[y]) =
    \max(\eta' \var[y]) + 0\),
  \end{center}
  hence the thesis holds with \(h =0 \leq \bound{\var :=k}\) and \(\var[z] = \var[y]\).
  
  % ----------------------------------------------
  
  \medskip
  
  \noindent
  \textbf{Case} (\(\var := \var[w] + k\))
  % 
  Take \(\eta \in \inte\) and assume
  \(\max(\semi{\var := \var[w] + k}\eta \var[y]) > \bound{\var :=
    \var[w] + k} = |k|\).
  %
  Recall that
  \(\semi{\var := \var[w] + k}\eta = \eta[\var \mapsto \eta \var[w] + k]\).
  
  We distinguish two cases:
  \begin{itemize}
    
  \item If \(\var[y] \neq \var\), then for all \(\eta' \sqsupseteq \eta\), we have
    \(\semi{\var := \var[w] + k}\eta' \var[y] = \eta' \var[y]\) and thus
    \begin{equation*}
      \max(\semi{\var := \var[w] + k}\eta' \var[y]) = \max(\eta' \var[y])
    \end{equation*}
    hence the thesis follows with
    \(h =0 \leq \bound{\var := \var[w] + k}\) and \(\var[z] = \var[y]\).
    
  \item 
    If \(\var = \var[y]\) then  for all \(\eta' \sqsupseteq \eta\), we have
    \(\semi{\var := \var[w] + k}\eta' \var[y] = \eta' \var[w] +
    k\) and thus
    \begin{equation*}
      \max(\semi{\var := \var[w] + k}\eta' \var[y]) = \max(\eta' \var[w]) +
      k
    \end{equation*}
    hence, the thesis follows with \(h = k\) (recall that
    \(k \leq |k| = \bound{\var := \var[w] + k}\)) and
    \(\var[z] = \var[w]\).
  \end{itemize}

  % ----------------------------------------------
  
  % \medskip
  
  % \noindent
  % \textbf{Case} (\(\var := \var[w] - k\))
  % % 
  % Take \(\eta \in \Int\) and assume
  % \(\max(\semi{\var := \var[w] - k}\eta \var[y]) > \bound{\var := \var[w] - k} = k\).
  % %
  % Recall that
  % \(\semi{\var := \var[w] - k}\eta = \eta[\var \mapsto \eta \var[w] - k]\).
  
  % We distinguish two cases:
  % \begin{itemize}
    
  % \item If \(\var[y] \neq \var\), then for all
  %   \(\eta' \in \Int\) such that
  %   \(\eta \sqsubseteq \eta'\), we have
  %   \(\semi{\var := \var[w] - k}\eta' \var[y] = \eta' \var[y]\) and thus
  %   \begin{center}
  %     \(\max(\semi{\var := \var[w] - k}\eta' \var[y]) = \max(\eta' \var[y])\).
  %   \end{center}
  %   hence the thesis holds, with
  %   \(h =0 \leq \bound{\var := \var[w] - k}\) and \(\var[z] = \var[y]\).
    
  % \item If \(\var = \var[y]\) then for all \(\eta' \in \Int\) such
  %   that \(\eta \sqsubseteq \eta'\), we have
  %   \(\semi{\var := \var[w] - k}\eta' \var[y] = \eta' \var[w] - k\) and
  %   thus
  %   \begin{center}
  %     \(\max(\semi{\var := \var[w] - k}\eta' \var[y]) = \max(\eta' \var[w]) -
  %     k\).
  %   \end{center}
  %   Note that the assumption \(\max(\semi{\var := \var[w] - k}\eta \var[y]) > k\) and thus
  %   \(\max(\semi{\var := \var[w] - k}\eta' \var[y]) > k\) ensures that subtraction is not
  %   truncated on the maximum.
    
  %   Hence the thesis holds, with \(h = -k\), hence \(|h| = \bound{\var := \var[w] - k}\),
  %   and \(\var[z] = \var[w]\).
  % \end{itemize}
  
  % ----------------------------------------------
  
  \medskip
  
  \noindent
  \textbf{Case} (\(\com_1 \ndet \com_2\))
  % 
  Take \(\eta \in \inte\) and assume
  \(\max(\semi{\com_1 \ndet \com_2}\eta) > \bound{\com_1 \ndet \com_2}
  = \bound{\com_1} + \bound{\com_2}\).  Recall that
  \(\semi{\com_1 \ndet \com_2} \eta = \semi{\com_1}\eta \sqcup
  \semi{\com_2}\eta\).
  % 
  Hence, since
  \(\max(\semi{\com_1 \ndet \com_2} \eta \var[y]) \neq \infty\), we
  have that
  \(\max(\semi{\com_1} \eta \var[y]) \neq \infty \neq
  \max(\semi{\com_2}\eta \var[y])\).  Moreover
  \begin{align*}
    \max(\semi{\com_1 \ndet \com_2}\eta \var[y])& =  \max(\semi{\com_1}\eta \var[y] \sqcup \semi{\com_2}\eta \var[y]) \\ 
    & = \max \{ \max(\semi{\com_1}\eta \var[y]), \max(\semi{\com_2}\eta \var[y])\} % & \text{by \ref{inc:hp2}} 
  \end{align*}

  Thus
  \(\max(\semi{\com_1 \ndet \com_2}\eta \var[y]) =
  \max(\semi{\com_i}\eta \var[y])\) for some \(i \in \{1,2\}\). We can
  assume, without loss of generality, that the maximum is realized by
  the first component, i.e.,
  \(\max(\semi{\com_1 \ndet \com_2}\eta \var[y]) =
  \max(\semi{\com_1}\eta \var[y]) > \bound{\com_1 \ndet
    \com_2}\). Hence we can use the inductive hypothesis on \(\com_1\)
  and state that there exists \(h \in \mathbb{Z}\) with
  \(|h| \leq \bound{\com_1}\) and \(\var[z] \in \Var\) such that
  \(\max(\semi{\com_1}\eta \var[y]) = \max(\eta \var[z]) + h\) and for
  all \(\eta' \in \inte\), \(\eta \sqsubseteq \eta'\),
  \[
  \max(\semi{\com_1}\eta' \var[y]) \geq \max(\eta' \var[z]) + h
  \]

  Therefore 
  \[
  \max(\semi{\com_1 \ndet \com_2}\eta \var[y])
  = \max(\semi{\com_1}\eta \var[y]) = \max(\eta \var[z]) + h
  \]
  %
  and and for
  all \(\eta' \in \inte\), \(\eta \sqsubseteq \eta'\),
  \begin{align*}
    \max(\semi{\com_1 \ndet \com_2}\eta' \var[y])
    &= \max\{ \max(\semi{\com_1}\eta' \var[y]),  \max(\semi{\com_2}\eta' \var[y])\}\\
    % 
    & \geq \max(\semi{\com_1}\eta' \var[y])\\
    %
    & \geq \max(\eta' \var[z]) + h
  \end{align*}
  with \(|h| \leq \bound{\com_1} \leq \bound{\com_1 \ndet \com_2}\), as desired.

  % ----------------------------------------------
  
  \medskip
  
  \noindent
  \textbf{Case} (\(\com_1 \seq \com_2\))
  % 
  Take \(\eta \in \inte\) and assume
  \(\max(\semi{\com_1 \seq \com_2}\eta\var[y]) > \bound{\com_1 \seq
    \com_2} = \bound{\com_1} + \bound{\com_2}\).
  
  Recall that
  \(\semi{\com_1 \seq \com_2} \eta = \semi{\com_2}(\semi{\com_1}\eta)\).
  %
  If we define
  \begin{center}
    \(\semi{\com_1} \eta = \eta_1\)
  \end{center}
  since \(\max(\semi{\com_2}\eta_1 \var[y]) \neq \infty\) and
  \(\max(\semi{\com_2} \eta_1 \var[y]) > \bound{\com_1 \seq \com_2}
  \geq \bound{\com_2}\), by inductive hypothesis on \(\com_2\), there
  are \(|h_2| \leq \bound{\com_2}\) and \(\var[w] \in \Var\) such that
  \(\max(\semi{\com_2} \eta_1 \var[y]) = \max(\eta_1 \var[w]) + h_2\)
  and for all \(\eta_1' \in \inte\) with
  \(\eta_1 \sqsubseteq \eta_1'\)
  \begin{equation}
    \label{eq:seq2}
    \max(\semi{\com_2} \eta_1' \var[y]) \geq \max(\eta_1' \var[w]) + h_2
    \tag{\dag}
  \end{equation}
  
  Now observe that
  \(\max(\semi{\com_1}\eta \var[w]) = \max(\eta_1 \var[w]) >
  \bound{\com_1}\). Otherwise, if it were \(\max(\eta_1 \var[w]) \leq
  \bound{\com_1}\) we would have
  \begin{center}
    \(\max(\semi{\com_2}\eta_1 \var[y]) = \max(\eta_1 \var[w]) + h_2 \leq
    \bound{\com_1} + \bound{\com_2} = \bound{\com_1 \seq \com_2}\),
  \end{center}
  violating the hypotheses. Moreover,
  \(\semi{\com_1}\eta \var[w] \neq \infty\), otherwise we would have
  \(\max(\semi{\com_2} \eta_1 \var[y]) = \max(\eta_1 \var[w]) + h_2 =
  + \infty\), contradicting the hypotheses.  Therefore we can apply
  the inductive hypothesis also to \(\com_1\) and deduce that there
  are \(|h_1| \leq \bound{\com_1}\) and \(\var[w]' \in \Var\) such
  that
  \(\max(\semi{\com_1} \eta \var[w]) = \max(\eta \var[w]') + h_1\) and
  for all \(\eta' \in \inte\) with \(\eta \sqsubseteq \eta'\)
  \begin{equation}
    \label{eq:seq1}
    \max(\semi{\com_1} \eta' \var[w]) \geq \max(\eta' \var[w]') + h_1
    \tag{\ddag}
  \end{equation}

  Now, for all \(\eta' \in \inte\) with \(\eta \sqsubseteq \eta'\) we have that:
  \begin{align*}
    \max(\semi{\com_1 \seq \com_2}\eta \var[y])
    &= \max(\semi{\com_2}(\semi{\com_1}\eta) \var[y])\\
    %
    & = \max(\semi{\com_2}\eta_1 \var[y])\\
    %
    & = \max(\eta_1 \var[w]) + h_2\\
    %
    & = \max(\semi{\com_1}\eta \var[w]) + h_2\\
    %
    & = \max(\eta \var[w]')+h_1 + h_2
  \end{align*}
  and
  \begin{align*}
    \max(\semi{\com_1 \seq \com_2}\eta' \var[y]) & = & \\ 
    \max(\semi{\com_2}(\semi{\com_1}\eta') \var[y]) & \geq & \\ 
    \max(\semi{\com_1}\eta' \var[w]) + h_2 & \geq & 
    \text{by \eqref{eq:seq2}, since } \eta_1 = \semi{\com_1}\eta \sqsubseteq \semi{\com_1}\eta' \text{ , by monotonicity} \\
    (\max(\eta' \var[w]') +h_1) + h_2 & & \text{by \eqref{eq:seq1}}\
  \end{align*}

  Thus, the thesis holds with \(h= h_1+h_2\), as
  \(|h| =|h_1 +h_2| \leq |h_1| + |h_2| \leq \bound{\com_1} +
  \bound{\com_2} = \bound{\com_1 \seq \com_2}\), as needed.

  % ----------------------------------------------------
  
  \medskip
  
  \noindent
  \textbf{Case} (\(\fix{\com}\)) 
  %
  Let \(\eta \in \inte\) such that
  \(\max(\semi{\fix{\com}} \eta \var[y]) \neq + \infty\). Recall that
  \(\semi{\fix{\com}} \eta = \lfp{\lambda \mu. (\semi{\com} \mu
    \sqcup\eta)}\). Observe that the least fixpoint of
  \(\lambda \mu. (\semi{\com} \mu \sqcup\eta)\) coincides with the
  least fixpoint of
  \(\lambda \mu. (\semi{\com} \mu \sqcup \mu) = \lambda \mu. \semi{
    \com\ndet \tru} \mu\) above \(\eta\). Hence, if
  \begin{itemize}
  \item \(\eta_0 \veq \eta\),
  \item for all \(i\in \nat\),
    \(\eta_{i+1} \veq \semi{\com} \eta_i \sqcup \eta_i= \semi{\com \ndet
    \tru} \eta_i \sqsupseteq \eta_i\),
  \end{itemize}
  then we define an increasing chain \(\{\eta_i\}_{i\in \nat}\subseteq \inte\) such that
  \[ 
  \semi{\fix{\com}} \eta = \textstyle\bigsqcup_{i \in \nat} \eta_i.
  \]
  Since \(\max(\semi{\fix{\com}}) \eta \var[y] \neq \infty\), we have that
  for all \(i\in \nat\), \(\eta_i \var[y] \neq \infty\). Moreover,
  \(\textstyle\bigsqcup_{i \in \nat} \eta_i\) on \(\var[y]\) is
  finitely reached in the chain \(\{\eta_i\}_{i\in \nat}\), i.e.,
  there exists \(m \in \mathbb{N}\) such that for all \(i \geq m+1\)
  \[
  \semi{\fix{\com}} \eta \var[y] = \eta_{i} \var[y].
  \]
  %\(\eta_m \neq \eta_{m+1}\) and 

  The inductive hypothesis holds for \(\com\) and \(\tru\), hence for
  \(\com \ndet \tru\), therefore for all \(\var \in \Var\) and
  \(j \in \{0,1, \ldots, m\}\), if \(\max(\eta_{j+1} \var) >\bound{\com\ndet \tru} =\bound{\com}\) then
  there exist \(\var[z] \in \Var\) and \(h \in \mathbb{Z}\) such that \(|h| \leq \bound{\com}\) and 
  \begin{enumerate}[label=(\alph*)]
  \item \(\infty \neq \max(\eta_{j+1} \var) = \max(\eta_j \var[z]) + h\),
  \item \(\forall \eta' \sqsupseteq \eta_j.
    \max(\semi{\com \ndet \tru}\eta'\var) \geq \max(\eta' \var[z]) + h\).
  \end{enumerate}
  To shortly denote that the two conditions (a) and (b) hold, we write
  \[
  (\var[z],j) \to_{h} (\var, j+1)
  \]
  %   when 
  %  \(j\in  \{0, \ldots, m \}\), \(\var\in \Var\) and 
  %  %\(\max(\eta_{j+1} \var) >\bound{\com}\) hold, so that the existence of
  %   \(h \in \mathbb{Z}\) and \(\var[z] \in \Var\) such that 
  %   \(\max(\eta_{j+1} \var) = \max(\eta_j \var[z]) + h\) follows.
  %   
  %  
  
  Now, assume that for some variable \(\var[y] \in \Var\)
  \[\max(\semi{\fix{\com}} \eta \var[y]) = \max(\eta_{m+1} \var[y]) >
  \bound{\fix{\com}} = (n+1) \bound{\com}\]
  where \(n=|\mathit{vars}(\com)|\). 
  We want to show that the thesis holds, i.e., that there exist
  \(\var[z] \in \Var\) and \(h \in \mathbb{Z}\) with
  \(|h| \leq \bound{\fix{\com}}\) such that:
  \begin{equation}\label{eq:maxendmaxstart}
    \max(\semi{\fix{\com}} \eta \var[y]) = \max(\eta \var[z]) + h
    \tag{i}
  \end{equation}
  and for all \(\eta' \sqsupseteq \eta\),
  \begin{equation}\label{eq:maxmiddlemaxend}
    \max(\semi{\fix{\com}} \eta' \var[y]) \geq \max(\eta' \var[z]) + h
    \tag{ii}
  \end{equation}
  
  Let us consider \eqref{eq:maxendmaxstart}. We first observe that we
  can define a path
  \begin{equation}
    \label{eq:fix}
    \sigma \veq (\var[y]_0, 0) \to_{h_0} (\var[y]_1, 1) \to_{h_1}
    \ldots \to_{h_m} (\var[y]_{m+1}, m+1)
  \end{equation}
  such that \(\var[y]_{m+1}=\var[y]\) and for all \(j \in \{0,\ldots, m+1\}\), 
  \(\var[y]_j\in \Var\)  and
  \(\max(\eta_{j} \var[y]_{j}) > \bound{\com}\).
  %, so that for \(j\neq m+1\), \(|h_j|\leq \bound{\com}\).
  In fact, if, by contradiction, this is not the case, there would
  exist an index \(i \in \{0,\ldots, m\}\) (as
  \(\max(\eta_{m+1} \var[y]_{m+1}) > \bound{\com}\) already holds)
  such that \(\max(\eta_{i} \var[y]_{i}) \leq \bound{\com}\), while
  for all \(j \in \{i+1,\ldots, m+1\}\),
  \(\max(\eta_{j} \var[y]_j) > \bound{\com}\).  Thus, in such a case,
  we consider the nonempty path:
  \[\pi \veq (\var[y]_i, i) \to_{h_i} (\var[y]_{i+1}, i+1) \to_{h_{i+1}} \ldots
  \to_{h_m} (\var[y]_{m+1}, m+1)\]
  and we have that:
  \begin{align*}
    &\Sigma_{j=i}^m h_j =\\ 
    &\Sigma_{j=i}^m \max(\eta_{j+1} \var[y]_{j+1}) - \max(\eta_j \var[y]_{j}) =\\
    &\max(\eta_{m+1} \var[y]_{m+1}) - \max(\eta_i \var[y]_{i}) =\\
    &\max(\eta_{m+1} \var[y]) - \max(\eta_i \var[y]_{i})>\\
    &  (n+1) \bound{\com} - \bound{\com} = n \bound{\com}
  \end{align*}
  with \(|h_j| \leq \bound{\com}\) for \(j \in \{i,\ldots,
  m\}\). Hence we can apply Lemma~\ref{le:cycles} to the projection
  \(\pi_p\) of the nodes of this path \(\pi\) to the variable
  component to deduce that \(\pi_p\) has a subpath which is a cycle
  with a strictly positive weight.  More precisely, there exist
  \(i \leq k_1 < k_2 \leq m+1\) such that
  \(\var[y]_{k_1} = \var[y]_{k_2}\) and
  \(h = \Sigma_{j=k_1}^{k_2-1} h_j > 0\). If we denote
  \(\var[w] = \var[y]_{k_1} = \var[y]_{k_2}\), then we have that
  \begin{align*}
    \max(\eta_{k_2} \var[w]) & =  h_{k_2-1}  + \max(\eta_{k_2-1} \var[w]) \\
                             & =  h_{k_2-1} + h_{k_2-2} + \max(\eta_{k_2-2} \var[w])  \\
                             & = \Sigma_{j=k_1}^{k_2-1} h_j + \max(\eta_{k_1} \var[w])  \\
                             & = h +  \max(\eta_{k_1} \var[w])  
  \end{align*}
  Thus,
  \[\max(\semi{\com \ndet \tru}^{k_2-k_1} \eta_{k_1} \var[w]) = \max(\eta_{k_1}
  \var[w]) + h\] 
  Observe that for all \(\eta' \sqsupseteq \eta_{k_1}\)
  \begin{equation}\label{prop2}
    \max(\semi{\com \ndet \tru}^{k_2-k_1} \eta' \var[w]) \geq \max(\eta'
    \var[w]) + h
  \end{equation}
  This property \eqref{prop2} can be shown by induction on \(k_2-k_1 \geq 1\).
  
  \noindent    
  Then, an inductive argument allows us to show that for all \(r \in \mathbb{N}\):
  \begin{equation}\label{nopp}
    \max(\semi{\com \ndet \tru}^{r(k_2-k_1)} \eta_{k_1} \var[w]) \geq \max(\eta_{k_1}
    \var[w]) + r h
  \end{equation}  
  In fact, for \(r=0\) the claim trivially holds. Assuming the
  validity for \(r\geq 0\) then we have that
  \begin{align*}    
    & \max(\semi{\com \ndet \tru}^{(r+1)(k_2-k_1)} \eta_{k_1} \var[w]) =\\
    &\max(\semi{\com \ndet \tru}^{k_2-k_1} ( \semi{\com \ndet \tru}^{r(k_2-k_1)} \eta_{k_1}) \var[w]) \geq & \mbox{by \eqref{prop2} as \(\eta_{k_1} \sqsubseteq \semi{\com \ndet \tru}^{r(k_2-k_1)} \eta_{k_1}\) }\\
    &\max(\semi{\com \ndet \tru}^{r(k_2-k_1)} \eta_{k_1} \var[w]) + h \geq & \mbox{by inductive hypothesis}\\
    &  \max(\eta_{k_1}\var[w])  + rh + h
      \geq 
      \max(\eta_{k_1}\var[w])  + (r+1)h
  \end{align*}

  \noindent
  However, This would contradict the hypothesis
  \(\semi{\fix{\com}} \eta \var[y] \neq \infty\). In fact the
  inequality \eqref{nopp} would imply
  \begin{align*}
    \semi{\fix{\com}} \eta \var[w]
    & = \bigsqcup_{i \in \mathbb{N}} \semi{\com
      \ndet \tru}^i \eta \var[w] =\\ 
    & =  \bigsqcup_{i \in \mathbb{N}} \semi{\com \ndet
      \tru}^i \eta_{k_1} \var[w]\\ 
    & = \bigsqcup_{r \in \mathbb{N}} \semi{\com \ndet
      \tru}^{r(k_2-k_1)} \eta_{k_1} \var[w]\\
    & = +\infty
  \end{align*}

  Now, from \eqref{eq:fix} we deduce that for all
  \(\eta' \sqsupseteq \eta_{k_1}\), for \(j \in \{ k_1, \ldots, m\}\),
  if we let \(\mu_{k_1} = \eta'\) and
  \(\mu_{j+1} = \semi{\com \ndet \tru} \mu_j\), we have that
  \(\max(\mu_{j+1} \var[y]_{j+1}) \geq \max(\mu_{j+1} \var[y]_{j}) +
  h_j\) and thus
  \[
  \semi{\com \ndet \tru}^{m-k_1+1} \eta' \var[y] = \mu_{m+1}
  \var[y]_{m+1} \geq 
  \max(\var[y]_{k_1}) + \Sigma_{i=k_1}^m h_i = \max(\eta' \var[w]) + \Sigma_{i=k_1}^m h_i
  \]
  Since \(\eta' = \semi{\fix{\com}} \eta \sqsupseteq \eta_{k_1}\) we conclude
  \begin{align*}
    \semi{\fix{\com}} \eta \var[y]
    & = \semi{\com \ndet \tru}^{m-k_1+1} \semi{\fix{\com}} \eta \var[y]\\
    & \geq +\infty + \Sigma_{i=k_1}^m h_i = +\infty
  \end{align*}
  contradicting the assumption.
  
  \noindent
  Therefore, the path \(\sigma\) of \eqref{eq:fix} must exist, and
  consequently
  \[\max(\semi{\fix{\com}} \eta \var[y]) = \max(\eta_{m+1} \var[y]) =
    \max(\eta \var[y]_0) + \Sigma_{i=0}^m h_i\] and
  \(\Sigma_{i=0}^m h_i \leq \bound{\fix{\com}}=(n+1)\bound{\com}\),
  otherwise we could use the same argument above for inferring the
  contradiction \(\max(\semi{\fix{\com}} \eta \var[y]) = +\infty\).

  \medskip

  Let us now show \eqref{eq:maxmiddlemaxend}. Given
  \(\eta' \sqsupseteq \eta\) from \eqref{eq:fix} we deduce that for
  all \(j \in \{ 0, \ldots, m\}\), if we let \(\mu_0 = \eta'\) and
  \(\mu_{j+1} = \semi{\com \ndet \tru} \mu_j\), we have that
  \[
    \max(\mu_{j+1} \var[y]_{j+1}) \geq \max(\mu_{j+1} \var[y]_{j}) +
    h_j. \]
  Therefore, since \(\semi{\fix{\com}} \eta' \sqsupseteq \mu_{m+1}\)
  (observe that the convergence of \(\semi{\fix{\com}} \eta' \) could
  be at an index greater than \(m+1\)), we conclude that:
  \[\max(\semi{\fix{\com}} \eta' \var[y]) \geq \max(\mu_{m+1}
  \var[y]) = \max(\mu_{m+1} \var[y]_{m+1}) \geq \max(\eta' \var[y]_0)
  + \Sigma_{i=0}^m h_i\] as desired.
\end{proof}

We can now notice that this proof also works for the \(\min\) value of
each variable's interval. I.e., the following property also holds:

% Dual formulation of the same lemma
\begin{corollary}\label{co:inc}
  Let \(\com \in \imp\).
  
  \noindent
  For all \(\eta \in \inte\) and
  \(\var[y] \in \Var\), if
  \(\min(\semi{\com}\eta\var[y]) \neq -\infty\) and
  \(\min(\semi{\com}\eta\var[y]) < -\lbound{\com}\) then there exist a
  variable \(\var[z] \in\Var\) and an integer \(h \in \z\) s.t.
  \(|h| \leq \lbound{\com}\) s.t. the following two properties hold:
  
  \begin{enumerate}[label=(\roman*)]
  \item \(\min(\semi{\com}\eta \var[y]) = \min(\eta \var[z]) + h\); \label{point1min}
  \item  for all \(\eta' \in \inte\), if \(\eta' \sqsupseteq \eta\)
    % and \(\semi{\com}\eta' \neq \top\)
    then
    \(\min(\semi{\com}\eta' \var[y]) \leq \min(\eta' \var[z]) + h\). \label{point2min}
  \end{enumerate}
\end{corollary}

\begin{proof}
  The full proof is available at Appendix~\ref{ap:proof}, as
  Lemma~\ref{co:inc2}. Intuitively the proof works by considering the
  integers \(\z\) with the reverse ordering \(<\) and a new bound,
  \(\lbound{\com}\), computed by considering the reverse ordering.
\end{proof}

% \noindent
% Lemma~\ref{le:inc} and its Corollary~\ref{co:inc} provide an
% effective algorithm for computing the interval semantics of
% commands. More precisely, given a command \(\com\), the corresponding
% finite set of variables \(\Var_{\com} \veq \mathit{vars}(\com)\), and
% an interval environment \(\rho : \Var_{\com} \to \Int\), we define
% \begin{align*}
%   \max(\rho) & \veq \max \{ \max(\rho(\var)) \mid \var \in \Var_{\com}\} \\
%   \min(\rho) & \veq \min \{ \min(\rho(\var)) \mid \var \in \Var_{\com}\}
% \end{align*}
% %
% Then, when computing \(\semi{\com} \rho\) on such \(\rho\)
% having a finite domain, we can restrict to a bounded interval domain
% \(\bA_{\com,\rho} \veq (\Var_{\com} \to \Int_{\com,\rho}) \cup
% \{ \top, \bot \}\) where
% \begin{equation*}
%   \Int_{\com,\rho} \veq \{ \interval{a}{b} \mid a, b \in \z\ \land\
%   \min(\rho) - \lbound{\com} \leq a \leq b \leq \max(\rho) + \bound{\com}\}
%   % \{\interval{a}{\infty} \mid a \in \nat \}
% \end{equation*}
% % We could also operate uniformly on all commands, defining the
% % semantics for \(\com\) in a domain with intervals bounded by
% % \(\max(\rho) +\bound{\com}\)
