% !TeX root = mod.tex
\section{Computing interval semantics}
\label{sec:computingint}

Lemma~\ref{le:inc} provides an effective algorithm for computing the
abstract semantics of commands provided a domain that respects
properties \ref{inc:hp1} and \ref{inc:hp2}. We can in fact verify that
the intervals respect such claims.

\begin{observation}[\ref{inc:hp1} holds on intervals]
  Let \(\iota \in \Int, S \in \poset{\z}\). If
  \(\max(\abstr(S)) = \infty\) and
  \({\iota \sqcap \abstr(S) \neq \bot}\) then
  \({max(\iota \sqcap \abstr(S)) = \max(\iota)}\) where
  \(S \in \poset{\Z}\) and \(\iota \in \Int\)
\end{observation}

\begin{proof}
  Let \(\iota = \interval{a}{b}\) and
  \(\abstr(S) = \interval{c}{\infty}\) and remember that
  \({\iota \sqcap \abstr(S) \neq \bot}\). Then
  \begin{equation*}
    \max([a,b] \sqcap \abstr(S)) = \max(\interval{\max\{a, c\}}{\min\{b , \infty\}}) = b = \max(\iota) \qedhere
  \end{equation*}
\end{proof}

\begin{observation}[\ref{inc:hp2} holds on intervals]
  Let \(\iota, \kappa \in \Int\), then
  \(\max(\iota \sqcup \kappa) = \max\{\max(\iota), \max(\kappa)\}\)
\end{observation}

\begin{proof}
  Let \(\iota = \interval{a}{b}\) and \(\kappa =
  \interval{c}{d}\). Then by definition
  \(\iota \sqcup \kappa = \interval{\min\{a,c\}}{\max\{b,d\}}\), and
  therefore
  \begin{equation*}
    \max(\iota \sqcup \kappa) = \max\{b,d\} = \max\{\max(\iota), \max(\kappa)\} \qedhere
  \end{equation*}
\end{proof}

This means that we can apply Lemma~\ref{le:inc} on the intervals
domain \(\inte\).  First, given a command \(\com\), the corresponding
finite set of variables \(\Var_{\com} \veq \mathit{vars}(\com)\), and
an interval environment \(\rho : \Var_{\com} \to \Int\), we define
\[\max(\rho) \veq \max \{ \max(\rho(\var)) \mid \var \in \Var_{\com}
  \}.\]
%
Then, when computing \(\semi[\inte]{\com} \rho\) on such \(\rho\)
having a finite domain, we can restrict to a bounded interval domain
\(\inte_{\com,\rho} \veq (\Var_{\com} \to \Int_{\com,\rho}) \cup
\{ \top, \bot \}\) where
\begin{equation*}
  \Int_{\com,\rho} \veq \{ \interval{a}{b} \mid a, b \in \z \land
  \min(\rho) - \lbound{\com} \leq a \leq b \leq \max(\rho) + \bound{\com}\}
\end{equation*}

% We could also operate uniformly on all commands, defining the
% semantics for \(\com\) in a domain with intervals bounded by
% \(\max(\rho) +\bound{\com}\)

\begin{lemma}
  Let \(\com\in \imp\) be a command. Then, for all finitely supported
  \(\rho : \Var \to \Int\), the abstract semantics
  \(\semi{\com} \rho \)
  % 
  % \semi{\fix{\com}} \rho & = \lfp{\lambda \rho'. (\semi{\com} \rho')
  % \sqcup \rho}
  computed in \(\inte\) and in \(\inte_{\com,\rho}\)
  coincide, i.e., 
  \begin{equation*}
    \semi[\inte]{\com}\rho = \semi[\inte_{\com,\rho}]{\com}\rho
  \end{equation*}
\end{lemma}

\begin{proof}
  The proof will proceed by induction on the command \(\com\). First,
  let's explore the base cases.

  \medskip

  \textbf{Case} (\(\var \in S\)).
  %
  Recall that
  % \begin{equation*}
  %   \bound{\var \in S} = \begin{cases}
  %     \min(\abstr(S)) & \text{if }\max(\abstr(S)) = \infty \\
  %     \max(\abstr(S)) & \text{otherwise}
  %   \end{cases}
  % \end{equation*}
  % and
  \begin{equation*}
    \semi[\inte]{\var\in S}\rho = \rho[\var \mapsto \rho\var \sqcap \abstr(S)]
  \end{equation*}
  by definition it holds that
  \(\max(\rho\var \sqcap \abstr(S)) \leq \max(\rho\var)\) and
  therefore
  \begin{equation*}
    \semi[\inte_{\com,\rho}]{\var\in S}\rho = \rho[\var \mapsto \rho\var \sqcap \abstr(S)]
  \end{equation*}
  which means that
  \begin{equation*}
    \semi[\inte]{\var\in S}\rho = \semi[\inte_{\com,\rho}]{\var\in S}\rho
  \end{equation*}

  \medskip

  \noindent
  \textbf{Case} (\(\var := k\)).
  %
  Recall that
  \begin{equation*}
    \semi[\inte]{\var := k}\rho = \rho[\var \mapsto \interval{k}{k}]
  \end{equation*}
  and also recall that
  \(k \leq k \leq \max(\rho) + k = \max(\rho) + \bound{\var :=
    k}\). Hence by definition
  \begin{equation*}
    \semi[\inte_{\com,\rho}]{\var := k}\rho = \rho[\var \mapsto \interval{k}{k}]
  \end{equation*}
  and therefore
  \begin{equation*}
    \semi[\inte]{\var := k}\rho = \semi[\inte_{\com,\rho}]{\var := k}\rho
  \end{equation*}

  \medskip

  \noindent
  \textbf{Case} (\(\var := \var[y] + k\)).
  % 
  Recall that
  \({\semi[\inte]{\var := \var[y] + k}\rho} = {\rho[\var \mapsto
    \rho\var[y] + k]}\) and
  \({\max(\rho\var + k)} \leq {\max(\rho) + k} = {\max(\rho) +
    \bound{\var := \var[y] + k}}\)
  and therefore by definition
  \begin{equation*}
    \semi[\inte_{\com,\rho}]{\var := \var[y] + k}\rho = \rho[\var \mapsto \rho\var[y] + k]
  \end{equation*}
  hence
  \begin{equation*}
    \semi[\inte_{\com,\rho}]{\var := \var[y] + k}\rho = \semi[\inte]{\var := \var[y] + k}\rho
  \end{equation*}
  %
  % \medskip
  % 
  % \noindent
  % \textbf{Case} (\(\var := \var[y] - k\)).
  % % 
  % Recall that
  % \({\semi[\inte]{\var := \var[y] - k}\rho} = {\rho[\var \mapsto
  %   \rho\var[y] - k]}\) and that
  % \({\max(\rho\var[y] - k)} \leq {\max(\rho) + k} = {\max(\rho) +
  %   \bound{\rho\var[y] - k}}\). Hence we can notice that
  % \({\semi[\inte_{\com,\rho}]{\var := \var[y] - k}\rho} = {\rho[\var
  %   \mapsto \rho\var[y] - k]}\) and therefore
  % \begin{equation*}
  %   \semi[\inte]{\var := \var[y] - k}\rho = \semi[\inte_{\com,\rho}]{\var := \var[y] - k}\rho
  % \end{equation*}

  \medskip

  \noindent
  \textbf{Case} (\(\com_1 + \com_2\)).
  % 
  Recall that
  \({\semi[\inte]{\com_1 + \com_2}\rho} = {\semi[\inte]{\com_1}\rho}
  \sqcup {\semi[\inte]{\com_2}\rho}\) and by inductive hypothesis
  \begin{align*}
    \semi[\inte]{\com_1}\rho & = \semi[\inte_{\com_1, \rho}]{\com_1}\rho \\
    \semi[\inte]{\com_2}\rho & = \semi[\inte_{\com_2, \rho}]{\com_2}\rho
  \end{align*}
  by definition
  \begin{equation*}
    \semi[\inte_{\com_1, \rho}]{\com_1}\rho \sqcup \semi[\inte_{\com_2, \rho}]{\com_2}\rho = \rho'
  \end{equation*}
  where \(\forall \var \in \Var\)
  \(\rho'\var = {\left(\semi[\inte_{\com_1,
        \rho}]{\com_1}\rho\right)(\var)} \sqcup
  {\left(\semi[\inte_{\com_2, \rho}]{\com_2}\rho\right)
    (\var)}\). \todo[inline]{\(\top\) e \(\bot\) possiamo escluderli
    perchè banalmente coincidono sempre} Recall that
  \begin{align}
    \inte_{\com_1, \rho} & = \{\interval{a}{b} \mid \min(\rho) - \lbound{\com_1} \leq a \leq b \leq \max(\rho) + \bound{\com_1}\} \label{ic1}\\
    \inte_{\com_2, \rho} & = \{\interval{a}{b} \mid \min(\rho) - \lbound{\com_2} \leq a \leq b \leq \max(\rho) + \bound{\com_2}\} \label{ic2}
  \end{align}
  Hence observe that for all \(\var\in\Var\) we can call
  \begin{align*}
    \iota & = \semi[\inte_{\com_1, \rho}]{\com_1}\rho & \text{where } \iota\var = \interval{a}{b} \text{ for some } a,b\in\z \\
    \kappa & = \semi[\inte_{\com_2, \rho}]{\com_2}\rho & \text{where }\kappa\var = \interval{c}{d} \text{ for some } c,d \in\z
  \end{align*}
  and by the definition of \(\sqcup\) for intervals
  \begin{equation*}
    \iota\var \sqcup \kappa\var = \interval{\min\{a,c\}}{\max\{b,d\}}
  \end{equation*}
  we can notice that by \eqref{ic1}
  \(b = {\max(\iota\var)} \leq {\max(\rho) + \bound{\com_1}}\) and by
  \eqref{ic2}
  \(d = {\max(\kappa\var)} \leq {\max(\rho) + \bound{\com_2}}\),
    therefore
  \begin{align*}
    \max(\iota\var\sqcup\kappa\var) & \leq \max(\rho) + \bound{\com_1} + \bound{\com_2} \\
                                    & = \max(\rho) + (\bound{\com_1} + \bound{\com_2}) \\
                                    & = \max(\rho) + \bound{\com_1 + \com_2}
  \end{align*}
  and dually
  \begin{align*}
    \min(\iota\var\sqcup\kappa\var) & \geq \min(\rho) - \lbound{\com_1} - \lbound{\com_2} \\
                                    & = \min(\rho) - (\lbound{\com_1} + \lbound{\com_2}) \\
                                    & = \min(\rho) - \lbound{\com_1 + \com_2}
  \end{align*}
  and therefore
  \begin{equation*}
    \min(\rho) - \lbound{\com_1 + \com_2} \leq \min(\iota\var \sqcup \kappa\var) \leq \max(\iota\var \sqcup \kappa\var) \leq \max(\rho) + \bound{\com_1 + \com_2}
  \end{equation*}
  which by definition of \(\inte_{\com_1 + \com_2, \rho}\) means
  \(\iota \sqcup \kappa \in \inte_{\com_1 + \com_2, \rho}\) and
  therefore
  \begin{equation*}
    \semi[\inte]{\com_1 \ndet \com_2}\rho = \semi[\inte]{\com_1}\rho \sqcup \semi[\inte]{\com_2}\rho = \semi[\inte_{\com_1, \rho}]{\com_1}\rho \sqcup \semi[\inte_{\com_2, \rho}]{\com_2}\rho = \semi[\inte_{\com_1 + \com_2, \rho}]{\com_1 + \com_2}\rho
  \end{equation*}
  
  
  \medskip

  \noindent
  \textbf{Case} (\(\com_1 ; \com_2\)).
  % 
  The goal in this case is to prove that
  \({\semi[\inte]{\com_1 \seq \com_2}\rho} = {\semi[\inte_{\com_1 \seq
      \com_2, \rho}]{\com_1 \seq \com_2}\rho}\). Let's start by
  recalling that
  \begin{equation}\label{eq:def1}
    {\semi[\inte]{\com_1 \seq \com_2}\rho} = {\semi[\inte]{\com_2}\left(\semi[\inte]{\com_1}\rho\right)}
  \end{equation}
  by inductive hypothesis we know that
  \({\semi[\inte]{\com_1}\rho} = {\semi[\inte_{\com_1,
      \rho}]{\com_1}\rho} = \rho'\) which means that for all
  \(\var\in\Var\), let
  \({\left(\semi[\inte_{\com_1, \rho}]{\com_1}\rho\right)}(\var) =
  \interval{a}{b}\). By definition of \(\inte_{\com_1,\rho}\)
  \begin{equation}\label{eq:b1}
    \min(\rho) - \lbound{\com_1} \leq a \leq b \leq \max(\rho) + \bound{\com_1}
  \end{equation}
  Let's now substitute \(\rho'\) in \eqref{eq:def1} and observe that
  \begin{equation}\label{eq:ind1}
    \semi[\inte]{\com_2}\rho' = \semi[\inte_{\com_2, \rho'}]{\com_2}\rho' \quad \text{by inductive hypothesis}
  \end{equation}
  Again this means that for every \(\var \in \Var\) by letting
  \({\left(\semi[\inte_{\com_2, \rho'}]{\com_2}\rho'\right)}(\var) =
  \interval{a}{b}\) we now that
  \begin{equation}\label{eq:b2}
    \min(\rho') - \lbound{\com_2} \leq a \leq b \leq \max(\rho') + \bound{\com_2}
  \end{equation}
  but by \eqref{eq:b1} we know that
  \({\min(\rho') = \min(\rho) - \lbound{\com_1}}\) and
  \({\max(\rho') = \max(\rho) + \bound{\com_1}}\) which means that if
  we substitute it in \eqref{eq:b2}
  \begin{equation*}
    \min(\rho) - \lbound{\com_1} - \lbound{\com_2} \leq a \leq b \leq \max(\rho) + \bound{\com_1} + \bound{\com_2}
  \end{equation*}
  which is exactly
  \begin{equation*}
    \min(\rho) - \lbound{\com_1 \seq \com_2} \leq a \leq b \leq \max(\rho) + \bound{\com_1 \seq \com_2}
  \end{equation*}
  which means that
  \(\inte_{\com_1 \seq \com_2, \rho} = \inte_{\com_2,
    \left(\semi[\inte_{\com_1, \rho}]{\com_1}\rho\right)} =
  \inte_{\com_2, \rho'}\) Therefore we can substitute \(\rho'\) in
  \eqref{eq:ind1} and deduce that
  \begin{equation*}
    \semi[\inte]{\com_2}\left(\semi[\inte]{\com_1}\rho\right) =
    \semi[\inte]{\com_2}\rho' = \semi[\inte_{\com_2,\rho'}]{\com_2}\rho' =
    \semi[\inte_{\com_1\seq\com_2}, \rho]{\com_2}\left(\semi[\inte_{\com_1, \rho}]{\com_1}\rho\right)
  \end{equation*}
  We can conclude by noticing that \(\inte_{\com_1,\rho}\) is entirely
  contained in \(\inte_{\com_1\seq \com_2, \rho}\) since by letting
  \({\semi[\inte_{\com_1,\rho}]{\com_1}\rho = \interval{a}{b}}\)
  \begin{align*}
    \min(\rho) - \lbound{\com_1\seq \com_2} & = \min(\rho) - \lbound{\com_1} - \lbound{\com_2} \\
                                            & \leq  \min(\rho) - \lbound{\com_1} \\
                                            & \leq a \\
                                            & \leq b \\
                                            & \leq \max(\rho) + \bound{\com_1}  \\
                                            & \leq \max(\rho) + \bound{\com_1 \ndet \com_2}
  \end{align*}
  And therefore
  \begin{align*}
     \semi[\inte]{\com_1\seq \com_2}\rho & = \semi[\inte]{\com_2}\left(\semi[\inte]{\com_1}\rho\right)\\
                                          & = \semi[\inte_{\com_1\seq\com_2}, \rho]{\com_2}\left(\semi[\inte_{\com_1\seq\com_2, \rho}]{\com_1}\rho\right) \\
                                          & = \semi[\inte_{\com_1\seq\com_2, \rho}]{\com_1\seq\com_2}\rho
  \end{align*}

  \medskip

  \noindent
  \textbf{Case} (\(\fix{\com}\)).
  % 
  Recall as we observed in Lemma~\ref{le:inc} that
  \({\semi[\inte]{\fix{\com}}\rho} = {\lfp(\lambda \mu
    . \semi[\inte]{\com \ndet \tru}\mu)}\) above \(\rho\). we can
  therefore build the chain
  \begin{align*}
    \rho_0 & \defin \rho \\
    \rho_{i+1} & \defin \semi[\inte]{\com}\rho_i \sqcup \rho_i = \semi[\inte]{\com \ndet \tru}\rho_i \sqsupseteq \rho_i
  \end{align*}
  and therefore
  \({\semi[\inte]{\fix{\com}}\rho} =
  {\textstyle\bigsqcup_{i\in\n}\rho_i}\). By inductive hypothesis we
  can notice that
  \begin{equation*}
    \semi[\inte]{\com \ndet \tru}\rho_i = \semi[\inte_{\com,\rho_i}]{\com \ndet \tru}\rho_i
  \end{equation*}
  which means that \(\forall \var\in\Var, i\in\n\) considering
  \(\rho_i\var = \interval{a}{b}\)
  \begin{equation*}
    \min(\rho_i) - \lbound{\com} \leq a \leq b \leq \max(\rho_i) + \bound{\com}
  \end{equation*}

\end{proof}
