\section{Program bounds}
\label{sec:bounds}

\begin{definition}[\textbf{Program bound}]
  \label{de:bound}
  The \emph{upper bound} associated with a command \(\com\in \imp\) is
  an integer number, denoted \(\bound{\com}\in \z\), defined
  inductively as follows:
  \begin{align*}
    % 
    \bound{\var \in S}  
    & \veq \begin{cases}
      \min(S) & \text{if } \max(S)=+\infty\\
      \max(S) & \text{if } \max(S)\in\z
    \end{cases}
    \\
    \bound{\var := k} 
    & \veq | k |\\
    %
    \bound{\var := \var[y] + k}
    & \veq | k |\\
    %
    \bound{\com_1 \ndet \com_2}
    & \veq \bound{\com_1} + \bound{\com_2}\\
    %
    \bound{\com_1 \seq \com_2}
    & \veq \bound{\com_1} + \bound{\com_2}\\
    % \bound{\kleene{\com}}
    % & \veq (|\varsof{\com}|+1) \bound{\com} \\ 
    \bound{\fix{\com}}
    & \veq (|\varsof{\com}|+1) \bound{\com} \\ 
  \end{align*}

  while the \emph{lower bound} associated with a command
  \(\com\in \imp\) is again an integer number, denoted
  \(\lbound{\com}\in \z\), defined inductively as follows:

  \begin{align*}
    % 
    \lbound{\var \in S}  
    & \veq \begin{cases}
      \max(S) & \text{if } \min(S)=-\infty\\
      \min(S) & \text{if } \min(S)\in \z
    \end{cases}
    \\
    % 
    \lbound{\var := k} 
    & \veq - | k |\\
    %
    \lbound{\var := \var[y] + k}
    & \veq - | k |\\
    %
    \lbound{\com_1 \ndet \com_2}
    & \veq \lbound{\com_1} + \lbound{\com_2}\\
    %
    \lbound{\com_1 \seq \com_2}
    & \veq \lbound{\com_1} + \lbound{\com_2}\\
    %
    % \lbound{\kleene{\com}}
    % & \veq (|\varsof{\com}|+1) \lbound{\com} \\ 
    %
    \lbound{\fix{\com}}
    & \veq (|\varsof{\com}|+1) \lbound{\com} \\ 
  \end{align*}

  where \(\varsof{\com}\) denotes the set of variables occurring in 
  \(\com\).
\end{definition}

We can notice that the two definitions of the bound \(\bound{\com}\)
and \(\lbound{\com}\) coincide, except for the filtering instruction
\(\var\in S\).  In order to compute the bound of a program we
introduce bounded environments, that we'll later use for
lemma~\ref{le:computebounds}

\begin{definition}[\textbf{Bound Environment}]
  \label{de:boundenv}
  A bound environment (benv for short) is a total function \(b:\Var
  \to \nat\). We define \(\benv \veq \{ b \mid b:\Var \to \nat\}\).
  Each command \(\com\in \imp\) induces a benv transformer
  \(\boundt{C}:\benv \to \benv\), which is defined inductively as
  follows:
  \begin{align*}
    %
    % boolean
    %
    \boundt{\var \in S}b  
    & \veq 
    \begin{cases} 
      b[\var[x]\mapsto b(\var[x])+\min(S)]& \text{if }\max(S)=\infty\\
      b[\var[x]\mapsto b(\var[x])+\max(S)]& \text{if }\max(S)\in \nat
    \end{cases}
    \\
    %
    % assignments
    %
    \boundt{\var := k}b 
    & \veq  b[\var[x]\mapsto b(\var[x])+k]\\
    %
    \boundt{\var := \var[y] + k}b
    & \veq  b[\var[x]\mapsto b(\var[x])+b(\var[y])+k]\\
    %
    % \boundt{\var := \var[y] - k}b
    % & \veq  b[\var[x]\mapsto b(\var[x]) + b(\var[y]) - k]\\
    %
    % commands
    %
    \boundt{\com_1 \ndet \com_2}b
    & \veq \lambda \var[x]. (\boundt{\com_1}b)(\var[x]) + (\boundt{\com_2}b)(\var[x])\\
    %
    \boundt{\com_1 \seq \com_2}b
    & \veq \lambda \var[x]. (\boundt{\com_1}b)(\var[x]) + (\boundt{\com_2}b)(\var[x])\\
    %& \veq \boundt{\com_2}(\boundt{\com_1}b) \\
    % 
    \boundt{\com^*}b
    &\veq  \lambda \var[x]. (|\varsof{\com}|+1)(\boundt{\com}b)(\var[x])
  \end{align*}    
  where \(\varsof{\com}\) denotes the set of variables occurring
  in \(\com\).
\end{definition}

\begin{lemma}\label{le:computebounds}
  For all \(\com \in \imp\),
  \(\bound{\com}=\sum_{\var\in \vars{\com}}
  \left(\boundt{\com}b_0\right)(\var)\), with \(b_0\veq \lambda x.0\).
\end{lemma}
\begin{proof}
  For the following proof we work by induction on \(\com\in \imp\):

  \paragraph*{Base cases:} \mbox{} \\
  
  \medskip

  \textbf{Case} (\(\var \in S\))
  %
  \begin{align*}
    \bound{\var\in S} & = \begin{cases}
      \min(S) & \text{if } \max(S) = \infty \\
      \max(S) & \text{otherwise}
    \end{cases} & \\
    \boundt{\var \in S}b_0 & = \begin{cases}
      b_0[\var \mapsto 0 + \min(S)] & \text{if } \max(S) = \infty \\
      b_0[\var \mapsto 0 + \max(S)] & \text{if } \max(S) \in \n
    \end{cases} &
  \end{align*}
  
  and since \(\var\) is the only variable in \(\varsof{\var \in S}\),
  \(\bound{\var\in S} = \sum_{\vars{\var \in S}}\left(\boundt{\var \in
      S}b_0\right)\var\)
  
  \medskip
  
  \textbf{Case} (\(\var := k\))
  %
  just notice that
  \(\bound{\var := k} = k = \sum_{\var[y] \in \vars{\com}} b_0[\var \mapsto b_0 + k]
  = \boundt{\var := k }b_0\)

  \medskip

  \textbf{Case} (\(\var := \var[y] + k\))
  
  \medskip

  \noindent
  \paragraph*{Inductive cases:}\mbox{}\\
  
  \medskip
  
  \textbf{Case} (\(\com_1 \ndet \com_2\))
  %
  \begin{align*}
    \bound{\com_1 \ndet \com_2} & = &\\
    \bound{\com_1} + \bound{\com_2} & = & \text{by inductive hypothesis}\\
      \sum_{\var\in \vars{\com_1}} (\boundt{\com}b_0)(\var) + \sum_{\var\in \vars{\com_2}} (\boundt{\com}b_0)(\var) & = &\\
    \sum_{\var\in \vars{\com_1}\cap \vars{\com_2}} (\boundt{\com_1}b_0)(\var)+(\boundt{\com_2}b_0)(\var) & \;+ & \\
    \sum_{\var\in \vars{\com_1}\smallsetminus \vars{\com_2}} (\boundt{\com_1}b_0)(\var) & \;+ & \\
    \sum_{\var\in \vars{\com_2}\smallsetminus \vars{\com_1}} (\boundt{\com_2}b_0)(\var) & = & \\
    \boundt{\com_1 \ndet \com_2}b_0 & &
  \end{align*}

  \medskip
  
  \textbf{Case} (\(\com_1 ; \com_2\))
  %
  identical to \((\com_1 \ndet \com_2)\);

  \medskip
  
  \textbf{Case} (\(\com^*\))
  % 
  \begin{align*}
    \bound{\com^*} & = &\\
    |\vars{C}+1|\bound{\com} & = & \text{by inductive hypothesis} \\
    |\vars{C}+1|\sum_{\var\in \vars{\com}} (\boundt{\com}b_0)(\var) & = &\\
    \sum_{\var\in \vars{\com}} |\vars{C}+1|(\boundt{\com}b_0)(\var) & = &\\
    % \lambda \var[x]. (\boundt{\com_1}b)(\var[x]) + (\boundt{\com_2}b_0)(\var[x]) =&\\
      \boundt{\fix{\com}}b_0 & &
  \end{align*}
\end{proof}
