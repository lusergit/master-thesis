\section{Analysis assumptions}
\label{sec:assumptions}


Given some abstract domain \(\bA\) we define an \emph{analysis over}
\(\bA\) as the function \(\semi{\cdot} : \imp \to \bA \to \bA\), which
maps each \(\imp\) command to a continuous function over the \(\bA\)
lattice.  The analysis is inductively defined as follows:

\begin{definition}[Abstract interpretation]\label{de:ainterpret}
  The \emph{interval semantics} of \(\imp\) is the strict (i.e.,
  preserving \(\bot\))
  % and co-strict (i.e., preserving \(\top\))
  extension of the following function
  \(\semi{\cdot}: \imp \to \bA \to \bA\). For all
  \(\eta: \Var \to \bA_*\),
  \begin{align*}
    \semi{\var \in S}\eta 
    & \veq  
      \begin{cases}
        \eta[\var \mapsto \eta(\var)\sqcap \abstr[\Int](S)] & \text{if }\eta(\var)\sqcap \abstr(S)\neq \bot \\
        \bot & \text{otherwise}
      \end{cases}\\
    % \semi{\tru}\eta 
    % & \veq \eta\\
    % \semi{\ff}\eta 
    % & \veq \bot\\
    \semi{\var := k}\eta 
    & \veq \eta[\var \mapsto \const{k}]\\
    \semi{\var := \var[y] + k}\eta 
    & \veq \eta[\var \mapsto \eta (\var[y]) + k]\\
    \semi{\var := \var[y] - k}\eta 
    & \veq \eta[\var \mapsto \eta (\var[y]) - k]\\
    \semi{\com_1 \ndet \com_2} \eta
    & \veq \semi{\com_1} \eta \sqcup \semi{\com_2} \eta\\
    \semi{\com_1 \seq \com_2} \eta
    & \veq \semi{\com_2} (\semi{\com_1} \eta)\\
    \semi{\kleene{\com}} \eta
    & \veq \textstyle \bigsqcup_{i \in \nat} \semi{\com}^i (\eta)\\
    \semi{\fix{\com}} \eta
    & \veq  \lfp(\lambda \mu. (\eta \sqcup \semi{\com} \mu))
  \end{align*}
\end{definition}

Notice that the filtering in Definition~\ref{de:intervalsem} is not
the best correct approximation. In particular the filtering
instruction \(\var\in S\) is performed first by abstracting the
numerical set \(S \subseteq \n\) with \(\abstr\) and then by
computing the greatest lower bound with \(\eta(\var)\). Best correct
approximation would consist in the opposite approach: compute the
concrete \(\concr(\eta\var)\) and subsequently compute the meet in
the domain \(\poset{\n}\):
\begin{equation*}
  \semi{\var\in S}\eta = \begin{cases}
    \eta[\var \mapsto \abstr[\Int](\concr(\eta\var) \cap S)]
    & \text{if } \concr(\eta\var) \cap S \neq \bot \\
    \bot & \text{otherwise}
  \end{cases}
\end{equation*}
This however would introduce a problem in proving Lemma~\ref{le:inc},
as Axiom~\ref{ax:max} would not be respected if we compute the
analysis using as abstract domain \(\bCnr\).  Notice that this
approximation coincides in fact with the best correct approximation in
one important case: \(\var \leq k\). This case is widely used in
programming and is already considered a good way of writing
analizer-friendly code. to support this thesis we can cite the fact
that it is part of the nasa programming guidelines on writing
analyzer-friendly code, as reported in \cite{nasa:ten}.