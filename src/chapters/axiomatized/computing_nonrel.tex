\section{Computing non-relational collecting semantics}\label{sec:computingnonrel}

In the following section we follow the same scheme used in
Section~\ref{sec:computing} in order to prove that we can also bonud
the non relational collecting domain to ensure convergence while
obtaining the most precise interval.  To start, we rely on
Definition~\ref{def:minmax} of \(\min\) and \(\max\) values of an
abstract environment \(\rho\) to bound the non relational collecting
domain \(\bCnr\) in this way:

\begin{definition}[Bounded non relational collecting domain]
  We define
  \(\bbCnr{k_1}{k_2} \defin \Var_{\com} \to \bounded{\poset[*]{\z}}{k_1}{k_2}\) where
  \begin{equation*}
    \bposet[*]{k_1}{k_2}{\z} = \{S \subseteq \z \mid S \neq \emptyset \land \forall x \in S \quad k_1 \leq x \leq k_2\} \cup \{\top\}
  \end{equation*}
\end{definition}
Notice that contrary to \(\binte{k_1}{k_2}\) we have no way of
rapresenting a diverging element. For bounded intervals we had the
\(\interval{a}{+\infty}\) and \(\interval{-\infty}{b}\) elements with
\(a,b\in\z\), but for arbitrary subsets of \(z\) we have to rely on a
smashed \(\top\) element.

\medskip

\noindent
We can already observe that by definition there are no infinite
ascending nor descending chains, as every chain is bounded by abobve
by some \(k_2 \in \z\) and below by some \(k_1\in\z\). Moreover, there
is a Galois connection between this abstract domain and its unbounded
counterpart \(\poset\z\). First let's define the concretization and
abstraction maps

\begin{definition}\label{def:abstrnrb}
  Let \(k_1,k_2\in\z\) s.t.\ \(k_1 \leq k_2\). We define a
  concretization map
  \(\concr[k_1,k_2] : \bposet[*]{k_1}{k_2}{\z} \to \poset[*]{\z}\) as
  the identity function
  \begin{equation*}
    \concr[k_1,k_2] = \id
  \end{equation*}
  siilarly we define an abstraction map
  \(\abstr[k_1,k_2] : \poset[*]{\z} \to \bposet[*]{k_1}{k_2}{\z}\) in
  the follwing way
  \begin{equation*}
    \abstr[k_1,k_2](S) = \begin{cases}
      S & \text{if } \min(S) \geq k_1 \land \max(S) \leq k_2 \\
      \top & \text{otherwise}
    \end{cases}
  \end{equation*}
\end{definition}

\begin{lemma}
  Given \(k_1, k_2 \in \z\) s.t.\ \(k_1\leq k_2\).
  \begin{equation*}
    \tuple{\poset[*]\z, \subseteq} \galois{\abstr[k_1,k_2]}{\id} \tuple{\bposet[*]{k_1}{k_2}{\z}, \subseteq}
  \end{equation*}
  i.e.,
  \(\tuple{\abstr[k_1,k_2], \poset[*]{\z},
    {\bposet[*]{k_1}{k_2}{\z}}, \id}\) is a Galois connection.
\end{lemma} 

\begin{proof}
  The proof consists in showing that \(\concr[k_1,k_2]\) and
  \(\abstr[k_1,k_2]\) satisfy the properties of
  Theorem~\ref{th:galoisprop}:
  \begin{enumerate}[label={(\arabic*)}]
  \item\label{itemone} \(\abstr[k_1,k_2]\), \(\id\) are monotonic;
  \item\label{itemtwo} \(\id \conc \abstr[k_1, k_2]\) is extensive, i.e.,
    \(\sigma \subseteq \abstr[k_1,k_2](\sigma)\) for all
    \(\sigma \in \poset[*]{\z}\);
  \item\label{itemthree} \(\abstr[k_1,k_2] \conc\concr\) is reductive, i.e.,
    \(\abstr[k_1,k_2](\sigma_b) \subseteq \sigma_b\) for all
    \(\sigma_b \in \bposet[*]{k_1}{k_2}{\z}\).
  \end{enumerate}
  To start let's prove~\ref{itemone}. Of course \(\id\) is monotone by
  definition. For \(\abstr[k_1,k_2]\) we have to prove that given any
  \(\sigma, \tau \in \poset[*]{\z}\) s.t.\ \(\sigma \subseteq \tau\)
  it holds that
  \(\abstr[k_1,k_2](\sigma) \subseteq \abstr[k_1,k_2](\tau)\). Notice
  that since \(\sigma \subseteq \tau\) it holds that
  \(\max(\sigma) \leq \max(\tau)\) and
  \(\min(\sigma) \geq \min(\tau)\), which means by
  Definition~\ref{def:abstrnrb}
  \(\abstr[k_1,k_2]{\sigma} \subseteq \abstr[k_1,k_2]{\tau}\), which
  is our thesis.

  \medskip

  \noindent
  Both~\ref{itemtwo} and~\ref{itemthree} follow from
  Definition~\ref{def:abstrnrb}. For~\ref{itemtwo} for all
  \(\sigma \in \poset[*]{\z}\) either
  \(\abstr[k_1,k_2]({\sigma}) = \sigma\) or
  \(\abstr[k_1,k_2]({\sigma}) = \top\), hence in both cases
  \(\sigma \subseteq \abstr[k_1,k_2]({\sigma})\)
  holds. For~\ref{itemthree}, for all
  \(\sigma_b \in \bposet[*]{k_1}{k_2}{\z}\) it holds that
  \(\max(\sigma_b) \leq k_2\) and \(\min(\sigma_b) \geq k_1\), hence
  \begin{equation}\label{eq:equality2}
    \abstr[k_1,k_2]({\sigma_b}) = \sigma_b
  \end{equation}
  and therefore \(\abstr[k_1,k_2]({\sigma_b}) \subseteq \sigma_b\)
  holds.  Moreover,~\eqref{eq:equality2} means that for all
  \(\sigma_b \in \bposet[*]{k_1}{k_2}{\z}\) that
  \(\abstr[k_1,k_2] \conc \id = \id\), which means by
  Definition~\ref{def:insertion} that we formed a Galois insertion:
  \begin{equation*}
    \tuple{\poset[*]{\z}, \subseteq} \galoiS{\abstr[k_1,k_2]}{\id} \tuple{\bposet[*]{k_1}{k_2}{\z}, \subseteq}
  \end{equation*}
\end{proof}

Notice that since \(\bCnr\) and \(\bbCnr{k_1}{k_2}\) are respectively
the point-wise lifing of \(\poset[*]{\z}\) and
\(\bposet[*]{k_1}{k_2}{\z}\) there is also a Galois insertion between
them:
\begin{equation*}
  \tuple{\bCnr, \ovdot\subseteq}
  \galoiS{\abstr[k_1,k_2]}{\id}
  \tuple{\bbCnr{k_1}{k_2}, \ovdot\subseteq}
\end{equation*}
Since we have a galois connection between the non relational
collecting domain \(\bCnr\) and its bounded version
\(\bbCnr{k_1}{k_2}\) we can define an abstract inductive semantics
which is sound by construction:
\begin{definition}[\textbf{Abstract bounded non relational collecting semantics}]
  Let \(k_1, k_2 \in \z\) s.t.\ \(k_1\leq k_2\).  We define basic
  expressions over the bounded non relational collecting semantics
  \({\absem[\bbCnr{k_1}{k_2}]{\cdot} : \expr \to \bbCnr{k_1}{k_2} \to
    \bbCnr{k_1}{k_2}}\) as
  \begin{equation*}
    \absem[\bbCnr{k_1}{k_2}]{\com[e]} \defin \abstr[k_1,k_2] \left(\absem[\bCnr]{\com[e]} \right)
  \end{equation*}
  i.e.\ the best correct abstraction.
\end{definition}

\begin{lemma}[\textbf{Bounded non relational collecting is sound}]\label{le:soundnr}
  Let \(k_1, k_2 \in \z\) s.t.\ \(k_1\leq k_2\). For all
  \(\abstract\eta \in \bbCnr{k_1}{k_2}\) it holds that
  \begin{equation*}
    \left(\semnr{\com} \conc \id \right)\abstract\eta \ovdot\subseteq \left(\id \conc \bsemnr{k_1}{k_2}{\com}\right) \abstract\eta
  \end{equation*}
  i.e., \(\bsemnr{k_1}{k_2}{\cdot}\) is sound w.r.t.\ \(\semnr{\cdot}\).
\end{lemma}

\begin{proof}
  The proof floowis from the fact that \(\bsemnr{k_1}{k_2}{\cdot}\) is
  defined as the bca on basic expressions over \(\bCnr\) and there is
  a Galois connection
  \begin{equation*}
    \tuple{\bCnr, \ovdot\subseteq} \galoiS{\abstr[k_1,k_2]}{\id} \tuple{\bbCnr{k_1}{k_2}, \ovdot\subseteq}
  \end{equation*}
  Hence by Lemma~\ref{le:bcainducessoundness} our thesis 
  \begin{equation*}
    \left(\semnr{\com} \conc \id \right) \abstract\eta \ovdot\subseteq \left(\id \conc \bsemnr{k_1}{k_2}{\com}\right) \abstract\eta
  \end{equation*}
  holds.
\end{proof}

The next step is to notice that as a consequence of bounds and
Lemma~\ref{le:incnr} also bounded non collecting relational semantics
can compute the same result of its unbounded counterpart, provided
that the second one does not diverge.

\begin{theorem}
  Let \(\com\in\imp\) be a program. Then, for all finitely supported
  \(\eta : \Var_{\com} \to \bCnr\) and \(k_1, k_2 \in \z\) s.t.\
  \(\bCnr_{\com,\eta} \preceq \bbCnr{k_1}{k_2}\), i.e.,
  \(k_1 \leq \min(\eta) - \nlbound{\com}\) and
  \(k_2 \geq \max(\eta) + \nbound{\com}\) then for all
  \(\var\in\Var_{\com}\):
  \begin{equation*}
    |\semnr{\com}\eta\var| \neq +\infty \quad \Rightarrow \quad \semnr{\com}\eta\var = \bsemnr{k_1}{k_2}{\com}\eta\var
  \end{equation*}
  i.e., if the analysis over \(\bCnr\) for some variable \(\var\) does
  not diverge, then the analysis over \(\bbCnr{k_1}{k_2}\) for that
  variable converges to the same result.
\end{theorem}

\begin{proof}
  The proof will proceed by induction on the program \(\com\),
  covering first the base cases of \(\expr\) expressions and then the
  inductive cases of \(\imp\). Notice that because of
  Lemma~\ref{le:soundnr}
  \(\semnr{\com} \eta \ovdot\subseteq \bsemnr{k_1}{k_2}{\com}\eta\)
  already holds for every \(k_1, k_2 \in \z\) s.t.\ \(k_1\leq k_2\),
  hence what we have to prove for every case is that
  \(|\semnr{\com}\eta\var| \neq +\infty \Rightarrow \semnr{\com} \eta
  \var \ovdot\supseteq \bsemnr{k_1}{k_2}{\com}\eta\var\).
  \begin{inductive}
    \case{\(\var\in I\)} In this case we have to prove that for all
    \(\var[y] \in \Var_{\com}\)
    \[|\semnr{\var\in I}\eta\var[y]| \neq + \infty \Rightarrow
      \semnr{\var\in I}\eta \var[y] \ovdot\supseteq
      \bsemnr{k_1}{k_2}{\var\in I}\eta\var[y]\] with
    \(k_1\leq \min(\eta) - \nlbound{\var \in I}\) and
    \(k_1\geq \max(\eta) + \nbound{\var \in I}\). Notice that since
    \(\semnr{\var \in I}\eta\) updates only the variable \(\var\), the
    only case we're interested in is when \(\var[y] = \var\). This
    applies also for next base cases, hence we will omit this
    observation in future base cases.

    \case{\(\var := k\)} In this case

    \case{\(\var := \var[y] + k\)} In this case
  \end{inductive}
\end{proof}
