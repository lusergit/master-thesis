\section{Computing non-relational collecting semantics}\label{sec:computingnonrel}

In this section, we demonstrate the ability to compute a further
abstraction of non-relational collecting semantics to infer the
termination of non-relational collecting analysis. We illustrate that
merely bounding individual variables, as done with intervals, is
insufficient. Intuitively, we can no longer deduce that when one
variable diverges, it does not impact the behavior of other variables.
To start, we rely on Definition~\ref{def:minmax} of \(\min\) and
\(\max\) values of an abstract environment \(\rho\) to bound the non
relational collecting domain \(\bCnr\) in this way:

\begin{definition}[Bounded non-relational collecting domain]
  We define
  \(\bbCnr{\low}{\upp} \defin \Var_{\com} \to \bounded{\poset[*]{\z}}{\low}{\upp}\) where
  \begin{equation*}
    \bposet[*]{\low}{\upp}{\z} = \{S \subseteq \z \mid S \neq \emptyset \land \forall x \in S \quad \low \leq x \leq \upp\} \cup \{\top\}
  \end{equation*}
\end{definition}
Notice that differently from what happened in the case of intervals,
for the domain \(\bCnr\) we have no appropriate way of representing an
unbounded element: unbounded intervals could be represented with
elements with the shape
\(\interval{a}{+\infty}, \interval{-\infty}{b}\) with \(a,b\in
\z\). For arbitrary subsets of \(\z\) we have instead to rely on a
single \(\top\) element.

\medskip

\noindent
A first observation we can make is that by definition there are no
infinite ascending nor descending chains, as every chain is bounded by
above by some \(\upp \in \z\) and below by some \(\low\in\z\). Moreover,
there is a Galois connection between this abstract domain and its
unbounded counterpart \(\poset\z\). First let's define the
concretization and abstraction maps

\begin{definition}\label{def:abstrnrb}
  Let \(\low,\upp\in\z\) s.t.\ \(\low \leq \upp\). We define a
  concretization map
  \(\sconcr[\low,\upp] : \bposet[*]{\low}{\upp}{\z} \to \poset[*]{\z}\) as
  the identity function
  \begin{equation*}
    \sconcr[\low,\upp] = \id
  \end{equation*}
  similarly we define an abstraction map
  \(\sabstr[\low,\upp] : \poset[*]{\z} \to \bposet[*]{\low}{\upp}{\z}\) in
  the following way
  \begin{equation*}
    \sabstr[\low,\upp](S) = \begin{cases}
      S & \text{if } \inf(S) \geq \low \land \sup(S) \leq \upp \\
      \top & \text{otherwise}
    \end{cases}
  \end{equation*}
\end{definition}

\begin{lemma}
  Given \(\low, \upp \in \z\) s.t.\ \(\low\leq \upp\).
  \begin{equation*}
    \tuple{\poset[*]\z, \subseteq} \galois{\sabstr[\low,\upp]}{\sconcr[\low,\upp]} \tuple{\bposet[*]{\low}{\upp}{\z}, \subseteq}
  \end{equation*}
  i.e.,
  \(\tuple{\sabstr[\low,\upp], \poset[*]{\z},
    {\bposet[*]{\low}{\upp}{\z}}, \sconcr[\low,\upp]}\) is a Galois connection.
\end{lemma} 

\begin{proof}
  The proof consists in showing that \(\sconcr[\low,\upp]\) and
  \(\sabstr[\low,\upp]\) satisfy the properties of
  Theorem~\ref{th:galoisprop}:
  \begin{enumerate}[label={(\arabic*)}]
  \item\label{itemone} \(\sabstr[\low,\upp]\), \(\id\) are monotonic;
  \item\label{itemtwo} \(\id \conc \sabstr[\low, \upp]\) is extensive, i.e.,
    \(\sigma \subseteq \sabstr[\low,\upp](\sigma)\) for all
    \(\sigma \in \poset[*]{\z}\);
  \item\label{itemthree} \(\sabstr[\low,\upp] \conc\sconcr\) is reductive, i.e.,
    \(\sabstr[\low,\upp](\sigma_b) \subseteq \sigma_b\) for all
    \(\sigma_b \in \bposet[*]{\low}{\upp}{\z}\).
  \end{enumerate}
  To start let's prove~\ref{itemone}. Of course \(\id\) is monotone by
  definition. For \(\sabstr[\low,\upp]\) we have to prove that given any
  \(\sigma, \tau \in \poset[*]{\z}\) s.t.\ \(\sigma \subseteq \tau\)
  it holds that
  \(\sabstr[\low,\upp](\sigma) \subseteq \sabstr[\low,\upp](\tau)\). Notice
  that since \(\sigma \subseteq \tau\) it holds that
  \(\max(\sigma) \leq \max(\tau)\) and
  \(\min(\sigma) \geq \min(\tau)\), which means by
  Definition~\ref{def:abstrnrb}
  \(\sabstr[\low,\upp]{\sigma} \subseteq \sabstr[\low,\upp]{\tau}\), which
  is our thesis.

  \medskip

  \noindent
  Both~\ref{itemtwo} and~\ref{itemthree} follow from
  Definition~\ref{def:abstrnrb}. For~\ref{itemtwo} for all
  \(\sigma \in \poset[*]{\z}\) either
  \(\sabstr[\low,\upp]({\sigma}) = \sigma\) or
  \(\sabstr[\low,\upp]({\sigma}) = \top\), hence in both cases
  \(\sigma \subseteq \sabstr[\low,\upp]({\sigma})\)
  holds. For~\ref{itemthree}, for all
  \(\sigma_b \in \bposet[*]{\low}{\upp}{\z}\) it holds that
  \(\max(\sigma_b) \leq \upp\) and \(\min(\sigma_b) \geq \low\), hence
  \begin{equation}\label{eq:equality2}
    \sabstr[\low,\upp]({\sigma_b}) = \sigma_b
  \end{equation}
  and therefore \(\sabstr[\low,\upp]({\sigma_b}) \subseteq \sigma_b\)
  holds.  Moreover,~\eqref{eq:equality2} means that for all
  \(\sigma_b \in \bposet[*]{\low}{\upp}{\z}\) that
  \(\sabstr[\low,\upp] \conc \id = \id\), which means by
  Definition~\ref{def:insertion} that we formed a Galois insertion:
  \begin{equation*}
    \tuple{\poset[*]{\z}, \subseteq}
    \galoiS{\sabstr[\low,\upp]}{\sconcr[\low,\upp]}
    \tuple{\bposet[*]{\low}{\upp}{\z}, \subseteq}
    \qedhere
  \end{equation*}
\end{proof}
Notice that since \(\bCnr\) and \(\bbCnr{\low}{\upp}\) are respectively
the point-wise lifting of \(\poset[*]{\z}\) and
\(\bposet[*]{\low}{\upp}{\z}\) there is also a Galois insertion between
them:
\begin{equation*}
  \tuple{\bCnr, \ovdot\subseteq}
  \galoiS{\abstr[\low,\upp]}{\concr[\low,\upp]}
  \tuple{\bbCnr{\low}{\upp}, \ovdot\subseteq}
\end{equation*}
Where
\(\abstr[\low,\upp](\eta) = \lambda \var . \sabstr[\low,\upp](\eta\var)\)
and \(\concr[\low,\upp] = \id\).  Since we have a Galois connection
between the non relational collecting domain \(\bCnr\) and its bounded
version \(\bbCnr{\low}{\upp}\) we can define an abstract inductive
semantics which is sound by construction:
\begin{definition}[Bounded non-relational collecting semantics]
  Let \(\low, \upp \in \z\) s.t.\ \(\low\leq \upp\).  We define basic
  expressions over the bounded non relational collecting semantics
  \({\absem[\bbCnr{\low}{\upp}]{\cdot} : \expr \to \bbCnr{\low}{\upp} \to
    \bbCnr{\low}{\upp}}\) as
  \begin{equation*}
    \absem[\bbCnr{\low}{\upp}]{\com[e]} \defin \abstr[\low,\upp] \conc \absem[\bCnr]{\com[e]}
  \end{equation*}
  i.e.\ the best correct abstraction.
\end{definition}

\begin{lemma}[Bounded non-relational collecting is sound]\label{le:soundnr}
  Let \(\low, \upp \in \z\) s.t.\ \(\low\leq \upp\). For all
  \(\abstract\eta \in \bbCnr{\low}{\upp}\) it holds that
  \begin{equation*}
    \left(\semnr{\com} \conc \id \right)\abstract\eta \ovdot\subseteq \left(\id \conc \bsemnr{\low}{\upp}{\com}\right) \abstract\eta
  \end{equation*}
  i.e., \(\bsemnr{\low}{\upp}{\cdot}\) is sound w.r.t.\ \(\semnr{\cdot}\).
\end{lemma}

\begin{proof}
  The proof follows from the fact that \(\bsemnr{\low}{\upp}{\cdot}\) is
  defined as the bca on basic expressions over \(\bCnr\) and there is
  a Galois connection
  \begin{equation*}
    \tuple{\bCnr, \ovdot\subseteq} \galoiS{\abstr[\low,\upp]}{\concr[\low,\upp]} \tuple{\bbCnr{\low}{\upp}, \ovdot\subseteq}
  \end{equation*}
  Hence by Lemma~\ref{le:bcainducessoundness} our thesis 
  \begin{equation*}
    \left(\semnr{\com} \conc \id \right) \abstract\eta \ovdot\subseteq \left(\id \conc \bsemnr{\low}{\upp}{\com}\right) \abstract\eta
  \end{equation*}
  holds.
\end{proof}

By using \(\low, \upp\) properly, we can introduce a notion of order
between bounded non relational collecting domain. More in detail,
given \(a,b,c,d \in \z\) s.t.\ \(a\leq b\) and \(c \leq d\). Then
\(\preceq\) is a relation order s.t.
\begin{equation*}
  \bbCnr{a}{b} \preceq \bbCnr{c}{d} \iff a \leq c \land d \leq b
\end{equation*}

We bounded our analysis the same way we did with interval analysis in
Definition~\ref{def:boundedint}. This initial solution however has a
problem. Consider the following code snippet:
\begin{lstlisting}[language=Imp,caption=Snippet where bounded analysis diverges from the unbounded counterpart, label=code3]
  /* Pa */ 
  x := 0
  y := 0
  /* Pb */ 
  while (x < 1)
    x := x + 1
    y := y + 2
  /* Pc */ 
  if (y = 1)
    x := 2
\end{lstlisting}

This example highlights the main problem of just considering the
bounds for each variable. When a variable exceeds the program bound
and reaches \(\top\) this implies a loss of information, which leads
to a difference between the non-relational collecting semantics and
its bounded counterpart, even when the first one stays inside the
bounds.

To highlight the problem let's compute the exact semantics
\(\semnr{\com[P]}\), to better address the problem, consider
\(\com[P] = {\com[P]}_a\seq {\com[P]}_b \seq {\com[P]}_c\) and an
initial environment \([\var\mapsto\top,\var[y]\mapsto\top] = \eta\).
\begin{description}
\item[\(\semnr{{\com[P]}_a}\eta\)] infers the invariant
  \(\eta_a = [\var\mapsto \{0\}, \var[y]\mapsto\{0\}]\). This is our
  starting point.
\item[\(\semnr{{\com[P]}_b}\eta_a\)] infers the invariant
  \(\eta_b = [\var\mapsto\{0,1\},
  \var[y]\mapsto\mathbb{P}\cap\n]\). The analysis filters for all
  elements greater than \(1\) for the variable \(\var\), while
  variable \(\var[y]\) grows indefinitely, each time increasing by
  \(2\), which means that it can assume any positive even number,
  hence \(\mathbb{P} \cap \n\).
\item[\(\semnr{{\com[P]}_c}\eta_b\)] infers the final result
  \begin{equation*}
    \eta_c = [\var\mapsto\{0,1\}, \var[y]\mapsto\mathbb{P}\cap\n]
  \end{equation*}
  The guard \(\var[y] = 1\) in the \col{if} statement filters for
  \(\mathbb{P}\cap\n\cap\{1\} = \emptyset\) and therefore the variable
  \(\var\) is never assigned to 2, leaving the result unchanged.
\end{description}

Let's consider instead an hypothetical analysis over
\(\bbCnr{\low}{\upp}\), with \(\low,\upp\) chosen accordingly to the
program.
\begin{description}
\item[\(\bsemnr{\low}{\upp}{{\com[P]}_b}\eta_a\)] infers
  \(\eta_b' = [\var\mapsto\{0,1\}, \var[y]\mapsto\top]\), since at
  some point \(\var[y]\) will exceed the bound, hence the iteration
  will widen to \(\top\).
  
\item[\(\bsemnr{\low}{\upp}{{\com[P]}_c}\eta_b'\)] now has to filer
  again for the guard \(\var[y] = 1\), but in this case
  \(\top \cap \{1\} = \{1\}\), hence the final invariant becomes
  \begin{equation*}
    \eta_c' = [\var\mapsto\{0,1,2\}, \var[y] \mapsto \top]
  \end{equation*}
\end{description}
hence \(\eta_c'\neq\eta_c\), which means that
\(\bsemnr{\low}{\upp}{\cdot}\) and \(\semnr{\cdot}\) diverge even when
some variable in \(\semnr{\cdot}\) stays inside the program bounds.

We hypothesize that it is feasible to deduce the exact infinite
invariant, given that all the necessary information for its generation
is syntactically accessible. Prior research on this topic, such as the
work of~\cite{Lefaucheux2024}, primarily addresses Presburger
arithmetic, which is beyond the scope of this thesis.

For this reason our approach consists in smashing the \(\top\) element
of our analysis. Remember that the original problem we want to solve
(roughly) is the non-termination of the analysis,

% che accade quando l'analisi di una variabile diverge all'interno di
% un loop usando le iterazioni di kleene.

\begin{definition}[Smashed \(\top\) non realtional collecting]
  Let
  \begin{equation*}
    \bposet{\low}{\upp}{\z} \defin \{S \subseteq \z \mid S \neq \emptyset \land \forall x \in S \quad \low \leq x \leq \upp\}.
  \end{equation*}
  We define \(\btbCnr{\low}{\upp}\) as
  \begin{equation*}
    \btbCnr{\low}{\upp} \defin (\Var \to \bposet{\low}{\upp}{\z}) \cup \{\bot, \top\}
  \end{equation*}
\end{definition}

we can build a Galois connection with \(\bbCnr{\low}{\upp}\) for some
fixed \(\low, \upp \in \z\):

\begin{definition}
  Let \(\low, \upp \in \z\) s.t.\ \(\low \leq \upp\),
  \(\eta \in \bbCnr{\low}{\upp}\) and
  \(\overline\eta \in \btbCnr{\low}{\upp}\). Then the abstraction map
  \({\tabstr[\low,\upp]} : \bbCnr{\low}{\upp} \to \btbCnr{\low}{\upp}\) is
  defined as
  \begin{equation*}
    {\tabstr[\low,\upp]}(\eta) = \begin{cases}
      \top & \text{if } \exists\var\in\Var \text{ s.t. } \eta\var = \top \\
      \eta & \text{otherwise}
    \end{cases}
  \end{equation*}
  while concretization map
  \({\tconcr[\low,\upp]} : \btbCnr{\low}{\upp} \to \bbCnr{\low}{\upp}\)
  is defined as
  \begin{align*}
    \tconcr[\low,\upp](\top) & = \lambda \var \in \Var . \top \\
    \tconcr[\low,\upp](\overline\eta) & = \overline\eta
  \end{align*}
\end{definition}

We can now define base expressions based on the bca with the bounded
non relational collecting semantics:

\begin{definition}\label{def:boundedtop}
  Let \(\com[e] \in \expr\). The semantics of base expressions over
  \(\btbCnr{\low}{\upp}\) is defined as
  \begin{align*}
    \absem[\btbCnr{\low}{\upp}]{\com[e]}(\top) \defin & \top \\
    \absem[\btbCnr{\low}{\upp}]{\com[e]}(\eta) \defin & \left({\tabstr[\low,\upp]}\conc \absem[\bbCnr{\low}{\upp}]{\com[e]}\right)\eta
  \end{align*}
\end{definition}
Notice that this is \emph{not} the b.c.a.\ on the basic expressions:
the \(\top\) element is preserved. This is proved in
Lemma~\ref{le:btsemnrtop}.

Once again, \(\semi[\btbCnr{\low}{\upp}]{\cdot}\) is defined accordingly
to the abstract inductive semantics of Definition~\ref{def:abstrsem}.
Notice that contrary to latter definition of \(\bbCnr{\low}{\upp}\) in
this case we have a smashed top element. The idea is that whenever a
variable diverges we infer that the whole precise analysis diverges,
in order to solve Problem~\ref{problem1} and decide analysis
termination.  For simplicity, from now on we will refer to
\(\semi[\btbCnr{\low}{\upp}]{\cdot}\) as \(\btsemnr{\low}{\upp}{\cdot}\).

We preliminarly prove a simple but useful property of
\(\btsemnr{\low}{\upp}{\cdot}\): it preserves the \(\top\) element.

\begin{lemma}[\(\btsemnr{\low}{\upp}{\cdot}\) preserves \(\top\)]\label{le:btsemnrtop}
  Let \(\low, \upp \in \z\) s.t.\ \(\low \leq \upp\) and \(\com\in\imp\)
  \begin{equation*}
    \btsemnr{\low}{\upp}{\com}\top = \top
  \end{equation*}
\end{lemma}

\begin{proof}
  We proceed by induction on the program \(\com\) to prove that
  \begin{equation*}
    \btsemnr{\low}{\upp}{\com}\top = \top
  \end{equation*}
  \begin{inductive}
    \case{\(\com[e]\)} In the case of base expressions
    \(\com[e]\in\expr\) the Lemma follows from the definition of
    base expressions over the top element.
    \begin{align*}
      \btsemnr{\low}{\upp}{\com[e]}\top & = \absem[\btbCnr{\low}{\upp}]{\com[e]}(\top) \\
                                      & = \top & \text{by Definition~\ref{def:boundedtop}}
    \end{align*}

    \case{\(\com_1 \ndet \com_2\)} In this case
    \begin{align*}
      \btsemnr{\low}{\upp}{\com_1\ndet\com_2}\top & = \btsemnr{\low}{\upp}{\com_1}\top \ovdot\cup \btsemnr{\low}{\upp}{\com_2}\top \\
                                                & = \top \ovdot\cup \top & \text{By inductive hypothesis} \\
                                                & = \top
    \end{align*}
    \case{\(\com_1 \seq \com_2\)} In this case
    \begin{align*}
      \btsemnr{\low}{\upp}{\com_1\seq\com_2}\top & = \btsemnr{\low}{\upp}{\com_2}\left(\btsemnr{\low}{\upp}{\com_1}\top\right) \\
                                               & = \btsemnr{\low}{\upp}{\com_2}\top & \text{By induction on }\com_1 \\
                                               & = \top & \text{By induction on }\com_2
    \end{align*}
    \case{\(\fix\com\)} In this case
    \begin{align*}
      \btsemnr{\low}{\upp}{\fix\com}\top & = \lfp\left(\lambda \mu . \top \ovdot\cup \btsemnr{\low}{\upp}{\com}\mu\right) \ovdot\supseteq \top & \text{By definition of }\lfp \\
                                       & = \top & \text{By definition of }\top
    \end{align*}
  \end{inductive}
\end{proof}

\begin{theorem}
  Let \(\com\in\imp\) be a program. Then, for all finitely supported
  \(\eta \in \btbCnr{\low}{\upp}\) and \(\low, \upp \in \z\) s.t.\
  \(\bCnr_{\com,\eta} \preceq \bbCnr{\low}{\upp}\), i.e.,
  \(\low \leq \min(\eta) - \nlbound{\com}\) and
  \(\upp \geq \max(\eta) + \nbound{\com}\) then
  \begin{equation*}
    \btsemnr{\low}{\upp}{\com}\eta \neq \top
    \quad
    \implies
    \quad
    \semnr{\com}\eta = \btsemnr{\low}{\upp}{\com}\eta
  \end{equation*}
  i.e., if the analysis over \(\btbCnr{\low}{\upp}\) does not diverge,
  then the analysis over \(\bCnr\) converges to the same result.
\end{theorem}

\begin{proof}
  The proof will proceed by induction on the program \(\com\),
  covering first the base cases of \(\expr\) expressions and then the
  inductive cases of \(\imp\). Notice that because of
  Lemma~\ref{le:soundnr}
  \begin{equation*}
    \semnr{\com}\tconcr[\low,\upp](\overline\eta)
    \ovdot\subseteq
    \semi[\bbCnr{\low}{\upp}]{\com}\tconcr[\low,\upp](\overline\eta)
    \ovdot\subseteq
    \tconcr[\low,\upp](\btsemnr{\low}{\upp}{\com}\overline\eta)
    % \ovdot\subseteq \tconcr[\low,\upp]\left(\btsemnr{\low}{\upp}{\com}\eta\right)
  \end{equation*}
  already holds for every \(\low, \upp \in \z\) s.t.\ \(\low\leq \upp\)
  and \(\overline\eta \in \btbCnr{\low}{\upp}\), hence what we have to
  prove for every case is that
  \begin{equation*}
    \btsemnr{\low}{\upp}{\com}\eta \neq \top \implies \semnr{\com} \eta \ovdot\supseteq \btsemnr{\low}{\upp}{\com}\eta
  \end{equation*}
  First notice that it cannot be \(\overline\eta = \top\), otherwise
  by Lemma~\ref{le:btsemnrtop} \(\btsemnr{\low}{\upp}{\com}\top = \top\)
  and therefore the hypothesis
  \(\btsemnr{\low}{\upp}{\com}\top \neq \top\) is not
  respected. Furthermore, notice that
  \(\btsemnr{\low}{\upp}{\com}\overline\eta \neq \top\) implies that
  \(\tconcr[\low,\upp](\btsemnr{\low}{\upp}{\com}\overline\eta) =
  \btsemnr{\low}{\upp}{\com}\overline\eta\), due to the definition of
  \(\tconcr[\low,\upp]\).
  \begin{inductive}
    \case{\(\com[e]\)} In this case we have to prove that
    \begin{equation*}
      \btsemnr{\low}{\upp}{\com[e]}\overline{\eta} \neq \top \implies
      \semnr{\com[e]}\tconcr[\low,\upp](\overline\eta) = \tconcr[\low,\upp](\btsemnr{\low}{\upp}{\com[e]}\overline\eta)
    \end{equation*}
    First, if \(\overline\eta = \bot\) we can notice that
    \begin{equation*}
      (\tabstr[\low\upp] \conc \abstr[\low,\upp] \conc \semnr{\com[e]})\bot
      = \bot =
      \semnr{\com[e]}\tconcr[\low,\upp](\bot)
    \end{equation*}
    which is our thesis.

    The last case is when \(\top \neq \overline\eta\neq\bot\) and
    therefore for all \(\var\in\Var\)
    \(\overline\eta\var \in \bposet{\low}{\upp}{\z}\). In this case by
    Lemma~\ref{le:incnr} and Lemma~\ref{le:decnr} for all
    \(\var[y]\in\Var\) both
    \begin{align*}
      \max(\semnr{\com[e]}\tconcr[\low,\upp](\overline\eta)\var[y]) & \leq \max(\tconcr[\low,\upp](\overline\eta)) + \bound{\com[e]} \leq \max(\tconcr[\low,\upp](\overline\eta)) + \nbound{\com[e]} = \upp \\
      \min(\semnr{\com[e]}\tconcr[\low,\upp](\overline\eta)\var[y]) & \geq \min(\tconcr[\low,\upp](\overline\eta)) - \lbound{\com[e]} \geq \min(\tconcr[\low,\upp](\overline\eta)) - \nlbound{\com[e]} = \low
    \end{align*}
    hence by definition of \(\abstr[\low,\upp]\) and
    \(\tabstr[\low,\upp]\)
    \begin{equation*}
      \left(\tabstr[\low,\upp] \conc \abstr[\low,\upp]\right)\left(\semnr{\com[e]}\tconcr[\low,\upp](\overline\eta)\right)
      =
      \semnr{\com[e]}\tconcr[\low,\upp](\overline\eta)
    \end{equation*}
    which is our thesis.

    \case{\(\com_1 \ndet \com_2\)} In this case we have to prove that
    \begin{equation*}
      \btsemnr{\low}{\upp}{\com_1 \ndet \com_2}\eta \neq \top
      \implies
      \semnr{\com_1 \ndet \com_2}\eta = \btsemnr{\low}{\upp}{\com_1 \ndet\com_2}\eta
    \end{equation*}
    with \(\low \leq \min(\eta) - \nlbound{\com_1 \ndet \com_2}\) and
    \(\upp \geq \max(\eta) + \nbound{\com_1 \ndet \com_2}\). First we
    can notice that since
    \(\btsemnr{\low}{\upp}{\com_1 \ndet \com_2}\eta\var =
    \btsemnr{\low}{\upp}{\com_1}\eta\var \cup
    \btsemnr{\low}{\upp}{\com_2}\eta\var\) our hypothesis
    \({\btsemnr{\low}{\upp}{\com_1 \ndet \com_2}\eta\var \neq \top}\)
    implies both \({\btsemnr{\low}{\upp}{\com_1}\eta\var \neq \top}\) and
    \({\btsemnr{\low}{\upp}{\com_2}\eta\var \neq \top}\). Hence by choice
    of \(\low\) and \(\upp\) we can use the inductive hypothesis and
    state that
    \begin{align*}
      \semnr{\com_1}\eta\var & = \btsemnr{\low}{\upp}{\com_1}\eta\var \\
      \semnr{\com_2}\eta\var & = \btsemnr{\low}{\upp}{\com_2}\eta\var
    \end{align*}
    and by closure over \(\cup\)
    \begin{equation*}
      \semnr{\com_1 \ndet\com_2}\eta\var =
      \semnr{\com_1}\eta\var \cup \semnr{\com_2}\eta\var =
      \btsemnr{\low}{\upp}{\com_1}\eta\var \cup \btsemnr{\low}{\upp}{\com_2}\eta\var =
      \btsemnr{\low}{\upp}{\com_1 \ndet\com_2}\eta\var
    \end{equation*}
    which is our thesis.

    \case{\(\com_1 \seq \com_2\)} In this case we have to prove that
    \begin{equation*}
      \btsemnr{\low}{\upp}{\com_1\seq\com_2}\eta \neq \top
      \implies
      \semnr{\com_1 \seq\com_2} \eta = \btsemnr{\low}{\upp}{\com_1 \seq\com_2}\eta
    \end{equation*}
    for all \(\low \leq \min(\eta) -\nlbound{\com_1\seq\com_2}\) and
    \(\upp \geq \max(\eta) + \nbound{\com_1\seq\com_2}\). First let's
    recall that
    \begin{equation*}
      \btsemnr{\low}{\upp}{\com_1\seq\com_2}\overline\eta
      =
      \btsemnr{\low}{\upp}{\com_2}\left(\btsemnr{\low}{\upp}{\com_1}\overline\eta\right)
    \end{equation*}
    and we are under the hypothesis
    \(\btsemnr{\low}{\upp}{\com_1\seq\com_2}\overline\eta \neq \top\),
    which means that
    \(\btsemnr{\low}{\upp}{\com_2}\overline\eta\neq \top \neq
    \btsemnr{\low}{\upp}{\com_2}\overline\eta'\) where
    \(\overline\eta' = \btsemnr{\low}{\upp}{\com_1}\overline\eta\),
    otherwise by Lemma~\ref{le:btsemnrtop} we would contradict the
    hypothesis
    \(\btsemnr{\low}{\upp}{\com_1\ndet\com_2}\overline\eta\neq\top\).
    Then, by inductive hypothesis
    \(\semnr{\com_1}\overline\eta =
    \btsemnr{\low}{\upp}{\com_1}\overline\eta\) for all
    \(\low \leq \min(\eta) - \nlbound{\com_1}\) and
    \(\upp \geq \max(\eta) + \nbound{\com_2}\). We can now call
    \(\overline\eta' = \btsemnr{\low}{\upp}{\com_1}\overline\eta\) and
    by inductive hypothesis again:
    \begin{equation*}
      \btsemnr{\low}{\upp}{\com_2}\eta' \neq \top
      \implies
      \semnr{\com_2}\eta' = \btsemnr{\low}{\upp}{\com_2}\eta'
    \end{equation*}
    for all \(\low \leq \min(\eta') - \nlbound{\com_2}\) and
    \(\upp \geq \max(\eta') + \nbound{\com}\). Notice that
    \(\min(\eta') \geq \min(\eta) - \nlbound{\com_1}\) and
    \(\max(\eta') \leq \max(\eta) + \nbound{\com_1}\) and therefore we
    can chose
    \(\low \leq \min(\eta) - \nlbound{\com_1} - \nlbound{\com_2}\) and
    \(\upp \geq \max(\eta) + \nbound{\com_1} + \nbound{\com_2}\). Since
    both inductive hypothesis hold, then following holds
    \begin{equation*}
      \btsemnr{\low}{\upp}{\com_1\seq\com_2}\eta \neq \top
      \implies
      \semnr{\com_1\seq\com_2}\eta = \btsemnr{\low}{\upp}{\com_1\seq\com_2}\eta
    \end{equation*}
    which is our thesis.
    
    \case{\(\fix\com\)} In this case we want to prove that
    \begin{equation*}
      \btsemnr{\low}{\upp}{\fix\com}\overline\eta \neq \top
      \implies
      \semnr{\fix\com}\overline\eta = \btsemnr{\low}{\upp}{\fix\com}\overline\eta
    \end{equation*}
    for all \(\low \geq \min(\overline\eta) - \nlbound{\fix\com}\) and
    \(\upp \leq \max(\overline\eta) + \nbound{\fix\com}\). Recall that by
    Lemma~\ref{le:soundnr} it always holds that
    \begin{equation*}
      \semnr{\fix\com}\overline\eta \ovdot\subseteq \btsemnr{\low}{\upp}{\fix\com}\overline\eta
    \end{equation*}
    We have therefore to prove that 
    \begin{equation}\label{eq:geq}
      \btsemnr{\low}{\upp}{\fix\com}\overline\eta \neq \top
      \implies
      \semnr{\fix\com}\overline\eta \ovdot\supseteq \btsemnr{\low}{\upp}{\fix\com}\overline\eta
    \end{equation}
    for all \(\low \leq \min(\overline\eta) - \nlbound{\fix\com}\) and
    \(\upp \geq \max(\overline\eta) + \nbound{\fix\com}\). To start notice that
    according to Lemma~\ref{le:sugar}
    \(\semnr{\fix\com}\overline\eta = \semnr{\kleene{(\com\ndet\tru)}}\overline\eta\),
    hence we can alternatively prove that
    \begin{equation*}
      \btsemnr{\low}{\upp}{\fix\com}\overline\eta\neq \top
      \implies
      \semnr{\fix\com}\overline\eta
      \ovdot\supseteq
      \bigcup_{i\in\n} {\left(\btsemnr{\low}{\upp}{\com \ndet \tru}\right)}^i\overline\eta
    \end{equation*}
    which implies Equation~\ref{eq:geq}. To start we will initially
    prove that for every \(i\in\n\) it holds that
    \begin{equation*}
      \semnr{\fix\com}\overline\eta \neq \top
      \implies
      %
      \semnr{\fix\com}\overline\eta \ovdot\supseteq
      {\left(\btsemnr{\low}{\upp}{\com \ndet \tru}\right)}^i\overline\eta
    \end{equation*}
    to then prove the first one by closure over \(\cup\). We will
    prove it by induction on \(i\):
    \begin{description}
      
    \item[Case] (\(i=0\)). In this case we have to prove that
      \begin{equation*}
        \btsemnr{\low}{\upp}{\fix\com}\overline\eta \neq \top
        \implies
        \semnr{\fix\com}\overline\eta \ovdot\supseteq \overline\eta
      \end{equation*}
      We can notice that by definition of \(\semnr{\fix\com}\) the
      thesis holds.
      
    \item[Case] (\(i \implies i+1\)). In this case we have to prove
      that
      \begin{equation*}
        \semnr{\fix\com}\overline\eta \neq \top
        \implies
        \semnr{\fix\com}\overline\eta \supseteq {\left(\btsemnr{\low}{\upp}\com\ndet\tru\right)}^{i+1}\overline\eta
      \end{equation*}
      First we can notice that
      \begin{align*}
        \semnr{\com\ndet\tru}(\semnr{\fix\com}\overline\eta) & = \semnr{\com}(\semnr{\fix\com}\overline\eta) \cup \semnr{\fix\com}\overline\eta \\
                                                    & = \semnr{\com}(\lfp(\lambda \mu . \overline\eta \cup \semnr{\com}\mu)) \cup \semnr{\fix\com}\overline\eta \\
                                                    & = \overline\eta \cup \semnr{\com}(\lfp(\lambda \mu . \overline\eta \cup \semnr{\com}\mu)) \cup \semnr{\fix\com}\overline\eta & \text{since } \overline\eta \subseteq \lfp(\lambda \mu . \overline\eta \cup \semnr{\com}\mu)\\
                                                    & = (\lambda \mu . \overline\eta \cup \semnr{\com}\mu)(\lfp(\lambda \mu . \overline\eta \cup \semnr{\com}\mu)) \cup \semnr{\fix\com}\overline\eta \\
                                                    & = (\lfp(\lambda \mu . \overline\eta \cup \semnr{\com}\mu)) \cup \semnr{\fix\com}\overline\eta & \text{by def.\ of } \lfp\\
                                                    & = \semnr{\fix\com}\overline\eta \cup \semnr{\fix\com}\overline\eta \\
                                                    & = \semnr{\fix\com}\overline\eta
      \end{align*}
      % can notice that since by hypothesis
      % \(|\semnr{\fix\com}\overline\eta\var| \neq \infty\), we can leverage
      % Lemma~\ref{le:incnr} and state that for all
      % \(\var\in \Var_{\com}\) the follwing hold
      % \begin{align*}
      %   \max(\semnr{\fix\com}\overline\eta\var) & \leq \max(\overline\eta) + \bound{\fix\com} \\
      %   \min(\semnr{\fix\com}\overline\eta\var) & \geq \min(\overline\eta) - \lbound{\fix\com}
      % \end{align*}
      Now we can preliminarly observe that by calling
      \(\beta = \semnr{\fix\com}\overline\eta\)
      \begin{equation}\label{eq:comfixbound}
       % \forall \var\in\Var \quad  \beta\var \neq\top \implies 
       \semnr{\com\ndet\tru}\beta \supseteq \btsemnr{\low}{\upp}{\com\ndet\tru}\beta
      \end{equation}
      In fact, for all \(\var\in \Var\)
      \begin{align*}
        \max(\semnr{\com\ndet\tru}\beta\var) & \leq \max(\beta) + \bound{\com\ndet\tru} = \max(\beta) + \bound{\com} \\
                                             & \leq \max(\overline\eta) + \bound{\fix\com} + \bound{\com} & \text{by Lemma~\ref{le:incnr}} \\
                                             & \leq \max(\overline\eta) + (n+2)\bound{\com} \\
                                             & \leq \max(\overline\eta) + (n+2)\nbound{\com} \\
                                             & \leq \max(\overline\eta) + \nbound{\fix\com} = \upp
      \end{align*}
      similarly for the min value
      \begin{equation*}
        \min(\semnr{\com\ndet\tru}\beta\var) \geq \min(\overline\eta) - \nlbound{\fix\com} = \low
      \end{equation*}
      hence
      \begin{align*}
        \beta = \semnr{\com\ndet\tru}\beta & \supseteq \btsemnr{\low}{\upp}{\com\ndet\tru}\beta & \text{by~\eqref{eq:comfixbound}}\\
                                           & \supseteq \btsemnr{\low}{\upp}{\com\ndet\tru}{\left(\btsemnr{\low}{\upp}{\com\ndet\tru}\right)}^i\overline\eta & \text{by induction on } i\\
                                           & = {\left(\btsemnr{\low}{\upp}{\com\ndet\tru}\right)}^{i+1}\overline\eta
      \end{align*}
    \end{description}
    Hence our thesis, for all \(i \in \n\)
    \begin{equation*}
      \btsemnr{\low}{\upp}{\fix\com}\overline\eta \top \implies
      % 
      \semnr{\fix\com}\overline\eta \supseteq
      {\left(\btsemnr{\low}{\upp}{\com\ndet\tru}\right)}^i\overline\eta
    \end{equation*}
    holds.  We can now conclude by noticing that our original thesis
    \begin{equation*}
      \btsemnr{\low}{\upp}{\fix\com}\overline\eta \neq \top \implies
      % 
      \semnr{\fix\com}\overline\eta \supseteq
      \bigcup_{i\in\n}{\left(\btsemnr{\low}{\upp}{\com\ndet\tru}\right)}^i\overline\eta =
      \btsemnr{\low}{\upp}{\fix\com}\overline\eta
    \end{equation*}
    also holds.
  \end{inductive}
\end{proof}

The latter theorem is a result similar to the result for the interval
domain with Theorem~\ref{th:bounded}. In its essence it states that
when doing static analysis with abstract interpretation using the non
relational collecting domain \(\bCnr\) for some program
\(\com\in\imp\) and an initial environment \(\eta \in \bCnr\) we can
consider a bounded version of the domain \(\btbCnr{\low}{\upp}\) with
\(\low = \min(\eta) - \nlbound{\com}\) and
\(\upp = \max(\eta) + \nbound{\com}\) (hence computed accordingly to
\(\com\) and \(\eta\)). Such domain has no infinite ascending or
descending chains, hence the Kleene iteration method is guaranteed to
halt in finite time. The result however is not as precise as the
bounded version of the interval analysis: while for intervals we could
compute the exact invariant, for the non-relational collecting domain
this is the case only when the bounded semantics is not
\(\top\). Intuitively this happens when all invariants inferred in the
computation are finite, i.e., if the analysis over \(\bCnr\) halts in
finite time. To put it differently, we can decide weather the analysis
over \(\bCnr\) halts in finite time or not.
