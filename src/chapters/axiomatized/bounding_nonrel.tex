\section{Bounded non-relational collecting
  semantics}\label{sec:boundingnonrel}

In this section on non-relational collecting semantics, we demonstrate
that while it is not possible to compute the most precise invariant
for any given program in \(\imp\), it is feasible to compute a more
abstract semantics. This abstract semantics provides valuable insights
into the termination behavior of the non-relational collecting
analyzer. Specifically, we show that the two abstract semantics align
when the abstract analyzer for the non-relational collecting semantics
terminates.

For an easier reading, we will refer to \(\semi[\bCnr]{\cdot}\) with
the same notation we used in \secref{sub:nonrelprop}:
\(\semnr{\cdot} = \semi[\bCnr]{\cdot}\).

\begin{lemma}\label{le:incnr}
  Let \(\com\in \imp\).
  % Then consider a function \({\max_{\Int} : \Int \to \z}\) s.t. for
  % all \({a\in\Int}\) it holds that \(x \leq \max_\Int(a)\) for all
  % \({x \in \concr[\Int](a)}\).
  For all \(\eta \in \bCnr\) and \(\var[y] \in \Var\), if
  \(\max(\semnr{\com} \eta \var[y]) \neq +\infty\) and
  \(\max(\semnr{\com}\eta \var[y]) > \bound{\com}\) then there exist a
  variable \(\var[z] \in \Var\) and an integer \(h \in \mathbb{Z}\)
  such that \(|h| \leq \bound{\com}\) and the following two properties
  hold:
  \begin{enumerate}[label=(\roman*)]
  \item\label{point1nrdef}
    \(\max(\semnr{\com}\eta \var[y]) = \max(\eta \var[z]) + h\);
  \item\label{point2nrdef} for all \(\eta' \in \bCnr\), if
    \(\eta' \ovdot\supseteq \eta\)
    % and \(\semnr{\com}\eta' \neq \top\)
    then
    \(\max(\semnr{\com}\eta' \var[y]) \geq \max(\eta' \var[z]) + h\).
  \end{enumerate}
  % dually, if \(\min(\semnr{\com}\eta\var[y]) \neq - \infty\) and
  % \(\min(\semnr{\com}\eta\var[y]) < \lbound{\com}\) then there exists a
  % variable \(\var[w] \in \Var\) and an integer \(i \in \z\) s.t.
  % \(|i| \leq \lbound{\com}\) and the following two properties hold:
  % \begin{enumerate}[label=(\roman*)]
  % \item \(\min(\semnr{\com}\eta \var[y]) = \min(\eta \var[w]) + i\); \label{point1min}
  % \item  for all \(\eta' \in \Int\), if \(\eta' \sqsupseteq \eta\)
  %   %   and \(\semnr{\com}\eta' \neq \top\)
  %   then
  %   \(\min(\semnr{\com}\eta' \var[y]) \geq \max(\eta' \var[w]) + i\). \label{point2min}
  % \end{enumerate}
\end{lemma}

\begin{proof}
  The proof is left in Appendix~\ref{ap:proofleincrnr}, since it is
  analogous to the proof of Lemma~\ref{le:inc}.
\end{proof}

\begin{remark}
  The key point is that in the base case \((\var \in I)\), we can draw
  similar conclusions as we did for intervals. This is because the
  filtering occurs on an interval \(I \in \inte\) rather than on an
  arbitrary decidable set. It is important to note that if we were to
  allow a generic decidable set in a guard, the result would not
  hold. Specifically, let \(S \in \poset{\z}\) be a decidable set and
  consider \(\var[y] = \var\); under these conditions, we observe that
  \begin{equation}
    \max(\semnr{\var\in S}\eta\var[y]) = \max(\eta[\var \mapsto \eta\var \cap S]\var) = \max(\eta\var \cap S)
  \end{equation}
  Since \(S\) is generally non-convex what happens is that from
  \(\eta \var \cap S \neq \emptyset\) and \(\max(S) = +\infty\) we can
  only deduce
  \begin{equation}
    \max(\eta\var \cap S) \leq \max(\eta\var)
  \end{equation}
  and not equality, providing a potential counterexample to the
  Lemma. For example consider the program \((\var\in\mathbb{P})\)
  % \begin{lstlisting}[caption=Potential counterexample, label=code2, language=Imp]
  %   x > 5;\end{lstlisting}
  where \(\mathbb{P}\) is the set of even numbers and an initial
  environment \({\eta \defin [\var\mapsto \mathbb{D} \cup \{2\}]}\),
  where \(\mathbb{D}\) is the set of odd numbers. Then
  \begin{equation*}
    {\semnr{\var\in\mathbb{P}}\eta\var} = \{2\}
  \end{equation*}
  and \(\max({\semnr{\var\in\mathbb{P}}\eta\var}) = 2\), while
  \(\bound{\semnr{\var\in\mathbb{P}}} = 0\).

  \medskip

  \noindent
  Hence both
  \(\max({\semnr{\var\in\mathbb{P}}\eta\var}) \neq +\infty\) and
  \(\max({\semnr{\var\in\mathbb{P}}\eta\var}) >
  \bound{\semnr{\var\in\mathbb{P}}}\) hold. The lemma would state that
  \(\exists \var[w] \in \Var\) and
  \(h \in \z \mid |h| \leq \bound{\var\in\mathbb{P}}\) s.t.
  \begin{enumerate}[label=(\roman*)]
  \item\label{eq:equality}
    \(\max(\semnr{\var\in \mathbb{P}}\eta\var) = \max(\eta\var[w]) + h\)
  \item
    \(\forall \eta'\sqsupseteq \eta \quad
    \max(\semnr{\var\in\mathbb{P}}\eta'\var[y]) \geq
    \max(\eta'\var[w]) + h\)
  \end{enumerate}
  and we can already see that~\ref{eq:equality} is false. In fact the
  only variable in the program is \(\var\), hence \(\var[w] = \var\),
  but \(\max(\eta\var) = +\infty\) and \(\forall h \in \z\) it happens
  that \(+\infty + h = +\infty\), hence
  \begin{equation*}
    \max(\semnr{\var\in\mathbb{P}}\eta\var) = 2 = +\infty = \max(\eta\var) + h
  \end{equation*}
  which is false.

  \todo[inline]{Fare esempio sul quaderno}
\end{remark}

The same applies for the increment on the lower bound, in a similar
fashion as for the intervals:

\begin{lemma}\label{le:decnr}
  Let \(\com\in \imp\).
  % Then consider a function \({\max_{\Int} : \Int \to \z}\) s.t. for
  % all \({a\in\Int}\) it holds that \(x \leq \max_\Int(a)\) for all
  % \({x \in \concr[\Int](a)}\).
  For all \(\eta \in \bCnr\) and \(\var[y] \in \Var\), if
  \(\min(\semnr{\com} \eta \var[y]) \neq -\infty\) and
  \(\min(\semnr{\com}\eta \var[y]) < -\bound{\com}\) then there exist
  a variable \(\var[z] \in \Var\) and an integer \(h \in \mathbb{Z}\)
  such that \(|h| \leq \bound{\com}\) and the following two properties
  hold:
  \begin{enumerate}[label=(\roman*)]
  \item\label{point1nrmin} \(\min(\semnr{\com}\eta \var[y]) = \min(\eta \var[z]) + h\); 
  \item\label{point2nrmin} for all \(\eta' \in \bCnr\), if
    \(\eta' \ovdot\supseteq \eta\)
    % and \(\semnr{\com}\eta' \neq \top\)
    then
    \(\min(\semnr{\com}\eta' \var[y]) \leq \min(\eta' \var[z]) + h\).
  \end{enumerate}
  % dually, if \(\min(\semnr{\com}\eta\var[y]) \neq - \infty\) and
  % \(\min(\semnr{\com}\eta\var[y]) < \lbound{\com}\) then there exists a
  % variable \(\var[w] \in \Var\) and an integer \(i \in \z\) s.t.
  % \(|i| \leq \lbound{\com}\) and the following two properties hold:
  % \begin{enumerate}[label=(\roman*)]
  % \item \(\min(\semnr{\com}\eta \var[y]) = \min(\eta \var[w]) + i\); \label{point1min}
  % \item  for all \(\eta' \in \Int\), if \(\eta' \sqsupseteq \eta\)
  %   %   and \(\semnr{\com}\eta' \neq \top\)
  %   then
  %   \(\min(\semnr{\com}\eta' \var[y]) \geq \max(\eta' \var[w]) + i\). \label{point2min}
  % \end{enumerate}
\end{lemma}

The proof is again left in Appendix~\ref{ap:minnr} as
Lemma~\ref{le:decnr2}.
