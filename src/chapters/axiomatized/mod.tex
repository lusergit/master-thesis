% !TeX root = ../mod.tex
\chapter{Program bounds and analysis termination}\label{ch:axiomatized}

In the last chapter we defined two abstract domains: the interval
domain and the non-relational collecting domain. Both are
non-relational domains, in the sense that they do not model the
relation between variables, but their possible values. Trough the
abstract inductive semantics of Definition~\ref{def:abstrsem} we
defined a simple abstract interpreter for the \(\imp\) language,
provided base expressions and an abstract domain \(\bA\). In
Chapter~\ref{ch:framework} we proved that the \(\imp\) language is
Turing-complete and therefore by Rice's theorem all non-trivial
semantic properties of programs written in such language are not
decidable.  
% Calcolare l'invariante esatto di un programma non è decidibile,
% appunto perchè è una proprietà semantica non triviale
Exact invariants fall into this category, as they tie the initial
state of a machine and a program to the output the program would
return.
% Se invece usiamo invarianti astratti?
% Abstract invariants, i.e., maps \(\Var \to \bA\), could be computable,
% as the loss of information might be enough to allow for computation

With this context, lets formalize the problem we will solve:

\begin{problem}[Exact analysis computation]\label{problem1}
  Given a program \(\com\in \imp\) and an abstract domain \(\bA\), for
  all \(\eta \in \bA\), compute
  \begin{equation*}
    \semi[\bA]{\com} \eta
  \end{equation*}
\end{problem}

In this chapter we argue that for the language \(\imp\) the abstract
semantics is computable for the domain \(\inte\), while for the domain
\(\bCnr\) we can only decide the termination of the analyzer.

\medskip

Observe that the exact computation of the abstract invariant provides,
already for our simple language, a precision which is not obtainable
with (basic) widening and narrowing. In the example in
Code~\ref{code1} if we consider the abstract domain \(\inte\), the
semantics maps \(\var\) and \(\var[y]\) to \(\interval{0}{2}\) and
\(\interval{6}{8}\) respectively, while widening/narrowing to
\(\interval{0}{+\infty}\) and \(\interval{6}{+\infty}\).

\begin{lstlisting}[caption=Code sample where analysis of
  \(\fix{\com}\) is less precise than \(\com^*\), label=code1,
  language=Imp]
  x:=0;
  y:=0;
  while (x<=5) do
     if (y=0) then
        y=y+1;
     endif;
     if (x=0) then
        x:=y+7;
     endif;
  done;
\end{lstlisting}

Of course, for the collecting semantics this is not the case. Already
computing a finite upper bound for loop invariants when they are
finite is impossible as this would allow to decide termination, as we
have seen in \secref{sec:finiteness}.


The main idea, based on previous research is to \emph{bound} the
domain \(\bA\). Each program is associated to a bound, an ideal value
above which for each variable we can safely assume that the program
diverges.  First, given a program, we associate the program with a
\emph{lower bound} and an \emph{upper bound}. The rough idea is that,
whenever a variable is beyond its bound, the behavior of the program
with respect to that variable becomes stable. % We also introduce an
% \emph{increment bound} which captures the largest increment or
% decrement that can affect a variable.

%% \subimport{./}{assumptions}
\subimport{./}{bounds}
\subimport{./}{computing}
\subimport{./}{computing_intervals}
\subimport{./}{bounding_nonrel}
\subimport{./}{computing_nonrel}
