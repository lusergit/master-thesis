\chapter{Program bounds and analysis termination}
\label{chap:axiomatized}

%% Framework, introduzione al linguaggio e alle sue proprietà, si
%% definisce la semantica e si dimostrano un paio di cose

In this chapter we argue that for the language \(\imp\) the abstract
semantics is computable in finite time without widening for abstract
domains with some properties.

Observe that the exact computation provides, already for our simple
language, a precision which is not obtainable with (basic) widening
and narrowing. In the example below if we consider the intervals
abstract domain, the semantics maps \(\var\) and \(\var[y]\) to
\([0,2]\) and \([6,8]\) resp., while widening/narrowing to
\([0,\infty]\) and \([6,\infty]\)

\begin{verbatim}
x:=0;
y:=0;
while (x<=5) do
   if (y=0) then
      y=y+1;
   endif;
   if (x==0) then
      x:=y+7;
   endif;
done;
end
\end{verbatim}

Of course, for the collecting semantics this is not the case. Already
computing a finite upper bound for loop invariants when they are
finite is impossible as this would allow to decide termination, as we
have seen in section \ref{sec:finiteness}. First let's formalize the
problem we want to solve.

\begin{problem}[Analysis termination]\label{problem1}
  Given a program \(\com\in \imp\) and an abstract domain \(\bA\) with
  \(\eta \in \bA\), decide
  \begin{equation*}
    \semi{\com} \eta =^? \top
  \end{equation*}
\end{problem}

To do so we present a novel technique, based on the idea of
\emph{bounds}. Each program is associated to a bound, an ideal value
above which for each variable we cannot guarantee convergence, and
therefore we can safely assume that the program diverges.  First,
given a program, we associate each variable with a \emph{single
  bound}, which captures both both an \emph{upper bound} and a
\emph{lower bound} for the variable. The rough idea is that, whenever
a variable is within its bound, the behavior of the program with
respect to that variable becomes stable. % We also introduce an
% \emph{increment bound} which captures the largest increment or
% decrement that can affect a variable.

%% \subimport{./}{assumptions}
\subimport{./}{bounds}
\subimport{./}{computing}
\subimport{./}{endintervals}