\section{Recursion theory}
\label{sec:recursionth}
This first section aims to provide background and terminology for the
parts in recursion theory that will follow. More in detail, we'll take
some notation from \cite{cutland1980computability} and introduce some
new notation based on the same book.

We start with functions: total and partial functions are essential to
recursion theory:

\begin{definition}[Total and partial functions]
  Let \(X,Y\) be two sets. We denote by \[X \to Y\] the set of all
  total functions from \(X\) to \(Y\).  And by
  \[X \hto Y\] the set of all partial functions from \(X\) to
  \(Y\).
\end{definition}
\noindent

Partial functions are actually functions from a subset
\(S \subseteq X\) which is called the \emph{natural domain} of \(f\).

\begin{definition}[Domain of partial functions]
  Let \(f : X \hto Y\). We write \(f(x)\downarrow\) to indicate that
  \(f\) is defined on \(x\), and \(f(x)\uparrow\) to indicate that
  \(f\) is undefined on \(x\). Hence
  \[dom(f) = \{x \in X \mid f(x)\downarrow\}\]
\end{definition}

% \begin{notation}[Implication overloading]
%   If \(S\subseteq Y\) then by \(f(x) \in S\) we mean \(f(x)\downarrow
%   \Rightarrow f(x) \in S\)
% \end{notation}

% \begin{notation}[Function Equality]
%   If \(f,g : X \hto Y\) then by \(f=g\) we mean that \(dom(f) =
%   dom(g)\) and for any \(x \in dom(f) = dom(g)\), \(f(x) = g(x)\).
% \end{notation}

We then need, mostly in \secref{sec:functionsimp} to talk about
partial recursive functions and their properties. We therefore define
partial recursive and total recursive functions as follows:

\begin{notation}[partial and total recursive
  functions]\label{bg:partialrec}
  By \(\partialrec[k]\) we denote the set of partial recursive
  functions on natural numbers, while by \(\n \tor \n\) we denote the
  set of partial recursive functions on natural numbers.
\end{notation}

We also need to talk about decidable properties and decidable sets. We
therefore introduce the notion of recursive and recursively enumerable
sets.

\begin{definition}[Recursively enumerable and recursive sets]
  A set \(A \subseteq \n^k\) is \emph{recursively enumerable} (r.e. or
  semi-decidable) if \(A = dom(f)\) for some \(f \in \partialrec[k]\).
  
  A set \(A\subseteq \n\) is a recursive set if both \(A\) and its
  complement \(\overline{A} = \n \setminus A\) are semi-decidable,
  i.e., there exists some \(f \in \n \tor \n\) s.t.
  \[f = \lambda n . (n \in A) ? 1 : 0\]
\end{definition}

\begin{lemma}[Computable function over a recursive set]\label{le:comprec}
  Given \(f : A \tor B\), let the domain \(A\) to be recursive. \(B\)
  is at least r. e.
\end{lemma}

\begin{proof}
  \(f : A \tor B\) total recursive function over a recursive set
  \(A\). We can write the function
  \[\bigx_B = \lambda x . sg(\mu z . |f(z) - x|)\] which is computable
  as it is composition of computable functions. In other terms, this
  function
  \begin{equation*}
    \bigx_B(x) =
    \begin{cases}
      1 & x \in B \\
      \uparrow & \text{otherwise}
    \end{cases}
  \end{equation*}
  is the semi-decision function for \(B\)
\end{proof}

\begin{observation}\label{obs:comprec}
  In general, \(B\) is not recursive, as it would means that both \(A
  \leq_m B\) and \(B \leq_m A\), which is not always the case, but it
  is r.e., as we could always write the inverse function as in lemma
  \ref{le:comprec} and derive a semi-decision function for the image
  of the function.
\end{observation}
