\section{Recursion Theory}

Following \cite{ranzato:analysis}, we'll provide the general
terminology and notation fr computable functions in recursion theory,
as in
\cite{cutland1980computability,odifreddi1992classical,rogers1987theory}.

\begin{definition}[Total functions]
  Let \(X,Y\) be two sets. Then \[X \to Y\] is the set of all total
  functions from \(X\) to \(Y\).
\end{definition}

\begin{definition}[Partial functions]
  Let \(X,Y\) be two sets. Then \[X \hto Y\] is the set of all partial
  functions from \(X\) to \(Y\).
\end{definition}
\noindent

Partial functions are actually functions from a subset
\(S \subseteq X\) which is called the \emph{natural domain} of \(f\)

\begin{definition}[Domain of partial functions]
  Let \(f : X \hto Y\). \(f(x)\downarrow\) means that \(f\) is defined
  on \(x\), \(f(x)\uparrow\) means that \(f\) is undefined on
  \(x\). Hence \[dom(f) = \{x \in X \mid f(x)\downarrow\}\]
\end{definition}

% \begin{notation}[Implication overloading]
%   If \(S\subseteq Y\) then by \(f(x) \in S\) we mean \(f(x)\downarrow
%   \Rightarrow f(x) \in S\)
% \end{notation}

% \begin{notation}[Function Equality]
%   If \(f,g : X \hto Y\) then by \(f=g\) we mean that \(dom(f) =
%   dom(g)\) and for any \(x \in dom(f) = dom(g)\), \(f(x) = g(x)\).
% \end{notation}

We then need, mostly in section \ref{sec:functionsimp} to talk about
partial recursive functions and their properties, namely the
properties and conclusion that we derive from well known computability
results. We therefore define partial recursive functions as follows:

\begin{notation}[Set of partial recursive functions]\label{bg:partialrec}
  By \(\partialrec[k]\) we denote the set of partial recursive
  functions on natural numbers
\end{notation}

\begin{notation}[Set of total recursive functions]
  By \(\n \tor \n\) we denote the set of partial recursive functions
  on natural numbers
\end{notation}

We also need to talk about decidable properties and decidable sets. We
therefore introduce the notion of recursive and recursively enumerable
sets.

\begin{definition}[Recursively enumerable set]
  \(A \subseteq \n^k\) is \emph{recursively enumerable} (r.e. or
  semi-decidable) if \(A = dom(f)\) for some \(f \in \partialrec[k]\)
\end{definition}

\begin{definition}[Recursive set]
  \(A\subseteq \n\) is a recursive set if both \(A\) and its
  complement \(\overline{A} = \n \setminus A\) are semi-decidable,
  i.e., there exists some \(f \in \n \tor \n\) s.t. \[f = \lambda n
  . (n \in A) ? 1 : 0\]
\end{definition}

\begin{lemma}[Computable function over a recursive set]\label{le:comprec}
  Given \(f : A \tor B\), let the domain \(A\) to be recursive. \(B\)
  is at least r. e.
\end{lemma}

\begin{proof}
  \(f : A \tor B\) total recursive function over a recursive set
  \(A\). We can write the function \[\bigx_B = \lambda x . sg(\mu z
  . |f(z) - x|)\] which is computable as it is composition of
  computable functions. In other terms, this is function
  \[\bigx_B(x)
  = \begin{cases}
    1 & x \in B \\
    \uparrow & \text{otherwise}
  \end{cases}\]
  which is the semi-decision function for \(B\)
\end{proof}

\begin{observation}\label{obs:comprec}
  In general, \(B\) is not recursive, as it would means that both \(A
  \leq_m B\) and \(B \leq_m A\), which is not always the case, but it
  is r.e., as we could always write the inverse function as in lemma
  \ref{le:comprec} and derive a semi-decision function for the image
  of the function.
\end{observation}
