\section{Introduction and related work}\label{sec:prev}

Abstract interpretation has faced a difficult challenge since the
beginning: the termination of the analyzer. When viewed from another
perspective, every interpreter can be considered the most precise
possible static analyzer. However, the critical issue is its
non-termination. Cousot addresses this problem from the outset
in~\cite{patrickradhia:one} by introducing two operators that have
since become standard: Widening and narrowing. The former is used to
infer properties related to the termination of a loop, albeit at the
cost of analysis precision (while still maintaining
soundness). However, is it possible to forgo the use of widening
operators and still achieve a sound analysis with the same precision
as the one utilizing these operators?\ \cite{Gawlitza2009} is the
first to introduce and demonstrate that this is indeed feasible,
presenting an algorithm for calculating fixed-point equation systems
using operations and the abstract domain of intervals in polynomial
time.  It relies on a generalization of the Bellman-Ford algorithm
from~\cite{bellman1958algo} to find a lest solution for system of
equations with addition and least upper bound. The method is then
extended until the authors build a cubic time algorithm for the class
of interval equations (equations with variables in the interval
domain).
