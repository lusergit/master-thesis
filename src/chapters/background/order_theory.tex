\section{Order theory}

Within Theoretical Computer Science, especially in the field of
semantics, partial orders hold significant importance. They are
extensively employed in Abstract Interpretation, as highlighted in
\cite{mine:course}, serving different levels of the theory to model
core notions. These notions include the idea of approximation, where
certain analysis results may be less precise than others, creating a
partial order where some results are incomparable. Moreover, partial
orders are fundamental in conveying the concept of soundness: an
analysis is deemed sound if its result is more general than the actual
behavior. These mathematical notions, essential for discussions
surrounding the Abstract Interpretation formalism, primarily involve
order and lattice theory.

\begin{definition}[Partiall ordered set]
  Let \(X\) be a non-empty set, \(\sqsubseteq \subseteq X \times X\)
  be a reflexive, antisymmetric and transitive relation on that set,
  i.e., \(\forall x,y,z \in X\):

  \begin{enumerate}
  \item \(x \sqsubseteq x\) (reflexivity)
  \item \(x \sqsubseteq y \wedge y \sqsubseteq x \Rightarrow x = y\)
    (antisymmetry)
  \item \(x \sqsubseteq y \wedge y \sqsubseteq z \Rightarrow x
    \sqsubseteq z\) (transitivity)
  \end{enumerate}

  Then the tuple \(\langle X, \sqsubseteq\rangle\) is a
  \emph{partially ordered set} (POSet).
\end{definition}

\begin{definition}[Least upper bound]
  Let \(\langle X, \sqsubseteq \rangle\) be a POSet and let \(Z
  \subseteq X\). We say that \(\overline{z}\) is an \emph{upper bound}
  on \(Z\) if \(\forall z \in Z\) \(z \sqsubseteq \overline{z}\). It
  is the \emph{least upper bound} of \(Z\) (denoted as
  \(\lubof[X]{Z}\)) if \(\forall z' \in Z\) upper bounds on \(Z\),
  \(\overline{z} \sqsubseteq z'\).
\end{definition}

\begin{definition}[Greatest lower bound]
  Let \(\langle X, \sqsubseteq \rangle\) be a POSet and let \(Z
  \subseteq X\). We say that \(\overline{z}\) is a \emph{lower bound}
  on \(Z\) if \(\forall z \in Z\) \(\overline{z} \sqsubseteq z\). It
  is the \emph{greatest lower bound} of \(Z\) (denoted as
  \(\glbof[X]{Z}\)) if \(\forall z' \in Z\) upper bounds on \(Z\), \(z'
  \sqsubseteq \overline{z}\).
\end{definition}

Usually then we're talking about least and greatest lower bound the
bigger set is often implicit, and we therefore simply write
\(\lubof{Z}\) and \(\glbof{Z}\).

In abstract interpretation we often rely on special kinds of POSets,
where the existance of the greatest lower bound and the least upper
bound is ensured for each subset of the original POSet. These sets are
called complete lattices

\begin{definition}[Complete lattice]
  A POSet \(\langle X, \sqsubseteq\rangle\) is called a \emph{complete
  lattice} if \[\forall Y \subseteq X \quad \exists \lubof{Y} \wedge
  \exists \glbof{Y}\]
\end{definition}

We're also interested in join morphisms

\begin{definition}[Join Morphism]
  Let \(\langle X, \sqsubseteq_X \rangle, \langle Y, \sqsubseteq_Y
  \rangle\) be two lattices. Let \(f : X \to Y\) be a mappging from
  \(X\) to \(Y\). \(F\) is a \emph{join morphism} if \(\forall x_1,
  x_2 \in X\) \(f(x_1 \vee x_2) = f(x_1) \vee f(x_2)\). In this case
  we can also say that \(f\) \emph{preserves the joins}.
\end{definition}
