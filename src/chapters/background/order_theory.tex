\section{Order theory}\label{sec:ordertheory}

Within Theoretical Computer Science, especially in the field of
semantics, partial orders hold significant importance. They are
extensively employed in Abstract Interpretation, as highlighted in
\cite{mine:course}, serving different levels of the theory to model
core notions. These notions include the idea of approximation, where
certain analysis results may be less precise than others, creating a
partial order where some results are incomparable. Moreover, partial
orders are fundamental in conveying the concept of soundness: an
analysis is deemed sound if its result is an over-approximation of the
actual behavior. These mathematical notions, essential for discussions
surrounding the Abstract Interpretation formalism, primarily involve
order and lattice theory.

\begin{definition}[Partiall ordered set]
  Let \(X\) be a non-empty set, \(\sqsubseteq \subseteq X \times X\)
  be a reflexive, anti-symmetric and transitive relation on that set,
  i.e., \(\forall x,y,z \in X\):

  \begin{enumerate}
  \item \(x \sqsubseteq x\) (reflexivity)
  \item \(x \sqsubseteq y \wedge y \sqsubseteq x \Rightarrow x = y\)
    (antisymmetry)
  \item \(x \sqsubseteq y \wedge y \sqsubseteq z \Rightarrow x
    \sqsubseteq z\) (transitivity)
  \end{enumerate}

  Then the tuple \(\langle X, \sqsubseteq\rangle\) is a
  \emph{partially ordered set} (POSet).
\end{definition}

\begin{definition}[Least upper bound]
  Let \(\langle X, \sqsubseteq \rangle\) be a POSet and let
  \(Z \subseteq X\). We say that \(\overline{z} \in Z\) is an
  \emph{upper bound} of \(Z\) if \(\forall z \in Z\)
  \(z \sqsubseteq \overline{z}\). It is the \emph{least upper bound}
  of \(Z\) if \(\forall z'\) upper bounds of \(Z\),
  \(\overline{z} \sqsubseteq z'\).
\end{definition}

\begin{definition}[Greatest lower bound]
  Let \(\langle X, \sqsubseteq \rangle\) be a POSet and let
  \(Z \subseteq X\). We say that \(\overline{z}\in Z\) is a
  \emph{lower bound} of \(Z\) if \(\forall z \in Z\)
  \(\overline{z} \sqsubseteq z\). It is the \emph{greatest lower
    bound} of \(Z\) if \(\forall z'\) upper bounds of \(Z\),
  \(z' \sqsubseteq \overline{z}\).
\end{definition}

Usually then we are talking about least and greatest lower bound the
host set is often implicit, and we therefore simply write
\(\lub({Z})\) and \(\glb({Z})\).  In abstract interpretation we often
rely on special kinds of POSets, where the existence of the greatest
lower bound and the least upper bound is ensured for each subset of
the original POSet. These sets are called complete lattices

\begin{definition}[Complete lattice]
  A POSet \(\langle X, \sqsubseteq\rangle\) is called a \emph{complete
  lattice} if \[\forall Y \subseteq X \quad \exists \lubof{Y} \wedge
  \exists \glbof{Y}\]
\end{definition}

Complete lattices are a subset of the class of chain complete partial
ordered sets. These kinds of partial orders are defined using the
concept of chains:

\begin{definition}[Chain]
  Let \(\tuple{D, \sqsubseteq}\) be a partially ordered set. Then
  \(Y\subseteq D\) is a chain if for any \(y_1, y_2 \in Y\) it holds
  that
  \begin{equation*}
    y_1 \sqsubseteq y_2 \vee y_2 \sqsubseteq y_1
  \end{equation*}
\end{definition}

\begin{definition}[\textbf{CCPOs}] A chain complete partially ordered
  set (ccpo) is a poset \(\tuple{D, \sqsubseteq}\) such that every
  chain of \(D\) has a least upper bound.
\end{definition}

% We're also interested in join morphisms

% \begin{definition}[Join Morphism]
%   Let \(\langle X, \sqsubseteq_X \rangle, \langle Y, \sqsubseteq_Y
%   \rangle\) be two lattices. Let \(f : X \to Y\) be a mappging from
%   \(X\) to \(Y\). \(F\) is a \emph{join morphism} if \(\forall x_1,
%   x_2 \in X\) \(f(x_1 \vee x_2) = f(x_1) \vee f(x_2)\). In this case
%   we can also say that \(f\) \emph{preserves the joins}.
% \end{definition}

The last building block we will use in the following chapters is the
Kleene-Knaster-Tarski theorem. This theorem is a fundamental result in
order theory and provides a powerful tool for analyzing and
establishing the existence of fixed points in complete lattices. To
state it we need to first link functions and order theory with some
definitions

\begin{definition}[Monotone functions]
  Let \(\tuple{D, \sqsubseteq}\) and \(\tuple{D', \sqsubseteq'}\) be
  complete lattices. The total function \(f : D \to D'\) is
  \emph{monotone} if
  \begin{equation*}
    d_1 \sqsubseteq d_2 \Rightarrow f(d_1) \sqsubseteq' f(d_2)
  \end{equation*}
\end{definition}

% More than monotonicity we are interested in continuous functions on a
% complete lattice

Monotonicity however does not preserve upper bounds, just orders. In
particular if we take a chain \(Y \subseteq D\) of some ccpo
\(\tuple{D, \subseteq}\) and some monotone function \(f : D \to D\),
in general \(\sqcup\{f(d) \mid d \in Y\} \sqsubseteq f(\sqcup Y)\),
but not \(\sqcup\{f(d) \mid d \in Y\} = f(\sqcup Y)\). Therefore we
introduce the concept of continuity, functions that preserve both
order and upper bounds

\begin{definition}[Continuous functions]
  Let \(\tuple{X, \sqsubseteq}\) and \(\tuple{X', \sqsubseteq'}\) be
  ccpos. The total function \(f : D \to D'\) is \emph{continuous} if
  \begin{itemize}
  \item \(f\) is monotone;
  \item \(\sqcup'\{f(d) \mid d \in D\} = f(\sqcup X)\)
  \end{itemize}
\end{definition}

Continuous functions over ccpos are important for the Kleene
fixed-point theorem, usually attributed to Tarski from
\cite{tarski1955lattice}, which is also called Kleene iteration. It
gives us an iteration strategy to find the least fixpoint of a
function over a ccpo, provided that the function is continuous.

\begin{theorem}[Kleene fixed-point]\label{th:fixpoint}
  Let \(f : D \to D\) be a continuous function over a chain complete
  partial order \(\tuple{D , \sqsubseteq}\) with the lest element
  \(\bot\). Then
  \begin{equation*}
    \lfp(f) = \sqcup\{f^n(\bot) \mid n \in \n\}
  \end{equation*}
  where
  \begin{itemize}
  \item \(f^0 = id\)
  \item \(f^{n+1} = f \circ f^n \quad \forall n \in \n\)
  \end{itemize}
  is the least fix point of \(f\).
\end{theorem}
