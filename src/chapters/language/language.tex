\section{The Imp language}

In order to talk about program properties we need a language to
express such programs. We define the \(\imp\) language, made of
regular commands and based on Kozen’s Kleene algebra with tests,
described in \cite{kozen1997kleene}.  We denote by \(\n\) the set of
naturals with the usual order, extended with the top element
\(+\infty\), s.t.  \(n \leq +\infty \quad \forall n \in \n\). We also
extend addition and subtraction by letting, for
\(z \in \n \quad +\infty + z = + \infty - z = + \infty\) and if
\(n \leq m\) \(n - m = 0\).  We focus on the following
non-deterministic language.
\begin{align*}
  \expr \ni \com[e] ::= & \; \var \in S \mid \tru \mid \ff \mid \var := k \mid \var := \var[y] + k \\
                        %% & \;  \mid \var := \var[y] - k \\
  \imp\starless \ni \com[D] ::= & \; \com[e] \mid \com[D + D] \mid \com[D ; D] \\
  \imp \ni \com[C] ::= & \; \com[D] \mid \com + \com \mid \com ; \com \mid \com^* \mid \fix{\com}
\end{align*}

where \(\var, \var[y] \in \Var\) a finite set of variables of
interest, i.e., the variables appearing in the considered program,
\(S\subseteq \n\) is (possibly empty) \emph{decidable} set of numbers,
\(a \in \n, b\in\n\cup\{+\infty\}, a\leq b, k \in \n\)
is any finite integer constant.
