\section{The Imp language}

In order to talk about program properties we need a language to
express such programs. We define the \(\imp\) language, made of
regular commands and based on Kozen’s Kleene algebra with tests,
described in~\cite{kozen1997kleene}.  We denote by \(\z\) the set of
naturals with the usual order, extended with the least and greatest
elements \(-\infty\) and \(+\infty\), s.t.
\(-\infty \leq z \leq +\infty\) for all \(z \in \z\). We also extend
addition and subtraction by letting, for all \(z \in \z\) it holds
that \(+\infty + z = + \infty - z = + \infty\) and
\(-\infty + z = -\infty + z = -\infty\).  We focus on the following
non-deterministic language.
\begin{align*}
  \expr \ni \com[e] ::= & \; \var \in I % \mid \tru \mid \ff 
                          \mid \var := k \mid \var := \var[y] + k % \mid \var := \var[y] - k 
  \\
  \imp\starless \ni \com[D] ::= & \; \com[e] \mid \com[D + D] \mid \com[D ; D] \\
  \imp \ni \com[C] ::= & \; \com[D] \mid \com + \com \mid \com ; \com \mid \com^* \mid \fix{\com}
\end{align*}

where \(\var, \var[y] \in \Var\) a finite set of variables of
interest, i.e., the variables appearing in the considered program,
\(I\in\Int\) an \emph{interval} (as defined in
Definition~\ref{def:int}),
\({a \in \z \cup \{-\infty\}}, {b \in \z \cup\{+\infty\}}, {a\leq b}\)
and \({k \in \z}\) is any finite integer constant.
