\section{Semantics}

In order to talk about program properties in our language, we first
need to define its \emph{semantics}. In the following section we
introduce both a collecting semantics in order to reason about program
\emph{invariants} and a small step semantics, in order to reason about
program \emph{execution}.% The first building block are
% \emph{environments}, maps from the set of variables to their numerical
% value: \[\] With
% environments we can describe the semantics of basic expressions, and
% later the semantics of comman \(\com\) in the \(\imp\) language.

\begin{definition}[Semantics of Basic Expressions]
  Let \emph{environments} be the maps from the set of variables to
  their numerical value:
  \(\env \defin \{\rho \mid \rho : \Var \to \z\}\). For basic
  expressions \(e \in \expr\) the \emph{concrete semantics}
  \(\bsem{ \cdot} : \expr \to \env \to \smashed{\env}\) is
  inductively defined by:
  \begin{align*}
    \bsem{\var\in S} \rho & \defin \begin{cases} \rho & \rho(\var)\in S \\ \bot & \text{otherwise} \end{cases} \\
    % \bsem{\var\in [a,b]} \rho & \defin \begin{cases} \rho & \rho(\var)\in [a,b] \\ \bot & \text{otherwise} \end{cases} \\
    % \bsem{\var\leq k} \rho & \defin \begin{cases} \rho & \rho(\var)\leq k \\ \bot & \text{otherwise} \end{cases} \\
    % \bsem{\var> k} \rho & \defin \begin{cases} \rho & \rho(\var) > k \\ \bot & \text{otherwise} \end{cases} \\
    \bsem{\tru} \rho & \defin \rho \\
    \bsem{\ff} \rho & \defin \bot \\
    \bsem{\var:= k} \rho & \defin \rho [\var\mapsto k] \\
    \bsem{\var:= \var[y] + k} \rho & \defin \begin{cases} \rho [\var\mapsto \rho(\var[y]) + k] & \rho \neq \bot \\ \bot & \text{otherwise}\end{cases} \\
    % \bsem{\var:= \var[y] - k} \rho & \defin \begin{cases} \rho [\var\mapsto \rho(\var[y]) - k] & \rho \neq \bot \\ \bot & \text{otherwise}\end{cases} \\
  \end{align*}
\end{definition}

The next building block is the concrete collecting semantics for the
language, it associates each program in \(\imp\) to a function which,
given a set of initial environments \(X\) ``collects'' the set of
final states produced by executing the program from \(X\).

\begin{definition}[Concrete collecting semantics]
  Let \(\dom \defin \langle \poset{\env} , \subseteq \rangle\) be a
  complete lattice called \emph{concrete collecting domain}. The
  \emph{concrete collecting semantics} for \(\imp\) is given by the
  total function \(\sem{\cdot} : \imp \to \dom \to \dom\) which maps
  each program \(\com \in \imp\) to a total function over the complete
  lattice \(\dom\), inductively defined as follows: given
  \(X \in \dom\)

  \begin{align*}
    \sem{\com[e]} X & \defin \{\bsem{\com[e]} \rho \mid \rho \in X,
                      \bsem{\com[e]} \rho \neq \bot\} \\
    \sem{\com[C_1] + \com[C_2]} X & \defin \sem{\com[C_1]} X \cup \sem{\com[C_2]} X \\
    \sem{\com[C_1] ; \com[C_2]} X & \defin \sem{\com[C_2]}(\sem{\com[C_1]} X) \\
    \sem{\com[C^*]} X & \defin \bigcup_{i \in \n} \sem{\com[C]}^i X \\
    \sem{\fix{C}} X & \defin \lfp(\lambda Y \in\poset{\env} . (X \cup \sem{\com}Y))
  \end{align*}
\end{definition}

We observe that the semantics we described is additive:

\begin{observation}[Semantics Additivity]
  Given \(\com\in\imp\), \(X,Y \in \dom\),
  \[\sem{\com}(X\cup Y) = \sem{\com}X \cup \sem{\com}Y\]
\end{observation}

\begin{proof}
  We will prove it by induction on the program \(\com\):

  \noindent
  \paragraph*{Base case: \\}
  \begin{itemize}
  \item \(\com \equiv \com[e]\) therefore
    \begin{align*}
      \sem{\com[e]}(X\cup Y) & = \{\seme{\com[e]}\rho \mid \rho \in X \cup Y, \seme{\com[e]}\rho \neq \bot\} \\
                             & = \{\seme{\com[e]}\rho \mid \rho \in X \vee \rho \in Y, \seme{\com[e]}\rho \neq \bot\} \\
                             & = \{\seme{\com[e]}\rho \mid \rho \in X , \seme{\com[e]}\rho \neq \bot\} \cup \{\seme{\com[e]}\rho \mid \rho \in Y , \seme{\com[e]}\rho \neq \bot\} \\
                             & = \sem{\com[e]}X \cup \sem{\com[e]}Y
    \end{align*}
  \end{itemize}
  
  \noindent
  \paragraph*{Inductive cases: \\}
  \begin{itemize}
  \item \(\com \equiv \com_1 + \com_2\) therefore
    \begin{align*}
      \sem{\com_1 + \com_2}(X\cup Y) & = \sem{\com_1}(X \cup Y) \cup \sem{\com_2}(X \cup Y) & \text{by definition}\\
                           & = \sem{\com_1}X \cup \sem{\com_1}Y \cup \sem{\com_2}X \cup \sem{\com_2}Y & \text{by inductive hypothesis} \\
                           & = \sem{\com_1 + \com_2} X \cup \sem{\com_1 + \com_2} Y
    \end{align*}
    
  \item \(\com \equiv \com_1; \com_2\) therefore
    \begin{align*}
      \sem{\com_1 ; \com_2}(X\cup Y) & = \sem{\com_2}(\sem{\com_1}(X \cup Y)) & \text{by definition}\\
                                     & = \sem{\com_2}(\sem{\com_1}X \cup \sem{\com_1}Y) & \text{by inductive hypothesis}\\
                                     & = \sem{\com_2}(\sem{\com_1}X) \cup \sem{\com_2}(\sem{\com_1}Y) & \text{by inductive hypothesis}
    \end{align*}
    
  \item \(\com \equiv \com^*\) therefore
    \[\sem{\com^*}(X\cup Y) = \bigcup_{i\in\n}\sem{C}^i(X\cup Y)\] in
    order to use the inductive hypothesis we have to show
    that
    \[\forall i \in \n \quad \sem{\com}^i(X \cup Y) = \sem{\com}^i X
      \cup \sem{\com}^i Y\] to do that, we work again by induction on
    \(i\):
    \begin{itemize}
    \item the base case is \(i = 0\) then \(X \cup Y = X \cup Y\).
    \item For the inductive case we need to show that
      \(i \Rightarrow i+1\):
      \begin{align*}
        \sem{\com}^{i+1}\left(X\cup Y\right) & = \sem{\com}\left(\sem{\com}^i \left(X \cup Y\right)\right) \\
                                             & = \sem{\com}\left(\sem{\com}^iX \cup \sem{\com}^iY\right) & \text{by induction hypothesis on } i \\
                                             & = \sem{\com}\left(\sem{\com}^iX\right) \cup \sem{\com}\left(\sem{\com}^iY\right) & \text{by induction hypothesis on } \com \\
                                             & = \sem{\com}^{i+1}X \cup \sem{\com}^{i+1}Y
      \end{align*}
    \end{itemize}
    Therefore we can use the inductive hypothesis internally and say
    \begin{align*}
      \sem{\com^*}\left(X\cup Y\right) & = \bigcup_{i\in\n}\sem{C}^i\left(X\cup Y\right) \\
                                       & = \bigcup_{i\in\n}\left(\sem{C}^iX \cup \sem{C}^iY\right) & \text{for the later statement} \\
                                       & = \left(\bigcup_{i\in\n}\sem{C}^iX\right) \cup \left(\bigcup_{i\in\n}\sem{C}^iY\right) \\
                                       & = \sem{\com^*}X \cup \sem{\com^*}Y \qedhere
    \end{align*}
  \end{itemize}
\end{proof}

We can also observe that a program induces a monotone function in the
concrete domain \(\dom\):

\begin{observation}
  Given a program \(\com\in\imp\), the semantics function
  \(\sem{\com} : \dom \to \dom\) is monotone.
\end{observation}

\begin{proof}
  We can prove this by induction on the program \(\com\in\imp\). Let
  \(X, Y\in\dom, X \subseteq Y\). We want to prove that
  \(\sem{\com}X \subseteq \sem{\com}Y\).

  \noindent
  \paragraph*{Base case: \\}
  \begin{itemize}
  \item \(\com \equiv \com[e]\) therefore
    \begin{align*}
      \sem{\com[e]}X & = \{\seme{\com[e]}\rho \mid \rho \in X , \seme{\com[e]}\rho \neq \bot\} \\
      \sem{\com[e]}Y & = \{\seme{\com[e]}\rho \mid \rho \in Y , \seme{\com[e]}\rho \neq \bot\} \\
    \end{align*}
    \(X\subseteq Y\) therefore \(\rho \in X \Rightarrow \rho \in Y\)
    which also means that
    \(\rho' \in \sem{\com[e]}X \Rightarrow \rho' \in \sem{\com[e]}Y\),
    which is a shorthand for
    \(\sem{\com[e]}X \subseteq \sem{\com[e]}Y\)
  \end{itemize}

  \noindent
  \paragraph*{Inductive cases: \\}
  \begin{itemize}
  \item \(\com \equiv \com_1 + \com_2\) therefore we need to show that
    \(\sem{com_1+\com_2}X \subseteq \sem{com_1+\com_2}Y \)
    \begin{align*}
      \sem{\com_1 + \com_2}X & = \sem{\com_1}X \cup \sem{\com_2}X \\
      \sem{\com_1 + \com_2}Y & = \sem{\com_1}Y \cup \sem{\com_2}Y \\
    \end{align*}
    by inductive hypothesis both
    \(\sem{\com_1}X \subseteq \sem{\com_1}Y\) and
    \(\sem{\com_2}X \subseteq \sem{\com_2}Y\) and therefore
    \(\sem{\com_1 + \com_2}X \subseteq \sem{\com_1 + \com_2}Y\).
  \item \(\com \equiv \com_1 ; \com_2\) therefore we need to show that
    \(\sem{com_1;\com_2}X \subseteq \sem{com_1;\com_2}Y \)
    \begin{align*}
      \sem{\com_1 ; \com_2}X & = \sem{\com_2}(\sem{\com_1}X) \\
      \sem{\com_1 ; \com_2}Y & = \sem{\com_2}(\sem{\com_1}Y) \\
    \end{align*}
    By induction hypothesis
    \(\sem{\com_1} X \subseteq \sem{\com_1}Y\), and by induction
    hypothesis again
    \(\sem{\com_2}(\sem{\com_1} X) \subseteq \sem{\com_2}(\sem{\com_1}
    Y)\) which means
    \(\sem{\com_1 + \com_2} X \subseteq \sem{\com_1 + \com_2}Y\).
  \item \(\com \equiv \com^*\) therefore we need to show that
    \(\sem{\com^*}X \subseteq \sem{\com^*}Y \).
    \begin{align*}
      \sem{\com^*}X & = \bigcup_{i\in\n}\sem{\com}^i X \\
      \sem{\com^*}Y & = \bigcup_{i\in\n}\sem{\com}^i Y
    \end{align*}
    what we need to prove is that
    \[\forall j \in \n \quad \bigcup_{i=0}^j\sem{\com}^i X \subseteq
      \bigcup_{i=0}^j\sem{\com}^i Y\]
    we can do this by induction on \(j\):
    \begin{itemize}
    \item \(j=0\) therefore \(X\subseteq Y\) which is true by
      hypothesis.
    \item Now we need to work on the inductive case
      \(j \Rightarrow j+1\). Notice that it holds that
      \begin{align*}
        \bigcup_{i=0}^{k+1}\sem{\com}^i X & = X \cup \bigcup_{i=1}^{k+1}\sem{\com}^i X & \text{by definition} \\
                                          & = X \cup \sem{\com}\left(\bigcup_{i=0}^{k}\sem{\com}^i X\right) & \text{by additivity}
      \end{align*}
      and also for \(Y\)
      \begin{equation*}
        \bigcup_{i=0}^{k+1}\sem{\com}^i Y = Y \cup \sem{\com}\left(\bigcup_{i=0}^{k}\sem{\com}^i Y\right)
      \end{equation*}
      Also notice that
      \begin{enumerate}[label=(\roman*)]
      \item \(X \subseteq Y\) by hypothesis;
      \item
        \(\bigcup_{i=0}^{k}\sem{\com}^i X \subseteq
        \bigcup_{i=0}^{k}\sem{\com}^i Y\) by inductive hypothesis;
      \item 
        \(\sem{\com}\left(\bigcup_{i=0}^{k}\sem{\com}^i X\right)
        \subseteq \sem{\com}\left(\bigcup_{i=0}^{k}\sem{\com}^i
          Y\right)\) by additivity.
      \end{enumerate}
      Therefore
      \[\bigcup_{i=0}^{k+1}\sem{\com}^i X = X \cup
        \sem{\com}\left(\bigcup_{i=0}^{k}\sem{\com}^i X\right)
        \subseteq Y \cup \sem{\com}\left(\bigcup_{i=0}^{k}\sem{\com}^i
          Y\right) = \bigcup_{i=0}^{k+1}\sem{\com}^i Y \]
    \end{itemize}
  \end{itemize}
\end{proof}

Since concrete semantics is additive, the Kleene star (\(\com^*\)) and
the fixpoint (\(\fix{\com}\)) have the same concrete semantics
\(\sem{\com^*} = \sem{\fix{\com}}\). In order to notice this, let
\(X \in \dom, f = \lambda Y \in\dom . \left(X \cup
  \sem{\com}Y\right)\) and recall that \(f^0 X = X\) and 
\(f^{n+1} X = X \cup \sem{\com}\left(f^n X\right)\).
\begin{align*}
  \sem{\fix{\com}}X = \lfp(f) & = \bigcup \{ f^n \; \bot \mid n \in \n \} & \text{by Knaster-Tarsky theorem} \\
                              & = \bigcup_{i\in\n} \left( X \cup \sem{\com}^iX \right) & \text{by latter observation} \\
                              & = X \cup \left( X \cup \sem{\com}X \right) \cup \left( X \cup \sem{\com}^2X \right) \cup \dots \\
                              & = \bigcup_{i\in\n} \sem{\com}^i X \\
                              & = \sem{\com^*}X
\end{align*}

This will not be the case for the abstract semantics (cf. example
\ref{ex:fix}), where the Kleene star can be more precise than the
fixpoint semantics, but harder to compute and, as such, less suited
for analysis. For the concrete semantics, however, since they are the
same in the next proofs we only explore the case \(\com^*\) since it
captures also \(\fix{\com}\). Since for a given program \(\com\) and a
set of initial states \(X \in \dom\) the collecting semantics
\(\sem{\com}X\) expresses properties that hold at the end of the
execution of \(\com\) we will in the following chapters usually refer
to \(\sem{\com}X\) as program \emph{invariant}.

\begin{notation}[Singleton shorthand]
  Sometimes we need to consider the semantics over the singleton set
  \(\{\rho\}\). In these cases we will write \(\sem{\com}\rho\)
  meaning \(\sem{\com}\{\rho\}\).
\end{notation}

%% \begin{definition}[Program length]
%%   Given a program \(C\in\imp\) its length is recursively defined as
%%   follows:
%%   \begin{align*}
%%     len(e)         & \defin 1 \\
%%     len(C_1 + C_2) & \defin len(C_1) + len(C_2) + 1 \\
%%     len(C_1;C_2)   & \defin len(C_1) + len(C_2) + 1 \\
%%     len(C^*)       & \defin len(C) + 1
%%   \end{align*}
%% \end{definition}

\subsection{Syntactic sugar}\label{sub:sugar}
We define some syntactic sugar for the language. In the next chapters
we will often use the syntactic sugar instead of its real equivalent
for the sake of simplicity.
\begin{align*}
  \var \in \interval{a}{b} & = \var \in S & \text{with } S = \interval{a}{b}, \text{ decidable}\\
  \var \leq k & = \var \in (-\infty, k]\\
  \var > k & = \var \in [k+1, + \infty) \\
  \var \in S_1 \vee \var \in S_2 & = (\var \in S_1) + (\var \in S_2) \\ 
  \var \in S_1 \wedge \var \in S_2 & = (\var \in S_1);(\var \in S_2) \\
  \var \not\in S & = \var \in \neg S \\
  \ifte{b}{C_1}{C_2} & = (\com[e];\com_1) + (\com[\neg e];\com_2) \\
  \while{b}{C} & = \fix{\com[e];\com};\neg\com[e] \\
  \pplus{\var} & = \var := \var + 1
\end{align*}

\subsection{Small step semantics}\label{sub:sos}

Now that we have defined the collecting semantics to express program
properties, we need the small step semantics to talk about program
execution. We start by defining \emph{program states}:
\(\state \defin \imp \times \env\) tuples of programs and program
environments.  With states we can define our small step semantics:

\begin{definition}[Small step semantics]\label{def:sosem}
  The small step transition relation for the language \(\imp\)
  \(\to : \state \times (\state \cup \env)\) is defined by the
  following rules:
  \begin{equation*}
    \infer[\text{expr}]
          {\stt{\com[e], \rho} \to \bsem{e}\rho}
          {\bsem{e}\rho \neq \bot}
  \end{equation*}
  
  \begin{equation*}
    \infer[\text{sum}_1]
    {\stt{\com[C_1 + C_2], \rho} \to \stt{\com[C_1], \rho}}
    {} \quad
    \infer[\text{sum}_2]
          {\stt{\com[C_1 + C_2], \rho} \to \stt{\com[C_2], \rho}}
          {}
  \end{equation*}
  
  \begin{equation*}
    \infer[\text{comp}_1]
          {\stt{\com[C_1; C_2], \rho} \to \stt{\com[C_1'; C_2], \rho'}}
          {\stt{\com[C_1], \rho} \to \stt{\com[C_1'], \rho'}} \quad
    \infer[\text{comp}_2]
          {\stt{\com[C_1; C_2], \rho} \to \stt{\com[C_2], \rho'}}
          {\stt{\com[C_1], \rho} \to \rho' }
  \end{equation*}

  \begin{equation*}
    \infer[\text{star}]
          {\stt{\com[C^*], \rho} \to \stt{\com[C;C^*], \rho}}
          {} \quad
    \infer[\text{star}_{\text{fix}}]
          {\stt{\com[C^*], \rho} \to \rho}
          {}
  \end{equation*}
\end{definition}
\noindent
In the following chapters we will usually use the following notation
to talk about program execution:
\begin{itemize}
\item \(\to^+\) is the transitive closure of the \(\to\) relation;
\item \(\to^*\) is the reflexive and transitive closure of the \(\to\)
  relation.
\end{itemize}
\noindent
With the following lemma we introduce a link between the small step
semantics and the concrete collecting semantics: the invariant of a
program is the collection of all the environments the program halts on
when executing.

\begin{lemma}\label{le:link}
  For any \(\com\in\imp, X \in \poset{\env}\)
  \[\sem{\com}X = \{\rho' \in \env \mid \rho \in X, \stt{\com, \rho}
    \to^* \rho'\}\]
\end{lemma}

where \(\to^*\) is the reflexive and transitive closure of the \(\to\)
relation.

\begin{proof}
  by induction on \(\com\):
  \paragraph*{Base case:\\}
  \(\com \equiv \com[e]\) \newline
  \(\sem{e}X = \{\bsem{e}\rho \mid \rho \in X \wedge \bsem{e}\rho \neq
  \bot \}\),
  \(\forall \rho \in X . \stt{\com[e], \rho} \to \bsem{e}\rho\) if
  \(\bsem{e}\rho \neq \bot\), and because of the expr rule
  \[\sem{e}X = \{\bsem{e}\rho \mid \rho \in X \wedge \bsem{e}\rho \neq
    \bot \} = \{\rho' \in\env \mid \rho \in X \stt{\com[e], \rho} \to
    \rho'\}\]
  \paragraph*{Inductive cases:\\}
  \begin{itemize}
  \item \(\com \equiv \com_1 + \com_2\) \newline
    \(\sem{\com_1 + \com_2}X = \sem{\com_1}X \cup \sem{\com_2}X\),
    \(\forall \rho \in X . \stt{\com_1 + \com_2, \rho} \to
    \stt{\com_1, \rho} \vee \stt{\com_1 + \com_2, \rho} \to
    \stt{\com_2, \rho}\) respectively according to rules
    sum\textsubscript{1} and sum\textsubscript{2}. By inductive
    hypothesis
    \[\sem{\com_1}X = \{\rho' \in\env \mid \rho \in X ,
      \stt{\com_1,\rho} \to^* \rho'\} \quad \sem{\com_2}X = \{\rho'
      \in\env \mid \rho \in X , \stt{\com_2,\rho} \to^* \rho'\}\]
    Therefore
    \begin{align*}
      \sem{\com_1 + \com_2}X & = \sem{\com_1}X \cup \sem{\com_2}X & \text{(by definition)}\\
      & = \{\rho' \in\env \mid \rho \in X . \stt{\com_1,\rho} \to^* \rho'\} \cup \{\rho' \in\env \mid \rho \in X , \stt{\com_2,\rho} \to^* \rho'\} & \text{(by ind. hp)}\\
      & = \{\rho' \in\env \mid \rho \in X . \stt{\com_1,\rho} \to^* \rho' \vee \stt{\com_2,\rho} \to^* \rho'\} \\
      & = \{\rho' \in\env \mid \rho \in X . \stt{\com_1 + \com_2, \rho} \to^* \rho'\}
    \end{align*}
  \item \(\com \equiv \com_1;\com_2\) \newline
    \(\sem{\com_1;\com_2}X = \sem{\com_2}(\sem{\com_1}X)\). By inductive hp
    \(\sem{\com_1}X = \{\rho' \in\env \mid \rho \in X, \stt{\com_1,
      \rho} \to^* \rho'\} = Y\), by inductive hp again
    \(\sem{\com_2}Y = \{\rho' \in\env \mid \rho \in Y, \stt{\com_2,
      \rho} \to^* \rho'\}\). Therefore
    \begin{align*}
      \sem{\com_1;\com_2}X & = \sem{\com_2}(\sem{\com_1}X) & \text{(by definition)}\\
      & = \{\rho' \in\env \mid \rho'' \in \{\rho''' \mid \rho \in X, \stt{\com_1, \rho}  \to^* \rho'''\}, \stt{\com_2, \rho''} \to^* \rho'\} & \text{(by ind. hp)}\\
      & = \{\rho' \in\env \mid \rho \in X . \stt{\com_1, \rho} \to^* \rho'' \wedge \stt{\com_2, \rho''} \to^* \rho'\} & \text{(by composition lemma)}\\
      & = \{\rho' \in\env \mid \rho \in X . \stt{\com_1;\com_2, \rho} \to^* \rho'\}
    \end{align*}
  \item \(\com \equiv \com^*\) \newline
    \(\sem{\com^*}X = \cup_{i\in\n}\sem{\com}^i X\)
    \begin{align*}
      \sem{\com^*}X & = X \cup \sem{\com}X \cup \sem{\com}^2X \cup \dots & \text{(by definition)}\\
      & = X \cup \{\rho' \in\env \mid \rho \in X . \stt{\com,\rho} \to^* \rho'\} \cup %% \{\rho' \in\poset{\env} \mid \rho \in X, \stt{\com;\com,\rho} \to^* \rho'\} \cup
      \dots & \text{(by ind. hp)}\\
      & = \cup_{i\in\n} \{\rho' \in\env \mid \rho \in X . \stt{\com^i, \rho} \to^*  \rho'\} \\
      & = \{\rho' \in\env \mid \rho \in X . \vee_{i\in\n} \stt{\com^i, \rho} \to^* \rho'\} \\
      & = \{\rho' \in\env \mid \rho \in X . \stt{\com^*, \rho} \to^* \rho'\}
    \end{align*} \qedhere
  \end{itemize}
\end{proof}

Notice that
\(\sem{C}X = \emptyset \iff \nexists \rho'\in\env, \rho \in X \mid
\stt{\com[C],\rho} \to^* \rho'\), in other words the collecting
semantics of some program \(\com\) starting from some states
\(X\in\dom\) is empty iff the program never halts on some state
\(\rho'\). Another observation is that due to non-determinism a
program can halt on multiple final states, or have one branch of
execution that halts on some final state, while the other never halts
on any final state. Non-determinism implies that there are two
different types of termination, intuitively a program can
\emph{always} halt or \emph{partially} halt. We will better explore
this concept in the next chapter.
