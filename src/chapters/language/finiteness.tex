\section{Deciding invariant finiteness}\label{sec:finiteness}

\begin{lemma}\label{le:finiteness}
  If \(\com[D]\in\imp\starless\), and \(X \in \poset{env}\) is
  finite, then
  \begin{enumerate}[label=(\roman*).]
  \item \(\sem{D}X\) is finite;
  \item \(\forall \rho \in X\) \(\trans{\com[D]}{\rho}\ahalts\)
  \item \(\sizeof{\trans{\com[D]}{\rho}} < \infty\) for all
    \(\rho \in X\).
  \end{enumerate}
\end{lemma}

\begin{proof}
  By induction on the program \(\com[D]\):
  \paragraph*{Base case:\\}
  \(\com[D] \equiv e\), therefore
  \begin{enumerate}[label=(\roman*).]
  \item \label{it:1}
    \(\sem{e}X = \{\bsem{e}\rho \mid \rho \in X , \bsem{e}\rho \neq
    \bot\}\), which is finite, since \(X\) is finite;
  \item \label{it:2} by expr rule \(\forall \rho \in X\) either
    \(\stt{e,\rho} \to \bsem{e}\rho\) or \(\stt{e,\rho} \not\to\). In
    both cases there are no infinite paths, and therefore
    \(\trans{\com[e]}{\rho} \ahalts\);
  \item Notice that
    \(\trans{\com[e]}{\rho} = \{\tau \in \Path^\infty \mid \tau_0 =
    \stt{\com[e], \rho}\}\) for all \(\rho \in X\), therefore
    \(\sizeof{\trans{\com[e]}{\rho}} = \sizeof{X} < \infty\) because
    of \ref{it:1}
  \end{enumerate}
  
  \paragraph*{Inductive cases:\\}
  \begin{enumerate}
  \item \(\com[D]\equiv \com[D_1 + D_2]\), therefore
    \begin{enumerate}[label=(\roman*).]
    \item \(\sem{D_1 + D_2} X = \sem{D_1}X \cup \sem{D_2}X\). By
      inductive hypothesis, both \(\sem{D_1}X, \sem{D_2}X\) are
      finite, as they are sub expressions of \(\com[D]\). Since the
      union of finite sets is finite, \(\sem{D_1 + D_2}X\) is finite;
    \item by inductive hypothesis again \(\forall \rho \in X\)
      \(\trans{\com[D]_1}{\rho}\ahalts\) and
      \(\trans{\com[D]_2}{\rho}\ahalts\). By sum\(_1\) rule
      \(\stt{\com_1 + \com_2, rho} \to \stt{\com_1, \rho}\) and by
      sum\(_2\) \(\stt{\com_1 + \com_2, rho} \to \stt{\com_2,
        \rho}\). Therefore \(\trans{\com_1 + \com_2}{\rho}\ahalts\).
    \item For the latter argument, since both
      \(\trans{\com[D_1]}{\rho}\) and \(\trans{\com[D_2]}{\rho}\) are
      finite and composed of finite paths
      \(\sizeof{\trans{(\com[D]_1 + \com[D]_2)}{\rho}} < \infty\).
    \end{enumerate}
  \item \(\com[D]\equiv \com[D_1; D_2]\), therefore
    \begin{enumerate}[label=(\roman*).]
    \item \(\sem{D_1;D_2}X = \sem{D_2}(\sem{D_1}X)\). By inductive
      hypothesis \(\sem{D_1}X = Y\). By inductive hypothesis again
      \(\sem{D_2}Y\) is finite;
    \item by inductive hypothesis both \(\forall \rho \in X\)
      \(\trans{\com[D]_1}{\rho}\ahalts\) and \(\forall \rho' \in Y\)
      \(\trans{\com[D]_2}{\rho'}\ahalts\), therefore by composition
      lemma \(\trans{\com[D]_1;\com[D]_2}{\rho} \ahalts\)
    \item by inductive hypothesis both
      \(\sizeof{\trans{\com_1}{\rho}} < \infty\) and
      \(\sizeof{\trans{\com_2}{\rho'}} < \infty\)
      \(\forall \rho \in X, \rho'\in \sem{\com_1}X\), todo
    \end{enumerate}
  \end{enumerate}
\end{proof}

\begin{lemma}\label{le:infiniteness}
  Given \(\com[D]\in\imp\starless\), and \(X \in \poset{env}\), the
  predicate "\(\sem{D^*}X\) is finite" is undecidable.
\end{lemma}

\begin{proof}
  Suppose we can decide \(\sem{D^*}X\) is finite. We show that we in
  that case we would be also able to decide whether
  \(\trans{\com[D]^*}{\rho}\ahalts\) for some \(\rho \in X\), which is
  undecidable.%% Let \(\rho =\envi{x_1 \mapsto a_1, \dots, x_k \mapsto
    %% a_k}\) \(X = \{\rho\}\) with \(a_1, \dots, a_k \in \z\).
  \begin{itemize}
  \item In case \(\sem{D^*}X\) is infinite, then it must be that
    \(\forall k \in \n\)
    \[\sem{D}^{k+1}X \nsubseteq \bigcup_{i=0}^k\sem{D}^iX\] otherwise if
    for some \(k\) \(\sem{D}^{k+1}X \subseteq \cup_{i=0}^k\sem{D}^iX\)
    it would mean that  todo
    
    we would reach a fixpoint and \(\sem{D^*}X\) would be
    finite. Since each application of \(\com[D]\) must create an
    nonempty set of new environments, we can build the inductive
    sequence
    \begin{align*}
      Y_0 & = X \\
      Y_{k+1} & = (\sem{D}Y_k)\setminus Y_k
    \end{align*}
    where
    \(\forall \rho' \in Y_{k+1} \exists \rho \in Y_k \mid \rho' \in
    \sem{D}\rho\) by definition. This means that there must be at
    least one \(\rho_1\in X\) that produces an infinite path
    \[\stt{\com[D]^*, \rho_1}\to^*\stt{\com[D]^*, \rho_2}\to^*\dots \]
    which produces new environments at each application of
    \(\com[D]\): \(\rho_1, \rho_2, \dots \mid \forall i,j \in \n\)
    \(\rho_i \neq \rho_j\) and therefore
    \(\trans{\com[D]^*}{\rho_1} \nehalts\) which means that
    \(\trans{\com[D]^*}{\rho_1}\ahalts\) is false.
  \item In case \(\sem{D^*}X\) is finite we can search for infinite
    paths in \(\trans{\com[D]^*}{\rho}\). Since
    \(\sem{\com[D]^*}X = \bigcup_{\rho\in X}\sem{\com[D]^*}\{\rho\}\),
    for every \(\rho' \in X\), \(\sem{\com[D]}\{\rho\}\) is
    finite. Consider the states in \(\sem{\com[D]}\{\rho\}\). For each
    one of them either
    \begin{itemize}
    \item
      \(\stt{\com[D], \rho'} \to^* \stt{\com[D'], \rho''} \not\to\),
      for every \(\to\) step we can apply the comp\(_1\) rule and
      therefore we would build
      \[\stt{\com[D];\com[D]^*, \rho'} \to^* \stt{\com[D']; \com[D]^*,
          \rho''} \not\to\] which is a finite path, therefore is not
      interesting.
    \item or \(\stt{\com[D], \rho'} \to^* \rho''\) then by composition
      lemma
      \(\stt{\com[D]; \com[D]^*, \rho'} \to^* \stt{\com[D]^*,
        \rho''}\) and by star\(_{\text{fix}}\)
      \(\stt{\com[D]^*, \rho''} \to \rho''\) and so
      \(\rho'' \in \sem{\com[D]^*}\{\rho\}\).
    \end{itemize}
  \end{itemize}
\end{proof}
