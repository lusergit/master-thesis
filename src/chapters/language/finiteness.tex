\section{Deciding invariant finiteness}

\begin{lemma}\label{le:finiteness}
  Given \(C\in\imp\) where the \(^*\) operator does not appear, and a
  finite \(X \in \poset{env}\), the predicate "\(\sem{C}X\) is finite"
  is decidable.
\end{lemma}

\begin{proof}
  By induction on the program \(C\):
  \paragraph*{Base case:\\}
  \(C \equiv e\), therefore \(\sem{e}X = \{\bsem{e}\rho \mid \rho \in
  X , \bsem{e}\rho \neq \bot\}\), which is finite, since \(X\) is
  finite.
  
  \paragraph*{Inductive cases:\\}
  \begin{enumerate}
  \item \(C\equiv C_1 + C_2\), therefore \(\sem{C_1 + C_2} X =
    \sem{C_1}X \cup \sem{C_2}X\). By inductive hypothesis, both
    \(\sem{C_1}X, \sem{C_2}X\) are finite, as they're sub expressions
    of \(C\). Since the union of finite sets is finite, \(\sem{C_1 +
      C_2}X\) is finite.
  \item \(C\equiv C_1; C_2\), therefore \(\sem{C_1;C_2}X =
    \sem{C_2}(\sem{C_1}X)\). By inductive hypothesis \(\sem{C_1}X =
    Y\) is finite. Again by inductive hypothesis \(\sem{C_2}Y\) is
    finite.
  \end{enumerate}
\end{proof}

\begin{lemma}\label{le:infiniteness}
  Given \(C\in\imp\) where the \(^*\) operator does not appear, and a
  finite \(X \in \poset{env}\), the predicate "\(\sem{C^*}X\) is
  finite" is undecidable.
\end{lemma}

\begin{proof}
  %% \(\sem{C^*}X = \cup_{i\in\n}\sem{C}^iX\). Because of lemma
  %% \ref{le:finiteness}, all \(\sem{C}^iX\) are finite, but their union
  %% might not.
\end{proof}
