\section{Deciding invariant finiteness}

\begin{lemma}\label{le:finiteness}
  If \(\com\in\imp_\starless\), and a finite \(X \in \poset{env}\)
  then \[\sem{C}X \text{ is finite}\]
\end{lemma}

\begin{proof}
  By induction on the program \(C\):
  \paragraph*{Base case:\\}
  \(\com \equiv e\), therefore \(\sem{e}X = \{\bsem{e}\rho \mid \rho
  \in X , \bsem{e}\rho \neq \bot\}\), which is finite, since \(X\) is
  finite.
  
  \paragraph*{Inductive cases:\\}
  \begin{enumerate}
  \item \(\com\equiv \com[C_1 + C_2]\), therefore \(\sem{C_1 + C_2} X =
    \sem{C_1}X \cup \sem{C_2}X\). By inductive hypothesis, both
    \(\sem{C_1}X, \sem{C_2}X\) are finite, as they're sub expressions
    of \(C\). Since the union of finite sets is finite, \(\sem{C_1 +
      C_2}X\) is finite.
  \item \(C\equiv C_1; C_2\), therefore \(\sem{C_1;C_2}X =
    \sem{C_2}(\sem{C_1}X)\). By inductive hypothesis \(\sem{C_1}X =
    Y\) is finite. Again by inductive hypothesis \(\sem{C_2}Y\) is
    finite.
  \end{enumerate}
\end{proof}

\begin{lemma}\label{le:infiniteness}
  Given \(\com\in\imp_\starless\), and a finite \(X \in \poset{env}\),
  the predicate "\(\sem{C^*}X\) is finite" is undecidable.
\end{lemma}

\begin{proof}
  Suppose we can decide \(\sem{C^*}X\) is finite. We show that we in
  that case we would be also able to decide wether \(\com^*(a_1,
  \dots, a_k)\ahalts\), which is undecidable. Let \(X = \{\envi{x_1
    \mapsto a_1, \dots, x_k \mapsto a_k}\}\) with \(a_1, \dots, a_k
  \in \z\).
  \begin{itemize}
  \item In case \(\sem{\com^*}X\) is infinite, then it must be that
    \(\forall i \in \n\) \(\sem{C}^{k+1}X \nsubseteq
    \cup_{i=0}^k\sem{\com}^iX\), otherwise we would reach a fixpoint
    and \(\sem{\com^*}X\) would be finite. Since each application of
    \(\com\) must create an unempty set of new environments we can
    build the inductive sequence of sets of environments
    \begin{align*}
      Y_0 & = X \\
      Y_{k+1} & = (\sem{C}Y_k)\setminus Y_k
    \end{align*}
    and \(\forall \rho' \in Y_{k+1} \exists \rho \in Y_k \mid \rho'
    \in \sem{C}\{\rho\}\) by definition. This means that there is an
    infinite path in the transition system, and therefore
    \(\com[C^*](a_1, \dots, a_k) \nehalts\) which means that
    \(\com[C^*] (a_1, \dots, a_k)\ahalts\) is false.
  \item In case \(\sem{C^*}X is finite\) then we can notice that
  \end{itemize}
\end{proof}
