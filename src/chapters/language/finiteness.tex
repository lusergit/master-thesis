\section{Deciding invariant finiteness}\label{sec:finiteness}

In this section we argue that even the finiteness of the semantics of
some program on some initial states is undecidable. We show that if we
were able to establish whether \(\sem{\com}X\) is finite for some
program \(\com\in\imp\) and some initial states \(X\in \dom\), we
could decide whether \(\stt{\com, \rho}\ahalts\) for all
\(\rho \in X\), which instead is known to be undecidable. The first
step is showing that if we have a program where the \(^*\) operator
does not appear, then the program can only produce a finite amount of
finite derivation sequences.

\begin{lemma}\label{le:finiteness}
  If \(\com[D]\in\imp\starless\), and \(X \in \poset{env}\) is
  finite, then
  \begin{enumerate}[label=(\roman*).,ref=\thetheorem.(\roman*)]
  \item\label{eq:fini} \(\sem{D}X\) is finite;
  \item\label{eq:finii} \(\forall \rho \in X\) \(\stt{\com[D], \rho}\ahalts\)
  \item\label{eq:finiii} \(\sizeof{\trans{\com[D]}{\rho}} < \infty\) for all
    \(\rho \in X\).
  \end{enumerate}

  where by \(\trans{\com[D]}{\rho}\) we mean the set of all derivation
  sequences starting from \(\stt{\com[D], \rho}\) in the transition
  system.
\end{lemma}

\begin{proof}
  By induction on the program \(\com[D]\):
  \paragraph*{Base case:\\}
  \(\com[D] \equiv e\), therefore
  \begin{enumerate}[label=(\roman*).]
  \item \label{it:1}
    \(\sem{e}X = \{\bsem{e}\rho \mid \rho \in X , \bsem{e}\rho \neq
    \bot\}\), which is finite, since \(X\) is finite;
  \item \label{it:2} by expr rule \(\forall \rho \in X\) either
    \(\stt{e,\rho} \to \bsem{e}\rho\) or \(\stt{e,\rho} \not\to\). In
    both cases there are no infinite derivation sequences, and therefore
    \(\trans{\com[e]}{\rho} \ahalts\);
  \item Notice that \(\forall \rho \in X\) either by the expr rule
    \(\stt{\com[e], \rho} \to \seme{\com[e]}\rho\) or
    \(\stt{\com[e], \rho}\not\to\) therefore
    \[\sizeof{\trans{\com[e]}{\rho}} \leq \sizeof{X} < \infty\]
  \end{enumerate}
  
  \paragraph*{Inductive cases:\\}
  \begin{enumerate}
  \item \(\com[D]\equiv \com[D_1 + D_2]\), therefore
    \begin{enumerate}[label=(\roman*).]
    \item \(\sem{D_1 + D_2} X = \sem{D_1}X \cup \sem{D_2}X\). By
      inductive hypothesis, both \(\sem{D_1}X, \sem{D_2}X\) are
      finite, as they are sub expressions of \(\com[D]\). Since the
      union of finite sets is finite, \(\sem{D_1 + D_2}X\) is finite;
    \item by inductive hypothesis again \(\forall \rho \in X\)
      \(\stt{\com[D]_1, \rho}\ahalts\) and
      \(\stt{\com[D]_2, \rho}\ahalts\). By sum\(_1\) rule
      \(\stt{\com[D]_1 + \com[D]_2, \rho} \to \stt{\com[D]_1, \rho}\)
      and by sum\(_2\)
      \(\stt{\com[D]_1 + \com[D]_2, \rho} \to \stt{\com[D]_2,
        \rho}\). Therefore
      \(\stt{\com[D]_1 + \com[D]_2}{\rho}\ahalts\).
    \item For the latter argument, since both
      \(\stt{\com[D]_1}{\rho}\) and \(\stt{\com[D_2]}{\rho}\) are
      finite and composed of finite derivation sequences
      \(\sizeof{\trans{\com[D]_1 + \com[D]_2}{\rho}} < \infty\).
    \end{enumerate}
  \item \(\com[D]\equiv \com[D_1; D_2]\), therefore
    \begin{enumerate}[label=(\roman*).]
    \item \(\sem{D_1;D_2}X = \sem{D_2}(\sem{D_1}X)\). By inductive
      hypothesis \(\sem{D_1}X = Y\). By inductive hypothesis again
      \(\sem{D_2}Y\) is finite;
    \item by inductive hypothesis both \(\forall \rho \in X\)
      \(\stt{\com[D]_1, \rho}\ahalts\) and \(\forall \rho' \in Y\)
      \(\stt{\com[D]_2,\rho'}\ahalts\), therefore by composition
      lemma \(\stt{\com[D]_1;\com[D]_2, \rho} \ahalts\)
    \item by inductive hypothesis both
      \(\sizeof{\trans{\com[D]_1}{\rho}} < \infty\) and
      \(\sizeof{\trans{\com[D]_2}{\rho'}} < \infty\)
      \(\forall \rho \in X, \rho'\in \sem{\com[D]_1}X\). For all
      derivation sequences starting from \(\stt{\com[D]_1, \rho}\)
      where \[\stt{\com[D]_1, \rho} \to^* \rho'\] with
      \(\rho'\in \sem{\com[D]_1}X\) we can apply the composition lemma
      and state that
      \[\stt{\com[D]_1;\com[D]_2, \rho} \to^* \stt{\com[D]_2, \rho'} \quad
        \forall \rho \in X\] from there we can notice that since
      \(\sizeof{\stt{\com[D]_2, \rho'}} < \infty\) then
      \(\sizeof{\stt{\com[D]_1;\com[D]_2, \rho'}} < \infty\)
    \end{enumerate}
  \end{enumerate}
\end{proof}


In order to prove that finiteness is undecidable we need the following
Lemma:

\begin{lemma}\label{le:contain}
  Let \(\com[D] \in \imp\starless\) and \(\rho \in \env\). If
  \begin{equation}\label{eq:hp}
    \sem{D}^{k+1}\rho \subseteq \bigcup_{i=0}^k\sem{D}^i\rho \quad \text{for some } k \in \n
  \end{equation}
  then
  \begin{equation}\label{eq:what}
    \forall j \in \n \quad \sem{D}^{k+1+j}\rho \subseteq \bigcup_{i=0}^k\sem{D}^i\rho
  \end{equation}
  and therefore
  \(\sem{\com[D]^*}\rho \subseteq \cup_{i=0}^k\sem{D}^i\rho\)
\end{lemma}
\begin{proof}
  We can show \eqref{eq:what} by induction on \(j\):
  \begin{itemize}
  \item if \(j=0\) then we want to show that
    \(\sem{D}^{k+1}\rho \subseteq \cup_{i=0}^k\sem{D}^i\rho\), which is
    true by hypothesis \eqref{eq:hp};
    
  \item In the inductive case we have to show that if the
    statement holds for \(j\), it also holds for \(j+1\). We know
    that
    \begin{align*}
      \bigcup_{i=0}^k\sem{\com[D]}^i\rho & = \bigcup_{i=0}^{k+1}\sem{\com[D]}^i\rho & \text{since by \eqref{eq:hp} } \sem{\com[D]}^{k+1}\rho \subseteq \cup_{i=0}^k\sem{\com[D]}^i\rho \\
                                         & = \rho \cup \bigcup _{i=1}^{k+1}\sem{\com[D]}^i \rho \\
                                         & = \rho \cup \sem{\com[D]}\left( \bigcup _{i=0}^k\sem{\com[D]}^i \rho\right) & \text{by additivity of } \sem{\com[D]}
    \end{align*}
    By inductive hypothesis
    \begin{equation*}
      \sem{\com[D]}^{k+1+j}\rho \subseteq \bigcup_{i=0}^k\sem{\com[D]}^i\rho
    \end{equation*}
    so, by monotonicity of \(\sem{\com[D]}\)
    \begin{equation*}
      \sem{\com[D]}\left( \sem{\com[D]}^{k+1+j}\rho\right) \subseteq \sem{\com[D]}\left(\bigcup_{i=0}^k\sem{\com[D]}^i\rho \right)
    \end{equation*}
    and therefore
    \begin{equation*}
      \sem{\com[D]}^{(k+1)+(j+1)}\rho \subseteq \left(\bigcup_{i=1}^{k+1}\sem{\com[D]}^i\rho \right) \subseteq \rho \cup \left(\bigcup_{i=1}^{k+1}\sem{\com[D]}^i\rho \right) = \bigcup_{i=0}^{k+1}\sem{\com[D]}^i\rho = \bigcup_{i=0}^{k}\sem{\com[D]}^i\rho
    \end{equation*}
  \end{itemize}
\end{proof}

We also need to recall Konig's Lemma from \cite{konig1926lemma}:

\begin{lemma}[König's Lemma]\label{le:konig}
  Let \(T\) be a rooted tree with an infinite number of nodes,
  each with a finite number of children.  Then \(T\) has a branch
  of infinite length.
\end{lemma}

With Lemma~\ref{le:contain} and Lemma~\ref{le:konig} we can prove the
following.

\begin{lemma}\label{le:infiniteness}
  Given \(\com[D]\in\imp\starless\), and \(\rho \in \env\), the
  predicate "\(\sem{D^*}\rho\) is finite" is undecidable.
\end{lemma}

\begin{proof}
  We work by contradiction, showing that if we know whether
  \(\sem{\com}\rho\) is finite or infinite we can decide
  \(\stt{\com, \rho}\ahalts\).
  \begin{itemize}
  \item Suppose that \(\sem{D^*}\rho\) is infinite, then we can
    observe that because Lemma~\ref{le:contain}

    \begin{equation}\label{eq:noncontain}
      \forall k \in \n \quad \sem{D}^{k+1}\rho \nsubseteq \bigcup_{i=0}^k\sem{D}^i\rho
    \end{equation}

    Otherwise
    \(\sem{\com[D]^*}\rho \subseteq
    \bigcup_{i\in\n}\sem{\com[D]}^i\rho\) and we would contraddict the
    hypothesis of \(\sem{\com[D]^*}\rho\) being infinite.
    % quindi lo statement deve essere falso
    Therefore \(\forall k \in \n\)
    \(\sem{\com[D]}^{k+1}\rho \not\subseteq \bigcup_{i=0}^k
    \sem{\com[D]}^i\rho\), otherwise
    \(\sem{\com[D]^*}\rho \subseteq \bigcup_{i=0}^k\sem{\com[D]}\rho\) which
    is impossible since the right term is a finite quantity.
    % No , definiamo l'albero dove ogni livello è composto dai
    % seguenti stati
    With this observation we build the tree \(\tuple{\env, \to^D}\),
    where \(\to^D \subseteq \env \times \env\) and
    \(\rho' \to^D \rho''\) if \(\stt{\com[D], \rho'} \to^*
    \rho''\). We define by the following rule the levels of the tree:
    \begin{align*}
      Y_0 & = \{\rho\} \\
      Y_{k+1} & = \left(\sem{\com[D]}^{k+1}\rho\right) \setminus \left( \bigcup_{i=0}^k \sem{\com[D]}\rho \right)
    \end{align*}
    Where \(Y_0\) is the singleton set containing the root \(\rho\)
    and the \(k\)-th level is made of the environments in the \(Y_k\)
    set. Figure \ref{fig:tree} shows a tree of \(\to^D\) relations and
    visualizes the levels \(Y_k\). We can therefore make the following
    observations:
    \begin{enumerate}[label=(\roman*)]
    \item The tree is rooted in \(\rho \in Y_0\). In fact
      \(\forall \rho' \in Y_1\) \(\rho \to^D \rho'\) by definition and
      \(\forall \rho'''\in Y_{k+1} \exists \rho'' \in Y_k \mid \rho''
      \to^D \rho'''\);
    \item since \(\forall k\in\n\)
      \(\sem{\com[D]}^{k+1}\rho \not\subseteq
      \cup_{i=0}^k\sem{\com[D]}^i\rho\) by \eqref{eq:noncontain}, each
      level \(Y_k\) is non empty. Each level is also finite because of
      Lemma~\ref{eq:fini}. Therefore there is an infinite
      quantity of levels, where each node has a finite quantity of
      children.
    \end{enumerate}

    what is left to do is show that there is a derivation sequence
    from \(\stt{\com[D]^*, \rho}\) of infinite length.
    \begin{figure}
      \centering
      \begin{tikzpicture}[->,>=stealth]
        \tikzset{node distance = .7cm}

        \node (rho)                     {\(\rho\)};
        \node (dots1)  [below=of rho]   {\(\dots\)};
        \node (rho11)  [left=of dots1]  {\(\rho'\)};
        \node (rho12)  [right=of dots1] {\(\rho''\)};
        \node (dots2)  [below=of dots1] {\(\dots\)};
        \node (phan2)  [right=of dots2] {\(\phantom{\rho''}\)};
        \node (rho21)  [left=of dots2]  {\(\rho'''\)};
        \node (vdot1)  [below=of rho21] {\(\vdots\)};
        \node (vdot2)  [below=of dots2] {\(\vdots\)};
        \node (phan3)  [right=of vdot2] {\(\phantom{\rho''}\)};

        \draw[red, thick, rounded corners]
        (rho11.south west) rectangle (rho12.north east);
        \draw[red, thick, rounded corners]
        (rho21.south west) rectangle (phan2.north east);
        \draw[red, thick, rounded corners]
        (vdot1.south west) rectangle (phan3.north east);
        \node[right=of rho12] at (rho12) {\(Y_1\)};
        \node[right=of phan2] at (phan2) {\(Y_2\)};
        \node[right=of phan3] at (phan3) {\(Y_k\)};

        \draw
        (rho) edge (rho11) edge (dots1) edge (rho12)
        (rho11) edge (rho21)
        (dots1) edge (dots2)
        (rho12) edge (dots2)
        (rho21) edge (vdot1)
        (dots2) edge (vdot2);
      \end{tikzpicture}
      \caption{Example of \(\to^D\) relations between elements of
        \(\env\).}\label{fig:tree}
    \end{figure}

    We can do this by using König's Lemma~\ref{le:konig} and deduce
    that there exists an infinite derivation sequence from \(\rho\) of
    \(\to^D\) relations
    \begin{equation*}
      \rho \to^D \rho' \to^D \rho'' \to^D \dots
    \end{equation*}
    Where each element belongs to a different level \(Y_k\), and
    therefore is different from every other environment appearing in
    the sequence. Observe that for all \(\rho', \rho''\in\env\)
    s.t. \(\rho' \to^D \rho''\) since
    \(\stt{\com[D], \rho'} \to^* \rho''\) we can apply
    Corollary~\ref{co:comp} of Lemma~\ref{le:comp} and derive that
    \(\stt{\com[D];\com[D]^*, \rho'} \to^* \stt{\com[D]^*, \rho''}\)
    and because of the star rule
    \(\stt{\com[D]^*, \rho'} \to \stt{\com[D];\com[D]^*, \rho'}\). We
    can therefore say that
    \begin{equation*}
      \stt{\com[D]^*, \rho'} \to^* \stt{\com[D]^*, \rho''}
    \end{equation*}
    Therefore, there exists an infinite derivation sequence  
    \begin{equation*}
      \stt{\com[D]^*, \rho} \to^*\stt{\com[D]^*, \rho'} \to^* \stt{\com[D]^*, \rho''} \to^* \dots
    \end{equation*}
    which means \(\stt{\com[D]^*, \rho}\nehalts\) and therefore
    \(\stt{\com[D]^*, \rho} \ahalts\) is false.
    
  \item if instead \(\sem{D^*}\rho\) is finite, then we can
    reduce total termination to the presence of some cycle in one of
    the derivation sequences starting from
    \(\stt{\com[D]^*, \rho}\). The statement we want to prove is the
    following:
    \begin{equation*}
      \text{if }\sem{\com[D]^*}\rho \text{ is finite, then } \stt{\com[D]^*, \rho} \ahalts \iff \text{ no derivation sequence starting from } \stt{\com[D]^*, \rho} \text{ has cycles}
    \end{equation*}
    \begin{itemize}
    \item[\((\Rightarrow)\)] In this case we want to prove that if
      \(\sem{\com[D]^*}\) is finite and
      \(\stt{\com[D], \rho} \ahalts\) then there are no cycles in
      any derivation sequence starting from \(\stt{\com[D],
        \rho}\). To do so we work by contradiction. Suppose there is
      some derivation sequence starting from
      \(\stt{\com[D]^*, \rho}\) with some cycle
      \begin{equation*}
        \stt{\com[D]^*, \rho} \to^* \stt{\com[D]^*, \rho'} \to^+ \stt{\com[D]^*, \rho'} \to^* \rho''
      \end{equation*}
      with \(\rho''\neq \rho, \rho'\), then we can notice that also
      the infinite derivation sequence
      \begin{equation*}
        \stt{\com[D]^*, \rho} \to^* \stt{\com[D]^*, \rho'} \to^+ \stt{\com[D]^*, \rho'} \to^+ \stt{\com[D]^*, \rho'} \to^+ \dots
      \end{equation*}
      is part of the transition system for \(\stt{\com[D], \rho}\),
      and therefore \(\stt{\com[D]^*, \rho} \ahalts\) is false which
      is absurd.
    \item[\((\Leftarrow)\)] In this case we want to prove that if
      \(\sem{\com[D]^*}\rho\) is finite and there are no cycles in any
      derivation sequence starting from \(\stt{\com[D], \rho}\) then
      \(\stt{\com[D], \rho} \ahalts\). We work again by
      contradiction. Suppose that we have an infinite derivation
      sequence starting from \(\stt{\com[D]^*, \rho}\). It must be
      that \(\forall i,j \in \n\) \(i \neq j\), \(\rho_i \neq \rho_j\)
      with \(\rho_0 = \rho\), otherwise there would be a cycle, which
      would contraddict the hypothesis. Therefore the derivation
      sequence has the shape
      \begin{equation*}
        \stt{\com[D]^*, \rho} \to^* \stt{\com[D]^*, \rho_1} \to^* \stt{\com[D]^*, \rho_2} \to^* \stt{\com[D]^*, \rho_3} \to^* \dots
      \end{equation*}
      We can notice that for all \(\rho' \in \{\rho, \rho_1, \dots\}\)
      and for the star\(_{\text{fix}}\) rule,
      \(\stt{\com[D]^*, \rho'} \to \rho'\) and therefore
      \(\rho'\in \sem{\com[D]^*}\rho\). This would mean that
      \(\sem{\com[D]^*}\rho\) is infinite, which is absurd.
    \end{itemize}

    To conclude we can observe that there is a finite amount of
    reachable states from \(\stt{\com[D]^*, \rho}\). Where by reachable
    we mean that there exists some derivation sequence ending up in
    that state.

    We can notice that starting from any state
    \(\stt{\com[D]^*, \rho'}\) with \(\rho'\in \sem{\com[D]^*}\rho\)
    we have 2 possibilities:
    \begin{itemize}
    \item we either apply the \starfix rule, resulting in a finite
      derivation sequence
      \begin{equation*}
        \stt{\com[D]^*, \rho'} \to \rho'
      \end{equation*}
      and therefore in a finite number of reached states;
    \item or we apply the star rule
      \begin{equation*}
        \stt{\com[D]^*, \rho'} \to \stt{\com[D];\com[D]^*, \rho'}
      \end{equation*}
      by lemma~\ref{le:finiteness} we know that
      \(\stt{\com[D], \rho'}\ahalts\) and
      \(\sizeof{\trans{\com[D]}{\rho'}} < \infty\), therefore there is
      a finite number of environments \(\rho''\)
      s.t. \(\stt{\com[D],\rho'} \to^* \rho''\). For each one of them
      we can use the composition lemma and observe that
      \begin{equation*}
        \stt{\com[D];\com[D]^*, \rho'} \to^* \stt{\com[D]^*, \rho''}
      \end{equation*}
      Ending up in a state \(\stt{\com[D]^*, \rho''}\) where we can
      apply the same reasoning
    \end{itemize}
    Therefore starting from any state \(\stt{\com[D]^*, \rho'}\) with
    \(\rho'\in \sem{\com[D]^*}\rho\) (in particular \(\rho\)), we
    either terminate our derivation sequence or we end up in some
    state \(\stt{\com[D]^*, \rho'}\) again, with
    \(\rho'\in\sem{\com[D]^*}\rho\). Since there is a finite amount of
    states \(\rho'\in\sem{\com[D]^*}\rho\), the number of reachable
    states from \(\stt{\com[D]^*, \rho}\) is finite.
  \end{itemize}
\end{proof}
