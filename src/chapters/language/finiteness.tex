\section{Deciding invariant finiteness}\label{sec:finiteness}

In this section we argue that even the finiteness of the semantics of
some program on some initial states is undecidable. We show that by
knowing wether \(\sem{\com}X\) is finite for some program
\(\com\in\imp\) and some initial states \(X\in \dom\), we can decide
wether \(\trans{\com}{\rho}\ahalts\) for all \(\rho \in X\). The first
step is showing that fi we have a program where the \(^*\) operator
does not appear, then the program can only produce a finite amount of
finite paths. In other words its transition system is a tree for every
initial state \(\stt{\com[D], \rho}\) for all \(\rho \in X\).

\begin{lemma}\label{le:finiteness}
  If \(\com[D]\in\imp\starless\), and \(X \in \poset{env}\) is
  finite, then
  \begin{enumerate}[label=(\roman*).]
  \item \(\sem{D}X\) is finite;
  \item \(\forall \rho \in X\) \(\trans{\com[D]}{\rho}\ahalts\)
  \item \(\sizeof{\trans{\com[D]}{\rho}} < \infty\) for all
    \(\rho \in X\).
  \end{enumerate}
\end{lemma}

\begin{proof}
  By induction on the program \(\com[D]\):
  \paragraph*{Base case:\\}
  \(\com[D] \equiv e\), therefore
  \begin{enumerate}[label=(\roman*).]
  \item \label{it:1}
    \(\sem{e}X = \{\bsem{e}\rho \mid \rho \in X , \bsem{e}\rho \neq
    \bot\}\), which is finite, since \(X\) is finite;
  \item \label{it:2} by expr rule \(\forall \rho \in X\) either
    \(\stt{e,\rho} \to \bsem{e}\rho\) or \(\stt{e,\rho} \not\to\). In
    both cases there are no infinite paths, and therefore
    \(\trans{\com[e]}{\rho} \ahalts\);
  \item Notice that
    \(\trans{\com[e]}{\rho} = \{\tau \in \Path^\infty \mid \tau_0 =
    \stt{\com[e], \rho}\}\) for all \(\rho \in X\), therefore
    \(\sizeof{\trans{\com[e]}{\rho}} = \sizeof{X} < \infty\) because
    of \ref{it:1}
  \end{enumerate}
  
  \paragraph*{Inductive cases:\\}
  \begin{enumerate}
  \item \(\com[D]\equiv \com[D_1 + D_2]\), therefore
    \begin{enumerate}[label=(\roman*).]
    \item \(\sem{D_1 + D_2} X = \sem{D_1}X \cup \sem{D_2}X\). By
      inductive hypothesis, both \(\sem{D_1}X, \sem{D_2}X\) are
      finite, as they are sub expressions of \(\com[D]\). Since the
      union of finite sets is finite, \(\sem{D_1 + D_2}X\) is finite;
    \item by inductive hypothesis again \(\forall \rho \in X\)
      \(\trans{\com[D]_1}{\rho}\ahalts\) and
      \(\trans{\com[D]_2}{\rho}\ahalts\). By sum\(_1\) rule
      \(\stt{\com_1 + \com_2, rho} \to \stt{\com_1, \rho}\) and by
      sum\(_2\) \(\stt{\com_1 + \com_2, rho} \to \stt{\com_2,
        \rho}\). Therefore \(\trans{\com_1 + \com_2}{\rho}\ahalts\).
    \item For the latter argument, since both
      \(\trans{\com[D_1]}{\rho}\) and \(\trans{\com[D_2]}{\rho}\) are
      finite and composed of finite paths
      \(\sizeof{\trans{(\com[D]_1 + \com[D]_2)}{\rho}} < \infty\).
    \end{enumerate}
  \item \(\com[D]\equiv \com[D_1; D_2]\), therefore
    \begin{enumerate}[label=(\roman*).]
    \item \(\sem{D_1;D_2}X = \sem{D_2}(\sem{D_1}X)\). By inductive
      hypothesis \(\sem{D_1}X = Y\). By inductive hypothesis again
      \(\sem{D_2}Y\) is finite;
    \item by inductive hypothesis both \(\forall \rho \in X\)
      \(\trans{\com[D]_1}{\rho}\ahalts\) and \(\forall \rho' \in Y\)
      \(\trans{\com[D]_2}{\rho'}\ahalts\), therefore by composition
      lemma \(\trans{\com[D]_1;\com[D]_2}{\rho} \ahalts\)
    \item by inductive hypothesis both
      \(\sizeof{\trans{\com_1}{\rho}} < \infty\) and
      \(\sizeof{\trans{\com_2}{\rho'}} < \infty\)
      \(\forall \rho \in X, \rho'\in \sem{\com_1}X\), todo
    \end{enumerate}
  \end{enumerate}
\end{proof}

\begin{lemma}\label{le:infiniteness}
  Given \(\com[D]\in\imp\starless\), and \(\rho \in \env\), the
  predicate "\(\sem{D^*}\rho\) is finite" is undecidable.
\end{lemma}

\begin{proof}
  We work by contraddiction, showing that if we know wether
  \(\sem{\com}\rho\) is finite or infinite we can decide
  \(\trans{\com}{\rho}\ahalts\).
  \begin{itemize}
  \item In case \(\sem{D^*}\rho\) is infinite, then we make the
    following observation:

    \begin{observation}\label{obs:noncontained}
      \begin{equation*}
        \forall k \in \n \quad \sem{D}^{k+1}\rho \nsubseteq \bigcup_{i=0}^k\sem{D}^i\rho
      \end{equation*} 
    \end{observation}
    The latter observation follows from the following lemma:
    \begin{lemma}\label{le:contain}
      Let \(\com[D] \in \imp\starless\) \(\rho \in \env\). If
      \begin{equation}\label{eq:hp}
        \sem{D}^{k+1}\rho \subseteq \cup_{i=0}^k\sem{D}^i\rho \quad \text{for some } k \in \n
      \end{equation}
      then
      \begin{equation}\label{eq:what}
        \forall j \in \n \quad \sem{D}^{k+1+j}\rho \subseteq \cup_{i=0}^k\sem{D}^i\rho
      \end{equation}
      and therefore
      \(\sem{\com[D]^*}\rho \subseteq \cup_{i=0}^k\sem{D}^i\rho\)
    \end{lemma}
    \begin{proof}
      We can show \eqref{eq:what} by induction on \(j\):
      \begin{itemize}
      \item if \(j=0\) then we want to show that
        \(\sem{D}^{k+1}\rho \subseteq \cup_{i=0}^k\sem{D}^i\rho\), which is
        true by hypothesis \eqref{eq:hp};
        
      \item In the inductive case we have to show that if the
        statement holds for \(j\), it also holds for \(j+1\). We know
        that
        \begin{align*}
          \bigcup_{i=0}^k\sem{\com[D]}^i\rho & = \bigcup_{i=0}^{k+1}\sem{\com[D]}^i\rho & \text{since by \eqref{eq:hp} } \sem{\com[D]}^{k+1}\rho \subseteq \cup_{i=0}^k\sem{\com[D]}^i\rho \\
                                          & = \rho \cup \bigcup _{i=1}^{k+1}\sem{\com[D]}^i \rho \\
                                          & = \rho \cup \sem{\com[D]}\left( \bigcup _{i=0}^k\sem{\com[D]}^i \rho\right) & \text{by additivity}
        \end{align*}
        By inductive hypothesis
        \begin{equation*}
          \sem{\com[D]}^{k+1+j}\rho \subseteq \bigcup_{i=0}^k\sem{\com[D]}^i\rho
        \end{equation*}
        so, by monotonicity of \(\sem{\com[D]}\)
        \begin{equation*}
          \sem{\com[D]}\left( \sem{\com[D]}^{k+1+j}\rho\right) \subseteq \sem{\com[D]}\left(\bigcup_{i=0}^k\sem{\com[D]}^i\rho \right)
        \end{equation*}
        and therefore
        \begin{equation*}
          \sem{\com[D]}^{(k+1)+(j+1)}\rho \subseteq \left(\bigcup_{i=1}^{k+1}\sem{\com[D]}^i\rho \right) \subseteq \rho \cup \left(\bigcup_{i=1}^{k+1}\sem{\com[D]}^i\rho \right) = \bigcup_{i=0}^{k+1}\sem{\com[D]}^i\rho = \bigcup_{i=0}^{k}\sem{\com[D]}^i\rho
        \end{equation*}
      \end{itemize}
    \end{proof}

    % quindi lo statement deve essere falso
    therefore \(\forall k \in \n\)
    \(\sem{\com[D]}^{k+1}\rho \not\subseteq \bigcup_{i=0}^k
    \sem{\com[D]}^i\rho\), otherwise
    \(\sem{\com[D]^*}\rho \subseteq \bigcup_{i=0}^k\sem{\com[D]}\rho\) which
    is impossible since the right term is a finite quantity.
    % No , definiamo l'albero dove ogni livello è composto dai
    % seguenti stati
    With this observation we build the tree \(\tuple{\env, \to^D}\),
    where \(\to^D \subseteq \env \times \env\) and if
    \(\rho' \to^D \rho''\) then \(\stt{\com[D], \rho'} \to^*
    \rho''\). We define by the following rule the levels of the tree:
    \begin{align*}
      Y_0 & = \{\rho\} \\
      Y_{k+1} & = \left(\sem{\com[D]}^{k+1}\rho\right) \setminus \left( \bigcup_{i=0}^k \sem{\com[D]}\rho \right)
    \end{align*}
    Where \(Y_0\) is the singleton set containing the root \(\rho\)
    and the \(k\)-th level is made of the environments in the \(Y_k\)
    set. Figure \ref{fig:tree} shows a tree of \(\to^D\) relations and
    visualizes the levels \(Y_k\). We can therefore make the following
    observations:
    \begin{enumerate}[label=(\roman*)]
    \item The tree is rooted in \(\rho \in Y_0\). In fact
      \(\forall \rho' \in Y_1\) \(\rho \to^D \rho'\) by definition and
      \(\forall \rho'''\in Y_{k+1} \exists \rho'' \in Y_k \mid \rho''
      \to^D \rho'''\);
    \item because of observation \ref{obs:noncontained} each level
      \(Y_k\) is non empty and finite, therefore there is an infinite
      quantity of levels, where each node has a finite quantity of
      children;
    \item\label{it:three} for all \(\rho', \rho''\in\env\)
      s.t. \(\rho' \to^D \rho''\) since
      \(\stt{\com[D], \rho'} \to^* \rho''\) we can apply the corollary
      (\ref{co:comp}) of the composition lemma (\ref{le:comp}) and
      state that
      \(\stt{\com[D];\com[D]^*, \rho'} \to^* \stt{\com[D]^*, \rho''}\)
      and because of the star rule
      \(\stt{\com[D]^*, \rho'} \to \stt{\com[D];\com[D]^*, \rho'}\) We
      can therefore say that
      \begin{equation*}
        \stt{\com[D]^*, \rho'} \to^* \stt{\com[D]^*, \rho''}
      \end{equation*}
    \end{enumerate}

    what is left to do is show that there's a path from
    \(\stt{\com[D]^*, \rho}\) of infinite length.  To do so we need
    König's Lemma from \cite{konig1926lemma}:
    
    \begin{lemma}[König's Lemma]\label{le:konig}
      Let \(T\) be a rooted tree with an infinite number of nodes,
      each with a finite number of children.  Then \(T\) has a branch
      of infinite length.
    \end{lemma}

    \begin{figure}
      \centering
      \begin{tikzpicture}[->,>=stealth]
        \tikzset{node distance = .7cm}

        \node (rho)                     {\(\rho\)};
        \node (dots1)  [below=of rho]   {\(\dots\)};
        \node (rho11)  [left=of dots1]  {\(\rho'\)};
        \node (rho12)  [right=of dots1] {\(\rho''\)};
        \node (dots2)  [below=of dots1] {\(\dots\)};
        \node (phan2)  [right=of dots2] {\(\phantom{\rho''}\)};
        \node (rho21)  [left=of dots2]  {\(\rho'''\)};
        \node (vdot1)  [below=of rho21] {\(\vdots\)};
        \node (vdot2)  [below=of dots2] {\(\vdots\)};
        \node (phan3)  [right=of vdot2] {\(\phantom{\rho''}\)};

        \draw[red, thick, rounded corners]
        (rho11.south west) rectangle (rho12.north east);
        \draw[red, thick, rounded corners]
        (rho21.south west) rectangle (phan2.north east);
        \draw[red, thick, rounded corners]
        (vdot1.south west) rectangle (phan3.north east);
        \node[right=of rho12] at (rho12) {\(Y_1\)};
        \node[right=of phan2] at (phan2) {\(Y_2\)};
        \node[right=of phan3] at (phan3) {\(Y_k\)};

        \draw
        (rho) edge (rho11) edge (dots1) edge (rho12)
        (rho11) edge (rho21)
        (dots1) edge (dots2)
        (rho12) edge (dots2)
        (rho21) edge (vdot1)
        (dots2) edge (vdot2);
      \end{tikzpicture}
      \caption{Example of \(\to^D\) relations between elements of
        \(\env\).}\label{fig:tree}
    \end{figure}

    We can therefore use König's Lemma (\ref{le:konig}) and deduce
    that there exists an infinite path from \(\rho\) of \(\to^D\)
    relations
    \begin{equation*}
      \rho \to^D \rho' \to^D \rho'' \to^D \dots
    \end{equation*}
    Where each element belongs to a different level \(Y_k\), and
    therefore is different from every other evnironment appearing in
    the sequence. Because of observation \ref{it:three} there exists a
    path
    \begin{equation*}
      \stt{\com[D]^*, \rho} \to^*\stt{\com[D]^*, \rho'} \to^* \stt{\com[D]^*, \rho''} \to^* \dots
    \end{equation*}
    which is infinite, which means \(\trans{\com[D]^*}{\rho}\nehalts\)
    and therefore \(\trans{\com[D]^*}{\rho} \ahalts\) is false.
    
  \item In case \(\sem{D^*}\rho\) is finite we can reduce total
    termination to the presence of some cycle in one of the paths
    starting from \(\trans{\com[D]^*}{\rho}\). The statement we want
    to prove is
    \begin{equation*}
      \text{if }\sem{\com[D]^*}\rho \text{ is finite, then } \trans{\com[D]^*}{\rho} \ahalts \iff \trans{\com[D]^*}{\rho} \text{ is acyclic}
    \end{equation*}
    \begin{itemize}
    \item[\((\Rightarrow)\)] In this case we want to prove that if
      \(\sem{\com[D]^*}\) is finite and
      \(\trans{\com[D]}{\rho} \ahalts\) then there are no cycles in
      any path starting from \(\stt{\com[D], \rho}\). To do so we work
      by contraddiction. Suppose there is some path starting from
      \(\trans{\com[D]^*}{\rho}\) with some cycle
      \begin{equation*}
        \stt{\com[D]^*, \rho} \to^* \stt{\com[D]^*, \rho'} \to^+ \stt{\com[D]^*, \rho'} \to^* \rho''
      \end{equation*}
      with \(\rho''\neq \rho, \rho'\), then we can notice that also
      the infinite path
      \begin{equation*}
        \stt{\com[D]^*, \rho} \to^* \stt{\com[D]^*, \rho'} \to^+ \stt{\com[D]^*, \rho'} \to^+ \stt{\com[D]^*, \rho'} \to^+ \dots
      \end{equation*}
      is part of the transition systsem for \(\stt{\com[D], \rho}\),
      and therefore \(\trans{\com[D]^*}{\rho} \ahalts\) is false which
      is absurd.
    \item[\((\Leftarrow)\)] In this case we want to prove that if
      \(\sem{\com[D]^*}\rho\) is finite and there are no cycles in any
      path starting from \(\stt{\com[D], \rho}\) then
      \(\trans{\com[D]}{\rho} \ahalts\). We work again by
      contraddiction. Suppose that we have an infinite path starting
      from \(\trans{\com[D]^*}{\rho}\). It must be that
      \(\forall i,j \in \n\) \(i \neq j\), \(\rho_i \neq \rho_j\) with
      \(\rho_0 = \rho\), otherwise there would be a cycle, which is
      not the case by hypothesis. Therefore the path would have the
      shape
      \begin{equation*}
        \stt{\com[D]^*, \rho} \to^* \stt{\com[D]^*, \rho_1} \to^* \stt{\com[D]^*, \rho_2} \to^* \stt{\com[D]^*, \rho_3} \to^* \dots
      \end{equation*}
      We can notice that for all \(\rho' \in \{\rho, \rho_1, \dots\}\)
      and for the star\(_{\text{fix}}\) rule,
      \(\stt{\com[D]^*, \rho'} \to \rho'\) and therefore
      \(\rho'\in \sem{\com[D]}\rho\). This would mean that
      \(\sem{\com[D]^*}\rho\) is infinite, which is absurd.
    \end{itemize}
  \end{itemize}
\end{proof}
