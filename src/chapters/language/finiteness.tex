\section{Deciding Finiteness}

Another information we might want to grasp from the concrete semantics
is weather the concrete invariant is finite or not. In this section we
argue that not even this information is obtainable, as it would allow
to decide for termination.

\begin{theorem}
  Given \(C\in\imp\) and an initial \(X \in 2^\env\), the
  predicate \[Q(C,X)= ``\sem{C}X \text{ is finite}" \] is
  undecidable.
\end{theorem}

\begin{proof}
  Consider \(C\in\imp, 2^\env \ni X = \{\rho\}\) a subset composed of
  a single state. The evaluation of \(\sem{C}\{\rho\}\) is exactly the
  strongest invariant for the trace \(\trace(C,\rho)\). If its
  finiteness was decidable, we could also decide for termination.

  Suppose we have an oracle \(\Omega = \lambda C , X . Q(C,X) ? 1 :
  0\). there are 2 cases for \(\Omega(C,\{\rho\})\):
  \begin{itemize}
  \item \(\Omega(C,\{\rho\}) = 0\): Then the most precise invariant
    for the program consists in an infinite quantity of states. The
    only way a program could produce infinite states is by not
    terminating.
  \item \(\Omega(C,\{\rho\}) = 1\): Then it means that the strongest
    invariant consists in a finite amount of states. We can build a
    procedure to decide \(C\) termination thanks to traces:
  \end{itemize}
\end{proof}
