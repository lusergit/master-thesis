\section{Deciding invariant finiteness}

\begin{lemma}\label{le:finiteness}
  If \(\com\in\imp_\starless\), and a finite \(X \in \poset{env}\)
  then \[\sem{C}X \text{ is finite}\]
\end{lemma}

\begin{proof}
  By induction on the program \(C\):
  \paragraph*{Base case:\\}
  \(\com \equiv e\), therefore \(\sem{e}X = \{\bsem{e}\rho \mid \rho
  \in X , \bsem{e}\rho \neq \bot\}\), which is finite, since \(X\) is
  finite.
  
  \paragraph*{Inductive cases:\\}
  \begin{enumerate}
  \item \(\com\equiv \cc{C_1 + C_2}\), therefore \(\sem{C_1 + C_2} X =
    \sem{C_1}X \cup \sem{C_2}X\). By inductive hypothesis, both
    \(\sem{C_1}X, \sem{C_2}X\) are finite, as they're sub expressions
    of \(C\). Since the union of finite sets is finite, \(\sem{C_1 +
      C_2}X\) is finite.
  \item \(C\equiv C_1; C_2\), therefore \(\sem{C_1;C_2}X =
    \sem{C_2}(\sem{C_1}X)\). By inductive hypothesis \(\sem{C_1}X =
    Y\) is finite. Again by inductive hypothesis \(\sem{C_2}Y\) is
    finite.
  \end{enumerate}
\end{proof}

\begin{lemma}\label{le:infiniteness}
  Given \(\com\in\imp_\starless\), and a finite \(X \in \poset{env}\),
  the predicate "\(\sem{C^*}X\) is finite" is undecidable.
\end{lemma}

\begin{proof}
  Suppose we can decide wether \(\sem{C^*}X\) is finite. We'll show
  that in both cases we can decide wether \(\com^*(a_1, \dots,
  a_k)\downarrow\), which we already show to be undecidable.
  \begin{itemize}
  \item Suppose we can decide wether \(\sem{C^*}X\) is infinite for
    \(\com\in\imp\) and \(X\in\poset{\env}\). Since \(\sem{C^*}X =
    \bigcup_{i\in\n}\sem{C}^i X\), \(\forall i \in \n \sem{C}^i X
    \equiv \sem{\underbrace{C;C;\dots;C}_{i\text{ times}}}X\) is
    finite because of lemma \ref{le:finiteness}. The only way we could
    end up with an infinite amount of states is by resulting in an
    infinite amount of different collections of environments for each
    \(\com\) application. In other words \(\nexists i,j \in \n \mid
    \sem{C}^i X = \sem{C}^{i+j}X\) and therefore \(\forall i,j\in\n
    \{\rho_t \in \env \mid \rho \in X , \stt{C^i, \rho} \to^* \rho_t\}
    \neq \{\rho_t \in \env \mid \rho \in X , \stt{C^{i+j}, \rho} \to^*
    \rho_t\}\). Therefore \(\com^*(a_1, \dots, a_k)\uparrow\).
  \item If we know instead that \(\sem{C^*}X\) is finite, we know that
    \(\red{\cc{C^*}}\) is finite (lemma \ref{le:reductionsfinite}),
    but we can observe that the set of states \(\{\stt{\cc{C'}, \rho'}
    \mid \stt{\cc{C}, \rho} \to^* \stt{\cc{C'}, \rho'}\}\)
    is \[\cc{C';C^*} \times X\] with \(\cc{C'}\in\red{\cc{C}}\) and
    \(X \subseteq \bigcup_{\cc{C''}\in\red{\cc{C}}} \sem{C''} \{\rho\}
    \). Since \(\com \in \imp_\starless\), \(\red{\com}\) is finite,
    \(\bigcup_{\cc{C''}\in\red{\cc{C}}} \sem{C''} \{\rho\}\) is finite
    and therefore \(X\) is finite.
  \end{itemize}
\end{proof}
