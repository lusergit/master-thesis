\section{Decidability}
The concrete semantics we described so far is able to express exactly
the most precide program invariant for every program. Unfortunately
this semantics is generally not computable, due to three reasons:

\begin{problem}[Domain rapresentation]\label{problem:domain}
  the concrete elements live in the complete lattice \(\dom = \langle
  2^\env, \subseteq \rangle\) which is infinite and therefore not
  computable rapresentable. The section \ref{decide:dom} explores what
  happens when we restrict our domain to decidable (recursive) sets
  and how we can mitigate this problem;
\end{problem}

\begin{problem}[Concrete operations]\label{problem:operations}
  the semantics operators for atomic statements \(\llp \cdot \rrp,
  \langle \cdot \rangle\) and the join operation \(\cup\) are not
  computable -- or, given a finite domain \(\dom_\alpha\), would be
  too costly to evaluate individually on each memory state;
\end{problem}

\begin{problem}[Domain Exploration non-Termination]\label{problem:exploration}
  the iterations needed in order to evaluate the semantics of loops
  for the denotational semantics, or in general solve the equational
  semantics may not be a procedure that converges in finite time --
  or, for a finite \(\dom_\alpha\) may require on finite, yet
  extremely long chains
\end{problem}

The general way is to sove the point 1 and 2 trough abstraction and
the 3rd point via convergence accelleration (namely the widening and
narrowing operations)

\subsection{Semantics decidability}\label{decide:sem}
The initial question is weather the collecting semantics \(\langle
\cdot \rangle : \imp \to \dom \to \dom\) termination is decidable or
not, i.e., if the following predicate \[Q(C,X) = ``\langle C \rangle
(X) \downarrow\text{''}\] is decidable or not. The intuitive result is
that the predicate is not decidable, as it asks weather program
termination is decidable or not, which is the famous halting problem
(from \cite{turing1936computable}).

\begin{theorem}[Termination is not decidable]
  The predicate \[Q(C,X) = ``\langle C \rangle (X)
  \downarrow\text{''}\] is not decidable
\end{theorem}

\begin{proof}
  For all \(C\in\imp\), we identify the partial recursive function
  \(\sem{C} : \n \htor \n\) which encodes for the collecting semantics
  of \(C\). if its termination was decidable we could decide for the
  termination of \(C\):
\end{proof}

\subsection{Domain rapresentation}\label{decide:dom}
%% Given a program \(C \in \imp\), we would like to be able to rapresent
%% the elements of the domain and codomain of the semantic function
%% \(\langle C \rangle : 2_\alpha^\env \to 2_\alpha^\env\) where \(\alpha
%% \in \{\) r.e., rec \(\}\), in the sense that we're restraining
%% ourselves either to a recursive subset of the whole powerset of
%% environments or to a recursively enumerable subset. The following
%% section aims to explore what happens in these 2 cases. We claim that
%% if we restrain ourselves a recursive subset of the power set of
%% states, the semantic function can only have as codomain a recursively
%% enumberable subset of states, this is due to known facts in
%% computability theory, while if we start from a recursively enumerable
%% set we end up again in a recursively enumerable set, ultimately
%% deciding \(\alpha\) r.e..
Problem \ref{problem:domain} can be solved by restraining our domain
\(\dom\) to its decidable subsets or recursively enumerable
subsets. If we restrain ourselves to recursive subsets however, the
image of the collecting semantics is composed of recursively
enumerable sets:

\begin{theorem}\label{th:domain}
  let \(C\in\imp, X\subseteq\env\), \(X\) \emph{decidable}, i.e.,
  \(\exists \bigx_X = \lambda x . (x \in X) ? 1 : 0\). \[Y = \langle C
  \rangle X\] is not decidable.
\end{theorem}

\begin{proof}\label{proof:domain}
  Recall observation \ref{obs:bijection}: in general a partial
  function over a decidable set returns a semi-decidable set,
  therefore \(Y\) is (in general) r.e.
\end{proof}

Because of the latter observation we can imagine that if we restrain
ouselves to decidable sets as domain of our semantics function, the
result will be anyway the set of recursively enumerable
domains \[\langle \cdot \rangle : \imp \to \dom_{rec} \hto
\dom_{r.e.}\]
