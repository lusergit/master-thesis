\section{Functions in Imp}
Since we're usually dealing with a finite number of free variables in
our programs, we can without loss of generality refer to (input)
variables as \(x_n\) with \(n\in\n\). Therefore the collections of
states \(X\in\poset{\env}\) will look like \[\envi{x_1\mapsto v_1, x_2
  \mapsto v_2, \dots, x_n\mapsto v_n, y \mapsto v_y, z \mapsto v_z,
  \dots}\] (since we're interested in finite programs, we can have
only a finite set of free variables per program).

\begin{notation}[Program input]
  Let \(C\in\imp\) be a program, \((a_1, \dots, a_k) \in \n^\omega\)
  be a sequence of natural numbers. We indicate the sequence of
  \(\to\) transitions starting from the state
  \(\stt{C,\envi{x_1\mapsto a_1, \dots, x_k\mapsto a_k}}\) as \[C(a_1,
  \dots, a_k)\]
\end{notation}

\begin{notation}[Existstential termination]
  We write \[\com[C](a_1, \dots, a_k) \ehalts \iff \sem{C}\{\envi{x_1
    \mapsto a_1, \dots, x_k \mapsto a_k}\}\] meaning that in the
  transition system at least one path leads to a final environment
  \(\rho_t\). Dually a programm \(\com[C](a_1, \dots, a_k) \nehalts\)
  iff there's no path that lead to a final environment \(\rho_t\).
\end{notation}

\begin{notation}[Universal termination]
  We write \[C(a_1, \dots, a_k) \ahalts \iff \forall \stt{C, \envi{x_1
      \mapsto a_1, \dots, x_k \mapsto a_k}} \to^* \rho_t \quad
  \text{for some } \rho_t\in \env\] meaning that in the transition
  system every path leads to a final environment \(\rho_t\). Dually a
  program \(\com[C](a_1, \dots, a_k) \nahalts\) iff at least one path
  does not lead to a final environment \(\rho_t\).
\end{notation}

\begin{notation}[Program output]
  We say \[C(a_1,\dots, a_n)\ehalts[b] \iff \exists \stt{C, \envi{x_1
      \mapsto a_1, \dots, x_k \mapsto a_k}} \to^* \rho_t \text{ s.t. }
  \rho_t(\var[y]) = b\] \[C(a_1,\dots, a_n)\ahalts[b] \iff \forall
  \stt{C, \envi{x_1 \mapsto a_1, \dots, x_k \mapsto a_k}} \to^* \rho_t
  \text{ s.t. } \rho_t(\var[y]) = b\]We consider the variable
  \(\var[y]\) as an output register for the program.
\end{notation}

\begin{observation}
  notice that this means, by lemma \ref{le:link} that \[C(a_1, \dots,
  a_k) \ehalts[b] \iff \exists \rho_t \in \sem{C}\{\envi{x_1 \mapsto
    a_1, \dots x_k \mapsto a_k}\} \; . \; \rho_t(y) = b\] \[C(a_1,
  \dots, a_k) \ahalts[b] \iff \forall \rho_t \in \sem{C}\{\envi{x_1
    \mapsto a_1, \dots x_k \mapsto a_k}\} \; . \; \rho_t(y) = b\]
\end{observation}


\begin{definition}[Imp computability]
  let \(f : \n^k \to \n\) be a function. \(f\) is Imp computable if

  \[\exists C \in\imp \mid \forall (a_1, \dots, a_k) \in \n^k \wedge
  b \in \n \] \[C(a_1, \dots, a_k) \downarrow b \iff (a_1, \dots,
  a_k) \in dom(f) \wedge f(a_1,\dots,a_k) = b\]
\end{definition}

We argue that the class of function computed by Imp is the same as the
set of partially recursive functions \(\partialrec\) (as defined in
\cite{cutland1980computability}). To do that we have to prove that it
contains the zero, successor and projection functions and it is closed
under composition, primitive recursion and unbounded minimalization.

\begin{lemma}[Imp functions richness]
  The class of Imp-computable function is rich.
\end{lemma}

\begin{proof}

  We'll proceed by proving that Imp has each and every one of the
  basic functions (zero, successor, projection). 

  \begin{itemize}
  \item The zero function:
    \begin{align*}
      z : \; & \n^k \to \n \\
      & (x_1, \dots, x_k) \mapsto 0
    \end{align*}
    is Imp-computable: \[z(a_1,\dots,a_k) \defin y := 0\]
  \item The successor function
    \begin{align*}
      s : \; & \n \to \n \\
      & x_1 \mapsto x_1 + 1
    \end{align*}
    is Imp-computable: \[s(a_1) \defin y := x_1 + 1\]
  \item The projection function
    \begin{align*}
      U_i^k : \; & \n^k \to \n \\
      & (x_1,\dots,x_k) \mapsto x_i
    \end{align*}
    is Imp-computable: \[U_i^k(a_1, \dots, a_k) \defin y := x_i + 0\]
  \end{itemize}

  We'll then prove that it is closed under composition, primitive
  recursion and unbounded minimalization.

  \begin{lemma}
    let \(f : \n^k \to \n\), \(g_1,\dots,g_k : \n^n \to \n\) and
    consider the composition
    \begin{align*}
      h : \; & \n^k \to \n \\
      & \vec{x} \mapsto f(g_1(\vec{x}), \dots, g_k(\vec{x}))
    \end{align*}
    \(h\) is Imp-computable.
  \end{lemma}
  \begin{proof}
    Since by hp \(f, g_n \forall n\in [1,k]\)
    are computable, we'll consider their programs \(F, G_n\forall n
    \in [1,k]\). Now consider the program
    \begin{center}
      \begin{tabular}{l}
        \(G_1(\vec{x})\);\\[0pt]
        \(y_1 := y + 0\);\\[0pt]
        \(G_2(\vec{x})\);\\[0pt]
        \(y_2 := y + 0\);\\[0pt]
        \(\dots\);\\[0pt]
        \(G_k(\vec{x})\);\\[0pt]
        \(y_k := y + 0\);\\[0pt]
        \(F(y_1, y_2,\dots, y_k)\);\\[0pt]
      \end{tabular}
    \end{center}
    Which is exactly \(h\). Therefore Imp is closed under generalised
    composition.
  \end{proof}

  \begin{lemma}
    Given \(f : \n^k \to \n\) and \(g : \n^{k+2} \to \n\) Imp
    computable, we argue that \(h : \n^{k+1} \to \n\)
    \[\begin{cases}
    h(\vec{x}, 0) = f(\vec{x}) \\
    h(\vec{x}, y+1) = g(\vec{x}, y, h(\vec{x}, y))
    \end{cases}\]
    defined trough primitive recursion is Imp-computable.
  \end{lemma}
  \begin{proof}
    We want a program to compute \(h : \n^{k+1} \to \n\). By
    hypothesis we have programs \(F, G\) to compute respectively \(f:
    \n^k \to \n\) and \(g : \n^{k+2} \to \n\). Consider the program
    \(H(\vec{x},x_{k+1})\):
    \begin{center}
      \begin{tabular}{l}
        \(s := 0;\)\\[0pt]
        \(F(\vec{x});\)\\[0pt]
        \((x_{k+1} \not\in [0,0]; G(\vec{x},s,y);s:=s+1;x_{k+1}:=x_{k+1}-1)^*;\)\\[0pt]
        \(x_{k+1} \in [0,0];\)\\[0pt]
      \end{tabular}

    \end{center}

    which computes exactly \(h\). Therefore Imp is closed under
    primitive recursion.
  \end{proof}

  \begin{lemma}
    Let \(f : \n^{k+1} \to \n\) be a Imp-computable function. Then the
    function \(h : \n^k \to \n\) defined trough unbounded
    minimalization
    \begin{equation}
      h(\vec{x}) = \mu y . f(\vec{x}, y) = \begin{cases}
        \text{least } z \text{ s.t. } & \begin{cases}
          f(\vec{x}, z) = 0 \\
	  f(\vec{x}, z) \downarrow \quad f(\vec{x},z')\neq 0 \quad \forall z < z' \\
	\end{cases} \\
        \uparrow                      & \text{otherwise}
      \end{cases}
    \end{equation}
    is Imp-computable.
  \end{lemma}

  \begin{proof}
    Let \(F\) be the program for the computable function \(f\)
    with ariety \(k+1\), \(\vec{x} = (x_1, x_2, \dots, x_k)\)
    . Consider the program \(H(\vec{x})\)
    \begin{center}
      \begin{tabular}{l}
        \(z := 0;\)\\[0pt]
        \(F(\vec{x},z);\)\\[0pt]
        \((y \not\in [0,0];z := z + 1;F(\vec{x},z))^*;\)\\[0pt]
        \(y\in [0,0];\)\\[0pt]
        \(y := z + 0;\)\\[0pt]
      \end{tabular}

    \end{center}
    Which outputs the least \(z\) s.t. \(F(\vec{x},z) \downarrow 0\),
    and loops forever otherwise. Imp is therefore closed under bounded
    minimalization.
  \end{proof}
  Since has the zero function, the successor function, the projections
  function and is closed under composition, primitive recursion and
  unbounded minimalization, the class of Imp-computable functions is
  rich.
\end{proof}

Since it is rich and \(\partialrec\) is the least class of rich
functions, \(\partialrec \subseteq \imp_f\) holds. Therefore we can
say \[f \in \partialrec[k] \Rightarrow \exists C \in \imp \mid C(a_1,
\dots, a_k)\downarrow b \iff f(a_1, \dots, a_k) \downarrow b\] From
this we get a couple of facts that derive from well known
computability results:
\begin{itemize}
\item deciding wether \(\sem{C}X \neq \emptyset\) (i.e., \(C(a_1,
  \dots, a_k) \ahalts\)) is the same as deciding \(x \in dom(f)\)
  for some \(f \in \partialrec[k]\), which is undecidable (from the
  input problem in \cite[p.~104]{cutland1980computability})
\item dually, deciding wether \(\sem{C}X = \emptyset\) (i.e., \(C(a_1,
  \dots, a_k)\nahalts\)) is also undecidable. The set of functions
  \(f\in \partialrec[k]\) s.t. \(f(x) \uparrow \forall x \in \n^k\) is
  not trivial and saturated, therefore it is not recursive (by Rice's
  theorem \cite{rice1953classes}).
\end{itemize}
