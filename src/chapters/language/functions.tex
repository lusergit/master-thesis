\section{Functions in Imp}\label{sec:functionsimp}
In the following section we argue that the set of functions is at
least a superset of the partially recursive functions described in
\cite{cutland1980computability}. This way we can derive some results
from well known computability results, without proving them from
scratch. We can do this by encoding partial recursive functions into
Imp programs.  Partial recursive functions are functions
\(\partialrec[k]\) with arity \(k\):
\begin{definition}[Partially recursive functions]
  The class \(\partialrec[k]\) of \emph{partially recursive functions}
  is the least class of functions on the natural numbers which
  contains
  \begin{enumerate}[label=(\alph*)]
  \item\label{parec:a} the zero function:
    \begin{align*}
      z : \; & \n^k \to \n \\
      & (x_1, \dots, x_k) \mapsto 0
    \end{align*}
  \item\label{parec:b} the successor function
    \begin{align*}
      s : \; & \n \to \n \\
      & x_1 \mapsto x_1 + 1
    \end{align*}
  \item\label{parec:c} the projection function
    \begin{align*}
      U_i^k : \; & \n^k \to \n \\
      & (x_1,\dots,x_k) \mapsto x_i
    \end{align*}
  \end{enumerate}
  and is closed under
  \begin{enumerate}[label=(\arabic*)]
  \item\label{parec:1} composition: given a function
    \(f : \partialrec[k]\) and functions
    \(g_1, \dots, g_k : \partialrec[n]\) the \emph{composition}
    \(h : \partialrec[n]\) is defined by
    \begin{equation*}
      h(\vec{x}) =
      \begin{cases}
        f(g_1(\vec{x}), \dots, g_k(\vec{x})) & \text{if } g_1(\vec{x}) \downarrow, \dots, g_k(\vec{x}) \downarrow \text{ and } f(g_1(\vec{x}), \dots, g_k(\vec{x})) \downarrow \\
        \uparrow & \text{otherwise}
      \end{cases}
    \end{equation*}
  \item\label{parec:2} primitive recursion: given
    \(f : \partialrec[k]\) and \(g : \partialrec[k+2]\) we define
    \(h : \partialrec[k+1]\) by \emph{primitive recursion} by
    \begin{equation*}
      \begin{cases}
        h(\vec{x}, 0) & = f(\vec{x}) \\
        h(\vec{x}, y + 1) & = g(\vec{x}, y, h(\vec{x}, y))
      \end{cases}
    \end{equation*}
  \item\label{parec:3} minimalization: given \(f : \partialrec[k+1]\),
    \(h : \partialrec[k]\) defined trough \emph{unbounded
      minimalization} is
    \begin{equation*}
      h(\vec{x}) = \mu y . f(\vec{x}, y) = \begin{cases}
        \text{least } z \text{ s.t. } & \begin{cases}
          f(\vec{x}, z) = 0 \\
	  f(\vec{x}, z) \downarrow \quad f(\vec{x},z')\neq 0 \quad \forall z < z' \\
	\end{cases} \\
        \uparrow                      & \text{otherwise}
      \end{cases}
    \end{equation*}
  \end{enumerate}
\end{definition}
\noindent
We also need to define what it means providing \((a_1, \dots, a_k)\)
as input for an Imp program. We do this by special input states and
variables: we can consider initial states
\(\rho = \envi{\var_1 \mapsto a_1, \dots, \var_k \mapsto a_k}\) where
each special variable \(\var_k\) maps to its initial value \(a_k\),
this way we can encode partial functions input into initial states for
a program \(\com\). Observe that since we are interested in finite
programs, it makes sense to consider only finite collections of free
variables.

%% \begin{notation}[Program input]
%%   Let \(C\in\imp\) be a program, \((a_1, \dots, a_k) \in \n^\omega\)
%%   be a sequence of natural numbers and \(\rho = \envi{x_1\mapsto a_1,
%%     \dots, x_k\mapsto a_k}\). We indicate the sequence of \(\to\)
%%   transitions starting from the state \(\stt{C,\rho}\) as \[\trans{C}{\rho}\]
%% \end{notation}
We also need to define what we mean by program output.
\begin{notation}[Program output]
  Let \(\env \ni \rho = \envi{\var_1\mapsto a_1, \dots, \var_n \mapsto
    a_n}\). We say
  \begin{align*}
    \trans{\com}{\rho}\ahalts[b] \iff & \forall \rho' \mid \stt{\com, \rho} \to^* \rho' \quad \rho'(\var[y]) = b \\
    \trans{\com}{\rho}\ehalts[b] \iff & \exists \rho' \mid \stt{\com, \rho} \to^* \rho' \quad \rho'(\var[y]) = b
  \end{align*}
  \(\com\) universally (partially) halts on \(b\) whenever for every
  (for some) final state \(\rho\) \(\rho(\var[y]) = b\). In other
  words we are using the special variable \(\var[y]\) as an output
  register.
\end{notation}

%% \begin{observation}
%%   notice that this means, by lemma \ref{le:link} that \[C(a_1, \dots,
%%   a_k) \ehalts[b] \iff \exists \rho' \in \sem{C}\{\envi{x_1 \mapsto
%%     a_1, \dots x_k \mapsto a_k}\} \; . \; \rho'(y) = b\] \[C(a_1,
%%   \dots, a_k) \ahalts[b] \iff \forall \rho' \in \sem{C}\{\envi{x_1
%%     \mapsto a_1, \dots x_k \mapsto a_k}\} \; . \; \rho'(y) = b\]
%% \end{observation}


\begin{definition}[Imp computability]
  let \(f : \n^k \to \n\) be a function. \(f\) is Imp computable if

  \[\exists \com \in\imp \mid \forall (a_1, \dots, a_k) \in \n^k \wedge
  b \in \n \] \[\trans{\com}{\rho} \ahalts[b] \iff (a_1, \dots, a_k)
  \in dom(f) \wedge f(a_1,\dots,a_k) = b\] with \(\rho = \envi{\var_1
    \mapsto a_1, \dots, \var_k \mapsto a_k}\).
\end{definition}

We argue that the class of function computed by Imp is the same as the
set of partially recursive functions \(\partialrec\) (as defined in
\cite{cutland1980computability}). To do that we have to prove that the
class of functions computed by the \(\imp\) language is a \emph{rich}
, i.e.

\begin{definition}[Rich class]
  A class of functions \(\mathcal{A}\) is said to be rich if it
  includes \ref{parec:a},\ref{parec:b} and \ref{parec:c} and it is
  closed under \ref{parec:1}, \ref{parec:2} and \ref{parec:3}.
\end{definition}

\begin{lemma}[Imp functions richness]
  The class of Imp-computable function is rich.
\end{lemma}

\begin{proof}

  We proceed by proving that Imp has each and every one of the basic
  functions (zero, successor, projection).

  \begin{itemize}
  \item The zero function:
    \begin{align*}
      z : \; & \n^k \to \n \\
      & (x_1, \dots, x_k) \mapsto 0
    \end{align*}
    is Imp-computable: \[z(a_1,\dots,a_k) \defin y := 0\]
  \item The successor function
    \begin{align*}
      s : \; & \n \to \n \\
      & x_1 \mapsto x_1 + 1
    \end{align*}
    is Imp-computable: \[s(a_1) \defin y := x_1 + 1\]
  \item The projection function
    \begin{align*}
      U_i^k : \; & \n^k \to \n \\
      & (x_1,\dots,x_k) \mapsto x_i
    \end{align*}
    is Imp-computable: \[U_i^k(a_1, \dots, a_k) \defin y := x_i + 0\]
  \end{itemize}

  We then prove that it is closed under composition, primitive
  recursion and unbounded minimalization.

  \begin{lemma}
    let \(f : \n^k \to \n\), \(g_1,\dots,g_k : \n^n \to \n\) and
    consider the composition
    \begin{align*}
      h : \; & \n^k \to \n \\
      & \vec{x} \mapsto f(g_1(\vec{x}), \dots, g_k(\vec{x}))
    \end{align*}
    \(h\) is Imp-computable.
  \end{lemma}
  \begin{proof}
    Since by hp \(f, g_n \forall n\in [1,k]\) are computable, we
    consider their programs \(F, G_n\forall n \in [1,k]\). Now
    consider the program
    \begin{center}
      \begin{tabular}{l}
        \(G_1(\vec{x})\);\\[0pt]
        \(y_1 := y + 0\);\\[0pt]
        \(G_2(\vec{x})\);\\[0pt]
        \(y_2 := y + 0\);\\[0pt]
        \(\dots\);\\[0pt]
        \(G_k(\vec{x})\);\\[0pt]
        \(y_k := y + 0\);\\[0pt]
        \(F(y_1, y_2,\dots, y_k)\);\\[0pt]
      \end{tabular}
    \end{center}
    Which is exactly \(h\). Therefore Imp is closed under generalized
    composition.
  \end{proof}

  \begin{lemma}
    Given \(f : \n^k \to \n\) and \(g : \n^{k+2} \to \n\) Imp
    computable, we argue that \(h : \n^{k+1} \to \n\)
    \[\begin{cases}
    h(\vec{x}, 0) = f(\vec{x}) \\
    h(\vec{x}, y+1) = g(\vec{x}, y, h(\vec{x}, y))
    \end{cases}\]
    defined trough primitive recursion is Imp-computable.
  \end{lemma}
  \begin{proof}
    We want a program to compute \(h : \n^{k+1} \to \n\). By
    hypothesis we have programs \(F, G\) to compute respectively \(f:
    \n^k \to \n\) and \(g : \n^{k+2} \to \n\). Consider the program
    \(H(\vec{x},x_{k+1})\):
    \begin{center}
      \begin{tabular}{l}
        \(s := 0;\)\\[0pt]
        \(F(\vec{x});\)\\[0pt]
        \((x_{k+1} \not\in [0,0]; G(\vec{x},s,y);s:=s+1;x_{k+1}:=x_{k+1}-1)^*;\)\\[0pt]
        \(x_{k+1} \in [0,0];\)\\[0pt]
      \end{tabular}

    \end{center}

    which computes exactly \(h\). Therefore Imp is closed under
    primitive recursion.
  \end{proof}

  \begin{lemma}
    Let \(f : \n^{k+1} \to \n\) be a Imp-computable function. Then the
    function \(h : \n^k \to \n\) defined trough unbounded
    minimalization
    \begin{equation}
      h(\vec{x}) = \mu y . f(\vec{x}, y) = \begin{cases}
        \text{least } z \text{ s.t. } & \begin{cases}
          f(\vec{x}, z) = 0 \\
	  f(\vec{x}, z) \downarrow \quad f(\vec{x},z')\neq 0 \quad \forall z < z' \\
	\end{cases} \\
        \uparrow                      & \text{otherwise}
      \end{cases}
    \end{equation}
    is Imp-computable.
  \end{lemma}

  \begin{proof}
    Let \(F\) be the program for the computable function \(f\)
    with arity \(k+1\), \(\vec{x} = (x_1, x_2, \dots, x_k)\)
    . Consider the program \(H(\vec{x})\)
    \begin{center}
      \begin{tabular}{l}
        \(z := 0;\)\\[0pt]
        \(F(\vec{x},z);\)\\[0pt]
        \((y \not\in [0,0];z := z + 1;F(\vec{x},z))^*;\)\\[0pt]
        \(y\in [0,0];\)\\[0pt]
        \(y := z + 0;\)\\[0pt]
      \end{tabular}

    \end{center}
    Which outputs the least \(z\) s.t. \(F(\vec{x},z) \downarrow 0\),
    and loops forever otherwise. Imp is therefore closed under bounded
    minimalization.
  \end{proof}
  Since has the zero function, the successor function, the projections
  function and is closed under composition, primitive recursion and
  unbounded minimalization, the class of Imp-computable functions is
  rich.
\end{proof}

Since it is rich and \(\partialrec\) is the least class of rich
functions, \(\partialrec \subseteq \imp_f\) holds. Therefore we can
say \[f \in \partialrec[k] \Rightarrow \exists \com \in \imp \mid
\trans{\com}{\rho}\ahalts b \iff f(a_1, \dots, a_k) \downarrow b\]
with \(\rho = \envi{\var_1 \mapsto a_1, \dots, \var_k \mapsto a_k}\).
From this we get a couple of facts that derive from well known
computability results:
\begin{corollary}
  \(\trans{\com}{\rho} \nahalts\) (i.e., \(\sem{C}X = \emptyset\)) is
  undecidable.
\end{corollary}

\begin{proof}
  The set of functions \(f\in \partialrec[k]\) s.t.
  \(f(x) \uparrow \forall x \in \n^k\) is not trivial and saturated,
  therefore it is not recursive by Rice's theorem
  \cite{rice1953classes}
\end{proof}

\begin{corollary}
  \(\trans{\com}{\rho} \ahalts\) is undecidable.
\end{corollary}
\begin{proof}
  The set of functions \(f\in \partialrec[k]\) s.t.
  \(f(x) \downarrow \forall x \in \n^k\) is not trivial and saturated,
  therefore it is not recursive by Rice's theorem
  \cite{rice1953classes};
\end{proof}