\section{Functions in Imp}
Since we usually deal with a finite number of free variables in our
programs, we can without loss of generality refer to (input) variables
as \(\var_n\) with \(n\in\n\). Therefore the collections of states
\(X\in\poset{\env}\) will look like \[\envi{\var_1\mapsto v_1, \var_2
  \mapsto v_2, \dots, \var_n\mapsto v_n, \var[y] \mapsto v_y, \var[z]
  \mapsto v_z, \dots}.\] Observe that since we're interested in finite
programs, it makes sense to consider only finite collections of free
variables.

%% \begin{notation}[Program input]
%%   Let \(C\in\imp\) be a program, \((a_1, \dots, a_k) \in \n^\omega\)
%%   be a sequence of natural numbers and \(\rho = \envi{x_1\mapsto a_1,
%%     \dots, x_k\mapsto a_k}\). We indicate the sequence of \(\to\)
%%   transitions starting from the state \(\stt{C,\rho}\) as \[\trans{C}{\rho}\]
%% \end{notation}

\begin{notation}[Partial termination]
  Let \(\rho = \envi{\var_1 \mapsto a_1, \var_2\mapsto a_2, \dots,
    \var_k\mapsto a_k}\). We say that a program \(\com\)
  \emph{partially halts} on some state \(\rho\) when there's at least
  one path of finite length in the transition system
  \(\trans{\com}{\rho}\) ending up in some state \(\rho'\): \[
  \trans{\com}{\rho} \ehalts \iff \exists k \in \n \mid \stt{\com,
    \rho} \to^k \rho'.\] Dually
  \[ \trans{\com}{\rho} \nahalts \iff \neg \; \trans{\com}{\rho} \ehalts \]
  a program always loops if there's no finite path in its transition
  system that leads to a final environment.
\end{notation}
 Example \ref{ex:partial} shows a program that partially halts, while
 example \ref{ex:neverhalts} shows a program that always loops.

\begin{notation}[Universal termination]
  Let \(\rho = \envi{\var_1 \mapsto a_1, \var_2\mapsto a_2, \dots,
    \var_k\mapsto a_k}\). We say that a program \(\com\)
  \emph{partially loops} on some state \(\rho\) when there's at least
  one path of infinite length in the transition system
  \(\trans{\com}{\rho}\): \[ \trans{\com}{\rho} \nehalts \iff \forall
  k \in \n \; \stt{\com, \rho} \to^k \stt{\com', \rho'} \quad
  \text{for some } \com'\in\imp, \rho'\in\env.\] Dually
  \begin{equation*}
    \trans{\com}{\rho} \ahalts \iff \neg \; \trans{\com}{\rho}
    \nehalts
  \end{equation*}
  a program \emph{universally halts} iff there's no infinite path in
  the transition systems.
\end{notation}

Notice that the absence of infinite paths implies that
\(\trans{\com}{\rho}\) is finite.  Example \ref{ex:partial} shows a
program that partially loops, while example \ref{ex:alwayshalt} shows
a program that universally halts.

%% Let's consider the following examples to better understand what this
%% means. The example \ref{ex:alwayshalt} shows a program that always
%% halts, and what it means for its transition system. Example
%% \ref{ex:neverhalts} shows a program that never halts. Finally example
%% \ref{ex:partial} shows a program that partially halts and partially
%% loops due to the presence of both a path that ends up in some state
%% \(\rho'\) and an infinite path.

\begin{example}\label{ex:alwayshalt}
  Consider the program
  \begin{equation*}
    \var := 0;
  \end{equation*}
  always halts, since \(\forall\rho \in \env, \rho \neq \bot\) builds
  the transition system \[\stt{\var := 0, \rho} \to \rho[\var \mapsto
    0]\] according to the expr rule in definition
  \ref{def:sosem}. Therefore \(\trans{(\var := 0)}{\rho} \ahalts\)
  \(\forall \rho \in \env\setminus\{\bot\}\).
\end{example}

\begin{example}\label{ex:neverhalts}
  Consider the program \(\com[P]\) \[(\var \geq 0;
  \pplus{\var})^*;\var < 0\] The program never haltsn \(\forall \rho
  \in \env\) s.t. \(\rho(\var) \geq 0\). In fact in these cases it
  builds the transition system in figure \ref{fig:tsysnhalt}. Which
  does not reach any final state \(\rho'\).
  \begin{figure}
    \begin{tikzpicture}[->,>=stealth]
      \tikzset{node distance = .5cm}
      \node (P) {\(\stt{(\var \geq 0; \pplus{\var})^*;\var < 0, \rho}\)};
      \node (zero)   [right=of P] {\(\stt{\var < 0, \rho} \not\to\)};
      \node (appzero) [below=of P] {\(\stt{\var \geq; 0\pplus{\var};(\var \geq 0; \pplus{\var})^*;\var < 0, \rho}\)};
      \node (appone) [below=of appzero] {\(\stt{(\var \geq 0; \pplus{\var})^*;\var < 0, \rho[\var \mapsto \rho(\var) + 1]}\)};
      \node (one) [right=of appone] {\(\stt{\var < 0, \rho[\var \mapsto \rho(\var) + 1]} \not\to\)};
      \node (appk) [below=of appone] {\(\stt{(\var \geq 0; \pplus{\var})^*;\var < 0, \rho[\var \mapsto \rho(\var) + k]}\)};
      \node (kei) [right=of appk] {\(\stt{\var < 0, \rho[\var \mapsto \rho(\var) + k]} \not\to\)};

      \draw
      (P) edge (zero) edge (appzero)
      (appzero) edge[to*] (appone)
      (appone) edge (one) edge[to*] (appk)
      (appk) edge (kei);
    \end{tikzpicture}
    \caption{Transition system of \((\var \geq 0; \pplus{\var})^*;\var
      < 0\)}\label{fig:tsysnhalt}
  \end{figure}
\end{example}

\begin{example}\label{ex:partial}
  Consider the program \[(\pplus{\var})^*\] it partially halts
  (\(\trans{(\pplus{\var})^*}{\rho}\ehalts\)), as according to the
  transition rule star\(_{\text{fix}}\) \(\exists \rho \in \env\) s.t.
  \[\infer[\text{star}_{\text{fix}}]{\stt{(\pplus{\var})^*, \rho} \to
    \rho}{\rho \neq \bot}\] But it also partially loops
  (\(\trans{(\pplus{\var})^*}{\rho}\nehalts\)). In fact we can build
  the infinite path\[\stt{(\pplus{\var})^*, \rho[\var\mapsto 0]} \to^*
  \stt{(\pplus{\var})^*, \rho[\var\mapsto 1]} \to^*
  \stt{(\pplus{\var})^*, \rho[\var\mapsto 2]} \to^* \dots\]
\end{example}


\begin{notation}[Program output]
  Let \(\env \ni \rho = \envi{\var_1\mapsto a_1, \dots, \var_n \mapsto
    a_n}\). We say
  \begin{align*}
    \trans{\com}{\rho}\ahalts[b] \iff & \forall \rho' \mid \stt{\com, \rho} \to^* \rho' \quad \rho'(\var[y]) = b \\
    \trans{\com}{\rho}\ehalts[b] \iff & \exists \rho' \mid \stt{\com, \rho} \to^* \rho' \quad \rho'(\var[y]) = b
  \end{align*}
\end{notation}

%% \begin{observation}
%%   notice that this means, by lemma \ref{le:link} that \[C(a_1, \dots,
%%   a_k) \ehalts[b] \iff \exists \rho' \in \sem{C}\{\envi{x_1 \mapsto
%%     a_1, \dots x_k \mapsto a_k}\} \; . \; \rho'(y) = b\] \[C(a_1,
%%   \dots, a_k) \ahalts[b] \iff \forall \rho' \in \sem{C}\{\envi{x_1
%%     \mapsto a_1, \dots x_k \mapsto a_k}\} \; . \; \rho'(y) = b\]
%% \end{observation}


\begin{definition}[Imp computability]
  let \(f : \n^k \to \n\) be a function. \(f\) is Imp computable if

  \[\exists \com \in\imp \mid \forall (a_1, \dots, a_k) \in \n^k \wedge
  b \in \n \] \[\trans{\com}{\rho} \ahalts[b] \iff (a_1, \dots, a_k)
  \in dom(f) \wedge f(a_1,\dots,a_k) = b\] with \(\rho = \envi{\var_1
    \mapsto a_1, \dots, \var_k \mapsto a_k}\).
\end{definition}

We argue that the class of function computed by Imp is the same as the
set of partially recursive functions \(\partialrec\) (as defined in
\cite{cutland1980computability}). To do that we have to prove that it
contains the zero, successor and projection functions and it is closed
under composition, primitive recursion and unbounded minimalization.

\begin{lemma}[Imp functions richness]
  The class of Imp-computable function is rich.
\end{lemma}

\begin{proof}

  We'll proceed by proving that Imp has each and every one of the
  basic functions (zero, successor, projection). 

  \begin{itemize}
  \item The zero function:
    \begin{align*}
      z : \; & \n^k \to \n \\
      & (x_1, \dots, x_k) \mapsto 0
    \end{align*}
    is Imp-computable: \[z(a_1,\dots,a_k) \defin y := 0\]
  \item The successor function
    \begin{align*}
      s : \; & \n \to \n \\
      & x_1 \mapsto x_1 + 1
    \end{align*}
    is Imp-computable: \[s(a_1) \defin y := x_1 + 1\]
  \item The projection function
    \begin{align*}
      U_i^k : \; & \n^k \to \n \\
      & (x_1,\dots,x_k) \mapsto x_i
    \end{align*}
    is Imp-computable: \[U_i^k(a_1, \dots, a_k) \defin y := x_i + 0\]
  \end{itemize}

  We'll then prove that it is closed under composition, primitive
  recursion and unbounded minimalization.

  \begin{lemma}
    let \(f : \n^k \to \n\), \(g_1,\dots,g_k : \n^n \to \n\) and
    consider the composition
    \begin{align*}
      h : \; & \n^k \to \n \\
      & \vec{x} \mapsto f(g_1(\vec{x}), \dots, g_k(\vec{x}))
    \end{align*}
    \(h\) is Imp-computable.
  \end{lemma}
  \begin{proof}
    Since by hp \(f, g_n \forall n\in [1,k]\)
    are computable, we'll consider their programs \(F, G_n\forall n
    \in [1,k]\). Now consider the program
    \begin{center}
      \begin{tabular}{l}
        \(G_1(\vec{x})\);\\[0pt]
        \(y_1 := y + 0\);\\[0pt]
        \(G_2(\vec{x})\);\\[0pt]
        \(y_2 := y + 0\);\\[0pt]
        \(\dots\);\\[0pt]
        \(G_k(\vec{x})\);\\[0pt]
        \(y_k := y + 0\);\\[0pt]
        \(F(y_1, y_2,\dots, y_k)\);\\[0pt]
      \end{tabular}
    \end{center}
    Which is exactly \(h\). Therefore Imp is closed under generalised
    composition.
  \end{proof}

  \begin{lemma}
    Given \(f : \n^k \to \n\) and \(g : \n^{k+2} \to \n\) Imp
    computable, we argue that \(h : \n^{k+1} \to \n\)
    \[\begin{cases}
    h(\vec{x}, 0) = f(\vec{x}) \\
    h(\vec{x}, y+1) = g(\vec{x}, y, h(\vec{x}, y))
    \end{cases}\]
    defined trough primitive recursion is Imp-computable.
  \end{lemma}
  \begin{proof}
    We want a program to compute \(h : \n^{k+1} \to \n\). By
    hypothesis we have programs \(F, G\) to compute respectively \(f:
    \n^k \to \n\) and \(g : \n^{k+2} \to \n\). Consider the program
    \(H(\vec{x},x_{k+1})\):
    \begin{center}
      \begin{tabular}{l}
        \(s := 0;\)\\[0pt]
        \(F(\vec{x});\)\\[0pt]
        \((x_{k+1} \not\in [0,0]; G(\vec{x},s,y);s:=s+1;x_{k+1}:=x_{k+1}-1)^*;\)\\[0pt]
        \(x_{k+1} \in [0,0];\)\\[0pt]
      \end{tabular}

    \end{center}

    which computes exactly \(h\). Therefore Imp is closed under
    primitive recursion.
  \end{proof}

  \begin{lemma}
    Let \(f : \n^{k+1} \to \n\) be a Imp-computable function. Then the
    function \(h : \n^k \to \n\) defined trough unbounded
    minimalization
    \begin{equation}
      h(\vec{x}) = \mu y . f(\vec{x}, y) = \begin{cases}
        \text{least } z \text{ s.t. } & \begin{cases}
          f(\vec{x}, z) = 0 \\
	  f(\vec{x}, z) \downarrow \quad f(\vec{x},z')\neq 0 \quad \forall z < z' \\
	\end{cases} \\
        \uparrow                      & \text{otherwise}
      \end{cases}
    \end{equation}
    is Imp-computable.
  \end{lemma}

  \begin{proof}
    Let \(F\) be the program for the computable function \(f\)
    with ariety \(k+1\), \(\vec{x} = (x_1, x_2, \dots, x_k)\)
    . Consider the program \(H(\vec{x})\)
    \begin{center}
      \begin{tabular}{l}
        \(z := 0;\)\\[0pt]
        \(F(\vec{x},z);\)\\[0pt]
        \((y \not\in [0,0];z := z + 1;F(\vec{x},z))^*;\)\\[0pt]
        \(y\in [0,0];\)\\[0pt]
        \(y := z + 0;\)\\[0pt]
      \end{tabular}

    \end{center}
    Which outputs the least \(z\) s.t. \(F(\vec{x},z) \downarrow 0\),
    and loops forever otherwise. Imp is therefore closed under bounded
    minimalization.
  \end{proof}
  Since has the zero function, the successor function, the projections
  function and is closed under composition, primitive recursion and
  unbounded minimalization, the class of Imp-computable functions is
  rich.
\end{proof}

Since it is rich and \(\partialrec\) is the least class of rich
functions, \(\partialrec \subseteq \imp_f\) holds. Therefore we can
say \[f \in \partialrec[k] \Rightarrow \exists \com \in \imp \mid
\trans{\com}{\rho}\ahalts b \iff f(a_1, \dots, a_k) \downarrow b\]
with \(\rho = \envi{\var_1 \mapsto a_1, \dots, \var_k \mapsto a_k}\).
From this we get a couple of facts that derive from well known
computability results:
\begin{itemize}
\item \(\sem{C}X = \emptyset\) (i.e., \(\trans{\com}{\rho} \nahalts\))
  is undecidable. The set of functions \(f\in \partialrec[k]\)
  s.t. \(f(x) \uparrow \forall x \in \n^k\) is not trivial and
  saturated, therefore it is not recursive (by Rice's theorem
  \cite{rice1953classes});
\item dually, \(\sem{C}X \neq \emptyset\) (i.e., \(\trans{\com}{\rho}
  \ehalts\)) is undecidable since is the negation of the latter
  statement;
\end{itemize}
