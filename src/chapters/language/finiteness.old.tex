\section{Deciding Finiteness}
Turns out that not even deciding weather the best invariant is finite
or not is decidable, as it would allow to decide (again) for
termination.

In order to do that we need a couple of preliminaries:

\begin{theorem}[Semantics of basic expressions reduces cardinality]\label{th:redcard}
  Given a basic expression \(e \in \expr\) and a Concrete invariant
  \(X \in \dom\) with \(|X| < \infty\)

  \[ | \langle e \rangle X | \leq |X| < \infty \]

  \begin{proof}
    Is sufficent to recall that \[\langle e \rangle X = \{ \llp e \rrp
    \rho \mid \rho \in X \wedge \llp e \rrp \rho \neq \bot \} \] we
    can see that according to the semantics either an environment in
    the initial set \(X\) is updated or it is discarded, as its update
    ends up being the \(\bot\) state. Since no new states are added to
    the image of \(\langle e \rangle X\), \(|\langle e \rangle X| \leq
    |X| < \infty\).
  \end{proof}

\end{theorem}

\begin{theorem}
  [Infinite invariant implies non-termination]
  \label{th:infinvnonterm}
  Let \(C\in \imp\) be program written in the \(\imp\) language, \(X\)
  a finite set of intial environments. If the invariant \(\langle C
  \rangle X\) is infinite, \(C\) does not halt.
  \begin{proof}
    We'll work by induction on the structure of the language.
    \paragraph*{Base cases\\}
    The only base case we have to conisder is when \(C = e\), which is
    already covered by theorem \ref{th:redcard}. \(\langle e \rangle
    X\) produces a finite amount fo states, therefore the statement
    holds vacuously.

    \paragraph*{Recursive cases\\}
    \begin{itemize}
    \item[\(C = C_1 + C_2\)] In this case
    \item[\(C = C_1;C_2\)] In this case
    \item[\(C = C_1^*\)] In this case 
    \end{itemize}
  \end{proof}
\end{theorem}

\begin{theorem}[Finiteness oracle]\label{th:finiteness}
  Given an oracle
  \[
  \Omega(X) = \begin{cases}
    1 & X \text{ is infinite} \\
    0 & X \text{ is finite}
  \end{cases}
  \]
  program termination is decidable.
  
  \begin{proof}
    Given the oracle \(\Omega\) as described in theorem
    \ref{th:finiteness} We could decide for the termination of \(C \in
    \imp\). \(\Omega(\langle C \rangle X)\) has 2 possible outcomes:
    \begin{itemize}
    \item[\textbf{1}] therefore by \ref{th:infinvnonterm}, \(C\)
      starting from environment \(X\) does not terminate.
    \item[\textbf{0}] In this case we can build a procedure to decide
      weather the program terminates or not. Intuitively we have to go
      trough a finite amount of states in order to find a cycle of
      states the program passes trough. If we find a cycle, then the
      program does not terminate, otherwise the program terinates.
      The procedure to check for a cycle in the trace of states is the
      algorithm \ref{alg:cycle}.
      \begin{algorithm}
        \caption{Termination procedure (kleene-term(C))}\label{alg:term}
        \label{alg:cycle}
        \begin{algorithmic}
          \State Environments \(\gets \{\rho\}\)
          \State NewEnv \(\gets \langle C \rangle\)Environments
          \While{NewEnv \(\not\in\) Environments \(\wedge\) \(b\)(NewEnv)}
          \State Environments \(\gets\) Environments \(\cup\) NewEnv
          \State NewEnv \(\gets \langle C \rangle \{\)NewEnv\(\}\)
          \EndWhile

          \If{NewEnv \(\in\) Environments} \State \Return 1
          \Comment{We found a cycle of environments, therefore \(P_1\)
            does not terminate} \EndIf

          \State \Return 0 \Comment{We otherwise exited the cycle
            because the guard on the final new state is not respected
            anymore, therefore the original program terminates}
        \end{algorithmic}
      \end{algorithm}

      The procedure must terminate, since the set of Environments to
      explore is bound to be finite by our oracle \(\Omega\).  The
      whole procedure to compute termination becomes algorithm
      \ref{alg:term}.
      \begin{algorithm}
        \caption{Termination procedure (term(C))}\label{alg:term}
        \begin{algorithmic}
          \If{\(\Omega(P, \rho ) = 0\)}
          \State \Return kleene-term(P)
          \EndIf
          \State \Return 1
        \end{algorithmic}
      \end{algorithm}
    \end{itemize}
  \end{proof}
\end{theorem}
