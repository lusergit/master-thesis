\section{Transition system}

With the set of states \(\state\), the set of environments \(\env\)
and the small step operational semantics \(\to\) we define a
transition system, this will be useful to define universal and partial
termination and to reason about program properties in the next
chapters.

\begin{definition}[Transition system] The transition system for the
  language Imp is
  \[\tsys \defin \tuple{\state \cup \env, \env, \to}\] where
  \begin{itemize}
  \item \(\state \cup \env\) is the set of configurations in the system;
  \item \(\env\) is the set of terminal states;
  \item \(\to\) is the small step semantics defined in
    Definition~\ref{def:sosem}, which describes the transition
    relations in the system.
  \end{itemize}
\end{definition}
\noindent
With the concept of derivation sequences we can define what we mean for
\emph{partial} and \emph{universal} termination.
\begin{definition}[Partial termination]
  Let \(\com \in \imp, \rho \in \env\). We say \(\com\)
  \emph{partially halts} on \(\rho\) when there is at least one
  derivation sequence of finite length in the transition system
  starting with \(\stt{\com,\rho}\) and ending in some state
  \(\rho'\):
  \[ \stt{\com, \rho} \ehalts \iff \exists k \in \n \mid \stt{\com,
      \rho} \to^k \rho'.\] Dually
  \[ \stt{\com, \rho} \nahalts \iff \neg \; \stt{\com, \rho}
    \ehalts \] a program \emph{always loops} if there is no finite
  derivation sequence in its transition system that leads to a final
  environment.
\end{definition}
\begin{definition}[Universal termination]
  Let \(\com \in \imp, \rho \in \env\). We say\(\com\) \emph{partially
    loops} on \(\rho\) when there is at least one derivation sequence
  of infinite length in the transition system starting from
  \(\stt{\com, \rho}\):
  \[ \stt{\com, \rho} \nehalts \iff \forall k \in \n \; \stt{\com,
      \rho} \to^k \stt{\com', \rho'} \quad \text{for some }
    \com'\in\imp, \rho'\in\env.\] Dually
  \begin{equation*}
    \stt{\com, \rho} \ahalts \iff \neg \; \stt{\com, \rho}
    \nehalts
  \end{equation*}
  a program \emph{universally halts} on \(\rho\) iff there is no
  infinite derivation sequence starting from \(\stt{\com, \rho}\) in
  the transition systems.
\end{definition}

Example~\ref{ex:partial} shows a program that partially halts, while
Example~\ref{ex:neverhalts} shows a program that always loops.  Notice
that the absence of infinite derivation sequences implies that \(\trans{\com}{\rho}\)
is finite.  Example \ref{ex:partial} shows a program that partially
loops, while example \ref{ex:alwayshalt} shows a program that
universally halts.

%% Let's consider the following examples to better understand what this
%% means. The example \ref{ex:alwayshalt} shows a program that always
%% halts, and what it means for its transition system. Example
%% \ref{ex:neverhalts} shows a program that never halts. Finally example
%% \ref{ex:partial} shows a program that partially halts and partially
%% loops due to the presence of both a derivation sequence that ends up in some state
%% \(\rho'\) and an infinite derivation sequence.

\begin{example}\label{ex:alwayshalt}
  Consider the program
  \begin{equation*}
    \var := 0;
  \end{equation*}
  it universally halts, since \(\forall\rho \in \env, \rho \neq \bot\)
  \[\stt{\var := 0, \rho} \to \rho[\var \mapsto 0]\] according to the
  expr rule in definition \ref{def:sosem}. Therefore
  \(\stt{(\var := 0), \rho} \ahalts\)
  \(\forall \rho \in \env\setminus\{\bot\}\).
\end{example}

\begin{example}\label{ex:neverhalts}
  Consider the program \(\com[P]\)
  \[(\var \geq 0; \pplus{\var})^*;\var < 0\] The program never halts on
  \(\forall \rho \in \env\) s.t. \(\rho(\var) \geq 0\). In fact in
  these cases it builds the transition system in figure
  \ref{fig:tsysnhalt}, where the infinite derivation sequence
  \[\stt{(\var \geq 0; \pplus{\var})^*; x < 0, \rho} \to^* \stt{(\var
      \geq 0; \pplus{\var})^*; x < 0, \rho[\var \mapsto \rho(\var) +
      1]} \to^* \dots\]
  
  \[\dots \to^* \stt{(\var \geq 0; \pplus{\var})^*; x < 0, \rho[\var
      \mapsto \rho(\var) + k]} \to^* \dots \] is always present.
  \begin{figure}
    \begin{tikzpicture}[->,>=stealth]
      \tikzset{node distance = .5cm}
      \node (P) {\(\stt{(\var \geq 0; \pplus{\var})^*;\var < 0, \rho}\)};
      \node (zero)   [right=of P] {\(\stt{\var < 0, \rho} \not\to\)};
      \node (appzero) [below=of P] {\(\stt{\var \geq; 0\pplus{\var};(\var \geq 0; \pplus{\var})^*;\var < 0, \rho}\)};
      \node (appone) [below=of appzero] {\(\stt{(\var \geq 0; \pplus{\var})^*;\var < 0, \rho[\var \mapsto \rho(\var) + 1]}\)};
      \node (one) [right=of appone] {\(\stt{\var < 0, \rho[\var \mapsto \rho(\var) + 1]} \not\to\)};
      \node (appk) [below=of appone] {\(\stt{(\var \geq 0; \pplus{\var})^*;\var < 0, \rho[\var \mapsto \rho(\var) + k]}\)};
      \node (kei) [right=of appk] {\(\stt{\var < 0, \rho[\var \mapsto \rho(\var) + k]} \not\to\)};

      \draw
      (P) edge (zero) edge (appzero)
      (appzero) edge[to*] (appone)
      (appone) edge (one) edge[to*] (appk)
      (appk) edge (kei);
    \end{tikzpicture}
    \caption{Transition system of \((\var \geq 0; \pplus{\var})^*;\var
      < 0\)}\label{fig:tsysnhalt}
  \end{figure}
\end{example}

\begin{example}\label{ex:partial}
  Consider the program \[(\pplus{\var})^*\] it partially halts
  (\(\stt{(\pplus{\var})^*, \rho}\ehalts\)), as according to the
  transition rule star\(_{\text{fix}}\) \(\exists \rho \in \env\) s.t.
  \[\infer[\text{star}_{\text{fix}}]{\stt{(\pplus{\var})^*, \rho} \to
    \rho}{\rho \neq \bot}\] But it also partially loops
(\(\stt{(\pplus{\var})^*, \rho}\nehalts\)). In fact we can build the
infinite derivation
sequence\[\stt{(\pplus{\var})^*, \rho[\var\mapsto 0]} \to^*
  \stt{(\pplus{\var})^*, \rho[\var\mapsto 1]} \to^*
  \stt{(\pplus{\var})^*, \rho[\var\mapsto 2]} \to^* \dots\]
\end{example}

\noindent
Other useful lemmas in the system are the composition and
decomposition lemma.

\begin{lemma}[Decomposition lemma]\label{le:decomp}
  If \(\stt{\com_1;\com_2, \rho} \to^k \rho''\), then there exists a
  state \(\rho'\) and a natural number \(k_1, k_2\)
  s.t. \(\stt{\com_1, \rho} \to^{k_1} \rho'\) and
  \(\stt{\com_2, \rho'} \to^{k_2}\rho''\), where \(k_1 + k_2 = k\)
\end{lemma}

\begin{proof}
  The proof is on induction on \(k\in\n\), i.e., by induction on the
  length of the derivation sequence.

  \medskip

  \noindent
  \textbf{Case} (\(k = 0\)).  Then
  \begin{equation*}
    \stt{\com_1\seq\com_2, \rho} \to^0 \rho''
  \end{equation*}
  holds vacuously since \(\stt{\com_1\seq\com_2, \rho}\) and
  \(\rho''\) are different.

  \medskip

  \noindent
  \textbf{Case} (\(k \implies k+1\)). Then
  \begin{equation*}
    \stt{\com_1\seq\com_2, \rho} \to^{k+1} \rho''
  \end{equation*}
  can be written as
  \begin{equation*}
    \stt{\com_1\seq\com_2, \rho} \to \gamma \to^{k} \rho''
  \end{equation*}
  for some configuration \(\gamma\). Now two cases apply, depending on
  the use of either comp\(_1\) or comp\(_2\) rules.

  \medskip

  \textbf{Case} \(\drule{comp_1}\).  Then
  \({\gamma = \stt{\com_1'\seq\com_2, \rho'}}\) and
  \({\stt{\com_1\seq\com_2, \rho} \to \stt{\com_1'\seq\com_2,
      \rho'}}\) because
  \({\stt{\com_1, \rho}\to\stt{\com_1', \rho'}}\). Therefore we have
  \begin{equation*}
    \stt{\com_1'\seq\com_2, \rho'} \to^k \rho''
  \end{equation*}
  Here we can use the induction hypothesis since the derivation
  sequence is shorter than the one we started with. Hence
  \(\exists \rho''\,' \in \env\) and natural numbers \(k_1, k_2\) s.t.
  \begin{equation*}
    \stt{\com_1'\seq\rho'}\to^{k_1}\rho''\,' \quad \wedge \quad \stt{\com_2, \rho''\,'} \to^{k_2} \rho''
  \end{equation*}
  where \(k_1 + k_2 = k\). Hence it holds that
  \begin{equation*}
    \stt{\com_1, \rho} \to^{k_1+1} \rho''\,'
  \end{equation*}
  and since \((k_1 + 1) + k_2 = k + 1\) it holds that
  \begin{equation*}
    \stt{\com_1\seq\com_2, \rho} \to^{k+1} \rho''
  \end{equation*}
  which is our thesis.

  \medskip

  \textbf{Case} \(\drule{comp_2}\). In this case
  \({\gamma = \stt{\com_2, \rho'}}\) because
  \({\stt{\com_1, \rho}\to\rho'}\) and it holds that
  \begin{equation*}
    \stt{\com_1\seq\com_2, \rho}\to\stt{\com_2, \rho'} \to^k \rho''
  \end{equation*}
  Hence our thesis follows by using the inductive hypothesis on
  \(\stt{\com_2, \rho'}\) and by choosing \({k_1 = 1}, {k_2 = k}\).
\end{proof}

From the latter theorem follows its corollary, which abstracts the
value of \(k\).

\begin{corollary}\label{co:decomp}
  If \(\stt{\com_1;\com_2, \rho} \to^* \rho''\) then \(\exists \rho'\)
  s.t.~\(\stt{\com_1, \rho} \to^* \rho'\) and
  \(\stt{\com_2, \rho'}\to^* \rho''\).
\end{corollary}

The second lemma states a similar but inverted property:

\begin{lemma}[Composition lemma]\label{le:comp}
  If \(\stt{\com_1, \rho} \to^k \rho'\) then
  \(\stt{\com_1; \com_2, \rho} \to^k \stt{\com_2, \rho'}\)
\end{lemma}

\begin{proof}
  The proof works again by induction on the length \(k\) of the
  derivation sequence:

  \medskip

  \noindent
  \textbf{Case} (\(k=0\)). In this case the statement vacuously holds
  as \(\stt{\com_1, \rho}\) and \(\rho'\) are different.

  \medskip

  \noindent
  \textbf{Case} (\(k \implies k + 1\)). In this case we have to
  prove that if \(\stt{\com_1, \rho} \to^{k+1} \rho'\) then
  \(\stt{\com_1\seq\com_2, \rho} \to^{k+1} \rho'\). To start we can
  notice that \(\stt{\com_1, \rho} \to^{k+1} \rho'\) means that we
  have 2 cases:
  \begin{enumerate}[label=(\arabic*)]
  \item either \(k = 0\), hence \(\stt{\com_1, \rho} \to \rho'\). But
    in this case we can use \(\drule{comp_2}\) and deduce that
    \begin{equation*}
      \infer{\stt{\com_1\seq\com_2, \rho} \to \stt{\com_2, \rho'}}{\stt{\com_1, \rho} \to \rho'}
    \end{equation*}
    
  \item Or \(k > 0\), which means
    \begin{equation}\label{eq:key}
      \stt{\com_1, \rho} \to \stt{\com_1', \rho''} \to^k \rho'.
    \end{equation}
    In this case we can use \(\drule{comp_1}\) and notice that
    \begin{equation*}
      \infer{\stt{\com_1\seq\com_2, \rho}\to\stt{\com_1'\seq\com_2, \rho''}}{\stt{\com_1, \rho} \to \stt{\com_1', \rho''}}
    \end{equation*}
    Now, by induction on \(k\) in~\eqref{eq:key} we know that
    \(\stt{\com_1'\seq\com_2, \rho''}\to^k\rho'\), hence
    \begin{equation*}
      \stt{\com_1\seq\com_2, \rho} \to \stt{\com_1'\seq\com_2, \rho''} \to^k \rho'
    \end{equation*}
    which is our thesis.
  \end{enumerate}
\end{proof}

\begin{corollary}\label{co:comp}
  If \(\stt{\com_1, \rho} \to^* \rho'\) then
  \(\stt{\com_1;\com_2, \rho}\to^* \stt{\com_2, \rho'}\).
\end{corollary}

%% \begin{lemma}\label{le:tosets}
%%   Given a program \(\com\in\imp\), for \(\red{\com}\) it holds that
%%   \begin{align*}
%%     \red{\com[e]} & = \emptyset \\
%%     \red{\com[C_1] + \com[C_2]} & = \{\com[C_1], \com[C_2]\}\cup\red{C_1}\cup\red{C_2} \\
%%     \red{\com[C_1];\com[C_2]} & = \{\com[C_2]\}\cup\red{C_2}\cup\{\com[C_1'];\com[C_2] \mid \com[C_1']\in\red{C_1}\} \\
%%     \red{C^*} & = \{\com[C;C^*]\} \cup \{\com[C';C^*] \mid \com[C']\in\red{\com[C]}\}
%%   \end{align*}
%% \end{lemma}

%% \begin{proof}
%%   By definition:
%%   \begin{itemize}
%%   \item \(\com[C]\equiv \com[e]\), and \(\red{\com[e]} = \{\com[C'] \mid
%%     \stt{\com[e], \rho} \to \stt{\com[C'], \rho'} \quad \text{for some }
%%     \rho\}\). But because of expr rule, either \(\stt{\com[e], \rho}
%%     \to \bsem{e}\rho\) (if \(\bsem{e}\rho \neq \bot\)) or
%%     \(\stt{\com[e], \rho} \not\to\) (if \(\bsem{e}\rho = \bot\)),
%%     therefore \(\nexists \com[C'] \mid \stt{\com[e], \rho} \to
%%     \stt{\com[C'], \rho'}\) for some \(\rho\), therefore \[\red{\com[e]}
%%     = \emptyset\]
%%   \item \(\com \equiv \com[C_1 + C_2]\) and \(\red{\com[C_1 + C_2]} =
%%     \{\com[C'] \mid \stt{\com[C_1 + C_2], \rho} \to \stt{\com[C'],
%%       \rho'}\) for some \(\rho\}\). By sum\(_1\) \(\stt{\com[C_1 +
%%         C_2], \rho} \to \stt{\com[C_1], \rho}\) and by sum\(_2\)
%%     \(\stt{\com[C_1 + C_2], \rho} \to \stt{\com[C_2],
%%       \rho}\). Therefore \[\red{\com[C_1 + C_2]} = \{\com[C_1],
%%     \com[C_2]\}\cup \red{\com[C_1]} \cup \red{\com[C_2]}\]
%%   \item \(\com \equiv \com[C_1;C_2]\). In this case by comp\(_1\) if
%%     \(\stt{\com[C_1], \rho}\to \stt{\com[C_1'], \rho'}\) then
%%     \(\stt{\com[C_1;C_2],\rho} \to \stt{\com[C_1';C_2], \rho'}\). By
%%     comp\(_2\) if \(\stt{\com[C_1], \rho}\to \rho'\) then
%%     \(\stt{\com[C_1;C_2],\rho} \to \stt{\com[C_2],
%%       \rho'}\). Therefore \[\red{C_1;C_2} = \{\com[C_2]\}\cup
%%     \red{C_2}\cup\{\com[C_1';C_2] \mid \com[C_1']\in\red{C_1}\}\]
%%   \item \(\com \equiv \com[C^*]\). In this case by star
%%     \(\stt{\com[C^*], \rho} \to \stt{\com[C;C^*], \rho}\) or by
%%     star\(_{\text{fix}}\) \(\stt{\com[C^*], \rho} \to
%%     \rho\). therefore \[\red{C^*} = \{\com[C;C^*]\} \cup
%%     \{\com[C';C^*]\mid \com[C']\in\red{C}\}\]
%%   \end{itemize}
%% \end{proof}

%% \begin{lemma}\label{le:reductionsfinite}
%%   \(\forall \com \in \imp\) \(\red{\com}\) is finite.
%% \end{lemma}

%% \begin{proof}
%%   We work by induction on \(\com\):

%%   \noindent
%%   \paragraph*{Base case:\\}
  
%%   \(\com \equiv \com[e]\). Because of lemma \ref{le:tosets}
%%   \(\red{\com[e]} = \emptyset\) which is finite.

%%   \noindent
%%   \paragraph*{Inductive cases:\\}
%%   \begin{itemize}
%%   \item \(\com \equiv \com[C_1 + C_2]\). By lemma \ref{le:tosets}
%%     \(\red{\com[C_1] + \com[C_2]} = \{\com[C_1], \com[C_2]\} \cup
%%     \red{\com[C_1]} \cup \red{\com[C_2]}\), which is finite as it is
%%     union of finite sets.
%%   \item \(\com \equiv \com[C_1;C_2]\) By lemma \ref{le:tosets}
%%     \(\red{\com[C_1]; \com[C_2]} =
%%     \{\com[C_2]\}\cup\red{C_2}\cup\{\com[C_1'];\com[C_2] \mid
%%     \com[C_1']\in\red{C_1}\}\). Both \(\red{\com[C_1]}\) and
%%     \(\red{\com[C_2]}\) are finite by inductive hypothesis, and
%%     therefore the union of finite sets is finite..
%%   \item \(\com \equiv \com[C^*]\). By lemma \ref{le:tosets}
%%     \(\red{\com[C^*]} = \{\com[C;C^*]\} \cup \{\com[C';C^*] \mid
%%     \com[C']\in\red{\com[C]}\}\). \(\red{\com[C]}\) is finite by
%%     inductive hypothesis, and therefore the union of finite sets is
%%     finite.
%%   \end{itemize}
%% \end{proof}
In order to better talk about the intermediate states in the execution
of a program we also introduce the notion of reducts:
\begin{definition}[Reducts]
  Let \(\imp^*\) denotes the set whose elements are statements in
  \(\imp\). The reduction function \(\mathsf{red} : \imp \to \imp^*\)
  is recursively defined by the following clauses:
  \begin{align*}
    \red{\com[e]} & \defin \{\com[e]\} \\
    \red{\com_1 + \com_2} & \defin \{\com_1 + \com_2\} \cup \red{\com_1} \cup \red{\com_2} \\
    \red{\com_1 ; \com_2} & \defin (\red{\com_1} ; \com_2) \cup \red{\com_2} \\
    \red{\com^*} & \defin \{\com^*\} \cup (\red{\com} ; \com^*)
  \end{align*}
  Where we overload the symbol \(\mathds{;}\) with the operator
  \(; : \imp^* \times \imp \to \imp^*\) defined by
  \begin{align*}
    \emptyset ; \com & \defin \emptyset \\
    \{\com_1, \dots, \com_k\} ; \com & \defin \{\com_1;\com, \dots, \com_k;\com\}
  \end{align*}
\end{definition}
\noindent
Notice that the set of reduction of any finite program \(\com\in\imp\)
is finite.
