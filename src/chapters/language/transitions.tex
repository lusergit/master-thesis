\section{Transition system}

With the set of states \(\state\), the set of environments \(\env\)
and the small operational semantics \(\to\) we define a transition
system:

\begin{definition}[Transition system]
  \[\tsys \defin \tuple{\state \cup \env, \env, \to}\] is a transition system for
  the language \(\imp\), where
  \begin{itemize}
  \item \(\state \cup \env\) is the set of configurations in the system;
  \item \(\env\) is the set of terminal states;
  \item \(\to\) is the small step semantics defined in definition
    \ref{def:sosem}, which describes the transition relations in the
    system.
  \end{itemize}
\end{definition}

\begin{definition}[Paths]
  Let \((\state \cup \env)^\infty \defin (\state \cup \env)^+ \cup
  (\state \cup \env)^\omega\) be the set of all infinitary sequences
  of states and environments (both finite and infinite). Then the set
  of \emph{paths} in the transition system is \[\Path^\infty \defin
  \{\ppath \in (\state \cup \env)^\infty \mid \forall i \in [ 1, |
    \ppath | ) . \ppath_i \to \ppath_{i+1}\}\]
\end{definition}

\begin{definition}
  Given \(\com \in \imp, \rho\in\env\) the \emph{paths} in the
  transition system starting from \(\stt{\com, \rho}\)
  are \[\trans{\com}{\rho} \defin \{\tau \in \Path^\infty \mid
  \ppath_0 = \stt{\com, \rho}\}\]
\end{definition}

%% \begin{notation}
%%   We use the notation \(\trans{\com}{\rho}\) to denote all the
%%   transitions starting from \(\stt{\com, \rho}\).
%% \end{notation}

\begin{definition}[Reductions]
  Let \(\imp^*\) denotes the set whose elements are finite, eventually
  empty, ordered sequences of statements in \(\imp\) where the empty
  sequence is denoted by \(\emptyseq\). Moreover let \(\conc : \imp^*
  \times \imp^* \to \imp^*\) be the list concatenation. The reduction
  function \(\mathsf{red} : \imp \to \imp^*\) is recursively defined by the
  following clauses:
  \begin{align*}
    \red{\com[e]} & \defin \sequence{\com[e]} \\
    \red{\com_1 + \com_2} & \defin \sequence{\com_1 + \com_2} \conc \red{\com_1} \conc \red{\com_2} \\
    \red{\com_1 ; \com_2} & \defin (\red{\com_1} \star \com_2) \conc \red{\com_2} \\
    \red{\com^*} & \defin \sequence{\com^*} \conc (\red{\com} \star \com^*)
  \end{align*}
  Where the operator \(\star : \imp^* \times \imp \to \imp^*\) is
  defined by
  \begin{align*}
    \emptyseq \star \com & \defin \emptyseq \\
    \sequence{\com_1, \dots, \com_k} \star \com & \defin \sequence{\com_1;\com, \dots, \com_k;\com}
  \end{align*}
\end{definition}

Notice that the set of reduction of any finite program \(\com\in\imp\) is finite.

\noindent
Other useful lemmas in the system are the composition and
decomposition lemma.

\begin{lemma}[Decomposition lemma]\label{le:decomp}
  If \(\stt{\com_1;\com_2, \rho} \to^k \rho''\), then there exists a
  state \(\rho'\) and a natural number \(k_1, k_2\)
  s.t. \(\stt{\com_1, \rho} \to^{k_1} \rho'\) and
  \(\stt{\com_2, \rho'} \to^{k_2}\rho''\), where \(k_1 + k_2 = k\)
\end{lemma}

\begin{corollary}\label{co:decomp}
  If \(\stt{\com_1;\com_2, \rho} \to^* \rho''\) then \(\exists \rho'\)
  s.t. \(\stt{\com_1, \rho} \to^* \rho'\) and
  \(\stt{\com_2, \rho'}\to^* \rho''\).
\end{corollary}

\begin{lemma}[Composition lemma]\label{le:comp}
  If \(\stt{\com_1, \rho} \to^k \rho'\) then
  \(\stt{\com_1; \com_2, \rho} \to^k \stt{\com_2, \rho'}\)
\end{lemma}

\begin{corollary}\label{co:comp}
  If \(\stt{\com_1, \rho} \to^* \rho'\) then
  \(\stt{\com_1;\com_2, \rho}\to^* \stt{\com_2, \rho'}\).
\end{corollary}

%% \begin{lemma}\label{le:tosets}
%%   Given a program \(\com\in\imp\), for \(\red{\com}\) it holds that
%%   \begin{align*}
%%     \red{\com[e]} & = \emptyset \\
%%     \red{\com[C_1] + \com[C_2]} & = \{\com[C_1], \com[C_2]\}\cup\red{C_1}\cup\red{C_2} \\
%%     \red{\com[C_1];\com[C_2]} & = \{\com[C_2]\}\cup\red{C_2}\cup\{\com[C_1'];\com[C_2] \mid \com[C_1']\in\red{C_1}\} \\
%%     \red{C^*} & = \{\com[C;C^*]\} \cup \{\com[C';C^*] \mid \com[C']\in\red{\com[C]}\}
%%   \end{align*}
%% \end{lemma}

%% \begin{proof}
%%   By definition:
%%   \begin{itemize}
%%   \item \(\com[C]\equiv \com[e]\), and \(\red{\com[e]} = \{\com[C'] \mid
%%     \stt{\com[e], \rho} \to \stt{\com[C'], \rho'} \quad \text{for some }
%%     \rho\}\). But because of expr rule, either \(\stt{\com[e], \rho}
%%     \to \bsem{e}\rho\) (if \(\bsem{e}\rho \neq \bot\)) or
%%     \(\stt{\com[e], \rho} \not\to\) (if \(\bsem{e}\rho = \bot\)),
%%     therefore \(\nexists \com[C'] \mid \stt{\com[e], \rho} \to
%%     \stt{\com[C'], \rho'}\) for some \(\rho\), therefore \[\red{\com[e]}
%%     = \emptyset\]
%%   \item \(\com \equiv \com[C_1 + C_2]\) and \(\red{\com[C_1 + C_2]} =
%%     \{\com[C'] \mid \stt{\com[C_1 + C_2], \rho} \to \stt{\com[C'],
%%       \rho'}\) for some \(\rho\}\). By sum\(_1\) \(\stt{\com[C_1 +
%%         C_2], \rho} \to \stt{\com[C_1], \rho}\) and by sum\(_2\)
%%     \(\stt{\com[C_1 + C_2], \rho} \to \stt{\com[C_2],
%%       \rho}\). Therefore \[\red{\com[C_1 + C_2]} = \{\com[C_1],
%%     \com[C_2]\}\cup \red{\com[C_1]} \cup \red{\com[C_2]}\]
%%   \item \(\com \equiv \com[C_1;C_2]\). In this case by comp\(_1\) if
%%     \(\stt{\com[C_1], \rho}\to \stt{\com[C_1'], \rho'}\) then
%%     \(\stt{\com[C_1;C_2],\rho} \to \stt{\com[C_1';C_2], \rho'}\). By
%%     comp\(_2\) if \(\stt{\com[C_1], \rho}\to \rho'\) then
%%     \(\stt{\com[C_1;C_2],\rho} \to \stt{\com[C_2],
%%       \rho'}\). Therefore \[\red{C_1;C_2} = \{\com[C_2]\}\cup
%%     \red{C_2}\cup\{\com[C_1';C_2] \mid \com[C_1']\in\red{C_1}\}\]
%%   \item \(\com \equiv \com[C^*]\). In this case by star
%%     \(\stt{\com[C^*], \rho} \to \stt{\com[C;C^*], \rho}\) or by
%%     star\(_{\text{fix}}\) \(\stt{\com[C^*], \rho} \to
%%     \rho\). therefore \[\red{C^*} = \{\com[C;C^*]\} \cup
%%     \{\com[C';C^*]\mid \com[C']\in\red{C}\}\]
%%   \end{itemize}
%% \end{proof}

%% \begin{lemma}\label{le:reductionsfinite}
%%   \(\forall \com \in \imp\) \(\red{\com}\) is finite.
%% \end{lemma}

%% \begin{proof}
%%   We work by induction on \(\com\):

%%   \noindent
%%   \paragraph*{Base case:\\}
  
%%   \(\com \equiv \com[e]\). Because of lemma \ref{le:tosets}
%%   \(\red{\com[e]} = \emptyset\) which is finite.

%%   \noindent
%%   \paragraph*{Inductive cases:\\}
%%   \begin{itemize}
%%   \item \(\com \equiv \com[C_1 + C_2]\). By lemma \ref{le:tosets}
%%     \(\red{\com[C_1] + \com[C_2]} = \{\com[C_1], \com[C_2]\} \cup
%%     \red{\com[C_1]} \cup \red{\com[C_2]}\), which is finite as it is
%%     union of finite sets.
%%   \item \(\com \equiv \com[C_1;C_2]\) By lemma \ref{le:tosets}
%%     \(\red{\com[C_1]; \com[C_2]} =
%%     \{\com[C_2]\}\cup\red{C_2}\cup\{\com[C_1'];\com[C_2] \mid
%%     \com[C_1']\in\red{C_1}\}\). Both \(\red{\com[C_1]}\) and
%%     \(\red{\com[C_2]}\) are finite by inductive hypothesis, and
%%     therefore the union of finite sets is finite..
%%   \item \(\com \equiv \com[C^*]\). By lemma \ref{le:tosets}
%%     \(\red{\com[C^*]} = \{\com[C;C^*]\} \cup \{\com[C';C^*] \mid
%%     \com[C']\in\red{\com[C]}\}\). \(\red{\com[C]}\) is finite by
%%     inductive hypothesis, and therefore the union of finite sets is
%%     finite.
%%   \end{itemize}
%% \end{proof}

