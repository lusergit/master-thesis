\section{Computing the interval semantics} 
\label{sec:computingint}

In this section we argue that for the language \(\imp\) the interval
abstract semantics is computable in finite time without widening.

Observe that the exact computation provides, already for our simple
language, a precision which is not obtainable with (basic) widening
and narrowing. In the example below the semantics maps \(\var\) and
\(\var[y]\) to \([0,2]\) and \([6,8]\) resp., while widening/narrowing
to \([0,\infty]\) and \([6,\infty]\)

\begin{verbatim}
x:=0;
y:=0;
while (x<=5) do
   if (y=0) then
      y=y+1;
   endif;
   if (x==0) then
      x:=y+7;
   endif;
done;
end
\end{verbatim}

Of course, for the collecting semantics this is not the case. Already
computing a finite upper bound for loop invariants when they are
finite is impossible as this would allow to decide termination, as we
have seen in section \ref{sec:finiteness}.

The main goal of this chapter is to provide an effective way of
computing interval semantics ensuring termination of the analyzer,
without relying on widening or narrowing operators. The problem we
want to solve is therefore the following:

\begin{problem}[Termination of interval analysis]\label{problem1}
  Given \(\com\in \imp\), \(\eta \in \bA\), decide: \(\semi{\com} \eta
  =^? \top\)
\end{problem}

To do so we present a novel technique, based on the idea of
\emph{bounds}. Each program is associated to a bound, an ideal value
above which for each variable we cannot guarantee convergence, and
therefore we can safely assume that the program diverges.  First,
given a program, we associate each variable with a \emph{single
  bound}, which captures both both an \emph{upper bound} and a
\emph{lower bound} for the variable. The rough idea is that, whenever
a variable is within its bound, the behavior of the program with
respect to that variable becomes stable. % We also introduce an
% \emph{increment bound} which captures the largest increment or
% decrement that can affect a variable.

\begin{definition}[\textbf{Program bound}]
  \label{de:bound}
  The \emph{bound} associated with a command \(\com\in \imp\) is a
  natural number, denoted \(\bound{\com}\in \nat\), defined
  inductively as follows:
  \begin{align*}
    %
    \bound{\var \in S}  
    & \veq \begin{cases}
      \min(S) & \text{if } \max(S)=\infty\\
      \max(S) & \text{if } \max(S)\in \nat
    \end{cases}
    \\
    %
    \bound{\var \in [a,b]}  
    & \veq \begin{cases}
      a & \text{if } b=\infty\\
      b & \text{if } b\in \nat
    \end{cases}
    \\
    % boolean
    %
    \bound{\var \leq k}  
    & \veq k\\
    %
    \bound{\var > k} 
    & \veq k\\
    %
    %
    \bound{\tru} 
    & \veq 0\\
    %
    \bound{\ff} 
    & \veq 0\\
    % \bound{\var > \var[y]}  
    % & \veq 0\\
    %
    %
    % assignements
    %
    \bound{\var := k} 
    & \veq k\\
    %
    \bound{\var := \var[y] + k}
    & \veq k\\
    %
    \bound{\var := \var[y] - k}
    & \veq k\\
    % 
    % \bound{\var := \var[y]} 
    % & \veq 0\\[3mm]
    %
    % commands
    %
    \bound{\com_1 \ndet \com_2}
    & \veq \bound{\com_1} + \bound{\com_2}\\
    %
    \bound{\com_1 \seq \com_2}
    & \veq \bound{\com_1} + \bound{\com_2}\\
    %% %% 
    %% \bound{\kleene{\com}}
    %% & \veq 2 \bound{\com}\\
    %% %% 
    \bound{\kleene{\com}}
    & \veq (|\varsof{\com}|+1) \bound{\com} \\ 
  \end{align*}
  where \(\varsof{\com}\) denotes the set of variables occurring in 
  \(\com\).
\end{definition}

In order to compute the bound of a program we introduce bounded
environments, that we'll later use for lemma~\ref{le:computebounds}

\begin{definition}[\textbf{Bound Environment}]
  \label{de:boundenv}
  A bound environment (benv for short) is a total function \(b:\Var
  \to \nat\). We define \(\benv \veq \{ b \mid b:\Var \to \nat\}\).
  Each command \(\com\in \imp\) induces a benv transformer
  \(\boundt{C}:\benv \to \benv\), which is defined inductively as
  follows:
  \begin{align*}
    %
    % boolean
    %
    \boundt{\var \in S}b  
    & \veq 
    \begin{cases} 
      b[\var[x]\mapsto b(\var[x])+\min(S)]& \text{if }\max(S)=\infty\\
      b[\var[x]\mapsto b(\var[x])+\max(S)]& \text{if }\max(S)\in \nat
    \end{cases}
    \\
    %
    % assignments
    %
    \boundt{\var := k}b 
    & \veq  b[\var[x]\mapsto b(\var[x])+k]\\
    %
    \boundt{\var := \var[y] + k}b
    & \veq  b[\var[x]\mapsto b(\var[x])+b(\var[y])+k]\\
    %
    \boundt{\var := \var[y] - k}b
    & \veq  b[\var[x]\mapsto b(\var[x]) + b(\var[y]) - k]\\
    %
    % commands
    %
    \boundt{\com_1 \ndet \com_2}b
    & \veq \lambda \var[x]. (\boundt{\com_1}b)(\var[x]) + (\boundt{\com_2}b)(\var[x])\\
    %
    \boundt{\com_1 \seq \com_2}b
    & \veq \lambda \var[x]. (\boundt{\com_1}b)(\var[x]) + (\boundt{\com_2}b)(\var[x])\\
    %& \veq \boundt{\com_2}(\boundt{\com_1}b) \\
    % 
    \boundt{\com^*}b
    &\veq  \lambda \var[x]. (|\varsof{\com}|+1)(\boundt{\com}b)(\var[x])
  \end{align*}    
  where \(\varsof{\com}\) denotes the set of variables occurring
  in \(\com\).
\end{definition}

\begin{lemma}\label{le:computebounds}
  For all \(\com \in \imp\), \(\bound{\com}=\sum_{\var\in \vars{\com}}
  (\boundt{\com}b_0)(\var)\), with \(b_0\veq \lambda x.0\).
\end{lemma}
\begin{proof}
  By induction on \(\com\in \imp\).
  \paragraph*{Base cases: \\}
  \noindent
  \begin{itemize}[label={}]
  \item \((\var \in S)\):
    \begin{align*}
      \bound{\var\in S} & = \begin{cases} \min(S) & \text{if } \max(S) = \infty \\ \max(S) & \text{otherwise} \end{cases} & \\
      \boundt{\var \in S}b_0 & = \begin{cases} b_0[\var \mapsto 0 + \min(S)] & \text{if } \max(S) = \infty \\ b_0[\var \mapsto 0 + \max(S)] & \text{if } \max(S) \in \n \end{cases} &
    \end{align*}
    and since \(\var\) is the only variable in \(\varsof{\var \in
      S}\), \(\bound{\var\in S} = \boundt{\var \in S}b_0(\var)\)
  \item \((\var \in \interval{a}{b}), (\var \leq k), (\var > k)\) are
    the same as the latter case;
  \item \((\tru), (\ff)\): notice that \(\varsof{\tru} = \varsof{\ff}
    = \emptyset\);
  \item \((\var := k)\): just notice that
    \(\bound{\var := k} = k = b_0(\var) + k = b_0[\var \mapsto b_0 +
    k] = \boundt{\var := k }b_0\) and \(\var\) is the only variable in
    \(\var := k\).
  \item \((\var := \var + k), (\var := \var - k)\) are analogous to
    the latter case.
  \end{itemize}

  \noindent
  \paragraph*{Inductive cases:\\}
  \begin{itemize}[label={}]
  \item[\((\com_1 \ndet \com_2)\)]
    \begin{align*}
      \bound{\com_1 \ndet \com_2} & = &\\
      \bound{\com_1} + \bound{\com_2} & = & \text{by inductive hypothesis}\\
      \sum_{\var\in \vars{\com_1}} (\boundt{\com}b_0)(\var) + \sum_{\var\in \vars{\com_2}} (\boundt{\com}b_0)(\var) & = &\\
      \sum_{\var\in \vars{\com_1}\cap \vars{\com_2}} (\boundt{\com_1}b_0)(\var)+(\boundt{\com_2}b_0)(\var) & \;+ & \\
      \sum_{\var\in \vars{\com_1}\smallsetminus \vars{\com_2}} (\boundt{\com_1}b_0)(\var) & \;+ & \\
      \sum_{\var\in \vars{\com_2}\smallsetminus \vars{\com_1}} (\boundt{\com_2}b_0)(\var) & = & \\
      \boundt{\com_1 \ndet \com_2}b_0 & &
    \end{align*}
  \item[\((\com_1; \com_2)\)] identical to \((\com_1 \ndet \com_2)\);
  \item[\((\com^*)\)]
    \begin{align*}
      \bound{\com^*} & = &\\
      |\vars{C}+1|\bound{\com} & = & \text{by inductive hypothesis} \\
      |\vars{C}+1|\sum_{\var\in \vars{\com}} (\boundt{\com}b_0)(\var) & = &\\
      \sum_{\var\in \vars{\com}} |\vars{C}+1|(\boundt{\com}b_0)(\var) & = &\\
      %\lambda \var[x]. (\boundt{\com_1}b)(\var[x]) + (\boundt{\com_2}b_0)(\var[x]) =&\\
      \boundt{\fix{\com}}b_0 & &
    \end{align*}
  \end{itemize}
\end{proof}

We next prove an easy graph-theoretic property which will later be
helpful.
% We focus on graphs with edges labelled over integer numbers. 
Consider a finite directed and edge-weighted graph
\(\langle X,\to\rangle\) where
\(\to \mathord{\subseteq} X \times \mathbb{Z} \times X\) and
\(x \to_h x'\) denotes that \((x, h, x') \in \mathbin{\to}\). Consider
a finite path in \(\langle X,\to\rangle\)
\[p= x_0 \to_{h_0} x_1 \to_{h_1} x_2 \to_{h_2} \ldots \to_{h_{\ell-1}}
  x_{\ell}\] where:
\begin{enumerate}[label=(\roman*).]
\item \(\ell\geq 1\)
\item the carrier size of \(p\) is \(s(p) \veq |\{x_0,...,x_\ell\}|\)
\item the weight of \(p\) is \(w(p) \veq \Sigma_{k=0}^{\ell-1} h_k\)
\item the length of \(p\) is \(|p|\veq \ell\)
\item given indices \(0 \leq i < j \leq \ell\), \(p_{i,j}\) denotes
  the subpath of \(p\) given by
  \(x_i \to_{h_i} x_{i+1} \to_{i+1} \cdots \to_{h_{j-1}} x_j\) whose
  length is \(j-i\); \(p_{i,j}\) is a cycle if \(x_i=x_j\).
\end{enumerate}



\begin{lemma}[\textbf{Positive cycles in weighted directed graphs}]
  \label{le:cycles}
  Let \(p\) be a finite path
  \[p = x_0 \to_{h_0} x_1 \to_{h_1} x_2 \to_{h_2} \cdots \to_{h_{\ell-1}} x_{\ell}\]
  with
  \(m \veq \max \{ |h_j| \mid j \in \{0, \ldots, \ell-1\}\}\in \nat\) and 
  \(w(p) > (|X|-1)m\). Then, \(p\) has a subpath which is a cycle having a 
  strictly positive weight.
\end{lemma}

\begin{proof}
  First note that \(w(p) = \Sigma_{k=0}^{\ell-1} h_k > (|X|-1)m\) implies that \(|p|=\ell \geq |X|\). 
  Then, we show our claim by induction on \(|p|=\ell \geq |X|\).


  \noindent
  \((|p|=|X|)\): Since the path \(p\) includes exactly \(|X|+1=\ell+1\) nodes, there exist
  indices \(0\leq i < j\leq \ell\) such that \(x_i=x_j\), i.e., \(p_{i,j}\) is a subpath of \(p\) which is a
  cycle. Moreover, since this cycle \(p_{i,j}\) includes at least one edge, we have that
  \begin{align*}
    w(p_{i,j}) = w(p) - (\Sigma_{k=0}^{i-1} h_k + \Sigma_{k=j}^{\ell-1}
    h_k) &> & \text{as \(w(p) > (|X|-1)m\)}\\
    \left(|X|-1\right)m - (\Sigma_{k=0}^{i-1} h_k + \Sigma_{k=j}^{\ell-1}
    h_k) &\geq & \text{as \(\Sigma_{k=0}^{i-1} h_k + \Sigma_{k=j}^{\ell-1}
        h_k \leq (\ell-1)m\)}\\
    (|X|-1)m - (\ell-1)m &= & \text{[as \(\ell=|X|\)]}  \\  
    (|X|-1)m - (|X|-1)m &= 0
  \end{align*}
  so that \(w(p_{i,j})>0\) holds. 

  \medskip
  \noindent

  \noindent
  \((|p|>|X|)\): Since the path \(p\) includes at least \(|X|+2\) nodes, as in
  the base case, we have that \(p\) has a subpath which is a cycle. Then, we consider a
  cycle \(p_{i,j}\) in \(p\), for some indices \(0\leq i < j\leq \ell\), which is maximal, i.e.,  such that
  if  \(p_{i',j'}\) is a cycle in \(p\),  for some 
  \(0\leq i' < j'\leq \ell\), then \(p_{i,j}\) is not a proper subpath of \(p_{i',j'}\).

  \noindent
  If \(w(p_{i,j})>0\) then we are done. Otherwise we have that  \(w(p_{i,j})\leq 0\) 
  and we consider the path \(p'\) obtained from \(p\) by stripping off the cycle \(p_{i,j}\), i.e.,
  \begin{center}
    \(p' \equiv \overbrace{x_0 \to_{h_0} x_1 \to_{h_1} \cdots \to_{h_{i-1}} x_i}^{p'_0,i} =
    \overbrace{
      x_j\to_{h_{j+1}} \ldots \to_{h_{\ell-1}} x_{\ell}}^{p'_{j+1,\ell}}\)
  \end{center}
  Since \(|p'| < |p|\) and
  \(w(p')=w(p)-w(p_{i,j})\geq w(p)> (|X|-1)m\), we can apply the
  inductive hypothesis on \(p'\).  We therefore derive that \(p'\) has
  a subpath \(q\) which is a cycle having strictly positive
  weight. This cycle \(q\) is either entirely in \(p'_{0,i}\) or in
  \(p'_{j+1,\ell}\), otherwise \(q\) would include the cycle
  \(p_{i,j}\) thus contradicting the maximality of \(p_{i,j}\). Hence,
  \(q\) is a cycle in the original path \(p\) having a strictly
  positive weight.
\end{proof}

\begin{lemma}
  \label{le:inc}
  Let \(\com\in \imp\). % and \(\var[y] \in \Var\)
  
  \noindent
  For all \(\eta \in \mathbb{A}\) and \(\var[y] \in \Var\), if
  \(\max(\semi{\com} \eta \var[y]) \neq \infty\) and
  \(\max(\semi{\com}\eta \var[y]) > \bound{\com}\) then there exist a
  variable \(\var[z] \in \Var\) and an integer \(h \in \mathbb{Z}\)
  such that \(|h| \leq \bound{\com}\) and the following two properties
  hold:
  \begin{enumerate}[label=(\roman*)]
  \item \(\max(\semi{\com}\eta \var[y]) = \max(\eta \var[z]) + h\); \label{point1}
  \item  for all \(\eta' \in \mathbb{A}\), if \(\eta' \sqsupseteq \eta\)
    % and \(\semi{\com}\eta' \neq \top\)
    then
    \(\max(\semi{\com}\eta' \var[y]) \geq \max(\eta' \var[z]) + h\). \label{point2}
  \end{enumerate}
\end{lemma}

\begin{proof}
  The proof is by structural induction on the command \(\com\in \imp\).
  %
  We preliminarly observe that we can safely assume
  \(\eta \neq \bot\).
  % In fact, if \(\eta = \top\) then \(\semi{\com} \eta = \top\), so that the
  % hypothesis \(\semi{\com} \eta \neq \top\) would not
  % be satisfied.
  In fact, if \(\eta = \bot\) then \(\semi{\com} \bot = \bot\) and
  thus \(\max(\semi{\com} \eta \var[y]) = 0 \leq \bound{\com}\),
  against the hypothesis
  \(\max(\semi{\com}\eta \var[y]) > \bound{\com}\). Moreover, when
  quantifying over \(\eta'\) such that \(\eta' \sqsupseteq \eta\) in
  \ref{point1}, if \(\max(\semi{\com}\eta' \var[y]) = \infty\)
  holds, then
  \(\max(\semi{\com}\eta' \var[y]) \geq \max(\eta' \var[z]) + h\)
  trivially holds, hence we will sometimes silently omit to consider
  this case.
  
  \medskip
  
  \noindent
  \textbf{Case} (\(\var \in S\))\\
  % 
  Take \(\eta \in \mathbb{A}\) and assume
  \(\infty\neq\max(\semi{\var \in S}\eta \var[y]) > \bound{\var \in S}\).
  %
  Clearly \(\semi{\var \in S}\eta \neq \bot\), otherwise we would get the contradiction
  \(\max(\semi{\var \in S}\eta \var[y]) =0 \leq \bound{\var \in S}\).
  
  We distinguish two cases:
  \begin{itemize}
    
  \item If \(\var[y] \neq \var\), then for all \(\eta' \in \mathbb{A}\)
    such that \(\eta \sqsubseteq \eta'\) it holds
    \(\bot \neq \semi{\var \in S}\eta' = \eta'[\var \mapsto \eta(\var)
      \sqcap \Int(S)]\) and thus
    \[
    \max(\semi{\var \in S}\eta' \var[y]) = \max(\eta' \var[y]) = \max(\eta' \var[y]) + 0
    \]
    hence the thesis follows with \(\var[z]=\var[y]\) and \(h = 0\).

  \item If \(\var[y] = \var[x]\) then  \(\eta(\var) \in \Int_*\) and
    \[ 
    \max(\semi{\var \in S}\eta \var[y]) = \max (\eta(\var) \sqcap
    \Int(S))
    \]
    Note that it cannot be \(\max(S)\in \nat\). Otherwise, by
    Definition~\ref{de:bound},
    \(\max (\eta(\var) \sqcap \Int(S)) \leq \max(S)= \bound{\var \in
      S}\), violating the assumption \(\max(\semi{\var \in S}\eta \var[y]) > \bound{\var \in S}\).
    Hence, \(\max(S) = \infty\) must hold and therefore
    \(\max (\eta(\var) \sqcap \Int(S)) = \max(\eta(\var))=
    \max(\eta(\var))+0\). It is immediate to check that the same holds
    for all
    \(\eta' \sqsupseteq \eta\), i.e.,
    \[
    \max (\eta'(\var) \sqcap \Int(S)) = \max(\eta'(\var))=
    \max(\eta'(\var))+0
    \]
    and thus the thesis follows with  \(\var[z]=\var[y]=\var[x]\) and \(h=0\).
  \end{itemize}  
  
  % \medskip
  
  % \noindent
  % \textbf{Case} (\(\tru\))
  % A consequence of the fact that \(\tru \equiv x\in \nat\). 

  % \medskip
  
  % \noindent
  % \textbf{Case} (\(\ff\))
  % A consequence of the fact that \(\ff \equiv x\in \varnothing\). 
  % %
  % %  Since \(\semi{\ff}\eta = \bot\) and thus
  % %  \begin{center}
  % %    \(\max(\semi{\ff}\eta \var[y]) = \max(\bot \var[y]) =0 \leq \bound{\ff}\).
  % %  \end{center}
  % %  the thesis vacuosly holds


  % % ----------------------------------------------
  
  \medskip
  
  \noindent
  \textbf{Case} (\(\var := k\))
  % 
  Take \(\eta \in \mathbb{A}\) and assume
  \(\max(\semi{\var := k}\eta \var[y]) > \bound{\var := k} = k\).

  Observe that it cannot be \(\var = \var[y]\). In fact, since
  \(\semi{\var := k}\eta = \eta[\var \mapsto \interval{k}{k}]\), we would have
  \(\semi{\var := k}\eta \var[y] = \interval{k}{k}\) and thus
  \begin{center}
    \(\max(\semi{\var := k}\eta \var[y]) = k  = \bound{\var := k}\).
  \end{center}
  violating the assumption.
  %
  Therefore, it must be \(\var[y] \neq \var\). Now, for all
  \(\eta' \sqsupseteq \eta\), we have
  \(\semi{\var := k}\eta' \var[y] = \eta' \var[y]\) and thus
  \begin{center}
    \(\max(\semi{\var := k}\eta' \var[y]) = \max(\eta' \var[y]) =
    \max(\eta' \var[y]) + 0\),
  \end{center}
  hence the thesis holds with \(h =0 \leq \bound{\var :=k}\) and \(\var[z] = \var[y]\).
  

  % ----------------------------------------------
  
  \medskip
  
  \noindent
  \textbf{Case} (\(\var := \var[w] + k\))
  % 
  Take \(\eta \in \mathbb{A}\) and assume
  \(\max(\semi{\var := \var[w] + k}\eta \var[y]) > \bound{\var := \var[w] + k} = k\).
  %
  Recall that
  \(\semi{\var := \var[w] + k}\eta = \eta[\var \mapsto \eta \var[w] + k]\).
  
  We distinguish two cases:
  \begin{itemize}
    
  \item If \(\var[y] \neq \var\), then for all \(\eta' \sqsupseteq \eta\), we have
    \(\semi{\var := \var[w] + k}\eta' \var[y] = \eta' \var[y]\) and thus
    \begin{center}
      \(\max(\semi{\var := \var[w] + k}\eta' \var[y]) = \max(\eta' \var[y])\).
    \end{center}
    hence the thesis follows with
    \(h =0 \leq \bound{\var := \var[w] + k}\) and \(\var[z] = \var[y]\).
    
  \item 
    If \(\var = \var[y]\) then  for all \(\eta' \sqsupseteq \eta\), we have
    \(\semi{\var := \var[w] + k}\eta' \var[y] = \eta' \var[w] +
    k\) and thus
    \begin{center}
      \(\max(\semi{\var := \var[w] + k}\eta' \var[y]) = \max(\eta' \var[w]) +
      k\).
    \end{center}
    
    Hence, the thesis follows with \(h = k \leq \bound{\var := \var[w] + k}\)
    and \(\var[z] = \var[w]\).
  \end{itemize}
  

  % ----------------------------------------------
  
  \medskip
  
  \noindent
  \textbf{Case} (\(\var := \var[w] - k\))
  % 
  Take \(\eta \in \mathbb{A}\) and assume
  \(\max(\semi{\var := \var[w] - k}\eta \var[y]) > \bound{\var := \var[w] - k} = k\).
  %
  Recall that
  \(\semi{\var := \var[w] - k}\eta = \eta[\var \mapsto \eta \var[w] - k]\).
  
  We distinguish two cases:
  \begin{itemize}
    
  \item If \(\var[y] \neq \var\), then for all
    \(\eta' \in \mathbb{A}\) such that
    \(\eta \sqsubseteq \eta'\), we have
    \(\semi{\var := \var[w] - k}\eta' \var[y] = \eta' \var[y]\) and thus
    \begin{center}
      \(\max(\semi{\var := \var[w] - k}\eta' \var[y]) = \max(\eta' \var[y])\).
    \end{center}
    hence the thesis holds, with
    \(h =0 \leq \bound{\var := \var[w] - k}\) and \(\var[z] = \var[y]\).
    
  \item If \(\var = \var[y]\) then for all \(\eta' \in \mathbb{A}\) such
    that \(\eta \sqsubseteq \eta'\), we have
    \(\semi{\var := \var[w] - k}\eta' \var[y] = \eta' \var[w] - k\) and
    thus
    \begin{center}
      \(\max(\semi{\var := \var[w] - k}\eta' \var[y]) = \max(\eta' \var[w]) -
      k\).
    \end{center}
    Note that the assumption \(\max(\semi{\var := \var[w] - k}\eta \var[y]) > k\) and thus
    \(\max(\semi{\var := \var[w] - k}\eta' \var[y]) > k\) ensures that subtraction is not
    truncated on the maximum.
    
    Hence the thesis holds, with \(h = -k\), hence \(|h| = \bound{\var := \var[w] - k}\),
    and \(\var[z] = \var[w]\).
  \end{itemize}
  
  % ----------------------------------------------
  
  \medskip
  
  \noindent
  \textbf{Case} (\(\com_1 \ndet \com_2\))
  % 
  Take \(\eta \in \mathbb{A}\) and assume
  \(\max(\semi{\com_1 \ndet \com_2}\eta) > \bound{\com_1 \ndet \com_2} =
  \bound{\com_1} + \bound{\com_2}\).

  Recall that
  \(\semi{\com_1 \ndet \com_2} \eta = \semi{\com_1}\eta \sqcup \semi{\com_2}\eta\).
  % 
  Hence, since \(\semi{\com_1 \ndet \com_2} \eta \var[y] \neq \infty\), we have
  that \(\semi{\com_1} \eta \var[y] \neq \infty \neq \semi{\com_2}\eta \var[y]\).

  Moreover
  \begin{align*}
    \max(\semi{\com_1 \ndet \com_2}\eta \var[y])& =  \max(\semi{\com_1}\eta \var[y] \sqcup \semi{\com_2}\eta \var[y]) \\ 
    & = \max \{ \max(\semi{\com_1}\eta \var[y]), \max(\semi{\com_2}\eta \var[y])\}
  \end{align*}

  Thus
  \(\max(\semi{\com_1 \ndet \com_2}\eta \var[y]) =
  \max(\semi{\com_i}\eta \var[y])\) for some \(i \in \{1,2\}\). We can
  assume, without loss of generality, that the maximum is realized
  by the first component, i.e.,
  \(\max(\semi{\com_1 \ndet \com_2}\eta \var[y]) =
  \max(\semi{\com_1}\eta \var[y])\). Hence, by inductive hypothesis on
  \(\com_1\), we have that there exists \(h \in \mathbb{Z}\) with
  \(|h| \leq \bound{\com_1}\) and \(\var[z] \in \Var\) such that
  \(\max(\semi{\com_1}\eta \var[y]) = \max(\eta \var[z]) + h\) and for
  all \(\eta' \in \mathbb{A}\), \(\eta \sqsubseteq \eta'\),
  \[
  \max(\semi{\com_1}\eta' \var[y]) \geq \max(\eta' \var[z]) + h
  \]

  Therefore 
  \[
  \max(\semi{\com_1 \ndet \com_2}\eta \var[y])
  = \max(\semi{\com_1}\eta \var[y]) = \max(\eta \var[z]) + h
  \]
  %
  and and for
  all \(\eta' \in \mathbb{A}\), \(\eta \sqsubseteq \eta'\),
  \begin{align*}
    \max(\semi{\com_1 \ndet \com_2}\eta' \var[y])
    &= \max\{ \max(\semi{\com_1}\eta' \var[y]),  \max(\semi{\com_2}\eta' \var[y])\}\\
    % 
    & \geq \max(\semi{\com_1\eta'} \var[y])\\
    %
    & \geq \max(\eta' \var[z]) + h
  \end{align*}
  with \(|h| \leq \bound{\com_1} \leq \bound{\com_1 \ndet \com_2}\), as desired.


  % ----------------------------------------------
  
  \medskip
  
  \noindent
  \textbf{Case} (\(\com_1 \seq \com_2\))
  % 
  Take \(\eta \in \mathbb{A}\) and assume
  \(\max(\semi{\com_1 \seq \com_2}\eta) > \bound{\com_1 \seq \com_2} =
  \bound{\com_1} + \bound{\com_2}\).
  
  Recall that
  \(\semi{\com_1 \seq \com_2} \eta = \semi{\com_2}(\semi{\com_1}\eta)\).
  %
  If we define
  \begin{center}
    \(\semi{\com_1} \eta = \eta_1\)
  \end{center}
  since \(\max(\com_2 \eta_1 \var[y]) \neq \infty\) and 
  \(\max(\com_2 \eta_1 \var[y]) > \bound{\com_1 \seq \com_2} \geq
  \bound{\com_2}\), by inductive hypothesis on \(\com_2\), there are
  \(|h_2| \leq \bound{\com_2}\) and \(\var[w] \in \Var\) such that
  \(\max(\semi{\com_2} \eta_1 \var[y]) = \max(\eta_1 \var[w]) + h_2\) and
  for all \(\eta_1' \in \mathbb{A}\) with \(\eta_1 \sqsubseteq \eta_1'\)
  \begin{equation}
    \label{eq:seq2}
    \max(\semi{\com_2} \eta_1' \var[y]) \geq \max(\eta_1' \var[w]) + h_2
    \tag{\dag}
  \end{equation}
  
  Now observe that
  \(\max(\semi{\com_1}\eta \var[w]) = \max(\eta_1 \var[w]) >
  \bound{\com_1}\). Otherwise, if it were \(\max(\eta_1 \var[w]) \leq
  \bound{\com_1}\) we would have
  \begin{center}
    \(\max(\semi{\com_2}\eta_1 \var[y]) = \max(\eta_1 \var[w]) + h_2 \leq
    \bound{\com_1} + \bound{\com_2} = \bound{\com_1 \seq \com_2}\),
  \end{center}
  violating the hypotheses. Moreover,
  \(\semi{\com_1}\eta \var[w] \neq \infty\), otherwise we would have
  \(\max(\semi{\com_2} \eta_1 \var[y]) = \max(\eta_1 \var[w]) + h_2 =
  \infty\), contradicting the hypotheses.  Therefore we can apply the
  inductive hypothesis also to \(\com_1\) and deduce that there are
  \(|h_1| \leq \bound{\com_1}\) and \(\var[w]' \in \Var\) such that
  \(\max(\semi{\com_1} \eta \var[w]) = \max(\eta \var[w]') + h_1\) and
  for all \(\eta' \in \mathbb{A}\) with \(\eta \sqsubseteq \eta'\)
  \begin{equation}
    \label{eq:seq1}
    \max(\semi{\com_1} \eta' \var[w]) \geq \max(\eta' \var[w]') + h_1
    \tag{\ddag}
  \end{equation}

  Now, for all \(\eta' \in \mathbb{A}\) with \(\eta \sqsubseteq \eta'\) we have that:
  \begin{align*}
    \max(\semi{\com_1 \seq \com_2}\eta \var[y])
    &= \max(\semi{\com_2}(\semi{\com_1}\eta) \var[y])\\
    %
    & = \max(\semi{\com_2}\eta_1 \var[y])\\
    %
    & = \max(\eta_1 \var[w]) + h_2\\
    %
    & = \max(\semi{\com_1}\eta \var[w]) + h_2\\
    %
    &= \max(\eta \var[w]')+h_1 + h_2
  \end{align*}
  %
  and
  \begin{align*}
    \max(\semi{\com_1 \seq \com_2}\eta' \var[y]) & = & \\ 
    \max(\semi{\com_2}(\semi{\com_1}\eta') \var[w]) & \geq & \\ 
    \max(\semi{\com_1}\eta' \var[w]') + h_2 & \geq & 
    \text{by \eqref{eq:seq2}, since } \eta_1 = \semi{\com_1}\eta \sqsubseteq \semi{\com_1}\eta' \text{ , by monotonicity} \\
    (\max(\eta' \var[y]) +h_1) + h_2 & & \text{by \eqref{eq:seq1}}\
  \end{align*}

  Thus, the thesis holds with \(h= h_1+h_2\), as
  \(|h| =|h_1 +h_2| \leq |h_1| + |h_2| \leq \bound{\com_1} +
  \bound{\com_2} = \bound{\com_1 \seq \com_2}\), as needed.



  % ----------------------------------------------------
  
  \medskip
  
  \noindent
  \textbf{Case} (\(\fix{\com}\)) 
  %
  Let \(\eta \in \mathbb{A}\) such that
  \(\semi{\fix{\com}} \eta \var[y] \neq \infty\). Recall that
  \(\semi{\fix{\com}} \eta = \lfp{\lambda \mu. (\semi{\com}
    \mu \sqcup\eta)}\). Observe that the least fixpoint of
  \(\lambda \mu. (\semi{\com}
  \mu \sqcup\eta)\) coincides with the least
  fixpoint of
  \(\lambda \mu. (\semi{\com} \mu  \sqcup \mu) = \lambda \mu. \semi{ \com\ndet \tru} \mu\) above \(\eta\). Hence, if
  \begin{itemize}
  \item \(\eta_0 \veq \eta\),
  \item for all \(i\in \nat\),
    \(\eta_{i+1} \veq \semi{\com} \eta_i \sqcup \eta_i= \semi{\com \ndet
    \tru} \eta_i \sqsupseteq \eta_i\),
  \end{itemize}
  then we define an increasing chain \(\{\eta_i\}_{i\in \nat}\subseteq \mathbb{A}\) such that
  \[ 
  \semi{\fix{\com}} \eta = \textstyle\bigsqcup_{i \in \nat} \eta_i.
  \]
  Since \(\semi{\fix{\com}} \eta \var[y] \neq \infty\), we have that
  for all \(i\in \nat\), \(\eta_i \var[y] \neq \infty\). Moreover,
  \(\textstyle\bigsqcup_{i \in \nat} \eta_i\) on \(\var[y]\) is
  finitely reached in the chain \(\{\eta_i\}_{i\in \nat}\), i.e.,
  there exists \(m \in \mathbb{N}\) such that for all \(i \geq m+1\)
  \[
  \semi{\fix{\com}} \eta \var[y] = \eta_{i} \var[y].
  \]
  %\(\eta_m \neq \eta_{m+1}\) and 

  The inductive hypothesis holds for \(\com\) and \(\tru\), hence for
  \(\com \ndet \tru\), therefore for all \(\var \in \Var\) and
  \(j \in \{0,1, \ldots, m\}\), if \(\max(\eta_{j+1} \var) >\bound{\com\ndet \tru} =\bound{\com}\) then
  there exist \(\var[z] \in \Var\) and \(h \in \mathbb{Z}\) such that \(|h| \leq \bound{\com}\) and 
  \begin{enumerate}[label=(\alph*)]
  \item \(\infty \neq \max(\eta_{j+1} \var) = \max(\eta_j \var[z]) + h\),
  \item \(\forall \eta' \sqsupseteq \eta_j.
    \max(\semi{\com \ndet \tru}\eta'\var) \geq \max(\eta' \var[z]) + h\).
  \end{enumerate}
  To shortly denote that the two conditions (a) and (b) hold, we write
  \[
  (\var[z],j) \to_{h} (\var, j+1)
  \]
  %   when 
  %  \(j\in  \{0, \ldots, m \}\), \(\var\in \Var\) and 
  %  %\(\max(\eta_{j+1} \var) >\bound{\com}\) hold, so that the existence of
  %   \(h \in \mathbb{Z}\) and \(\var[z] \in \Var\) such that 
  %   \(\max(\eta_{j+1} \var) = \max(\eta_j \var[z]) + h\) follows.
  %   
  %  
  
  Now, assume that for some variable \(\var[y] \in \Var\)
  \[\max(\semi{\fix{\com}} \eta \var[y]) = \max(\eta_{m+1} \var[y]) >
  \bound{\fix{\com}} = (n+1) \bound{\com}\]
  where \(n=|\mathit{vars}(\com)|\). 
  We want to show that the thesis holds, i.e., that there exist
  \(\var[z] \in \Var\) and \(h \in \mathbb{Z}\) with
  \(|h| \leq \bound{\fix{\com}}\) such that:
  \begin{equation}\label{eq:maxendmaxstart}
    \max(\semi{\fix{\com}} \eta \var[y]) = \max(\eta \var[z]) + h
    \tag{i}
  \end{equation}
  and for all \(\eta' \sqsupseteq \eta\),
  \begin{equation}\label{eq:maxmiddlemaxend}
    \max(\semi{\fix{\com}} \eta' \var[y]) \geq \max(\eta' \var[z]) + h
    \tag{ii}
  \end{equation}
  
  Let us consider \eqref{eq:maxendmaxstart}. We first observe that we
  can define a path
  \begin{equation}
    \label{eq:fix}
    \sigma \veq (\var[y]_0, 0) \to_{h_0} (\var[y]_1, 1) \to_{h_1}
    \ldots \to_{h_m} (\var[y]_{m+1}, m+1)
  \end{equation}
  such that \(\var[y]_{m+1}=\var[y]\) and for all \(j \in \{0,\ldots, m+1\}\), 
  \(\var[y]_j\in \Var\)  and
  \(\max(\eta_{j} \var[y]_{j}) > \bound{\com}\).
  %, so that for \(j\neq m+1\), \(|h_j|\leq \bound{\com}\).
  In fact, if, by contradiction, this is not the case, there would
  exist an index \(i \in \{0,\ldots, m\}\) (as
  \(\max(\eta_{m+1} \var[y]_{m+1}) > \bound{\com}\) already holds)
  such that \(\max(\eta_{i} \var[y]_{i}) \leq \bound{\com}\), while
  for all \(j \in \{i+1,\ldots, m+1\}\),
  \(\max(\eta_{j} \var[y]_j) > \bound{\com}\).  Thus, in such a case,
  we consider the nonempty path:
  \[\pi \veq (\var[y]_i, i) \to_{h_i} (\var[y]_{i+1}, i+1) \to_{h_{i+1}} \ldots
  \to_{h_m} (\var[y]_{m+1}, m+1)\]
  and we have that:
  \begin{align*}
    &\Sigma_{j=i}^m h_j =\\ 
    &\Sigma_{j=i}^m \max(\eta_{j+1} \var[y]_{j+1}) - \max(\eta_j \var[y]_{j}) =\\
    &\max(\eta_{m+1} \var[y]_{m+1}) - \max(\eta_i \var[y]_{i}) =\\
    &\max(\eta_{m+1} \var[y]) - \max(\eta_i \var[y]_{i})>\\
    &  (n+1) \bound{\com} - \bound{\com} = n \bound{\com}
  \end{align*}
  with \(|h_j| \leq \bound{\com}\) for \(j \in \{i,\ldots, m\}\). Hence
  we can apply Lemma~\ref{le:cycles} to the projection \(\pi_p\) of the nodes
  of this path \(\pi\) to the variable
  component 
  to deduce that \(\pi_p\) has a subpath which is a cycle with a strictly positive weight. 
  More precisely, there
  exist \(i \leq k_1 < k_2 \leq m+1\) such that
  \(\var[y]_{k_1} = \var[y]_{k_2}\) and
  \(h = \Sigma_{j=k_1}^{k_2-1} h_j > 0\). If we denote 
  \(\var[w] = \var[y]_{k_1} = \var[y]_{k_2}\), then we
  have  that   
  \begin{align*}
    \max(\eta_{k_2} \var[w]) & =  h_{k_2-1}  + \max(\eta_{k_2-1} \var[w]) \\
                             & =  h_{k_2-1} + h_{k_2-2} + \max(\eta_{k_2-2} \var[w])  \\
                             & = \Sigma_{j=k_1}^{k_2-1} h_j + \max(\eta_{k_1} \var[w])  \\
                             & = h +  \max(\eta_{k_1} \var[w])  
  \end{align*}
  Thus,
  \[\max(\semi{\com \ndet \tru}^{k_2-k_1} \eta_{k_1} \var[w]) = \max(\eta_{k_1}
  \var[w]) + h\] 
  Observe that for all \(\eta' \sqsupseteq \eta_{k_1}\)
  \begin{equation}\label{prop2}
    \max(\semi{\com \ndet \tru}^{k_2-k_1} \eta' \var[w]) \geq \max(\eta'
    \var[w]) + h
  \end{equation}
  This property \eqref{prop2} can be shown by induction on \(k_2-k_1 \geq 1\).
  
  \noindent    
  Then, an inductive argument allows us to show that for all \(r \in \mathbb{N}\):
  \begin{equation}\label{nopp}
    \max(\semi{\com \ndet \tru}^{r(k_2-k_1)} \eta_{k_1} \var[w]) \geq \max(\eta_{k_1}
    \var[w]) + r h
  \end{equation}  
  In fact, for \(r=0\) the claim trivially holds. Assuming the validity for \(r\geq 0\) then we have that
  \begin{align*}    
    & \max(\semi{\com \ndet \tru}^{(r+1)(k_2-k_1)} \eta_{k_1} \var[w]) =\\
    &\max(\semi{\com \ndet \tru}^{k_2-k_1} ( \semi{\com \ndet \tru}^{r(k_2-k_1)} \eta_{k_1}) \var[w]) \geq & \mbox{[by \eqref{prop2} as \(\eta_{k_1} \sqsubseteq \semi{\com \ndet \tru}^{r(k_2-k_1)} \eta_{k_1}\) ]}\\
    &\max(\semi{\com \ndet \tru}^{r(k_2-k_1)} \eta_{k_1} \var[w]) + h \geq & \mbox{[by inductive hypothesis}]\\
    &  \max(\eta_{k_1}\var[w])  + rh + h
    \geq 
    \max(\eta_{k_1}\var[w])  + (r+1)h
  \end{align*}

  \noindent
  However,
  This would contradict the hypothesis \(\semi{\fix{\com}} \eta \var[y] \neq \infty\). In fact the inequality \eqref{nopp} would imply
  \begin{align*}
    \semi{\fix{\com}} \eta \var[w]
    & = \bigsqcup_{i \in \mathbb{N}} \semi{\com
      \ndet \tru}^i \eta \var[w] =\\ 
    & =  \bigsqcup_{i \in \mathbb{N}} \semi{\com \ndet
      \tru}^i \eta_{k_1} \var[w]\\ 
    & = \bigsqcup_{r \in \mathbb{N}} \semi{\com \ndet
      \tru}^{r(k_2-k_1)} \eta_{k_1} \var[w]\\
    & = \infty
  \end{align*}


  Now, from \eqref{eq:fix} we deduce that for all \(\eta' \sqsupseteq \eta_{k_1}\), for 
  \(j \in \{ k_1, \ldots, m\}\), if we let \(\mu_{k_1} = \eta'\) and
  \(\mu_{j+1} = \semi{\com \ndet \tru} \mu_j\), we have that \(\max(\mu_{j+1} \var[y]_{j+1}) \geq \max(\mu_{j+1} \var[y]_{j}) + h_j\) and thus 
  \[
  \semi{\com \ndet \tru}^{m-k_1+1} \eta' \var[y] = \mu_{m+1}
  \var[y]_{m+1} \geq 
  \max(\var[y]_{k_1}) + \Sigma_{i=k_1}^m h_i = \max(\eta' \var[w]) + \Sigma_{i=k_1}^m h_i
  \]
  Since \(\eta' = \semi{\fix{\com}} \eta \sqsupseteq \eta_{k_1}\) we conclude
  \begin{align*}
    \semi{\fix{\com}} \eta \var[y]
    & = \semi{\com \ndet \tru}^{m-k_1+1} \semi{\fix{\com}} \eta \var[y]\\
    & \geq \infty + \Sigma_{i=k_1}^m h_i = \infty
  \end{align*}
  contradicting the assumption.
  
  \noindent
  Therefore, the path \(\sigma\) of \eqref{eq:fix} must exist, and consequently
  \[\max(\semi{\fix{\com}} \eta \var[y]) = \max(\eta_{m+1} \var[y]) = \max(\eta  \var[y]_0) + \Sigma_{i=0}^m h_i\]
  and
  \(\Sigma_{i=0}^m h_i \leq \bound{\fix{\com}}=(n+1)\bound{\com}\),
  otherwise we could use the same argument above for inferring the
  contradiction \(p\semi{\fix{\com}} \eta \var[y] = \infty\).

  \medskip

  Let us now show \eqref{eq:maxmiddlemaxend}. Given
  \(\eta' \sqsupseteq \eta\) from \eqref{eq:fix} we deduce that for
  all \(j \in \{ 0, \ldots, m\}\), if we let \(\mu_0 = \eta'\) and
  \(\mu_{j+1} = \semi{\com \ndet \tru} \mu_j\), we have that
  \[
  \max(\mu_{j+1} \var[y]_{j+1}) \geq \max(\mu_{j+1} \var[y]_{j}) + h_j.
  \]
  Therefore, since \(\semi{\fix{\com}} \eta' \sqsupseteq \mu_{m+1}\)
  (observe that the convergence of \(\semi{\fix{\com}} \eta' \) could
  be at an index greater than \(m+1\)), we conclude that:
  \[\max(\semi{\fix{\com}} \eta' \var[y]) \geq \max(\mu_{m+1}
  \var[y]) = \max(\mu_{m+1} \var[y]_{m+1}) \geq \max(\eta' \var[y]_0)
  + \Sigma_{i=0}^m h_i\] as desired.
\end{proof}

Lemma~\ref{le:inc} provides an effective algorithm for computing the
interval semantics of commands. More precisely, given a command
\(\com\), the corresponding finite set of variables
\(\Var_{\com} \veq \mathit{vars}(\com)\), and an interval environment
\(\rho : \Var_{\com} \to \Int\), we define
\[\max(\rho) \veq \max \{ \max(\rho(\var)) \mid \var \in \Var_{\com}
  \}.\]
%
Then, when computing \(\semi{\com} \rho\) on such \(\rho\)
having a finite domain, we can restrict to a bounded interval domain
\(\mathbb{A}_{\com,\rho} \veq (\Var_{\com} \to \Int_{\com,\rho}) \cup
\{ \top, \bot \}\) where
\[
  \Int_{\com,\rho} \veq \{ \interval{a}{b} \mid a, b \in \nat\ \land\
  a \leq b \leq \max(\rho) + \bound{\com}\} \cup
  \{\interval{a}{\infty} \mid a \in \nat \}
\]

% We could also operate uniformly on all commands, defining the
% semantics for \(\com\) in a domain with intervals bounded by
% \(\max(\rho) +\bound{\com}\)

\begin{lemma}
  Let \(\com\in \imp\) be a command. Then, for all finitely supported
  \(\rho : \Var \to \Int\), the abstract semantics
  \(\semi{\com} \rho \)
  % 
  % \semi{\fix{\com}} \rho & = \lfp{\lambda \rho'. (\semi{\com} \rho')
  % \sqcup \rho}
  computed in \(\bA\) and in \(\bA_{\com,\rho}\)
  coincide.
\end{lemma}

\begin{proof}
  What we want to prove is that given a command \(\com\in\imp\) and an
  initial environment \(\rho : \Var \to \Int\) we can build a complete
  lattice with finite ascending chains which contains all the results
  of the interval analysis.  We prove this by carrying a case analysis
  on each variable in \(\Var_{\com}\).  \medskip

  \noindent
  Let \(\var[y] \in \Var_{\com}\), we have two options:
  \begin{itemize}
  \item \(\max(\semi{\com}\rho\var[y]) = \infty\), then the result of
    the analysis \(\semi{\com}\rho\var[y] = \interval{a}{\infty}\)
    with \(a\in\n\), which is in \(\Int_{\com, \rho}\) by definition.
  \item \(\max(\semi{\com}\rho\var[y]) \neq \infty\). In this case we
    have again 2 possibilities:
    \begin{itemize}
    \item \(\max(\semi{\com}\rho\var[y]) \leq \bound{\com}\) and
      therefore by definition of \(\Int_{\com,\rho}\)
      \(\semi{\com}\rho\var[y] \in \Int_{\com, \rho}\).
      
    \item \(\max(\semi{\com}\rho\var[y]) > \bound{\com}\). In this
      case we can apply Lemma~\ref{le:inc} and notice that
      \begin{equation*}
        \exists \var[z] \in \Var_{\com}, h \in \z \mid \left|h\right| \leq \bound{\com} \wedge \max(\semi{\com}\rho\var[y]) = \max(\rho\var[z]) + h
      \end{equation*}
      and we can conclude by noticing that
      \(\max(\rho\var[z]) + h\leq \max(\rho) + \bound{\com}\), which
      means that \(\semi{\com}\rho\var[y] \in \Int_{\com, \rho}\).
    \end{itemize}
  \end{itemize}

  Since for every given variable \(\var[y]\) the analysis
  \(\semi{\com}\rho\var[y]\) is in \(\Int_{\com,\rho}\) then
  \[\semi{\com}\rho \in \bA_{\com,\rho}\]
\end{proof}