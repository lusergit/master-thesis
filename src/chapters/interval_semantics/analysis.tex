\section{Interval Analysis}

We define \emph{interval analysis} of the above language \(\imp\) in a
standard way, taking the best correct approximations (bca) for the
basic expressions in \(\expr\).

\begin{definition}[Integer intervals]
  We call
  \[ \Int \defin \{ \interval{a}{b} \mid a \in \z \cup
    \{-\infty\} \wedge b \in \z \cup \{+\infty\} \wedge a\leq b \} \cup
    \{\abstract{\bot}\} \] set of integer intervals.
\end{definition}

\begin{definition}[Concretization map]
  We define the \emph{concretization map} \(\gamma:\Int \to
  \poset{\z}\) as
  \begin{align*}
    \gamma(\interval{a}{b}) & \defin \{x\in \z \mid a \leq x \leq b\} \\
    \gamma(\bot) & \defin \emptyset
  \end{align*}
\end{definition}

\noindent
Observe that \(\tuple{\Int,\sqsubseteq}\) is a complete lattice where
for all \(I,J\in \Int\), \(I\sqsubseteq J\) iff
\(\gamma(I) \subseteq \gamma(J)\).


\begin{definition}[Abstract integer domain]
  Let \(\Int_* \defin \Int \setminus \{\abstract{\bot}\}\). The
  abstract domain \(\bA\) for program analysis is the variable-wise
  lifting of \(\Int\): \[ \bA \defin (\Var \to \Int_*) \cup \{
    \abstract{\bot} \} \]
\end{definition}

where the intervals for a given variable are always nonempty, while
\(\abstract{\bot}\) represents the empty set of environments.  Thus,
the corresponding concretization is defined as follows:

\begin{definition}[Interval concretization]
  We define the \emph{concretization map} for the abstract domain
  \(\bA\) \(\concr[\Int] : \bA \to \poset{\env}\) as
  \begin{align*}
    \concr[\Int](\bot) & \defin \emptyset \\
    \forall \eta\neq \bot \quad \concr[\Int](\eta) & \defin \{\rho \in \env \mid \forall x\in \Var \: \rho(x) \in \gamma(\eta(x))\} 
  \end{align*}
\end{definition}

\begin{observation}
  If we consider the ordering \(\sqsubseteq\) on \(\bA\)
  s.t. \[\forall \eta, \theta \in \bA \quad \eta \sqsubseteq \theta
    \iff \concr[\Int](\eta) \subseteq \concr[\Int](\theta)\] then
  \(\tuple{\bA, \sqsubseteq}\) is a complete lattice.
\end{observation}

\begin{definition}[Interval abstraction]
  We define the \emph{abstraction map} of some numerical set \(X
  \subseteq \z\) into the abstract domain \(\bA\): \(\abstr[\Int] :
  \poset{\z} \to \bA\) as \[\abstr[\Int](X) \defin \begin{cases}
    \abstract{\bot} & \text{if } X = \emptyset \\ \interval{\min
    X}{\max X} & \text{otherwise} \end{cases}\]
\end{definition}

Observe that since we have both a concretization map \(\concr[\Int]\)
and an abstraction map \(\abstr[\Int]\) we have built the Galois
Connection \[\tuple{\concr[\Int], \dom, \bA, \abstr[\Int]}\] between
the concrete domain \(\dom\) and the abstract domain \(\bA\),
resulting

\begin{definition}[Abstract operations]
  We define sound abstract operations in the \(\bA\) domain:
  \begin{align*}
    \interval{a}{b} \; \abstract{\cup} \; \interval{c}{d} & \defin \interval{\min(a,c)}{\max(b,d)} \\
    \interval{a}{b} \; \abstract{\cap} \; \interval{c}{d} & \defin \begin{cases} \interval{\max(a,c)}{\min(b,d)} & \text{if } \min < \max \\
      \abstract{\bot} & \text{otherwise} \end{cases}
  \end{align*}
  And sound abstract arithmetical operations:
  \begin{align*}
    \abstract{-}\; \interval{a}{b} & \defin \interval{-b}{-a} \\
    \interval{a}{b} \abstract{+} \interval{c}{d} & \defin \interval{a+c}{b+d} \\
    \interval{a}{b} \abstract{-} \interval{c}{d} & \defin \interval{a-c}{b-d} \\
    \interval{a}{b} \abstract{\times} \interval{c}{d} & \defin \interval{\min(ac,ad,bc,bd)}{\max(ac,ad,bc,bd)} \\
  \end{align*}
\end{definition}

\begin{definition}[Interval sharpening]
  \label{de:trunc}
  For a nonempty interval \(\interval{a}{b} \in \Int\) and \(c \in \Z\), we define
  two operations raising \(\uparrow\) the lower bound to \(c\) and lowering \(\downarrow\) the upper
  bound to \(c\), respectively:
  \begin{align*}
    &\truncL{\interval{a}{b}}{c} \veq 
      \begin{cases} 
        \interval{\max\{a,c\}}{b} & \text{if } c\leq b\\
        \bot & \text{if } c > b
      \end{cases}
    \\
    &\truncR{\interval{a}{b}}{c} \veq 
      \begin{cases}   
        \interval{a}{\min\{b,c\}} & \text{if } c\geq a\\
        \bot & \text{if } c < a 
      \end{cases} 
  \end{align*}
  Observe that \(\max(\truncR{\interval{a}{b}}{c})\leq c\) always holds. \qed   
  % \end{quote}
\end{definition}

\begin{definition}[Interval addition and subtraction]
  \label{de:add}
  For a nonempty interval \(\interval{a}{b} \in \Int\) and \(c \in \Z\) define
  \(\interval{a}{b} \pm c \veq \interval{a\pm c}{b\pm c}\) (recall that \(\pm \infty + c = \pm\infty - c = \pm\infty\)).  
  \qed
\end{definition}


Observe that for every interval \(\interval{a}{b} \in \Int\) and
\(c \in \Z\)
\begin{center}
  \(\max(\truncL{\interval{a}{b}}{c}) \leq b\)
  \qquad and \qquad
  \(\max(\truncR{\interval{a}{b}}{c}) \leq c\)
\end{center}
that trivially holds by 
defining \(\max (\bot)  \veq 0\) (i.e., \(0\) is the maximum of
an empty interval).
% Just observe that the resulting interval could be empty, i.e.\ \(\bot\), and .

The \emph{interval semantics} of \(\imp\) is defined as the strict
(i.e., preserving \(\bot\))
% and co-strict (i.e., preserving \(\top\))
extension of the following function \(\semi{\cdot}: \exp \cup \imp \to
\bA \to \bA\). For all \(\eta: \Var \to \Int_*\),

\begin{align*}
  % 
  \semi{\var \in S}\eta 
  & \veq  
    \begin{cases}
      \eta[\var \mapsto \eta(\var)\sqcap \abstr[\Int](S)] & \text{if
                                                            }\eta(\var)\sqcap \abstr[\Int](S)\neq \bot \\ \bot &
                                                                                                                 \text{otherwise}
    \end{cases}\\
    % 
  \semi{\var \in [a,b]}\eta 
  & \veq
    \begin{cases}
      \eta[\var \mapsto \eta(\var)\sqcap [a,b]]  & \text{if }\eta(\var)\sqcap [a,b]\neq \bot \\
      \bot & \text{otherwise}
    \end{cases}\\
    % boolean
    % 
  \semi{\var \leq k}\eta 
  & \veq 
    \begin{cases}
      \eta[\var \mapsto \truncR{\eta(\var)}{k}] & \text{if }\truncR{\eta(\var)}{k}\neq \bot \\
      \bot & \text{otherwise}
    \end{cases}\\
    % 
  \semi{\var > k}\eta 
  & \veq 
    \begin{cases}
      \eta[\var \mapsto \truncL{\eta(\var)}{k+1}] & \text{if }\truncR{\eta(\var)}{k}\neq \bot \\
      \bot & \text{otherwise}
    \end{cases}\\
    % 
  \semi{\tru}\eta 
  & \veq \eta\\
  % 
  \semi{\ff}\eta 
  & \veq \bot\\
  % 
  % \semi{\id}\eta & \veq \eta\\
  % 
  \semi{\var := k}\eta 
  & \veq \eta[\var \mapsto \interval{k}{k}]\\
  % 
  \semi{\var := \var[y] + k}\eta 
  & \veq \eta[\var \mapsto \eta (\var[y]) + k]\\
  % 
  \semi{\var := \var - k}\eta 
  & \veq \eta[\var \mapsto \eta (\var[y]) - k]\\
  % \semi{\var := \var[y]}\eta 
  % & = \eta[\var \mapsto \eta(\var[y])]\\[3mm]
  %
  % commands
  %
  \semi{\com_1 \ndet \com_2} \eta
  & \veq \semi{\com_1} \eta \sqcup \semi{\com_2} \eta\\
  %
  \semi{\com_1 \seq \com_2} \eta
  & \veq \semi{\com_2} (\semi{\com_1} \eta)\\
  %
  % 
  \semi{\kleene{\com}} \eta
  & \veq \textstyle \bigsqcup_{i \in \nat} \semi{\com}^i (\eta)\\
  %
  \semi{\fix{\com}} \eta
  & \veq  \lfp{\lambda \mu. (\eta \sqcup \semi{\com} \mu)}
\end{align*}

The semantics is well-defined, because of the following lemma:

\begin{lemma}\label{le:monotone}
  for all \(\com \in \imp\), \[\semi{\com}: \bA \to \bA\] is
  monotone.
\end{lemma}

\begin{proof}
  What we have to proof is that given \(\eta, \theta \in \bA\), with
  \(\eta \sqsubseteq \theta\) then \(\forall \com \in \imp\)
  \(\semi{\com}\eta \sqsubseteq \semi{\com}\theta\). We will work by
  induction on the grammar of \(\com\):
  
  \noindent
  \textbf{Base cases:}\ \\ \ We avoid cases where \(\eta = \bot\) and
  \(\semi{\com}\eta = \bot\) as \(\forall \theta \in \bA\) \(\bot
  \sqsubseteq \theta\) and it becomes trivially true.
  \begin{itemize}
  \item \(\com \equiv \var \in S\). Then
    \begin{align*}
      \semi{\var \in S}\eta & = \eta[x\mapsto\eta(x) \sqcap \Int(S)] \\
      \semi{\var \in S}\theta & = \theta[x\mapsto \theta(x) \sqcap \Int(S)]
    \end{align*}
    Since \(\eta(x) \sqcap \Int(S) \neq \bot\) and \(\eta \sqsubseteq
    \theta\), then \(\theta(x) \sqcap \Int(S) \neq \bot\). We can see
    that
    \begin{align*}
      \eta \sqsubseteq \theta \iff & \concr(\eta) \subseteq \concr(\theta) \\
      \iff & \{x\in\z \mid x \in \eta(\var)\} \subseteq \{x\in\z \mid x \in \theta(\var)\} \\
      \iff & \{x\in\z \mid x \in \eta(\var)\} \cap \{x \in \z \mid x \in \Int(S)\} \subseteq \{x\in\z \mid x \in \theta(\var)\} \cap \{x \in \z \mid x \in \Int(S)\} \\
      \iff & \{x \in \z \mid x \in \eta(\var) \wedge x \in \Int(S)\} \subseteq \{x \in \z \mid x \in \theta(\var) \wedge x \in \Int(S)\} \\
      \iff & \{x \in \z \mid x \in \eta(\var) \sqcap \Int(S)\} \subseteq \{x \in \z \mid x \in \theta(\var) \sqcap \Int(S)\} \\
      \iff & \concr[\Int](\eta[x\mapsto \eta(\var) \sqcap \Int(S)](\var)) \subseteq \concr[\Int](\theta[x\mapsto \theta(\var) \sqcap \Int(S)](\var)) \\
      \iff & \semi{\var \in S}\eta \sqsubseteq \semi{\var \in S}\theta
    \end{align*}
  \item for the base cases \(\var \in \interval{a}{b}, \var \leq k,
    \var > k\) we can use the same proceedings;
  \item \(\com \equiv \tru\). Then \(\semi{\tru}\eta = \eta
    \sqsubseteq \theta = \semi{\tru}\theta\);
  \item \(\com \equiv \ff\). Then \(\semi{\ff}\eta = \bot \sqsubseteq
    \bot = \semi{\ff}\theta\);
  \item \(\com \equiv \var := k\). Then
    \begin{align}\label{eq:implic}
      \begin{split}
        \eta \sqsubseteq \theta \iff & \concr[\Int](\eta) \subseteq \concr[\Int](\theta) \\
        \iff & \{\rho\in\env \mid \forall \var \in \Var \rho(\var) \in \concr(\eta(\var))\} \subseteq \{\rho\in\env \mid \forall \var \in \Var \rho(\var) \in \concr(\theta(\var))\} \\
        \iff & \forall \var \in \Var, \rho \in \env \quad \rho(\var) \in \concr(\eta(\var)) \Rightarrow \rho(\var) \in \concr(\theta(\var)) \\
      \end{split}
    \end{align}
    Notice that \[\semi{\var := k}\eta =
    \eta[\var\mapsto\interval{k}{k}]\] \[\semi{\var := k}\theta =
    \theta[\var\mapsto\interval{k}{k}]\] because of equation
    \ref{eq:implic} in this case we know that \(\forall \var[y]\in\Var
    ,\; \var[y] \neq \var \; \rho(\var[y]) \in \concr(\eta(\var[y]))
    \Rightarrow \rho(\var[y]) \in \concr(\theta(\var[y]))\). For
    \(\var\) it holds that \(\rho(\var) \in \concr(\interval{k}{k})
    \Rightarrow \rho(\var) \in \concr(\interval{k}{k})\) and therefore
    \begin{align*}
      \forall \var[y] \in \Var, \rho \in \env \quad & \rho(\var[y]) \in \concr(\eta[\var\mapsto\interval{k}{k}](\var[y])) \Rightarrow \rho(\var[y]) \in \concr(\theta[\var\mapsto\interval{k}{k}](\var[y])) \\
      \iff & \concr[\Int](\semi{\var:=k}\eta) \subseteq \concr[\Int](\semi{\var:=k}\theta) \\
      \iff & \semi{\var :=k}\eta \sqsubseteq \semi{\var := k}\theta
    \end{align*}
  \item For \(\com \equiv \var := \var[y] + k, \var := \var[y] - k\)
    the procedure is the same.
  \end{itemize}

  \noindent
  \textbf{Recursive cases:}\ \\ \
  \begin{itemize}
  \item \(\com \equiv \com_1 + \com_2\). Then
    \begin{align*}
      \semi{\com_1 + \com_2}\eta & = \semi{\com_1}\eta \sqcup \semi{\com_2}\eta & \\
      & \sqsubseteq \semi{\com_1}\theta \sqcup \semi{\com_2}\theta & \text{by inductive hp.} \\
      & = \semi{\com_1 + \com_2}\theta
    \end{align*}
  \item \(\com \equiv \com_1 ; \com_2\). Then
    \begin{align*}
      \semi{\com_1;\com_2}\eta = \; & \semi{\com_2}(\semi{\com_1}\eta) & \\
      & \alpha = \semi{\com_1}\eta \sqsubseteq \semi{\com_1}\theta = \beta & \text{by inductive hp.} \\
      & \semi{\com_2}\alpha \sqsubseteq \semi{\com_2}\beta & \text{by inductive hp.} \\
      & \semi{\com_2}(\semi{\com_1}\eta) \sqsubseteq \semi{\com_2}(\semi{\com_1}\theta) & \text{by substitution}
    \end{align*}
  \item \(\com^*\). Then by inductive hypothesis \(\forall i \in \n
    . \semi{\com}^i\eta \sqsubseteq \semi{\com}^i\theta\), which means
    \[\semi{\com^*}\eta = \bigsqcup_{i\in\n}\semi{\com}^i\eta \sqsubseteq
    \bigsqcup_{i\in\n}\semi{\com}^i\theta = \semi{\com^*}\theta.\]
  \end{itemize}
\end{proof}

\begin{example} \label{ex:fix}
  This is the case, for instance, the following program \(P\)
  represents the difference between the Kleene Star and the Fix
  operator:
\begin{verbatim}
  while x < 8 do
  if x = 2 then x := x+6;
  x := x-3
  if x <= 0 then x:=0
\end{verbatim}
starting with the finite interval \(\interval{3}{4}\) we get the
following loop invariants:

\begin{align*}
  \text{Kleene: } &\sqcup\{[3,4], [0,1], [0,0], [0,0], \ldots\} = [0,4]\\[5pt]
  \text{Fix: } & \sqcup\{\bot, [3,4], [0,4], [0,5], [0,5],\ldots\} = [0,5]
\end{align*}

\noindent
Both invariants are correct, because they over-approximate the most
precise concrete invariant \(\{0,1,3,4\}\), however the Kleene
invariant is strictly more precise than the Fix one.
\end{example}

\begin{lemma}[\(\fix{\com}\) \textbf{is syntactic sugar}]\label{lemma-ss}
  For all \(\eta\),
  \(\semi{\fix{\com}} \eta = \semi{\kleene{(\tru + \com)}} \eta\).
\end{lemma}

\begin{proof}
  Let us first show by induction that 
  \begin{equation}\label{prop2}
    \forall i\geq 0.\: (\eta \sqcup \tru \sqcup \semi{\com})^{i+1} \bot = (\tru \sqcup \semi{\com})^{i} \eta \tag{\(\sharp\)}
  \end{equation}

  \noindent
  \(i=0\): \( (\eta \sqcup \tru \sqcup \semi{\com})^{1} \bot = \eta \sqcup \bot \sqcup \semi{\com}\bot = \eta = 
  (\tru \sqcup \semi{\com})^{0} \eta\).
  
  %\medskip
  \noindent
  \(i+1\):  
  \begin{align*}
    (\tru \sqcup \semi{\com})^{i+1} \eta & = & \\
    (\tru \sqcup \semi{\com})((\tru \sqcup \semi{\com})^{i} \eta) & = & \\
    ((\tru \sqcup \semi{\com})^{i} \eta) \sqcup  \semi{\com}((\tru \sqcup \semi{\com})^{i} \eta) & = & \text{By induction}\\
    (\eta \sqcup \tru \sqcup \semi{\com})^{i+1} \bot \sqcup \semi{\com}((\eta \sqcup \tru \sqcup \semi{\com})^{i+1}\bot ) &= & \text{As } \eta \sqsubseteq (\eta \sqcup \tru \sqcup \semi{\com})^{i+1} \bot \\
    \eta \sqcup (\eta \sqcup \tru \sqcup \semi{\com})^{i+1} \bot \sqcup \semi{\com}((\eta \sqcup \tru \sqcup \semi{\com})^{i+1}\bot ) & = & \\
    (\eta \sqcup \tru \sqcup \semi{\com}) ((\eta \sqcup \tru \sqcup \semi{\com})^{i+1} \bot) & = & \\
    (\eta \sqcup \tru \sqcup \semi{\com})^{i+2} \bot & &
  \end{align*}

  Let us also show that:
  \begin{equation}\label{prop3}
    \lfp{\lambda \mu. (\eta \sqcup \semi{\com} \mu)} =
    \lfp{\lambda \mu. (\eta \sqcup \mu \sqcup \semi{\com} \mu)}\tag{\(\diamond\)}
  \end{equation}
  Observe that \(\lfp{\lambda \mu. (\eta \sqcup \semi{\com} \mu)} = \eta \sqcup  \semi{\com}(\lfp{\lambda \mu. (\eta \sqcup \semi{\com} \mu)})\), so that we have that:
  \[
  \eta \sqcup  \lfp{\lambda \mu. (\eta \sqcup \semi{\com} \mu)} \sqcup \semi{\com}(\lfp{\lambda \mu. (\eta \sqcup \semi{\com} \mu)})
  \sqsubseteq \lfp{\lambda \mu. (\eta \sqcup \semi{\com} \mu)}
  \]
  As a consequence, \(\lfp{\lambda \mu. (\eta \sqcup \mu \sqcup \semi{\com} \mu)}\sqsubseteq \lfp{\lambda \mu. (\eta \sqcup \semi{\com} \mu)}\) holds. The reverse inequality follows because, for all \(\mu\), 
  \(\eta \sqcup \semi{\com} \mu \sqsubseteq \eta \sqcup \mu \sqcup \semi{\com} \mu\).

  Then, we have that:
  \begin{align*}
    \semi{\fix{\com}} \eta & = \\
    \lfp{\lambda \mu. (\eta \sqcup \semi{\com} \mu)} & = &  \text{By \eqref{prop3}}\\
    \lfp{\lambda \mu. (\eta \sqcup \mu \sqcup \semi{\com} \mu)} & = & \text{By Knaster-Tarski Theorem} \\
    \bigsqcup_{i \in \nat} (\eta \sqcup \tru \sqcup \semi{\com})^i \bot & = \\
    \bot \sqcup \bigsqcup_{i \in \nat} (\eta \sqcup \tru \sqcup \semi{\com})^{i+1} \bot & = & \text{By \eqref{prop2}}\\
    \bigsqcup_{i \in \nat} (\tru \sqcup \semi{\com})^{i} \eta & = \\
    \semi{\kleene{(\tru + \com)}} \eta. &
  \end{align*}  
\end{proof}


\begin{theorem}[\textbf{Correctness}]
  For all \(\com \in \imp\) and \(\eta\in \bA\),
  \(\sem{\com}\concr(\eta) \subseteq \concr(\semi{\com}\eta)\) holds.
\end{theorem}

\begin{proof}
  by induction on \(\com\in\imp\):
  
  \noindent
  \paragraph*{Base cases:\\}
  \begin{itemize}
  \item \(\com \equiv \var \in S\):
    \begin{itemize}
    \item \(\sem{\var\in S}\concr[\Int](\eta) = \{\rho \in \env \mid
      \forall y \in \Var \; \rho(y) \in \concr(\eta(y))\} \cap \{\rho
      \in \env \mid \rho(\var) \in S\}\)
    \item \(\concr[\Int](\semi{\var\in S}\eta) = \{\rho \in \env \mid
      \forall y \in \Var \; \rho(y) \in \concr(\eta(y))\} \cap \{\rho
      \in \env \mid \rho(\var) \in \Int(S)\}\)
    \end{itemize}
    \(S\) is just decidable, not directly an interval, therefore in
    general \(S \subseteq \Int(S)\), and therefore \[\sem{\var \in
      S}\concr[\Int](\eta) \subseteq \concr[\Int](\semi{\var\in
      S}\eta);\]
  \item \(\com \equiv \var \in \interval{a}{b}, \var \leq k, \var >
    k\): is the same as the latter case;
  \item \(\com \equiv \tru\): \(\sem{\tru}\concr[\Int](\eta) =
    \concr[\Int](\eta)\), \(\concr[\Int](\semi{\tru}\eta) =
    \concr[\Int](\eta)\), and since \(\concr[\Int](\eta) \subseteq
    \concr[\Int](\eta)\) \[\sem{\tru}\concr[\Int](\eta) \subseteq
    \concr[\Int](\semi{\tru}\eta);\]
  \item \(\com \equiv \ff\): \(\sem{\ff}\concr[\Int](\eta) =
    \emptyset\), \(\concr[\Int](\semi{\ff}\eta) = \emptyset\) and
    therefore \[\sem{\ff}\concr[\Int](\eta) \subseteq
    \concr[\Int](\semi{\ff}\eta);\]
  \item \(\com \equiv \var := k\) therefore \(\sem{\var := k}
    \concr[\Int](\eta) = \{\rho \in \env \mid \forall y\in \Var . y
    \neq \var \Rightarrow \rho(y) \in \concr(\eta(y)), \rho(\var) \in
    \concr(\eta(\var) + k)\} = \concr[\Int](\semi{\var := k}\eta)\)
    therefore \[\sem{\var := k} \concr[\Int](\eta) \subseteq
    \concr[\Int](\semi{\var := k}\eta);\]
  \item \(\com \equiv \var := \var[y] + k, \var := \var[y] - k\) is
    the same as the latter case.
  \end{itemize}  
  \noindent
  \paragraph*{Inductive cases:\\}
  \begin{itemize}
  \item \(\com \equiv \com_1 + \com_2\), therefore \[\sem{\com_1 +
        \com_2}\concr[\Int](\eta) = \sem{\com_1}\concr[\Int](\eta) \cup
      \sem{\com_2}\concr[\Int](\eta)\] and \[\concr[\Int](\semi{\com_1 +
        \com_2}\eta) = \concr[\Int](\semi{\com_1}\eta \sqcup
      \semi{\com_2}\eta) = \concr[\Int](\semi{\com_1}\eta) \cup
      \concr[\Int](\semi{\com_2}\eta).\] By inductive hypothesis both
    \(\sem{\com_1}\concr[\Int](\eta) \subseteq
    \concr[\Int](\semi{\com_1}\eta)\) and
    \(\sem{\com_2}\concr[\Int](\eta) \subseteq
    \concr[\Int](\semi{\com_2}\eta)\), therefore \[\sem{\com_1 +
        \com_2}\concr[\Int](\eta) \subseteq \concr[\Int](\semi{\com_1 +
        \com_2}\eta);\]
  \item \(\com \equiv \com_1 ; \com_2\), therefore
    \(\sem{\com_1;\com_2}\concr[\Int](\eta) =
    \sem{\com_2}(\sem{\com_1}\concr[\Int](\eta))\), while
    \[\concr[\Int](\semi{\com_1; \com_2}\eta) =
      \concr[\Int] (\semi{\com_2}( \semi{\com_1}\eta)).\]
  \item \(\com \equiv \com^*\), therefore
    \(\sem{\com^*}\concr[\Int](\eta) =
    \bigcup_{i\in\n}\sem{\com}^i\concr[\Int](\eta)\), while
    \(\concr[\Int](\semi{\com^*}\eta) =
    \concr[\Int](\bigsqcup_{i\in\n}\semi{\com^*}\eta)\)
  \end{itemize}
\end{proof}

\begin{remark}
  Let us remark that in case we were interested in studying
  termination of the abstract interpreter, we could assume that the
  input of a program will always be a finite interval in such a way
  that non termination can be identified with the impossibility of
  converging to a finite interval for some variable. In fact, starting
  from an environment \(\eta\) which maps each variable to a finite
  interval, \(\semi{\com}\eta\) might be infinite on some variable
  when \(\com\) includes a either Kleene or fix iteration which does
  not converge in finitely many steps.
\end{remark}