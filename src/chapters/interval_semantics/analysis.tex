\section{Interval Analysis}

We define \emph{interval analysis} of the above language \(\imp\) in a
standard way, taking the best correct approximations (bca) for the
basic expressions in \(\expr\).
  
\begin{definition}[Integer intervals]
  We call
  \[ \Int \defin \{ \interval{a}{b} \mid a \in \z \cup
  \{-\infty\} \wedge b \in \z \cup \{+\infty\} \wedge a\leq b \} \cup
  \{\bot\} \] set of integer intervals.
\end{definition}

\begin{definition}[Concretization map]
  We define the \emph{concretization map} \(\gamma:\Int \to
  \poset{\z}\) as
  \begin{align*}
    \gamma(\interval{a}{b}) & \defin \{x\in \z \mid a \leq x \leq b\} \\
    \gamma(\bot) & \defin \emptyset
  \end{align*}
\end{definition}

\begin{observation}
  \(\tuple{\Int,\sqsubseteq}\) is a complete lattice where for all
  \(I,J\in \Int\), \(I\sqsubseteq J\) iff \(\gamma(I) \subseteq
  \gamma(J)\).
\end{observation}

\begin{definition}[Abstract integer domain]
  Let \(\Int_* \defin \Int \setminus \{\bot\}\). The abstract
  domain \(\bA\) for program analysis is the variable-wise
  lifting of \(\Int\): \[ \bA \defin (\Var \to \Int_*) \cup \{
  \bot \} \]
\end{definition}

where the intervals for a given variable are always nonempty, while
\(\bot\) represents the empty set of environments.  Thus, the
corresponding concretization is defined as follows:

\begin{definition}[Interval concretization]
  We define the \emph{concretization map} for the abstract domain
  \(\bA\) \(\concr[\Int] : \bA \to \poset{\env}\) as
  \begin{align*}
    \concr[\Int](\bot) &\defin \emptyset\\
    \forall \eta\neq \bot \quad \concr[\Int](\eta) &\defin \{\rho \in \env \mid \forall x\in \Var \: \rho(x) \in \gamma(\eta(x))\} 
  \end{align*}
\end{definition}

\begin{observation}
  If we consider the ordering \(\sqsubseteq\) on \(\bA\)
  s.t. \[\forall \eta, \theta \in \bA \quad \eta \sqsubseteq \theta
  \iff \concr[\Int](\eta) \subseteq \concr[\Int](\theta)\] then
  \(\tuple{\bA, \sqsubseteq}\) is a complete lattice.
\end{observation}

\begin{definition}[Interval sharpening]
  \label{de:trunc}
  For a nonempty interval \(\interval{a}{b} \in \Int\) and \(c \in \Z\), we define
  two operations raising \(\uparrow\) the lower bound to \(c\) and lowering \(\downarrow\) the upper
  bound to \(c\), respectively:
  \begin{align*}
    &\truncL{\interval{a}{b}}{c} \veq 
    \begin{cases} 
      \interval{\max\{a,c\}}{b} & \text{if } c\leq b\\
      \bot & \text{if } c > b
    \end{cases}
    \\
    &\truncR{\interval{a}{b}}{c} \veq 
    \begin{cases}   
      \interval{a}{\min\{b,c\}} & \text{if } c\geq a\\
      \bot & \text{if } c < a 
    \end{cases} 
  \end{align*}
  Observe that \(\max(\truncR{\interval{a}{b}}{c})\leq c\) always holds. \qed   
  %\end{quote}
\end{definition}

\begin{definition}[Interval addition and subtraction]
  \label{de:add}
  For a nonempty interval \(\interval{a}{b} \in \Int\) and \(c \in \Z\) define
  \(\interval{a}{b} \pm c \veq \interval{a\pm c}{b\pm c}\) (recall that \(\pm \infty + c = \pm\infty - c = \pm\infty\)).  
  \qed
\end{definition}


Observe that for every interval \(\interval{a}{b} \in \Int\) and
\(c \in \Z\)
\begin{center}
  \(\max(\truncL{\interval{a}{b}}{c}) \leq b\)
  \qquad and \qquad
  \(\max(\truncR{\interval{a}{b}}{c}) \leq c\)
\end{center}
that trivially holds by 
defining \(\max (\bot)  \veq 0\) (i.e., \(0\) is the maximum of
an empty interval).
%Just observe that the resulting interval could be empty, i.e.\ \(\bot\), and .

The \emph{interval semantics} of \(\imp\) is defined as the strict
(i.e., preserving \(\bot\))
% and co-strict (i.e., preserving \(\top\))
extension of the following function \(\semi{\cdot}: \exp \cup \imp \to
\bA \to \bA\). For all \(\eta: \Var \to \Int_*\),

\begin{align*}
  %
  \semi{\var \in S}\eta 
  & \veq  
  \begin{cases}
    \eta[\var \mapsto \eta(\var)\sqcap \Int(S)]  & \text{if }\eta(\var)\sqcap \Int(S)\neq \bot \\
    \bot & \text{otherwise}
  \end{cases}\\
  %
  \semi{\var \in [a,b]}\eta 
  & \veq  
  \begin{cases}
    \eta[\var \mapsto \eta(\var)\sqcap [a,b]]  & \text{if }\eta(\var)\sqcap [a,b]\neq \bot \\
    \bot & \text{otherwise}
  \end{cases}\\
  % boolean
  %
  \semi{\var \leq k}\eta 
  & \veq 
  \begin{cases}
    \eta[\var \mapsto \truncR{\eta(\var)}{k}] & \text{if }\truncR{\eta(\var)}{k}\neq \bot \\
    \bot & \text{otherwise}
  \end{cases}\\
  %
  \semi{\var > k}\eta 
  & \veq 
  \begin{cases}
    \eta[\var \mapsto \truncL{\eta(\var)}{k+1}] & \text{if }\truncR{\eta(\var)}{k}\neq \bot \\
    \bot & \text{otherwise}
  \end{cases}\\
  %
  \semi{\tru}\eta 
  & \veq \eta\\
  %
  \semi{\ff}\eta 
  & \veq \bot\\
  %
  %\semi{\id}\eta & \veq \eta\\
  %
  \semi{\var := k}\eta 
  & \veq \eta[\var \mapsto \interval{k}{k}]\\
  % 
  \semi{\var := \var[y] + k}\eta 
  & \veq \eta[\var \mapsto \eta (\var[y]) + k]\\
  % 
  \semi{\var := \var - k}\eta 
  & \veq \eta[\var \mapsto \eta (\var[y]) - k]\\
  % \semi{\var := \var[y]}\eta 
  % & = \eta[\var \mapsto \eta(\var[y])]\\[3mm]
  %
  % commands
  %
  \semi{\com_1 \ndet \com_2} \eta
  & \veq \semi{\com_1} \eta \sqcup \semi{\com_2} \eta\\
  %
  \semi{\com_1 \seq \com_2} \eta
  & \veq \semi{\com_2} (\semi{\com_1} \eta)\\
  %
  % 
  \semi{\kleene{\com}} \eta
  & \veq \textstyle \bigsqcup_{i \in \nat} \semi{\com}^i (\eta)\\
  %
  \semi{\fix{\com}} \eta
  & \veq  \lfp{\lambda \mu. (\eta \sqcup \semi{\com} \mu)}
\end{align*}

The semantics is well-defined, because of the following lemma:

\begin{lemma}\label{le:monotone}
   for all \(\com \in \imp\), \[\semi{\com}: \bA \to \bA\] is
   monotone.
\end{lemma}

\begin{proof}
  What we have to proof is that given \(\eta, \theta \in \bA\), with
  \(\eta \sqsubseteq \theta\) then \(\forall C \in \imp\)
  \(\semi{C}\eta \sqsubseteq \semi{C}\theta\). We'll work by induction
  on the grammar of \(C\):
  
  \noindent
  \textbf{Base cases:}\ \\ \ We avoid cases where \(\eta = \bot\) and
  \(\semi{C}\eta = \bot\) as \(\forall \theta \in \bA\) \(\bot
  \sqsubseteq \theta\) and it becomes trivially true.
  \begin{itemize}
  \item \(\com \equiv \var \in S\). Then
    \begin{align*}
      \semi{\var \in S}\eta & = \eta[x\mapsto\eta(x) \sqcap \Int(S)] \\
      \semi{\var \in S}\theta & = \theta[x\mapsto \theta(x) \sqcap \Int(S)]
    \end{align*}
    Since \(\eta(x) \sqcap \Int(S) \neq \bot\) and \(\eta \sqsubseteq
    \theta\), then \(\theta(x) \sqcap \Int(S) \neq \bot\). We can see
    that
    \begin{align*}
      \eta \sqsubseteq \theta \iff & \concr(\eta) \subseteq \concr(\theta) \\
      \iff & \{x\in\z \mid x \in \eta(\var)\} \subseteq \{x\in\z \mid x \in \theta(\var)\} \\
      \iff & \{x\in\z \mid x \in \eta(\var)\} \cap \{x \in \z \mid x \in \Int(S)\} \subseteq \{x\in\z \mid x \in \theta(\var)\} \cap \{x \in \z \mid x \in \Int(S)\} \\
      \iff & \{x \in \z \mid x \in \eta(\var) \wedge x \in \Int(S)\} \subseteq \{x \in \z \mid x \in \theta(\var) \wedge x \in \Int(S)\} \\
      \iff & \{x \in \z \mid x \in \eta(\var) \sqcap \Int(S)\} \subseteq \{x \in \z \mid x \in \theta(\var) \sqcap \Int(S)\} \\
      \iff & \concr[\Int](\eta[x\mapsto \eta(\var) \sqcap \Int(S)](\var)) \subseteq \concr[\Int](\theta[x\mapsto \theta(\var) \sqcap \Int(S)](\var)) \\
      \iff & \semi{\var \in S}\eta \sqsubseteq \semi{\var \in S}\theta
    \end{align*}
  \item for the base cases \(\var \in \interval{a}{b}, \var \leq k,
    \var > k\) we can use the same proceedings;
  \item \(\com \equiv \tru\). Then \(\semi{\tru}\eta = \eta
    \sqsubseteq \theta = \semi{\tru}\theta\);
  \item \(\com \equiv \ff\). Then \(\semi{\ff}\eta = \bot \sqsubseteq
    \bot = \semi{\ff}\theta\);
  \item \(\com \equiv \var := k\). Then
    \begin{align}\label{eq:implic}
      \begin{split}
        \eta \sqsubseteq \theta \iff & \concr[\Int](\eta) \subseteq \concr[\Int](\theta) \\
        \iff & \{\rho\in\env \mid \forall \var \in \Var \rho(\var) \in \concr(\eta(\var))\} \subseteq \{\rho\in\env \mid \forall \var \in \Var \rho(\var) \in \concr(\theta(\var))\} \\
        \iff & \forall \var \in \Var, \rho \in \env \quad \rho(\var) \in \concr(\eta(\var)) \Rightarrow \rho(\var) \in \concr(\theta(\var)) \\
      \end{split}
    \end{align}
    Notice that \[\semi{\var := k}\eta =
    \eta[\var\mapsto\interval{k}{k}]\] \[\semi{\var := k}\theta =
    \theta[\var\mapsto\interval{k}{k}]\] because of equation
    \ref{eq:implic} in this case we know that \(\forall \var[y]\in\Var
    ,\; \var[y] \neq \var \; \rho(\var[y]) \in \concr(\eta(\var[y]))
    \Rightarrow \rho(\var[y]) \in \concr(\theta(\var[y]))\). For
    \(\var\) it holds that \(\rho(\var) \in \concr(\interval{k}{k})
    \Rightarrow \rho(\var) \in \concr(\interval{k}{k})\) and therefore
    \begin{align*}
      \forall \var[y] \in \Var, \rho \in \env \quad & \rho(\var[y]) \in \concr(\eta[\var\mapsto\interval{k}{k}](\var[y])) \Rightarrow \rho(\var[y]) \in \concr(\theta[\var\mapsto\interval{k}{k}](\var[y])) \\
      \iff & \concr[\Int](\semi{\var:=k}\eta) \subseteq \concr[\Int](\semi{\var:=k}\theta) \\
      \iff & \semi{\var :=k}\eta \sqsubseteq \semi{\var := k}\theta
    \end{align*}
  \item For \(\com \equiv \var := \var[y] + k, \var := \var[y] - k\)
    the procedure is the same.
  \end{itemize}

  \noindent
  \textbf{Recusrsive cases:}\ \\ \
  \begin{itemize}
  \item \(\com \equiv \com_1 + \com_2\). Then
    \begin{align*}
      \semi{\com_1 + \com_2}\eta & = \semi{\com_1}\eta \sqcup \semi{\com_2}\eta & \\
      & \sqsubseteq \semi{\com_1}\theta \sqcup \semi{\com_2}\theta & \text{by inductive hp.} \\
      & = \semi{\com_1 + \com_2}\theta
    \end{align*}
  \item \(\com \equiv \com_1 ; \com_2\). Then
    \begin{align*}
      \semi{\com_1;\com_2}\eta = \; & \semi{\com_2}(\semi{\com_1}\eta) & \\
      & \alpha = \semi{\com_1}\eta \sqsubseteq \semi{\com_1}\theta = \beta & \text{by inductive hp.} \\
      & \semi{\com_2}\alpha \sqsubseteq \semi{\com_2}\beta & \text{by inductive hp.} \\
      & \semi{\com_2}(\semi{\com_1}\eta) \sqsubseteq \semi{\com_2}(\semi{\com_1}\theta) & \text{by substitution}
    \end{align*}
  \item \(\com^*\). Then by inductive hypothesis \(\forall i \in \n
    . \semi{\com}^i\eta \sqsubseteq \semi{\com}^i\theta\), which means
    \[\semi{\com^*}\eta = \bigsqcup_{i\in\n}\semi{\com}^i\eta \sqsubseteq
    \bigsqcup_{i\in\n}\theta = \semi{\com^*}\theta.\]
  \end{itemize}
\end{proof}

\begin{theorem}[\textbf{Correctness}]
  For all \(\com \in \imp\) and \(\eta\in \mathbb{A}\), 
  \(\sem{\com}\concr(\eta) \subseteq \concr(\semi{\com}\eta)\) holds. 
\end{theorem}

\todo{Sempre per induzione?}
\begin{proof}
  
\end{proof}

\begin{remark}
  Let us remark that in case we were interested in studying
  termination of the abstract interpreter, we could assume that the
  input of a program will always be a finite interval in such a way
  that nontermination can be identified with the impossibility of
  converging to a finite interval for some variable. In fact, starting
  from an environment \(\eta\) which maps each variable to a finite
  interval, \(\semi{\com}\eta\) might be infinite on some variable
  when \(\com\) includes a either Kleene or fix iteration which does
  not convernge in finitely many steps.
\end{remark}

\subsection{Computing the interval semantics} 

In this section we argue that for the language \(\imp\) the interval
abstract semantics is computable in finite time without widening.

Observe that the exact computation provides, already for our simple
language, a precision which is not obtainable with (basic) widening
and narrowing. In the example below the semantics maps \(\var\) and
\(\var[y]\) to \([0,2]\) and \([6,8]\) resp., while widening/narrowing
to \([0,\infty]\) and \([6,\infty]\)

\begin{verbatim}
x:=0;
y:=0;
while (x<=5) do
   if (y=0) then
      y=y+1;
   endif;
   if (x==0) then
      x:=y+7;
   endif;
done;
end
\end{verbatim}

Of course, for the collecting semantics this is not the case. Already
computing a finite upper bound for loop invariants when they are
finite is impossible as this would allow to decide termination, as
we've seen in section \ref{sec:finiteness}.

\begin{problem}[Termination of interval analysis]\label{problem1}
Given \(\com\in \imp\), \(\eta \in \bA\), decide: \(\sem{\com} \eta
=^? \top\)
\end{problem}

First, given a program, we associate each variable with a \emph{single
bound}, which captures both both an \emph{upper bound}, for which the
rough idea is that, whenever a variable is beyond that bound, the
behaviour of the program with respect to that variable becomes stable
and an \emph{increment bound} which captures the largest increment or
decrement that can affect a variable.

\begin{definition}[\textbf{Program bound}]
  \label{de:bound}
  The \emph{bound} associated with a command \(\com\in \imp\) is a
  natural number, denoted \(\bound{\com}\in \nat\), defined
  inductively as follows:
  \begin{align*}
    %
    \bound{\var \in S}  
    & \veq \begin{cases}
      \min(S) & \text{if } \max(S)=\infty\\
      \max(S) & \text{if } \max(S)\in \nat
    \end{cases}
    \\
    %
    \bound{\var \in [a,b]}  
    & \veq \begin{cases}
      a & \text{if } b=\infty\\
      b & \text{if } b\in \nat
    \end{cases}
    \\
    % boolean
    %
    \bound{\var \leq k}  
    & \veq k\\
    %
    \bound{\var > k} 
    & \veq k\\
    %
    %
    \bound{\tru} 
    & \veq 0\\
    %
    \bound{\ff} 
    & \veq 0\\
    % \bound{\var > \var[y]}  
    % & \veq 0\\
    %
    %
    % assignements
    %
    \bound{\var := k} 
    & \veq k\\
    %
    \bound{\var := \var[y] + k}
    & \veq k\\
    %
    \bound{\var := \var[y] - k}
    & \veq k\\
    % 
    % \bound{\var := \var[y]} 
    % & \veq 0\\[3mm]
    %
    % commands
    %
    \bound{\com_1 \ndet \com_2}
    & \veq \bound{\com_1} + \bound{\com_2}\\
    %
    \bound{\com_1 \seq \com_2}
    & \veq \bound{\com_1} + \bound{\com_2}\\
    %% %% 
    %% \bound{\kleene{\com}}
    %% & \veq 2 \bound{\com}\\
    %% %% 
    \bound{\kleene{\com}}
    & \veq (|\mathit{vars}(\com)|+1) \bound{\com} \\ 
  \end{align*}
  where \(\mathit{vars}(\com)\) denotes the set of variables occurring in 
  \(\com\).
\end{definition}

\begin{definition}[\textbf{Bound Environment}]
  \label{de:boundenv}
  A bound environment (benv for short) is a total function \(b:\Var \ra \nat\). We define 
  \(\benv \veq \{ b \mid b:\Var \ra \nat\}\).
  Each command 
  \(C\in \imp\) induces a benv transformer \(\boundt{C}:\benv \ra \benv\), which is defined inductively as follows:
  \begin{align*}
    %
    % boolean
    %
    \boundt{\var \in S}b  
    & \veq 
    \begin{cases} 
      b[\var[x]\mapsto b(\var[x])+\min(S)]& \text{if }\max(S)=\infty\\
      b[\var[x]\mapsto b(\var[x])+\max(S)]& \text{if }\max(S)\in \nat
    \end{cases}
    \\
    %
    % assignments
    %
    \boundt{\var := k}b 
    & \veq  b[\var[x]\mapsto b(\var[x])+k]\\
    %
    \boundt{\var := \var[y] + k}b
    & \veq  b[\var[x]\mapsto b(\var[x])+b(\var[y])+k]\\
    %
    \bound{\var := \var[y] - k}b
    & \veq  b[\var[x]\mapsto b(\var[x])+b(\var[y])+k]\\
    %
    % commands
    %
    \boundt{\com_1 \ndet \com_2}b
    & \veq \lambda \var[x]. (\boundt{\com_1}b)(\var[x]) + (\boundt{\com_2}b)(\var[x])\\
    %
    \boundt{\com_1 \seq \com_2}b
    & \veq \lambda \var[x]. (\boundt{\com_1}b)(\var[x]) + (\boundt{\com_2}b)(\var[x])\\
    %& \veq \boundt{\com_2}(\boundt{\com_1}b) \\
    % 
    \boundt{\fix{\com}}b
    &\veq  \lambda \var[x]. (|\mathit{vars}(\com)|+1)(\boundt{\com}b)(\var[x])
  \end{align*}    
  where \(\mathit{vars}(\com)\) denotes the set of variables occurring
  in \(\com\).
\end{definition}

\begin{lemma}
  For all \(\com \in \imp\), \(\bound{\com}=\sum_{\var\in \vars{\com}}
  (\boundt{\com}b_0)(\var)\), with \(b_0\veq \lambda x.0\).
\end{lemma}
\begin{proof}
  By induction on \(\com\in \imp\). The base cases are clear. 

  \noindent
  \((\com_1 \ndet \com_2)\):
  \begin{align*}
    \bound{\com_1 \ndet \com_2} =&\\
    \bound{\com_1} + \bound{\com_2} = & \quad \text{[by inductive hypothesis]}\\
    \sum_{\var\in \vars{\com_1}} (\boundt{\com}b_0)(\var) + \sum_{\var\in \vars{\com_2}} (\boundt{\com}b_0)(\var) =&\\
    \sum_{\var\in \vars{\com_1}\cap \vars{\com_2}} (\boundt{\com_1}b_0)(\var)+(\boundt{\com_2}b_0)(\var) +&\\
    \sum_{\var\in \vars{\com_1}\smallsetminus \vars{\com_2}} (\boundt{\com_1}b_0)(\var) +&\\
    \sum_{\var\in \vars{\com_2}\smallsetminus \vars{\com_1}} (\boundt{\com_2}b_0)(\var) =&\\
    \boundt{\com_1 \ndet \com_2}b_0 \phantom{=}&
  \end{align*}

  \noindent
  \((\com_1; \com_2)\): identical to \((\com_1 \ndet \com_2)\).

  \medskip
  \noindent
  \((\fix{\com})\):
  \begin{align*}
    \bound{\fix{\com}} =&\\
    |\vars{C}+1|\bound{\com} = & \quad \text{[by inductive hypothesis]}\\
    |\vars{C}+1|\sum_{\var\in \vars{\com}} (\boundt{\com}b_0)(\var)=&\\
    \sum_{\var\in \vars{\com}} |\vars{C}+1|(\boundt{\com}b_0)(\var)=&\\
    %\lambda \var[x]. (\boundt{\com_1}b)(\var[x]) + (\boundt{\com_2}b_0)(\var[x]) =&\\
    \boundt{\fix{\com}}b_0\phantom{=}&
  \end{align*}
\end{proof}
 
