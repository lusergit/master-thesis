\section{Interval Analysis}
The following chapter aims to introduce what we mean by interval
analysis, giving some background and definition and finally proving
the decidability of the analysis without the use of a widening
operator. Notice that this is not in contrast to the more general
results on concrete semantics, as interval analysis, falling under the
broader umbrella of static analysis aims to provide a sound but maybe
not complete analysis results.

\begin{definition}[Interval domain]
  We call by \[Int \defin \{[a,b] \mid a \in \z \cup \{-\infty\}
  \wedge b \in \z \cup \{+\infty\} \wedge a \leq b\}\]
  the interval domain.
\end{definition}

\begin{definition}[Concretization map]
  The concretization map \(\gamma : Int \to 2^\z\) is the
  following \[\gamma([a,b]) = \{x \in \z \mid a \leq x \leq b\}; \quad
  \gamma(\bot) = \emptyset\]
\end{definition}

\begin{definition}[Abstract domain]
  The We'll call by \(\biga\) the abstract domain \[\biga \defin (Var \to
  int_*) \cup \{\bot\}\]
\end{definition}

\begin{definition}[Abstract concretization]
  The abstract domain \(\biga\) concretization map is
  \begin{align}
    \gamma(\bot) & \defin \emptyset \\ \forall \eta \neq \bot, \quad
    \gamma(\eta) & \defin \{\rho \in \env \mid \forall x \in \var
    . \rho(x) \in \gamma(\eta(x))\}
  \end{align}
\end{definition}
