\subsection{Definition}
\label{sub:intervals}

We define \emph{interval analysis} of the above language \(\imp\) in a
standard way, taking the best correct approximations (bca) for the
basic expressions in \(\expr\).

\begin{definition}[Integer intervals]
  We call
  \[ \Int \defin \{ \interval{a}{b} \mid a \in \n \wedge b \in \n \cup
    \{+\infty\} \wedge a\leq b \} \cup \{\abstract{\bot}\} \] set of
  integer intervals.
\end{definition}

\begin{definition}[Concretization map]
  We define the \emph{concretization map} \(\gamma:\Int \to
  \poset{\n}\) as
  \begin{align*}
    \gamma(\interval{a}{b}) & \defin \{x\in \n \mid a \leq x \leq b\} \\
    \gamma(\bot) & \defin \emptyset
  \end{align*}
\end{definition}

\noindent
Observe that \(\tuple{\Int,\sqsubseteq}\) is a complete lattice where
for all \(I,J\in \Int\), \(I\sqsubseteq J\) iff
\(\gamma(I) \subseteq \gamma(J)\).


\begin{definition}[Abstract integer domain]
  Let \(\Int_* \defin \Int \setminus \{\abstract{\bot}\}\). The
  abstract domain \(\bA\) for program analysis is the variable-wise
  lifting of \(\Int\): \[ \bA \defin (\Var \to \Int_*) \cup \{
    \abstract{\bot} \} \]
\end{definition}

where the intervals for a given variable are always nonempty, while
\(\abstract{\bot}\) represents the empty set of environments.  Thus,
the corresponding concretization is defined as follows:

\begin{definition}[Interval concretization]
  We define the \emph{concretization map} for the abstract domain
  \(\bA\) \(\concr[\Int] : \bA \to \poset{\env}\) as
  \begin{align*}
    \concr[\Int](\bot) & \defin \emptyset \\
    \forall \eta\neq \bot \quad \concr[\Int](\eta) & \defin \{\rho \in \env \mid \forall x\in \Var \: \rho(x) \in \gamma(\eta(x))\} 
  \end{align*}
\end{definition}

\begin{observation}
  If we consider the ordering \(\sqsubseteq\) on \(\bA\)
  s.t. \[\forall \eta, \theta \in \bA \quad \eta \sqsubseteq \theta
    \iff \concr[\Int](\eta) \subseteq \concr[\Int](\theta)\] then
  \(\tuple{\bA, \sqsubseteq}\) is a complete lattice.
\end{observation}

\begin{definition}[Interval abstraction]
  We define the \emph{abstraction map} of some numerical set
  \(X \subseteq \n\) into the abstract domain \(\bA\):
  \(\abstr[\Int] : \poset{\n} \to \bA\) as
  \[\abstr[\Int](X) \defin \begin{cases} \abstract{\bot} & \text{if }
      X = \emptyset \\ \interval{\min(X)}{\max(X)} &
      \text{otherwise} \end{cases}\]
\end{definition}

Observe that since we have both a concretization map \(\concr[\Int]\)
and an abstraction map \(\abstr[\Int]\) we have built the Galois
Connection \[\tuple{\concr[\Int], \dom, \bA, \abstr[\Int]}\] between
the concrete domain \(\dom\) and the abstract domain \(\bA\),
resulting

\begin{definition}[Abstract operations]
  We define sound abstract lub and glb operations in the \(\bA\)
  domain:
  \begin{align*}
    \interval{a}{b} \; \acup \; \interval{c}{d} & \defin \interval{\min(a,c)}{\max(b,d)} \\
    \interval{a}{b} \; \acap \; \interval{c}{d} & \defin \begin{cases} \interval{\max(a,c)}{\min(b,d)} & \text{if } \min < \max \\
      \abstract{\bot} & \text{otherwise} \end{cases}
  \end{align*}
  And sound abstract arithmetical operations:
  \begin{align*}
    \abstract{-}\; \interval{a}{b} & \defin \interval{-b}{-a} \\
    \interval{a}{b} \abstract{+} \interval{c}{d} & \defin \interval{a+c}{b+d} \\
    \interval{a}{b} \abstract{-} \interval{c}{d} & \defin \interval{a-c}{b-d} \\
    \interval{a}{b} \abstract{\times} \interval{c}{d} & \defin \interval{\min(ac,ad,bc,bd)}{\max(ac,ad,bc,bd)} \\
  \end{align*}
\end{definition}

\begin{definition}[Interval sharpening]
  \label{de:trunc}
  For a nonempty interval \(\interval{a}{b} \in \Int\) and \(c \in \n\), we define
  two operations raising \(\uparrow\) the lower bound to \(c\) and lowering \(\downarrow\) the upper
  bound to \(c\), respectively:
  \begin{align*}
    &\truncL{\interval{a}{b}}{c} \veq 
      \begin{cases} 
        \interval{\max\{a,c\}}{b} & \text{if } c\leq b\\
        \bot & \text{if } c > b
      \end{cases}
    \\
    &\truncR{\interval{a}{b}}{c} \veq 
      \begin{cases}   
        \interval{a}{\min\{b,c\}} & \text{if } c\geq a\\
        \bot & \text{if } c < a 
      \end{cases} 
  \end{align*}
  Observe that \(\max(\truncR{\interval{a}{b}}{c})\leq c\) always holds. \qed   
  % \end{quote}
\end{definition}

\begin{definition}[Interval addition and subtraction]
  \label{de:add}
  For a nonempty interval \(\interval{a}{b} \in \Int\) and \(c \in \n\) define
  \(\interval{a}{b} \pm c \veq \interval{a\pm c}{b\pm c}\) (recall that \(\pm \infty + c = \pm\infty - c = \pm\infty\)).  
  \qed
\end{definition}


Observe that for every interval \(\interval{a}{b} \in \Int\) and
\(c \in \n\)
\begin{center}
  \(\max(\truncL{\interval{a}{b}}{c}) \leq b\)
  \qquad and \qquad
  \(\max(\truncR{\interval{a}{b}}{c}) \leq c\)
\end{center}
that trivially holds by 
defining \(\max (\bot)  \veq 0\) (i.e., \(0\) is the maximum of
an empty interval).
% Just observe that the resulting interval could be empty, i.e.\ \(\bot\), and .

\begin{definition}[Interval semantics]\label{de:intervalsem}
  The \emph{interval semantics} of \(\imp\) is the strict (i.e.,
  preserving \(\bot\))
  % and co-strict (i.e., preserving \(\top\))
  extension of the following function \(\semi{\cdot}: \imp \to
  \bA \to \bA\). For all \(\eta: \Var \to \Int_*\),
  \begin{align*}
    \semi{\var \in S}\eta 
    & \veq  
      \begin{cases}
        \eta[\var \mapsto \eta(\var)\sqcap \abstr[\Int](S)] & \text{if }\eta(\var)\sqcap \abstr[\Int](S)\neq \bot \\ \bot &
                                                                                                                            \text{otherwise}
      \end{cases}\\
    % \semi{\tru}\eta 
    % & \veq \eta\\
    % \semi{\ff}\eta 
    % & \veq \bot\\
    \semi{\var := k}\eta 
    & \veq \eta[\var \mapsto \interval{k}{k}]\\
    \semi{\var := \var[y] + k}\eta 
    & \veq \eta[\var \mapsto \eta (\var[y]) + k]\\
    \semi{\var := \var[y] - k}\eta 
    & \veq \eta[\var \mapsto \eta (\var[y]) - k]\\
    \semi{\com_1 \ndet \com_2} \eta
    & \veq \semi{\com_1} \eta \sqcup \semi{\com_2} \eta\\
    \semi{\com_1 \seq \com_2} \eta
    & \veq \semi{\com_2} (\semi{\com_1} \eta)\\
    \semi{\kleene{\com}} \eta
    & \veq \textstyle \bigsqcup_{i \in \nat} \semi{\com}^i (\eta)\\
    \semi{\fix{\com}} \eta
    & \veq  \lfp(\lambda \mu. (\eta \sqcup \semi{\com} \mu))
  \end{align*}
\end{definition}

Notice that the filtering in Definition~\ref{de:intervalsem} is not
the best correct approximation. In particular the filtering
instruction \(\var\in S\) is performed first by abstracting the
numerical set \(S \subseteq \n\) with \(\abstr[\Int]\) and then by
computing the greatest lower bound with \(\eta(\var)\). Best correct
approximation would consist in the opposite approach: compute the
concrete \(\concr(\eta\var)\) and subsequently compute the meet in
the \(\poset{\n}\) domain:
\begin{equation*}
  \semi{\var\in S}\eta = \begin{cases}
    \eta[\var \mapsto \abstr[\Int](\concr(\eta\var) \cap S)]
    & \text{if } \concr(\eta\var) \cap S \neq \bot \\
    \bot & \text{otherwise}
  \end{cases}
\end{equation*}
This however would introduce a problem in proving Lemma~\ref{le:inc},
as Axiom~\ref{ax:max} would not be respected.  Notice that this
approximation coincides in fact with the best correct approximation in
two important cases: \(\var \leq k\) and \(\var \geq k\). These two
cases are widely used in programming, namely we can cite the fact that
they are part of the NASA programming guidelines on writing
analyzer-friendly code as stated in \cite{nasa:ten}.
