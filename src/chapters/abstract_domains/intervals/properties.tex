\subsection{Properties}
\label{sub:intervalsprop}

We can immediately see how in the abstract interval domain, the
semantics of the Kleene star and the fixpoint operator is not the
same. This intuitively happens because the kleene star is the least
opper bound of a chain of interals, while the fix operator keeps
iterating over least upper bounds.

\begin{example} \label{ex:fix} This is the case, for instance, the
  following program \(P\) represents the difference between the Kleene
  Star and the Fix operator:
\begin{verbatim}
  while x < 8 do
  if x = 2 then x := x+6;
  x := x-3
  if x <= 0 then x:=0
\end{verbatim}
  starting with the finite interval \(\interval{3}{4}\) we get the
  following loop invariants:
  \begin{align*}
    \text{Kleene: } &\sqcup\{[3,4], [0,1], [0,0], [0,0], \ldots\} = [0,4]\\
    \text{Fix: } & \sqcup\{\bot, [3,4], [0,4], [0,5], [0,5],\ldots\} = [0,5]
  \end{align*}
  \noindent
  Both invariants are correct, because they over-approximate the most
  precise concrete invariant \(\{0,1,3,4\}\), however the Kleene
  invariant is strictly more precise than the Fix one.
\end{example}

\begin{lemma}[\(\fix{\com}\) \textbf{is syntactic sugar}]\label{le:sugar}
  For all \(\eta\),
  \(\semi[\inte]{\fix{\com}} \eta = \semi[\inte]{\kleene{(\tru + \com)}} \eta\).
\end{lemma}

\begin{proof}
  Let us first show by induction that 
  \begin{equation}\label{prop2}
    \forall i\geq 0.\: (\eta \sqcup \tru \sqcup \semi[\inte]{\com})^{i+1} \bot = (\tru \sqcup \semi[\inte]{\com})^{i} \eta \tag{\(\sharp\)}
  \end{equation}

  \noindent
  \(i=0\): \( (\eta \sqcup \tru \sqcup \semi[\inte]{\com})^{1} \bot = \eta \sqcup \bot \sqcup \semi[\inte]{\com}\bot = \eta = 
  (\tru \sqcup \semi[\inte]{\com})^{0} \eta\).
  
  % \medskip
  \noindent
  \(i+1\):  
  \begin{align*}
    (\tru \sqcup \semi[\inte]{\com})^{i+1} \eta & = & \\
    (\tru \sqcup \semi[\inte]{\com})((\tru \sqcup \semi[\inte]{\com})^{i} \eta) & = & \\
    ((\tru \sqcup \semi[\inte]{\com})^{i} \eta) \sqcup  \semi[\inte]{\com}((\tru \sqcup \semi[\inte]{\com})^{i} \eta) & = & \text{By induction}\\
    (\eta \sqcup \tru \sqcup \semi[\inte]{\com})^{i+1} \bot \sqcup \semi[\inte]{\com}((\eta \sqcup \tru \sqcup \semi[\inte]{\com})^{i+1}\bot ) &= & \text{As } \eta \sqsubseteq (\eta \sqcup \tru \sqcup \semi[\inte]{\com})^{i+1} \bot \\
    \eta \sqcup (\eta \sqcup \tru \sqcup \semi[\inte]{\com})^{i+1} \bot \sqcup \semi[\inte]{\com}((\eta \sqcup \tru \sqcup \semi[\inte]{\com})^{i+1}\bot ) & = & \\
    (\eta \sqcup \tru \sqcup \semi[\inte]{\com}) ((\eta \sqcup \tru \sqcup \semi[\inte]{\com})^{i+1} \bot) & = & \\
    (\eta \sqcup \tru \sqcup \semi[\inte]{\com})^{i+2} \bot & &
  \end{align*}

  Let us also show that:
  \begin{equation}\label{prop3}
    \lfp{\lambda \mu. (\eta \sqcup \semi[\inte]{\com} \mu)} =
    \lfp{\lambda \mu. (\eta \sqcup \mu \sqcup \semi[\inte]{\com} \mu)}\tag{\(\diamond\)}
  \end{equation}
  Observe that \(\lfp{\lambda \mu. (\eta \sqcup \semi[\inte]{\com} \mu)} = \eta \sqcup  \semi[\inte]{\com}(\lfp{\lambda \mu. (\eta \sqcup \semi[\inte]{\com} \mu)})\), so that we have that:
  \[
    \eta \sqcup  \lfp{\lambda \mu. (\eta \sqcup \semi[\inte]{\com} \mu)} \sqcup \semi[\inte]{\com}(\lfp{\lambda \mu. (\eta \sqcup \semi[\inte]{\com} \mu)})
    \sqsubseteq \lfp{\lambda \mu. (\eta \sqcup \semi[\inte]{\com} \mu)}
  \]
  As a consequence, \(\lfp{\lambda \mu. (\eta \sqcup \mu \sqcup \semi[\inte]{\com} \mu)}\sqsubseteq \lfp{\lambda \mu. (\eta \sqcup \semi[\inte]{\com} \mu)}\) holds. The reverse inequality follows because, for all \(\mu\), 
  \(\eta \sqcup \semi[\inte]{\com} \mu \sqsubseteq \eta \sqcup \mu \sqcup \semi[\inte]{\com} \mu\).

  Then, we have that:
  \begin{align*}
    \semi[\inte]{\fix{\com}} \eta & = \\
    \lfp{\lambda \mu. (\eta \sqcup \semi[\inte]{\com} \mu)} & = &  \text{By \eqref{prop3}}\\
    \lfp{\lambda \mu. (\eta \sqcup \mu \sqcup \semi[\inte]{\com} \mu)} & = & \text{By Knaster-Tarski Theorem} \\
    \bigsqcup_{i \in \nat} (\eta \sqcup \tru \sqcup \semi[\inte]{\com})^i \bot & = \\
    \bot \sqcup \bigsqcup_{i \in \nat} (\eta \sqcup \tru \sqcup \semi[\inte]{\com})^{i+1} \bot & = & \text{By \eqref{prop2}}\\
    \bigsqcup_{i \in \nat} (\tru \sqcup \semi[\inte]{\com})^{i} \eta & = \\
    \semi[\inte]{\kleene{(\tru + \com)}} \eta. &
  \end{align*}  
\end{proof}

\begin{remark}
  Let us remark that in case we were interested in studying
  termination of the abstract interpreter, we could assume that the
  input of a program will always be a finite interval in such a way
  that non termination can be identified with the impossibility of
  converging to a finite interval for some variable. In fact, starting
  from an environment \(\eta\) which maps each variable to a finite
  interval, \(\semi[\inte]{\com}\eta\) might be infinite on some variable
  when \(\com\) includes a either Kleene or fix iteration which does
  not converge in finitely many steps.
\end{remark}