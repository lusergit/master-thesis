We first define what the set of intervals \(\Int\) is and its
abstraction and concretization map to the powerset of integers.

\begin{definition}[Integer intervals]\label{def:int}
  We call
  \begin{equation*}
    \Int \defin \{ \interval{a}{b} \mid a \in \z \cup \{-\infty\}
    \wedge b \in \z \cup \{+\infty\} \wedge a\leq b \} \cup
    \{\bot\} 
  \end{equation*}
  the set of integer intervals. In the rest of the thesis we will
  write \(\top\) instead of \(\interval{-\infty}{+\infty}\)
\end{definition}

In order to later do the variable-wise lifting of the intervals domain
and relate it to the concrete environment \(\dom\) we need to define
concretization and abstraction maps for the intervals domain

\begin{definition}\label{def:concrint}
  We define the \emph{concretization map} \(\concr : \Int \to
  \poset{\z}\) as
  \begin{align*}
    \concr(\interval{a}{b}) & \defin \{x\in \z \mid a \leq x \leq b\} \\
    \concr(\bot) & \defin \emptyset
  \end{align*}

  And the \emph{abstraction map} \(\abstr:\poset{\z} \to \Int\) as
  \begin{equation*}
    \abstr(S) \defin
    \begin{cases}
      \bot & \text{if } S = \emptyset \\
      \interval{\inf(S)}{\sup(S)} & \text{otherwise}
    \end{cases}
  \end{equation*}
\end{definition}
The next step is to define some order on \(\Int\). For this purpose
we define a partial order \(\sqsubseteq\) based on the concretization
map.
\begin{definition}[Partial order on \(\Int\)]\label{def:intpo}
  Let \(I,J \in \Int\). Then
  \begin{equation*}
    I \sqsubseteq J \iff \concr(I) \subseteq \concr(J)
  \end{equation*}
\end{definition}
Observe that \(\tuple{\Int,\sqsubseteq}\) is a complete lattice. We
next characterize least upper bound and greatest lower bound on the
domain \(\Int\). Let \({\interval{a}{b}, \interval{c}{d}\in \Int}\)
\begin{align*}
  \interval{a}{b} \; \acup \; \interval{c}{d} & \defin \interval{\min(a,c)}{\max(b,d)} \\
  \interval{a}{b} \; \acap \; \interval{c}{d} & \defin
                                                \begin{cases} \
                                                  \interval{\max(a,c)}{\min(b,d)} & \text{if } \min(b,d) < \max(a,c) \\
                                                  \bot & \text{otherwise}
                                                \end{cases}
\end{align*}
The generalization to infinite sets is obtained in the obvious way, by
replacing \(\min\) and \(max\) with \(\inf\) and \(\sup\).
\noindent
The next building block is the definition of some more operations on
intervals, namely the addition and subtraction of an integer constant:

\begin{definition}[Interval addition and subtraction]
  For a nonempty interval \(\interval{a}{b} \in \Int\) and
  \(c \in \n\) define
  \(\interval{a}{b} \pm c \veq \interval{a\pm c}{b\pm c}\) (recall
  that conventionally \(\pm \infty + c = \pm\infty - c = \pm\infty\)).
\end{definition}

Notice that \(\concr\) and \(\abstr\) maps are concretization and
abstraction maps to and from \(\poset{\z}\). We can therefore notice
that there is a Galois connection between \(\poset{\z}\) and \(\Int\).

\begin{lemma}[\(\Int\) abstracts \(\poset{\z}\)]\label{le:inteposetz}
  Let \(\abstr,\concr\) be abstraction and concretization maps from
  Definition~\ref{def:concrint}. Then
  \(\tuple{\abstr, \poset{\z}, \Int, \concr}\) is a Galois connection.
\end{lemma}

\begin{proof}
  We once again rely on the characterization of Galois connections
  given by Theorem~\ref{th:alternate}:
  \begin{enumerate}[label = (\roman*)]
  \item \(\abstr\) and \(\concr\) are monotonic: let
    \(\iota, \kappa\in\inte\) s.t.\
    \(\iota \sqsubseteq \kappa \implies \concr(\iota) \subseteq
    \concr(\kappa)\) while if we let \(R,S \in \bCnr\) s.t.\
    \(R\subseteq S \implies \abstr(R) \sqsubseteq \abstr(S)\)
  \item \(\concr \conc \abstr\) is extensive. Notice that
    \(\forall R \in \bCnr\) it holds that
    \(R \sqsubseteq \{x\in\z \mid \inf(R) \leq x \leq \sup(R)\} =
    \concr(\abstr(R))\), which is our thesis.
  \item \(\abstr\conc\concr\) is reductive. Notice that
    \(\forall \iota \in \inte\) it holds that
    \(\abstr(\concr(\iota)) = \iota\), which implies that it is
    reductive.
  \end{enumerate}
  Therefore \(\tuple{\abstr, \poset{\z}, \Int, \concr}\) is a Galois
  connection.
\end{proof}

Moreover, since \(\concr \conc \abstr = \id\) we can deduce that
\(\tuple{\poset{\z}, \ovdot\subseteq} \galoiS{\abstr}{\concr}
\tuple{\Int, \sqsubseteq}\) is a Galois insertion
