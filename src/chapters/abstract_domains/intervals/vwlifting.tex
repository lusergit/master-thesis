\subsection{Variable-wise lifting}\label{sub:vwintervals}

We can therefore proceed to introduce the variable-wise lifting of the
\(\Int\) domain, building the abstract domain \(\inte\).

\begin{definition}[Abstract integer domain]
  Let \(\Int_* \defin \Int \setminus \{\bot\}\). The abstract domain
  \(\inte\) for program analysis is the variable-wise lifting of
  \(\Int\):
  \[ \inte \defin (\Var \to \Int_*) \cup \{ \bot \} \]
\end{definition}

In this domain, we define again abstraction and concretization maps,
building a Galois connection with the concrete domain. We do so by
overloading the \(\abstr\) and \(\concr\) functions, to refer also to
the abstraction and concretization of abstract environments.

\begin{definition}[Concretization and abstraction]\label{def:vwabstr}
  We define the concretization map \(\ovdot\concr : \inte \to \bCnr\)
  and the abstraction map \(\ovdot\abstr : \bCnr \to \inte\) as the
  point wise lifting of \(\concr, \abstr\) from
  Definition~\ref{def:concrint}:
  \begin{align*}
    \ovdot\concr(\eta) & \defin \lambda \var . \concr(\eta\var) \\
    \ovdot\abstr(\rho) & \defin \lambda \var . \abstr(\rho\var)
  \end{align*}
  for all \(\eta\in\inte, \rho\in\bCnr\).
\end{definition}

We can again define a notion of order for elements of \(\inte\) based
on the concretization map. We do by overloading the 
\(\sqsubseteq\) notation. Let \(\eta, \theta \in \inte\), then
\begin{equation*}
  \eta \sqsubseteq \theta \text{ iff } \dconcr(\eta) \ovdot\subseteq \dconcr(\theta)
\end{equation*}

Notice that because of \(\ovdot\concr\) definition
\begin{equation*}
  \eta \sqsubseteq \theta \iff \forall \var \in \Var \quad \eta(\var) \sqsubseteq \theta(\var)
\end{equation*}
i.e., two abstract environments are ordered if every variable's
interval of the first environment is contained in the interval of the
second abstract environment.  Also, the least upper bounds and
greatest lower bounds are obtained by lifting the \(\sqcup\) and
\(\sqcap\) operations, i.e., let \(\eta, \theta \in \inte\), then
\begin{align*}
  \eta \acap \theta = \sigma & \quad \text{if } \sigma(\var) = \eta(\var) \sqcap \theta(\var) \quad \forall \var\in\Var \\
  \eta \acup \theta = \sigma & \quad \text{if } \sigma(\var) = \eta(\var) \sqcup \theta(\var) \quad \forall \var\in\Var
\end{align*}

Again we can notice that \(\tuple{\inte, \sqsubseteq}\) is a complete
lattice, as for every two elements \(\eta,\theta\in\inte\) there
exists both \(\eta \sqcup \theta\) and \(\eta \sqcap \theta\).  With
these premises we can define our abstract inductive semantics on
intervals, by defining the base operations
\(\absem[\inte]{\cdot} : \expr \to \inte \to \inte\)

\begin{definition}[Base expressions on intervals]
  Let \(\eta \in \inte\) then the base expressions semantics
  \(\absem[\inte]{\cdot} : \expr \to \inte \to \inte\) is recursively
  defined as
  \begin{align*}
    \absem[\inte]{\var\in I}\eta & \defin
                                   \begin{cases}
                                     \eta[\var\mapsto\eta\var\sqcap I] & \text{if } \eta\var\sqcap I \neq \bot \\
                                     \bot & \text{otherwise}
                                   \end{cases} \\
    \absem[\inte]{\var := k}\eta & \defin \eta[\var\mapsto\interval{k}{k}] \\
    \absem[\inte]{\var := \var[y] + k}\eta & \defin \eta[\var\mapsto \eta\var[y] + k] \\
  \end{align*}
\end{definition}
With these base operations, \(\semi[\inte]{\cdot}\) is defined
accordingly to Definition~\ref{def:abstrsem}.  The next point is to
prove that the interval semantics \(\semi[\inte]{\cdot}\) is sound
w.r.t.\ the concrete semantics \(\sem\cdot\).

\begin{observation}\label{obs:alternative}
  An alternative characterization of \(\absem[\inte]{\cdot}\) views it
  as the b.c.a.\ of base expressions on \(\bCnr\):
  \begin{equation*}
    \absem[\inte]{\com[e]} = \ovdot\abstr \conc \absem[\bCnr]{\com[e]} \conc \ovdot\concr
  \end{equation*}
\end{observation}

\begin{lemma}[\(\inte\) abstracts \(\bCnr\)]
  Let \(\dabstr, \dconcr\) be the abstraction and concretization maps
  of Definition~\ref{def:vwabstr}, then
  \(\tuple{\dabstr, \dom, \inte, \dconcr}\) is a Galois connection.
\end{lemma}

\begin{proof}
  By Lemma~\ref{le:inteposetz}
  \(\tuple{\poset{\z}, \subseteq} \galoiS{\abstr}{\concr} \tuple{\Int,
    \sqsubseteq}\) and by Theorem~\ref{th:liftingins}
  \(\tuple{\bCnr, \ovdot\subseteq} \galoiS{\dabstr}{\dconcr}
  \tuple{\inte, \sqsubseteq}\)
\end{proof}

By latter Lemma and Observation~\ref{obs:alternative} we can deduce
the following

\begin{lemma}[intervals domain abstracts non-relational collecting semantics]\label{le:nonloso}
  \begin{equation*}
    \semnr{\com}\dconcr(\eta) \ovdot\subseteq \dconcr(\semi[\inte]{\com}\eta)
  \end{equation*}
\end{lemma}

\begin{proof}
  This follows from the fact that base expressions on intervals are
  characterised by the best correct approximation of base expressions
  on \(\bCnr\) and from Corollary~\ref{le:bcainducessoundness} b.c.a.\
  induces soundness.
\end{proof}

\begin{observation}
  Analysis on the intervals domain is sound to the concrete
  semantics. For all \(\eta\in\inte\)
  \begin{equation*}
    \sem{\com}\concr(\dconcr(\eta)) \subseteq \concr(\semnr{\com}\dconcr(\eta)) \subseteq \concr(\dconcr(\semi[\inte]{\com}\eta))
  \end{equation*}
\end{observation}

\begin{proof}
  The second inequality
  \(\concr(\semnr{\com}\dconcr(\eta)) \subseteq
  \concr(\dconcr(\semi[\inte]{\com}\eta))\) comes from
  Lemma~\ref{le:nonloso} and monotonicity of \(\concr\), while the
  first one is from Lemma~\ref{le:nrsoundness}
\end{proof}
  
