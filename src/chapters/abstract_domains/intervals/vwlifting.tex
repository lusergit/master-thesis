\subsection{Variable-wise lifting}
\label{sub:vwintervals}

We can therefore proceed to introduce the variable-wise lifting of the
\(\Int\) domain, building the abstract domain \(\inte\):

\begin{definition}[Abstract integer domain]
  Let \(\Int_* \defin \Int \setminus \{\bot\}\). The abstract domain
  \(\inte\) for program analysis is the variable-wise lifting of
  \(\Int\):
  \[ \inte \defin (\Var \to \Int_*) \cup \{ \bot \} \]
\end{definition}

In this domain, we define again abstraction and concretization maps,
building a Galois connection with the concrete domain. We do so by
overloading the \(\abstr\) and \(\concr\) functions, to refer also to
the abstraction and concretization of abstract environments.

\begin{definition}\label{def:vwabstr}
  We define the \emph{concretization map} of abstract environments
  \(\eta \in \inte\), i.e., \(\concr : \inte \to \poset{\env}\) as
  follows
  \begin{align*}
    \dconcr(\bot) & \defin \emptyset \\
    \dconcr(\eta) & \defin \{\rho \in \env \mid \forall \var \in \Var \quad \rho(\var) \in \concr(\eta\var)\}
  \end{align*}
  and the \emph{abstraction map} of sets of concrete environments
  \(X \in \poset{\env}\), i.e., \(\abstr : \poset{\env} \to \inte\) as
  \begin{align*}
    \dabstr(\emptyset) & \defin \bot \\
    \dabstr(X) & \defin \lambda \var \; . \; \abstr(\{\rho(\var) \mid \rho \in X\})
  \end{align*}
\end{definition}

We can again define a notion of order for elements of \(\inte\) based
on the concretization map. We do by overloading the 
\(\sqsubseteq\) notation. Let \(\eta, \theta \in \inte\), then
\begin{equation*}
  \eta \sqsubseteq \theta \text{ iff } \dconcr(\eta) \subseteq \dconcr(\theta)
\end{equation*}

Notice that because of the definition of the concretization map
(Definition~\ref{def:vwabstr})
\begin{equation*}
  \eta \sqsubseteq \theta \iff \forall \var \in \Var \quad \eta(\var) \sqsubseteq \theta(\var)
\end{equation*}
i.e., two abstract environments are ordered if every variable's
interval of the first environment is entirely contained in the
interval of the second abstract environment.  Again, we can define
least upper bounds and greatest lower bounds by lifting the \(\sqcup\)
and \(\sqcap\) operations. i.e., let \(\eta, \theta \in \inte\), then
\begin{align*}
  \eta \acap \theta = \sigma & \quad \text{if } \sigma(\var) = \eta(\var) \sqcap \theta(\var) \quad \forall \var\in\Var \\
  \eta \acup \theta = \sigma & \quad \text{if } \sigma(\var) = \eta(\var) \sqcup \theta(\var) \quad \forall \var\in\Var
\end{align*}

Again we can notice that \(\tuple{\inte, \sqsubseteq}\) is a complete
lattice, as for every two elements \(\eta,\theta\in\inte\) there
exists both \(\eta \sqcup \theta\) and \(\eta \sqcap \theta\).
