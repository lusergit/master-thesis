\section{Abstract semantics}\label{sec:abstractsem}

In order to talk about analysis over some abstract domain \(\bA\), we
preliminarly introduce the abstract semantic.

\begin{definition}\label{def:abstrsem}
  Given an abstract domain \(\bA\), with an abstraction map
  \({\alpha : \poset{\z} \to \bA}\) and a concretization map
  \({\concr : \bA \to \poset{\z}}\) the \emph{analysis semantics} over
  \(\bA\) is defined as the strict (i.e., preserving \(\bot\))
  extension of the following function
  \({\semi[\bA]{\cdot} : \imp \to \bA \to \bA}\). For all
  \(\eta \in \bA\)

  \begin{align*}
    \semi[\bA]{\var \in I}\eta 
    & \veq  
      \begin{cases}
        % \var \mapsto \eta(\var)\sqcap \abstr(S)
        \eta[\var \mapsto \abstr(\concr(\eta(\var)) \cap \concr[\Int](I))] & \text{if }\concr(\eta(\var)) \cap \concr[\Int](I) \neq \emptyset \\
        \bot & \text{otherwise}
      \end{cases}\\
    \semi[\bA]{\var := k}\eta 
    & \veq \eta[\var \mapsto \abstr(\{k\})]\\
    \semi[\bA]{\var := \var[y] + k}\eta 
    & \veq \eta[\var \mapsto \eta (\var[y]) + k]\\
    % \semi[\bA]{\var := \var[y] - k}\eta 
    % & \veq \eta[\var \mapsto \eta (\var[y]) - k]\\
    \semi[\bA]{\com_1 \ndet \com_2} \eta
    & \veq \semi[\bA]{\com_1} \eta \sqcup \semi[\bA]{\com_2} \eta\\
    \semi[\bA]{\com_1 \seq \com_2} \eta
    & \veq \semi[\bA]{\com_2} (\semi[\bA]{\com_1} \eta)\\
    \semi[\bA]{\kleene{\com}} \eta
    & \veq \textstyle \bigsqcup_{i \in \nat} {\left(\semi[\bA]{\com}\right)}^i (\eta)\\
    \semi[\bA]{\fix{\com}} \eta
    & \veq  \lfp(\lambda \mu. (\eta \sqcup \semi[\bA]{\com} \mu))
  \end{align*}
  where \(\concr[\Int]\) is the interval concretization map as defined
  in Definition~\ref{def:concrint} and \(I \in \Int\), defined in
  Definition~\ref{def:int}.
\end{definition}

From this definition we can observe that if we have two domains
\(\bA\) and \(\abstract{\bA}\) s.t.
\(\bA \galois{\abstr}{\concr} \abstract{\bA}\) according to some
anstraction and concretization maps \(\abstr\) and \(\concr\) then the
analysis over \(\abstract{\bA}\) is \emph{sound} w.r.t.\ the analysis
performed over \(\bA\), i.e., the following theorem holds:

\begin{theorem}[Abstraction soundness]\label{th:sound}
  Let
  \(\tuple{\bA, \sqsubseteq}, \tuple{\abstract{\bA},
    \abstract{\sqsubseteq}}\) be two domains equipped with their
  parial order s.t.\ \(\bA \galois{\abstr}{\concr} \abstract{\bA}\)
  for some abstraction and concretization maps \(\abstr, \concr\) and
  \(\com\in\imp\). Then for all
  \(\abstract{\eta} \in \abstract{\bA}\):
  \begin{equation*}
    ( \semi[\bA]{\com} \conc \gamma ) \; \abstract{\eta} \sqsubseteq (\gamma \conc \semi[\abstract{\bA}]{\com}) \; \abstract{\eta} 
  \end{equation*}
  i.e., the abstract analysis over \(\abstract{\bA}\) is sound w.r.t.\
  the analysis over \(\bA\).
\end{theorem}

\begin{proof}
  % Preliminare: Soundness viene preservata nei passi induttivi,
  % chiudendo per lub, concatenazione e fixpoint
\end{proof}

% Se S è un intervallo non abbiamo più bisogno di non fare la BCA.

% Notice that the filtering in Definition~\ref{def:abstrsem} is not the
% best correct approximation. In particular the filtering instruction
% \(\var\in S\) is performed first by abstracting the numerical set
% \(S \subseteq \n\) with \(\abstr\) and then by computing the greatest
% lower bound with \(\eta(\var)\). Best correct approximation would
% consist in the opposite approach: compute the concrete
% \(\concr(\eta(\var))\) and subsequently compute the meet in the
% \(\poset{\n}\) domain:
% \begin{equation*}
%   \semi{\var\in S}\eta = \begin{cases}
%     \eta[\var \mapsto \abstr(\concr(\eta\var) \cap S)] & \text{if } \concr(\eta\var) \cap S \neq \emptyset \\
%     \bot & \text{otherwise}
%   \end{cases}
% \end{equation*}
% This however would introduce a problem in proving Lemma~\ref{le:inc},
% as Hypothesis~\ref{inc:hp1} would not be respected.  Notice that this
% approximation coincides in fact with the best correct approximation in
% two important cases: \(\var \leq k\) and \(\var \geq k\) with
% \(\var\in\Var\) and some constat \(k\in\z\). These two cases are
% widely used in programming, namely we can cite the fact that they are
% part of the nasa programming guidelines on writing analyzer-friendly
% code as stated in \cite{nasa:ten}.
