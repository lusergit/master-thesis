\section{Abstract semantics}\label{sec:abstractsem}

In order to talk about analysis over some abstract domain \(\bA\), we
preliminarly introduce the abstract semantic.

\begin{definition}[\textbf{Abstract inductive semantic}]\label{def:abstrsem}
  Given a domain \(\bA\) and an abstract semantics
  \(\absem[\bA]{\cdot} : \expr \to \bA \to
  \bA\) for base expressions, the \emph{abstract inductive
    semantics} over \(\bA\) is defined as the strict (i.e.,
  preserving \(\bot\)) extension of the following function
  \({\semi[\bA]{\cdot} : \imp \to \bA \to
    \bA}\). For all \(\eta \in \bA\)

  \begin{align*}
    \semi[\bA]{\com[e]}\eta & \defin \absem[ \bA]{\com[e]}\eta\\
    % \semi[\bA]{\var \in I}\eta 
    % & \veq  
    % \begin{cases}
    %     %   \var \mapsto \eta(\var)\sqcap \abstr(S)
  %   \eta[\var \mapsto \abstr(\concr(\eta\var) \sqcap \concr[\Int](I))] & \text{if }\concr(\eta\var) \sqcap \concr[\Int](I) \neq \bot \\
    %     \bot & \text{otherwise}
    %   \end{cases}\\
    % \semi[\bA]{\var := k}\eta 
    % & \veq \eta[\var \mapsto \abstr(\{k\})]\\
    % \semi[\bA]{\var := \var[y] + k}\eta 
    % & \veq \eta[\var \mapsto \eta\var[y] + k]\\
    % % \semi[\bA]{\var := \var[y] - k}\eta 
    % % & \veq \eta[\var \mapsto \eta (\var[y]) - k]\\
    \semi[\bA]{\com_1 \ndet \com_2} \eta
    & \veq \semi[\bA]{\com_1} \eta \sqcup \semi[\bA]{\com_2} \eta\\
    \semi[\bA]{\com_1 \seq \com_2} \eta
    & \veq \semi[\bA]{\com_2} \left(\semi[\bA]{\com_1} \eta\right)\\
    \semi[\bA]{\kleene{\com}} \eta
    & \veq \textstyle \bigsqcup_{i \in \nat} {\left(\semi[\bA]{\com}\right)}^i (\eta)\\
    \semi[\bA]{\fix{\com}} \eta
    & \veq  \lfp(\lambda \mu. (\eta \sqcup \semi[\bA]{\com} \mu))
  \end{align*}
  % where \(\concr[\Int] : \Int \to \bA\) is a concretization map for intervals
  % \(I \in \Int\) defined in Definition~\ref{def:int} into \(\bA\).
\end{definition}

From this definition we can observe that soundness is preserved from
the base cases, i.e., if we have two domains \(\bA\) and
\(\abstract{\bA}\) s.t.
\(\bA \galois{\abstr}{\concr} \abstract{\bA}\) according to some
anstraction and concretization maps \(\abstr\) and \(\concr\) then the
analysis over \(\abstract{\bA}\) is \emph{sound} w.r.t.\ the analysis
performed over \(\bA\), provided that the base cases are sound.

\begin{theorem}[Abstraction soundness]\label{th:sound}
  Let \(C\in\imp\) and
  \(\tuple{\bA, \sqsubseteq}, \tuple{\abstract{\bA},
    \abstract{\sqsubseteq}}\) be two domains equipped with their
  parial order s.t.\ \(\bA \galois{\abstr}{\concr} \abstract{\bA}\)
  for some abstraction and concretization maps \(\abstr, \concr\) such
  that
  \begin{equation*}
    \forall  \com[e] \in \expr \quad (\absem[\bA]{\com[e]} \conc\concr) \abstract\eta \sqsubseteq (\concr \conc \absem[\abstract\bA]{\com[e]})
  \end{equation*}
  i.e., the analysis over the base cases are sound. Then for all
  \(\abstract{\eta} \in \abstract{\bA}\):
  \begin{equation*}
    ( \semi[\bA]{\com} \conc \gamma ) \; \abstract{\eta} \sqsubseteq (\gamma \conc \semi[\abstract{\bA}]{\com}) \; \abstract{\eta} 
  \end{equation*}
  i.e., the abstract analysis over \(\abstract{\bA}\) is sound w.r.t.\
  the analysis over \(\bA\).
\end{theorem}

\begin{proof}
  % Preliminare: Soundness viene preservata nei passi induttivi,
  % chiudendo per lub, concatenazione e fixpoint
  The proof will proceed again by induction on \(\com\).
  \begin{inductive}
    \case{\(\com[e]\)} In this case by hypothesis it holds that

    \begin{equation*}
      (\semi[\bA]{\com[e]} \conc \concr)\abstract\eta = (\absem[\bA]{\com[e]}\conc\concr)\abstract\eta \sqsubseteq (\concr\conc \absem[\abstract\bA]{\com[e]})\abstract\eta = (\concr\conc\semi[\abstract\bA]{\com[e]})\abstract\eta
    \end{equation*}
    Which is exactly our thesis.

    % For bca induces soundness
    % In this case we know that
    % \(\semi[\abstract\bA]{\com[e]}\abstract\eta =
    % \abstr(\semi[\bA]{\com[e]}\concr(\abstract\eta))\) and we have to
    % prove that
    % \[(\semi[\bA]{\com[e]} \conc \concr) \abstract\eta \sqsubseteq
    %   (\concr \conc \semi[\abstract\bA]{\com[e]})\abstract\eta\]
    % hence we can substitute and end up with
    % \[\semi[\bA]{\com[e]}(\concr(\abstract\eta)) \sqsubseteq
    %   \concr(\abstr(\semi[\bA]{\com[e]}\concr(\abstract\eta))).\] Here
    % we can call \(\rho = \semi[\bA]{\com[e]}(\concr(\abstract\eta))\)
    % and notice that by extensivity \(\forall \rho \in \bA\)
    % \begin{align*}
    %   \rho & \sqsubseteq (\concr \conc \abstr) \rho \\
    %   \semi[\bA]{\com[e]}(\concr(\abstract\eta)) & \sqsubseteq (\concr \conc \abstr) \left(\semi[\bA]{\com[e]}(\concr(\abstract\eta))\right)
    % \end{align*}
    % which is our thesis.
    
    \case{\(\com_1 \ndet \com_2\)} In this case by inductive
    hypothesis we know that both the following hold:
    \begin{align}
      \left(\semi[\bA]{\com_1} \conc \concr\right) \abstract\eta & \sqsubseteq \left(\concr \conc \semi[\abstract\bA]{\com_1}\right) \abstract\eta \label{eq:induct1}\\
      \left(\semi[\bA]{\com_2} \conc \concr\right) \abstract\eta & \sqsubseteq \left(\concr \conc \semi[\abstract\bA]{\com_2}\right) \abstract\eta \label{eq:induct2}
    \end{align}
    and
    \(\semi[\abstract\bA]{\com_1 \ndet\com_2}\abstract\eta =
    \semi[\abstract\bA]{\com_1}\abstract\eta \sqcup
    \semi[\abstract\bA]{\com_2}\abstract\eta\). What we have to prove
    is that
    \begin{equation*}
      \left(\semi[\bA]{\com_1 \ndet \com_2} \conc \concr \right) \abstract\eta
      \sqsubseteq
      \left(\concr \conc \semi[\abstract\bA]{\com_1 \ndet \com_2}\right) \abstract\eta
    \end{equation*}
    or, equivalently
    \begin{align*}
      \semi[\bA]{\com_1 \ndet \com_2}\left( \concr \abstract\eta \right) 
      & \sqsubseteq \concr\left(\semi[\abstract\bA]{\com_1 \ndet \com_2} \abstract\eta\right) \\
      \semi[\bA]{\com_1}\left( \concr \abstract\eta \right) \sqcup \semi[\bA]{\com_2}\left( \concr \abstract\eta \right)
      & \sqsubseteq \concr\left(\semi[\abstract\bA]{\com_1} \abstract\eta \sqcup \semi[\abstract\bA]{\com_2} \abstract\eta\right)
    \end{align*}
    Now we can notice that by Property~\ref{prop:four} of Galois connections
    \begin{equation}\label{eq:propconcr}
      \concr\left(\semi[\abstract\bA]{\com_1} \abstract\eta \sqcup \semi[\abstract\bA]{\com_2} \abstract\eta\right) =
      \bigsqcup\left\{\rho \in \bA \mid \abstr(\rho) \sqsubseteq \semi[\abstract\bA]{\com_1} \abstract\eta \sqcup \semi[\abstract\bA]{\com_2} \abstract\eta\right\}
    \end{equation}
    Now
    \begin{align*}
      \left(\abstr \conc \semi[\bA]{\com_1} \conc \concr\right) \abstract\eta & \sqsubseteq \left(\abstr \conc \concr \conc \semi[\abstract\bA]{\com_1}\right) \abstract\eta & \text{by monotonicity of } \abstr \text{ in~\eqref{eq:induct1}}\\
                                                                              & \sqsubseteq \left(\semi[\abstract\bA]{\com_1}\right) \abstract\eta & \text{by reductivity of }\abstr
    \end{align*}
    and the same applies for~\eqref{eq:induct2}. Hence because
    of~\eqref{eq:propconcr} we can observe that
    \begin{align*}
      \semi[\bA]{\com_1}\left( \concr \abstract\eta \right) \sqcup \semi[\bA]{\com_2}\left( \concr \abstract\eta \right)
      & \sqsubseteq
        \bigsqcup\left\{\rho \in \bA \mid \abstr(\rho) \sqsubseteq \semi[\abstract\bA]{\com_1} \abstract\eta \sqcup \semi[\abstract\bA]{\com_2} \abstract\eta\right\} \\
      & =
        \concr\left(\semi[\abstract\bA]{\com_1} \abstract\eta \sqcup \semi[\abstract\bA]{\com_2} \abstract\eta\right)
    \end{align*}
    which is our thesis.

    \case{\(\com_1\seq\com_2\)} In this case we have to prove that
    \begin{equation*}
      \left(\semi[\bA]{\com_1\seq\com_2} \conc \concr\right) \abstract \eta \sqsubseteq \left(\concr \conc \semi[\abstract\bA]{\com_1\seq\com_2}\right)\abstract\eta
    \end{equation*}
    or equivalently
    \begin{equation*}
      \left(\semi[\bA]{\com_2} \conc \semi[\bA]{\com_1} \conc \concr\right) \abstract \eta \sqsubseteq \left(\concr \conc \semi[\abstract\bA]{\com_2} \conc \semi[\abstract\bA]{\com_1}\right)\abstract\eta
    \end{equation*}
    Now we can notice that by inductive hypothesis
    \(\semi[\abstract\bA]{\com_1}\) and
    \(\semi[\abstract\bA]{\com_2}\) are sound abstractions of
    respectively \(\semi[\bA]{\com_1}\) and \(\semi[\bA]{\com_2}\),
    hence we have the hypothesis to apply Theorem~\ref{th:opcomp} and
    deduce that
    \begin{equation*}
      \left(\semi[\bA]{\com_2} \conc \semi[\bA]{\com_1} \conc \concr \right)
      \sqsubseteq
      \left(\concr \conc \semi[\abstract\bA]{\com_2} \conc \semi[\abstract\bA]{\com_1}\right)
    \end{equation*}
    which is our thesis.

    \case{\(\fix\com\)} In this case we have to prove that
    \begin{equation*}
      \left(\semi[\bA]{\fix\com} \conc \concr\right) \abstract\eta \sqsubseteq \left(\concr \conc \semi[\abstract\bA]{\fix\com}\right) \abstract\eta
    \end{equation*}
    we know by definition and Fixpoint Theorem~\ref{th:fixpoint} that
    \begin{equation*}
      \left(\concr \conc \semi[\abstract\bA]{\fix\com}\right) \abstract\eta
      =
      \concr(\lfp(\lambda \mu . \abstract\eta \sqcup \semi[\abstract\bA]{\com}\mu))
      =
      \concr\left(\bigsqcup \left\{{\left(\lambda \mu . \abstract\eta \sqcup \semi[\abstract\bA]{\com}\mu\right)}^n\abstract\bot \mid n\in\n\right\}\right)
    \end{equation*}
    \begin{equation*}
      \left(\semi[\bA]{\fix\com}\conc\concr\right) \abstract\eta
      =
      \lfp(\lambda \mu . \concr(\abstract\eta) \sqcup \semi[\bA]{\com}\mu) 
      =
      \bigsqcup \left\{{\left(\lambda \mu . \concr(\abstract\eta) \sqcup \semi[\bA]{\com}\mu\right)}^n\bot \mid n\in\n\right\}
    \end{equation*}
    We can now prove by induction on \(n\) that the two are in a
    \(\sqsubseteq\) relation. We do this by first proving that for all
    \(k \in \n\)
    \begin{equation*}
       f^k\bot \sqsubseteq \concr\left({(\abstract f)}^k \abstract\bot\right)
    \end{equation*}
    by induction on \(k\).
    \begin{description}
      % bot is preserved
    \item[Case] (\(n = 0\)). In this case
      \begin{equation*}
        \concr\left({\left(\lambda \mu . \abstract\eta \sqcup \semi[\abstract\bA]{\com}\mu\right)}^0\abstract\bot\right) = \concr(\abstract\bot)
      \end{equation*}
      \begin{equation*}
        {\left(\lambda \mu . \concr(\abstract\eta) \sqcup \semi[\bA]{\com}\mu\right)}^0\bot = \bot
      \end{equation*}
      hence \(\bot \sqsubseteq \concr(\abstract\bot)\) and our thesis
      holds.
      
    \item[Case] (\(n \Rightarrow n + 1\)). Let's call
      \(\abstract f = \lambda \mu . \abstract\eta \sqcup
      \semi[\abstract\bA]{\com}\mu\) and
      \(f = \lambda \mu . \concr(\abstract\eta) \sqcup
      \semi[\bA]{\com}\mu\). First we can observe that
      \begin{equation*}
        f^{n+1} \bot = f(f^n\bot) = \concr\left(\abstract\eta\right) \sqcup \semi[\bA]{\com}(f^n\bot)
      \end{equation*}
      \begin{equation*}
        {\abstract f}^{n+1} \abstract\bot = \abstract f({\abstract f}^n\bot) = \abstract\eta \sqcup \semi[\abstract\bA]{\com}({\abstract f}^n\abstract\bot)
      \end{equation*}
      now consider the following
      \begin{align*}
        \concr\left(\abstract\eta \sqcup \semi[\abstract\bA]{\com}({\abstract f}^n\abstract\bot)\right) & = \bigsqcup \left\{c \in \bA \mid \abstr(c) \sqsubseteq \abstract\eta \sqcup \semi[\abstract\bA]{\com}({\abstract f}^n\abstract\bot)\right\} & \text{by Property~\ref{prop:four} of Galois connections.} \\
        & \sqsupseteq \concr(\abstract\eta) \sqcup \concr\left(\semi[\abstract\bA]{\com}({\abstract f}^n \abstract\bot)\right) & \text{by } (\abstr\conc\concr) \text{ reductivity} \\
        &\sqsupseteq \concr(\abstract\eta) \sqcup \semi[\bA]{\com}\concr({\abstract f}^n \abstract \bot) & \text{by induction on }\com \\
        &\sqsupseteq \concr(\abstract\eta) \sqcup \semi[\bA]{\com}(f^n \bot) & \text{by induction on } n \\
      \end{align*}
      which is our thesis.
    \end{description}
    Hence \(f^n\bot \sqsubseteq \concr({\abstract f}^n \abstract\bot)\),
    for all \(n\in\n\). Therefore again by Property~\ref{prop:four} of
    Galois connections
    \begin{align*}
      \left(\semi[\bA]{\fix\com} \conc \concr \right)\abstract \eta & = \bigsqcup \left\{{\left(\lambda \mu . \concr(\abstract\eta) \sqcup \semi[\bA]{\com}\mu\right)}^n\bot \mid n \in\n\right\} \\
                                                                    & \sqsubseteq \concr\left(\bigsqcup \left\{{\left(\lambda \mu . \abstract\eta \sqcup \semi[\abstract\bA]{\com}\mu\right)}^n \abstract\bot \mid n \in\n\right\}\right) \\
                                                                    & = \left(\concr \conc \semi[\abstract\bA]{\fix\com}\right) \abstract \eta
    \end{align*}
    Which is our thesis.
  \end{inductive}
\end{proof}

We can also observe that best correct approximations (from
Definition~\ref{def:bca}) induce sound semantics:

\begin{lemma}[\textbf{bca induces soundness}]
  Let
  \(\tuple{\bA, \sqsubseteq}, \tuple{\abstract\bA,
    \abstract\sqsubseteq}\) be two abstract domains equipped with
  their respective orders s.t.\
  \(\bA \galois{\abstr}{\concr}\abstract\bA\). If for base expressions
  \(\absem[\abstract\bA]{\cdot} : \abstract\bA \to \abstract\bA\) is defined
  as
  \begin{equation*}
    \absem[\abstract\bA]{\com[e]} \abstract\eta  \defin \abstr\left(\absem[\bA]{\com[e]}\concr\left(\abstract\eta\right)\right)
  \end{equation*}
  Then
  \begin{equation*}
    \left(\semi[\bA]{\com} \conc\concr\right) \abstract\eta \sqsubseteq \left(\concr\conc \semi[\abstract\bA]{\com}\right)\abstract\eta
  \end{equation*}
  i.e., the abstract inductive semantics over \(\abstract\bA\) is
  sound w.r.t.\ the abstract inductive semantics over \(\bA\).
\end{lemma}

\begin{proof}
  The proof follows from Theorem~\ref{th:sound}, where we have to prove
  that for base cases the two semantics are sound.  In this case we
  know that
  \(\absem[\abstract\bA]{\com[e]}\abstract\eta =
  \abstr(\absem[\bA]{\com[e]}\concr(\abstract\eta))\) and we have to
  prove that
  \[(\absem[\bA]{\com[e]} \conc \concr) \abstract\eta \sqsubseteq
    (\concr \conc \absem[\abstract\bA]{\com[e]})\abstract\eta\]
  equivalently, we can substitute and notice that we can prove that
  \[\absem[\bA]{\com[e]}(\concr(\abstract\eta)) \sqsubseteq
    \concr(\abstr(\absem[\bA]{\com[e]}\concr(\abstract\eta))).\] Here
  we can call \(\rho = \absem[\bA]{\com[e]}(\concr(\abstract\eta))\)
  and notice that by extensivity \(\forall \rho \in \bA\)
  \begin{align*}
    \rho & \sqsubseteq (\concr \conc \abstr) \rho \\
    \absem[\bA]{\com[e]}(\concr(\abstract\eta)) & \sqsubseteq (\concr \conc \abstr) \left(\absem[\bA]{\com[e]}(\concr(\abstract\eta))\right)
  \end{align*}
  which is our thesis.
\end{proof}

% Se S è un intervallo non abbiamo più bisogno di non fare la BCA.

% Notice that the filtering in Definition~\ref{def:abstrsem} is not the
% best correct approximation. In particular the filtering instruction
% \(\var\in S\) is performed first by abstracting the numerical set
% \(S \subseteq \n\) with \(\abstr\) and then by computing the greatest
% lower bound with \(\eta(\var)\). Best correct approximation would
% consist in the opposite approach: compute the concrete
% \(\concr(\eta(\var))\) and subsequently compute the meet in the
% \(\poset{\n}\) domain:
% \begin{equation*}
%   \semi{\var\in S}\eta = \begin{cases}
%     \eta[\var \mapsto \abstr(\concr(\eta\var) \cap S)] & \text{if } \concr(\eta\var) \cap S \neq \emptyset \\
%     \bot & \text{otherwise}
%   \end{cases}
% \end{equation*}
% This however would introduce a problem in proving Lemma~\ref{le:inc},
% as Hypothesis~\ref{inc:hp1} would not be respected.  Notice that this
% approximation coincides in fact with the best correct approximation in
% two important cases: \(\var \leq k\) and \(\var \geq k\) with
% \(\var\in\Var\) and some constat \(k\in\z\). These two cases are
% widely used in programming, namely we can cite the fact that they are
% part of the nasa programming guidelines on writing analyzer-friendly
% code as stated in \cite{nasa:ten}.
