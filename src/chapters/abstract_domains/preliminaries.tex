\section{Abstract semantics}\label{sec:abstractsem}

In order to talk about analysis over some abstract domain \(\bA\), we
preliminarly introduce the abstract semantic.

\begin{definition}\label{def:abstrsem}
  Given two domains \(\bA, \abstract\bA\) with bottom elements
  \(\bot, \abstract\bot\) and an abstraction map
  \({\alpha : \bA \to \abstract\bA}\) and a concretization map
  \({\concr : \bA \to \abstract\bA}\) the \emph{analysis semantics}
  over \(\bA\) is defined as the strict (i.e., preserving \(\bot\))
  extension of the following function
  \({\semi[\abstract\bA]{\cdot} : \imp \to \abstract\bA \to
    \abstract\bA}\). For all \(\abstract\eta \in \abstract\bA\)

  \begin{align*}
    \semi[\abstract\bA]{\com[e]}\abstract\eta & \defin \abstr(\semi[\bA]{\com[e]}\concr(\abstract\eta)) \\
    % \semi[\abstract\bA]{\var \in I}\abstract\eta 
    % & \veq  
    %   \begin{cases}
    %     % \var \mapsto \abstract\eta(\var)\sqcap \abstr(S)
    %     \abstract\eta[\var \mapsto \abstr(\concr(\abstract\eta\var) \sqcap \concr[\Int](I))] & \text{if }\concr(\abstract\eta\var) \sqcap \concr[\Int](I) \neq \bot \\
    %     \bot & \text{otherwise}
    %   \end{cases}\\
    % \semi[\abstract\bA]{\var := k}\abstract\eta 
    % & \veq \abstract\eta[\var \mapsto \abstr(\{k\})]\\
    % \semi[\abstract\bA]{\var := \var[y] + k}\abstract\eta 
    % & \veq \abstract\eta[\var \mapsto \abstract\eta\var[y] + k]\\
    % % \semi[\abstract\bA]{\var := \var[y] - k}\abstract\eta 
    % % & \veq \abstract\eta[\var \mapsto \abstract\eta (\var[y]) - k]\\
    \semi[\abstract\bA]{\com_1 \ndet \com_2} \abstract\eta
    & \veq \semi[\abstract\bA]{\com_1} \abstract\eta \sqcup \semi[\abstract\bA]{\com_2} \abstract\eta\\
    \semi[\abstract\bA]{\com_1 \seq \com_2} \abstract\eta
    & \veq \semi[\abstract\bA]{\com_2} (\semi[\abstract\bA]{\com_1} \abstract\eta)\\
    \semi[\abstract\bA]{\kleene{\com}} \abstract\eta
    & \veq \textstyle \bigsqcup_{i \in \nat} {\left(\semi[\abstract\bA]{\com}\right)}^i (\abstract\eta)\\
    \semi[\abstract\bA]{\fix{\com}} \abstract\eta
    & \veq  \lfp(\lambda \mu. (\abstract\eta \sqcup \semi[\abstract\bA]{\com} \mu))
  \end{align*}
  % where \(\concr[\Int] : \Int \to \bA\) is a concretization map for intervals
  % \(I \in \Int\) defined in Definition~\ref{def:int} into \(\bA\).
\end{definition}

From this definition we can observe that if we have two domains
\(\bA\) and \(\abstract{\bA}\) s.t.
\(\bA \galois{\abstr}{\concr} \abstract{\bA}\) according to some
anstraction and concretization maps \(\abstr\) and \(\concr\) then the
analysis over \(\abstract{\bA}\) is \emph{sound} w.r.t.\ the analysis
performed over \(\bA\), i.e., the following theorem holds:

\begin{theorem}[Abstraction soundness]\label{th:sound}
  Let
  \(\tuple{\bA, \sqsubseteq}, \tuple{\abstract{\bA},
    \abstract{\sqsubseteq}}\) be two domains equipped with their
  parial order s.t.\ \(\bA \galois{\abstr}{\concr} \abstract{\bA}\)
  for some abstraction and concretization maps \(\abstr, \concr\) and
  \(\com\in\imp\). Then for all
  \(\abstract{\eta} \in \abstract{\bA}\):
  \begin{equation*}
    ( \semi[\bA]{\com} \conc \gamma ) \; \abstract{\eta} \sqsubseteq (\gamma \conc \semi[\abstract{\bA}]{\com}) \; \abstract{\eta} 
  \end{equation*}
  i.e., the abstract analysis over \(\abstract{\bA}\) is sound w.r.t.\
  the analysis over \(\bA\).
\end{theorem}

\begin{proof}
  % Preliminare: Soundness viene preservata nei passi induttivi,
  % chiudendo per lub, concatenazione e fixpoint
  The proof will proceed again by induction on \(\com\).
  \begin{inductive}
    \case{\(\com[e]\)} In this case we know that
    \(\semi[\abstract\bA]{\com[e]}\abstract\eta =
    \abstr(\semi[\bA]{\com[e]}\concr(\abstract\eta))\) and we have to
    prove that
    \[(\semi[\bA]{\com[e]} \conc \concr) \abstract\eta \sqsubseteq
      (\concr \conc \semi[\abstract\bA]{\com[e]})\abstract\eta\]
    hence we can substitute and end up with
    \[\semi[\bA]{\com[e]}(\concr(\abstract\eta)) \sqsubseteq
      \concr(\abstr(\semi[\bA]{\com[e]}\concr(\abstract\eta))).\] Here
    we can call \(\rho = \semi[\bA]{\com[e]}(\concr(\abstract\eta))\)
    and notice that by extensivity \(\forall \rho \in \bA\)
    \begin{align*}
      \rho & \sqsubseteq (\concr \conc \abstr) \rho \\
      \semi[\bA]{\com[e]}(\concr(\abstract\eta)) & \sqsubseteq (\concr \conc \abstr) \left(\semi[\bA]{\com[e]}(\concr(\abstract\eta))\right)
    \end{align*}
    which is our thesis.
    
    \case{\(\com_1 \ndet \com_2\)} In this case by inductive
    hypothesis we know that both the following hold:
    \begin{align}
      \left(\semi[\bA]{\com_1} \conc \concr\right) \abstract\eta & \sqsubseteq \left(\concr \conc \semi[\abstract\bA]{\com_1}\right) \abstract\eta \label{eq:induct1}\\
      \left(\semi[\bA]{\com_2} \conc \concr\right) \abstract\eta & \sqsubseteq \left(\concr \conc \semi[\abstract\bA]{\com_2}\right) \abstract\eta \label{eq:induct2}
    \end{align}
    and
    \(\semi[\abstract\bA]{\com_1 \ndet\com_2}\abstract\eta =
    \semi[\abstract\bA]{\com_1}\abstract\eta \sqcup
    \semi[\abstract\bA]{\com_2}\abstract\eta\). What we have to prove
    is that
    \begin{equation*}
      \left(\semi[\bA]{\com_1 \ndet \com_2} \conc \concr \right) \abstract\eta
      \sqsubseteq
      \left(\concr \conc \semi[\abstract\bA]{\com_1 \ndet \com_2}\right) \abstract\eta
    \end{equation*}
    or, equivalently
    \begin{align*}
      \semi[\bA]{\com_1 \ndet \com_2}\left( \concr \abstract\eta \right) 
      & \sqsubseteq \concr\left(\semi[\abstract\bA]{\com_1 \ndet \com_2} \abstract\eta\right) \\
      \semi[\bA]{\com_1}\left( \concr \abstract\eta \right) \sqcup \semi[\bA]{\com_2}\left( \concr \abstract\eta \right)
      & \sqsubseteq \concr\left(\semi[\abstract\bA]{\com_1} \abstract\eta \sqcup \semi[\abstract\bA]{\com_2} \abstract\eta\right)
    \end{align*}
    Now we can notice that by Property~\ref{prop:four} of Galois connections
    \begin{equation}\label{eq:propconcr}
      \concr\left(\semi[\abstract\bA]{\com_1} \abstract\eta \sqcup \semi[\abstract\bA]{\com_2} \abstract\eta\right) =
      \bigsqcup\left\{\rho \in \bA \mid \abstr(\rho) \sqsubseteq \semi[\abstract\bA]{\com_1} \abstract\eta \sqcup \semi[\abstract\bA]{\com_2} \abstract\eta\right\}
    \end{equation}
    Now
    \begin{align*}
      \left(\abstr \conc \semi[\bA]{\com_1} \conc \concr\right) \abstract\eta & \sqsubseteq \left(\abstr \conc \concr \conc \semi[\abstract\bA]{\com_1}\right) \abstract\eta & \text{by monotonicity of } \abstr \text{ in~\eqref{eq:induct1}}\\
                                                                              & \sqsubseteq \left(\semi[\abstract\bA]{\com_1}\right) \abstract\eta & \text{by reductivity of }\abstr
    \end{align*}
    and the same applies for~\eqref{eq:induct2}. Hence because
    of~\eqref{eq:propconcr} we can observe that
    \begin{align*}
      \semi[\bA]{\com_1}\left( \concr \abstract\eta \right) \sqcup \semi[\bA]{\com_2}\left( \concr \abstract\eta \right)
      & \sqsubseteq
        \bigsqcup\left\{\rho \in \bA \mid \abstr(\rho) \sqsubseteq \semi[\abstract\bA]{\com_1} \abstract\eta \sqcup \semi[\abstract\bA]{\com_2} \abstract\eta\right\} \\
      & =
        \concr\left(\semi[\abstract\bA]{\com_1} \abstract\eta \sqcup \semi[\abstract\bA]{\com_2} \abstract\eta\right)
    \end{align*}
    which is our thesis.

    \case{\(\com_1\seq\com_2\)} In this case we have to prove that
    \begin{equation*}
      \left(\semi[\bA]{\com_1\seq\com_2} \conc \concr\right) \abstract \eta \sqsubseteq \left(\concr \conc \semi[\abstract\bA]{\com_1\seq\com_2}\right)\abstract\eta
    \end{equation*}
    or equivalently
    \begin{equation*}
      \left(\semi[\bA]{\com_2} \conc \semi[\bA]{\com_1} \conc \concr\right) \abstract \eta \sqsubseteq \left(\concr \conc \semi[\abstract\bA]{\com_2} \conc \semi[\abstract\bA]{\com_1}\right)\abstract\eta
    \end{equation*}
    Now we can notice that by inductive hypothesis
    \(\semi[\abstract\bA]{\com_1}\) and
    \(\semi[\abstract\bA]{\com_2}\) are sound abstractions of
    respectively \(\semi[\bA]{\com_1}\) and \(\semi[\bA]{\com_2}\),
    hence we have the hypothesis to apply Theorem~\ref{th:opcomp} and
    deduce that
    \begin{equation*}
      \left(\semi[\bA]{\com_2} \conc \semi[\bA]{\com_1} \conc \concr \right)
      \sqsubseteq
      \left(\concr \conc \semi[\abstract\bA]{\com_2} \conc \semi[\abstract\bA]{\com_1}\right)
    \end{equation*}
    which is our thesis.

    \case{\(\fix\com\)} In this case we have to prove that
    \begin{equation*}
      \left(\semi[\bA]{\fix\com} \conc \concr\right) \abstract\eta \sqsubseteq \left(\concr \conc \semi[\abstract\bA]{\fix\com}\right) \abstract\eta
    \end{equation*}
    we know by definition and Fixpoint Theorem~\ref{th:fixpoint} that
    \begin{equation*}
      \left(\concr \conc \semi[\abstract\bA]{\fix\com}\right) \abstract\eta
      =
      \concr(\lfp(\lambda \mu . \abstract\eta \sqcup \semi[\abstract\bA]{\com}\mu))
      =
      \concr\left(\bigsqcup \left\{{\left(\lambda \mu . \abstract\eta \sqcup \semi[\abstract\bA]{\com}\mu\right)}^n\bot \mid n\in\n\right\}\right)
    \end{equation*}
    \begin{equation*}
      \left(\semi[\bA]{\fix\com}\conc\concr\right) \abstract\eta
      =
      \lfp(\lambda \mu . \concr(\abstract\eta) \sqcup \semi[\bA]{\com}\mu) 
      =
      \bigsqcup \left\{{\left(\lambda \mu . \concr(\abstract\eta) \sqcup \semi[\bA]{\com}\mu\right)}^n\bot \mid n\in\n\right\}
    \end{equation*}
    We can now prove by induction on \(n\) that the two are in a
    \(\sqsubseteq\) relation.
    \begin{description}
      % bot is preserved
    \item[Case] (\(n = 0\)). In this case 
      
    \item[Case] (\(n \Rightarrow n + 1\)).
    \end{description}
  \end{inductive}
\end{proof}

% Se S è un intervallo non abbiamo più bisogno di non fare la BCA.

% Notice that the filtering in Definition~\ref{def:abstrsem} is not the
% best correct approximation. In particular the filtering instruction
% \(\var\in S\) is performed first by abstracting the numerical set
% \(S \subseteq \n\) with \(\abstr\) and then by computing the greatest
% lower bound with \(\eta(\var)\). Best correct approximation would
% consist in the opposite approach: compute the concrete
% \(\concr(\eta(\var))\) and subsequently compute the meet in the
% \(\poset{\n}\) domain:
% \begin{equation*}
%   \semi{\var\in S}\eta = \begin{cases}
%     \eta[\var \mapsto \abstr(\concr(\eta\var) \cap S)] & \text{if } \concr(\eta\var) \cap S \neq \emptyset \\
%     \bot & \text{otherwise}
%   \end{cases}
% \end{equation*}
% This however would introduce a problem in proving Lemma~\ref{le:inc},
% as Hypothesis~\ref{inc:hp1} would not be respected.  Notice that this
% approximation coincides in fact with the best correct approximation in
% two important cases: \(\var \leq k\) and \(\var \geq k\) with
% \(\var\in\Var\) and some constat \(k\in\z\). These two cases are
% widely used in programming, namely we can cite the fact that they are
% part of the nasa programming guidelines on writing analyzer-friendly
% code as stated in \cite{nasa:ten}.
