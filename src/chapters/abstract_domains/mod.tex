\chapter{Abstract domains}\label{chap:abstractdomains}

%% Framework, introduzione al linguaggio e alle sue proprietà, si
%% definisce la semantica e si dimostrano un paio di cose

In the following chapters we present two domains that we will discuss:
the \emph{interval} domain and the \emph{non-relational} collecting
domain. The two domains are in the class of \emph{non-relational}
domains, meaning that they do not represent the relation between
variables. \todo{Expand} We are interested in these two domains as the
properties that we will discuss in Chapter~\ref{chap:axiomatized} will
apply to these domains, with some restrictions that we will discuss
later.  This chapter's sections are organized as follows:

\begin{itemize}
\item Section~\ref{sec:intervals} will talk about the interval domain,
  with its characterization in \S\ref{sub:intervals} and the domain
  properties in Section~\ref{sub:intervalsprop}.
\item Section~\ref{sec:nonrelational} will talk about the
  non-relational collecting abstraction, first by introducing the
  definition of the non-relational collecting domain in
  Section~\ref{sub:nonrel} and finally by showing some properties of
  the abstraction in Section~\ref{sub:nonrelprop}
\end{itemize}

\subimport{./}{preliminaries}
\subimport{./intervals}{mod}
\subimport{./nonrel}{mod}

%% old
%% \subimport{./}{assumptions}
%% \subimport{./}{decidability}
