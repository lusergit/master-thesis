\subsection{Variable-wise lifting}\label{sub:vwnonrel}

We can therefore proceed by introducing the variable-wise lifting of
the domain \(\poset{\z}\), building the abstract domain \(\bCnr\):

\begin{definition}[Abstract Non relational collecting domain]
  Let \(\poset[*]{\z} = \poset{\z} \setminus \{\emptyset\}\). The
  abstract domain \(\bCnr\) for program analysis is the variable-wise
  lifting of \(\poset{\z}\):
  \[ \bCnr \defin (\Var \to \poset[*]{\z}) \cup \{ \bot \} \]
\end{definition}

In this domain, we define again abstraction and concretization maps,
building a Galois connection with the concrete domain

\begin{definition}\label{def:vwgalois}
  We define the \emph{concretization map} of abstract environments
  \(\eta \in \bCnr\), i.e., \({\concr : \bCnr \to \poset{\env}}\) as
  follows
  \begin{align*}
    \concr(\bot) & \defin \emptyset \\
    \concr(\eta) & \defin \{\rho \in \env \mid \forall \var \in \Var \quad \rho(\var) \in \eta\var\}
  \end{align*}
  and the \emph{abstraction map} of sets of concrete environments
  \(X \in \poset{\env}\), i.e., \(\abstr : \poset{\env} \to \bCnr\) as
  \begin{align*}
    \abstr(\emptyset) & \defin \bot \\
    \abstr(X) & \defin \lambda \var \; . \; \{\rho(\var) \mid \rho \in X\}
  \end{align*}
\end{definition}

We can again define a notion of order for elements of \(\bCnr\) based
on the concretization map. Let \(\eta, \theta \in \bCnr\), then
\begin{equation*}
  \eta \ovdot\subseteq \theta \text{ iff } \concr(\eta) \subseteq \concr(\theta)
\end{equation*}

Notice that because of the definition of the concretization map
\begin{equation*}
  \eta \ovdot\subseteq \theta \iff \forall \var \in \Var \; \eta(\var) \ovdot\subseteq \theta(\var)
\end{equation*}
we can notice that \(\tuple{\bCnr, \ovdot\subseteq}\) is a complete
lattice, as for every two elements \(\eta,\theta\in\bCnr\) there
exists both \(\eta \ovdot\cup \theta\) and \(\eta \ovdot\cap \theta\)
by characterizing least upper bounds and greatest lower bounds as the
lifting of \(\sqcup\) and \(\sqcap\) operations. Let again
\(\eta, \theta \in \bCnr\), then
\begin{align*}
  \eta \ovdot\cap \theta = \sigma & \quad \text{if } \sigma(\var) = \eta(\var) \cap \theta(\var) \quad \forall \var\in\Var \\
  \eta \ovdot\cup \theta = \sigma & \quad \text{if } \sigma(\var) = \eta(\var) \cup \theta(\var) \quad \forall \var\in\Var
\end{align*}

Again With these premises we can now define base
operations on the non-relational collecting abstraction:
\begin{definition}[Base expressions on non-relational collecting]
  Let \(\eta \in \bCnr\) and \(\concr\) from
  Definition~\ref{def:concrint}. The base expressions semantics
  \(\absem[\bCnr]{\cdot} : \expr \to \bCnr \to \bCnr\) is recursively
  defined as
  \begin{align*}
    \absem[\bCnr]{\var\in I}\eta & \defin
                                   \begin{cases}
                                     \eta[\var\mapsto\eta\var\cap \concr(I)] & \text{if } \eta\var\cap \concr(I) \neq \emptyset \\
                                     \bot & \text{otherwise}
                                   \end{cases} \\
    \absem[\bCnr]{\var := k}\eta & \defin
                                   \begin{cases}
                                     \eta[\var\mapsto\{k\}] & \text{if } \eta\var\neq\top \\
                                     \eta & \text{otherwise}
                                   \end{cases} \\
    \absem[\bCnr]{\var := \var[y] + k}\eta & \defin
                                             \begin{cases}
                                               \eta[\var\mapsto\eta\var[y] + k] & \text{if } \eta\var\neq\top \\
                                               \eta & \text{otherwise}
                                             \end{cases}
  \end{align*}
\end{definition}
With these base operations, \(\semi[\bCnr]{\cdot}\) is defined
accordingly to Definition~\ref{def:abstrsem}.

Thanks to \(\abstr\) and \(\concr\) maps of
Definition~\ref{def:vwgalois} there is a Galois conncection between
the concrete domain \(\dom\) and the abstract domain \(\bCnr\).

\begin{lemma}[\(\bCnr\) abstracts \(\dom\)]
  Let \(\abstr, \concr\) be the abstraction and concretization maps
  from Definition~\ref{def:vwgalois}, then
  \(\tuple{\abstr, \dom, \bCnr, \concr}\) is a Galois connection.
\end{lemma}

\begin{proof}
  We use the characterization of Theorem~\ref{th:alternate}, which
  means proving
  \begin{enumerate}[label=(\roman*)]
  \item\label{proof:1} \(\abstr,\concr\) are monotonic;
  \item\label{proof:2} \(\concr \conc \abstr\) is extensive;
  \item\label{proof:3} \(\abstr\conc\concr\) is reductive.
  \end{enumerate}

  Lets start by proving~\ref{proof:1}. Let \(\eta, \theta \in \bCnr\)
  s.t.\ \(\eta\ovdot\subseteq\theta\). This means that
  \(\forall \var\in\Var\) \(\eta\var \subseteq \theta\var\) and
  therefore \(a\in\eta\var \implies a\in\theta\var\). By definition of
  \(\concr\) this means that
  \(\concr(\eta) \subseteq \concr(\theta)\). Let now
  \(R, S \in \poset{\env}\) s.t.\ \(\rho \subseteq \sigma\), which
  means that \(\rho \in R \implies \rho \in S\). By definition of
  \(\abstr\) it follows that
  \(\eta \in \abstr(R) \implies \eta \in \abstr(S)\) which means
  exactly \(\abstr(R) \ovdot\subseteq \abstr(S)\) which is our thesis.

  Let us now show~\ref{proof:2}. What we want to prove is that
  \(\forall R\in\poset(\env)\) it holds that
  \(R \subseteq \concr(\abstr(R))\). If
  \(R = \emptyset \implies \abstr(\emptyset) = \bot\) and therefore
  \(\concr(\bot) = \emptyset\) and therefore the statement
  holds. Otherwise
  \(R \neq \emptyset \implies \abstr(R) = \lambda x . \{\rho(\var)
  \mid \rho \in R\} = \eta\). Now,
  \(\concr(\eta) = \{\rho \in \env \mid \forall \var\in\Var \quad
  \rho(\var) \in \eta\var\}\), this means that
  \(\sigma \in R \implies \sigma \in \concr(\abstr(R))\), which is our
  thesis \(R \subseteq \concr(\abstr(R))\).

  We are left to show~\ref{proof:3}. Let \(\eta \in \bCnr\), we have 2
  cases. Either \(\eta = \bot\), which means
  \(\abstr(\concr(\bot)) = \abstr(\emptyset) = \bot \ovdot\subseteq
  \bot\), hence the statement holds. Otherwise
  \(\concr(\eta) = \{\rho \in \env \mid \forall \var\in\Var \;
  \rho\var \in \eta\var\} = R\) and
  \(\abstr(R) = \lambda\var.\{\rho\var \mid \rho\in R\} =\eta\), hence
  \(\abstr(\concr(\eta)) = \eta\), which implies our thesis
  \(\abstr \conc \concr\) is reductive, and therefore
  \(\tuple{\abstr, \dom, \bCnr, \concr}\) is a Galois connection.
\end{proof}

Notice that in particular \(\abstr\conc\concr = \id\), and therefore
there is a Galois insertion
\(\tuple{\dom, \subseteq} \galoiS{\abstr}{\concr}\tuple{\bCnr,
  \ovdot\subseteq}\) between the concrete domain \(\dom\) and the
abstract domain \(\bCnr\).

Moreover the abstract semantics \(\semnr{\cdot}\) is sound w.r.t.\ the
concrete semantics \(\sem{\cdot}\):

\begin{lemma}[Non-relational collecting soundess]\label{le:nrsoundness}
  Let \(\com\in\imp, \eta\in \bCnr\)
  \begin{equation*}
    \sem{\com}\concr(\eta) \subseteq \concr(\semnr{\com}\eta)
  \end{equation*}
  i.e., Non-relational collecting inductive abstract semantics is sound
  w.r.t.\ concrete semantics.
\end{lemma}

\begin{proof}
  Since there is a galois connection between abstract inductive
  semantics (notice that \(\sem\cdot = \semi[\dom]{\cdot}\)), We just
  need to show that the two semantics are sound on base operations.

  \begin{inductive}
    \case{\(\var\in I\)} In this case we need to show that
    \begin{equation*}
      \sem{\var\in I}\concr(\eta) \subseteq \concr(\semnr{\var \in I}\eta)
    \end{equation*}
    recall that
    \begin{align*}
      \sem{\var\in I}\concr(\eta) & = \{\bsem{\var\in I}\rho \mid \rho \in \concr(\eta) \; \bsem{\var\in I}\rho \neq \bot\} \\
      \concr(\semnr{\var\in I}\eta) & = \concr(\eta[\var\mapsto\eta\var\cap\concr[\inte](I)]) 
    \end{align*}
    to conclude notice that
    \begin{align*}
      \concr(\eta[\var\mapsto\eta\var\cap\concr[\inte](I)]) & = \{\rho \in \env \mid \forall \var\in\Var \; \rho\var\in\eta\var\} \\
                                                            & \supseteq \{\bsem{\var\in I}\rho \mid \rho \in \concr(\eta) \; \bsem{\var\in I}\rho \neq \bot\}
    \end{align*}
    which is our thesis.

    \case{\(\var := k\)} In this case we have to prove that
    \begin{equation*}
      \sem{\var := k}\concr(\eta) \subseteq \concr(\semnr{\var := k}\eta)
    \end{equation*}
    to do so just notice that
    \begin{align*}
      \sem{\var := k}\concr(\eta) & = \{\rho[\var\mapsto k] \mid \rho \in \concr(\eta)\} \\
                                  & \subseteq \{\rho \in \env \mid \forall \var[y] \in \Var \; \rho\var[y] \in \eta[\var\mapsto \{k\}]\var[y]\} \\
                                  & = \concr(\eta[\var\mapsto \{k\}])
    \end{align*}
    which is our thesis.

    \case{\(\var := \var[y] + k\)} In this case similarly to the
    latter case we can just notice that
    \begin{align*}
      \sem{\var := \var[y] + k}\concr(\eta) & = \{\rho[\var\mapsto \rho\var[y] + k] \mid \rho \in \concr(\eta)\} \\
                                            & \subseteq \{\rho \in \env \mid \forall \var[w] \in \Var \; \rho\var[w] \in \eta[\var\mapsto \eta\var[y] + k]\var[w]\} \\
                                            & = \concr(\eta[\var\mapsto \var[y] + k])
    \end{align*}
  \end{inductive}
  Since we are sound on the base cases, by Theorem~\ref{th:sound}
  \begin{equation*}
    \sem{\com}\concr(\eta) \subseteq \concr(\semnr{\com}\eta)
  \end{equation*}
\end{proof}
