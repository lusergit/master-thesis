\subsection{Variable-wise lifting}
\label{sub:vwintervals}

We can therefore proceed to introduce the variable-wise lifting of the
\(\poset{\z}\) domain, building the real abstract domain \(\bCnr\):

\begin{definition}[Abstract Non relational collecting domain]
  Let \(\poset{\z}_* = \poset{\z} \setminus \{\emptyset\}\). The
  abstract domain \(\bCnr\) for program analysis is the variable-wise
  lifting of \(\poset{\z}\):
  \[ \bCnr \defin (\Var \to \poset{\z}_*) \cup \{ \bot \} \]
\end{definition}

In this domain, we define again abstraction and concretization maps,
building a Galois connection with the concrete domain

\begin{definition}
  \label{def:vwgalios}
  We define the \emph{concretization map} of abstract environments
  \(\eta \in \bCnr\), i.e., \({\concr : \bCnr \to \poset{\env}}\) as
  follows
  \begin{align*}
    \concr(\bot) & \defin \emptyset \\
    \concr(\eta) & \defin \{\rho \in \env \mid \forall \var \in \Var \quad \rho(\var) \in \eta\var\}
  \end{align*}
  and the \emph{abstraction map} of sets of concrete environments
  \(X \in \poset{\env}\), i.e., \(\abstr : \poset{\env} \to \bCnr\) as
  \begin{align*}
    \abstr(\emptyset) & \defin \bot \\
    \abstr(X) & \defin \lambda \var \; . \; \{\rho(\var) \mid \rho \in X\}
  \end{align*}
\end{definition}

We can again define a notion of order for elements of \(\bCnr\) based
on the concretization map. We do by overloading the notation
\(\sqsubseteq\). Let \(\eta, \theta \in \bCnr\), then
\begin{equation*}
  \eta \sqsubseteq \theta \text{ iff } \concr(\eta) \subseteq \concr(\theta)
\end{equation*}

Notice that because of the definition of the concretization map
(Definition~\ref{def:vwgalios})
\begin{equation*}
  \eta \sqsubseteq \theta \iff \forall \var \in \Var \; \eta(\var) \sqsubseteq \theta(\var)
\end{equation*}
Again, we can define least upper bounds and greatest lower bounds by
lifting the \(\sqcup\) and \(\sqcap\) operations. Let again
\(\eta, \theta \in \bCnr\), then
\begin{align*}
  \eta \acap \theta = \sigma & \quad \text{if } \sigma(\var) = \eta(\var) \cap \theta(\var) \quad \forall \var\in\Var \\
  \eta \acup \theta = \sigma & \quad \text{if } \sigma(\var) = \eta(\var) \cup \theta(\var) \quad \forall \var\in\Var
\end{align*}

Again we can notice that \(\tuple{\bCnr, \sqsubseteq}\) is a complete
lattice, as for every two elements \(\eta,\theta\in\bCnr\) there
exists both \(\eta \sqcup \theta\) and \(\eta \sqcap \theta\).
