\subsection{Definition}
\label{sub:nonrel}

We first define \emph{non-relational collecting} analysis the the
\(\imp\) language in a standard way, taking again the best correct
approximation (bca) for the basic expressions in \(\expr\).

\begin{definition}[Non Relational Collecting domain]
  % Let \(\nrdom \defin \poset{\n} \setminus \{\emptyset\}\) and
  We define the non relational collecting environments as
  \begin{equation*}
    \envnr \defin \{\eta \mid \eta: \Var \to \poset{\z}\setminus \{\emptyset\}\} \cup \{\bot\}
  \end{equation*}
  And the non relational collecting domain as the complete lattice
  \(\bCnr \defin \tuple{\envnr,\dot{\subseteq}}\) where for all
  \(\eta,\eta'\in \envnr\)
  \begin{align*}
    \bot & \mathrel{\dot{\subseteq}} \eta\\
    \eta & \mathrel{\dot{\subseteq}} \eta' & \text{ if } \eta\var\subseteq \eta'\var \quad \forall \var\in \Var
  \end{align*}
\end{definition}

In order to have a proper Galois connection we also define abstraction
and concretization maps:

\begin{definition}[Non relational collecting abstraction]
  we define the \emph{non relational abstraction} map
  \begin{equation*}
    \alpha: \tuple{\poset{\env},\subseteq} \to \bCnr
  \end{equation*}
  as follows:
  \begin{equation*}
    \alpha(X) \defin 
      \begin{cases}
        \bot & \text{if }X=\varnothing \\
        \lambda \var. \{\rho(\var) \in \n \mid \rho \in X\} & \text{if }X\neq \varnothing\\
      \end{cases}
  \end{equation*}
\end{definition}

\begin{definition}[Non relational concretization]
  We define the \emph{concretization map}
  \[\gamma: \tuple{\envnr,\dot{\subseteq}} \to
  \tuple{\poset{\env},\subseteq}\] as follows:
  \begin{align*}
    \gamma(\bot) & \defin \varnothing \\
    \gamma(\eta) & \defin \{\rho : \Var \to \n \mid \forall \var \in \Var. \rho(\var)\in \eta(\var)\}
  \end{align*}
\end{definition}

With an abstraction and concretization map we have a Galois
connection, where \todo{finish}

we also define the least upper bound and the greatest lower bound on
elements of \(\bCnr\):

\begin{definition}[lub and glb in \(\bCnr\)]
  Let \(\eta,\theta \in \bCnr\). we define the \(\sqcup\) and
  \(\sqcap\) operations:
  \begin{align*}
    \eta \sqcup \theta & = \sigma & \text{if } \sigma\var = \eta\var \cup \theta\var \quad \forall \var \in \Var \\
    \eta \sqcap \theta & = \sigma & \text{if } \sigma\var = \eta\var \cap \theta\var \quad \forall \var \in \Var \\
  \end{align*}
\end{definition}

finally we can define non relational collecting analysis.

\begin{definition}[Non-Relational collecting semantics]\label{def:nonrel}
  The \emph{Non-Relational collecting semantics} of \(\imp\), is the
  strict (i.e., preserving \(\bot\))
  % and co-strict (i.e., preserving \(\top\))
  extension of the following function \(\semnr{\cdot}: \imp \to
  \bCnr \to \bCnr\). For all \(\eta \in \envnr\),

  \begin{align*}
    % 
    \semnr{\var \in S}\eta 
    & \defin  
      \begin{cases}
        \eta[\var \mapsto \eta(\var)\cap S] & \text{if }\eta(\var)\cap S\neq \bot \\ \bot & \text{otherwise}
      \end{cases}\\
    \semnr{\tru}\eta 
    & \defin \eta\\
    \semnr{\ff}\eta 
    & \defin \bot\\
    \semnr{\var := k}\eta 
    & \defin \eta[\var \mapsto \{k\}]\\
    \semnr{\var := \var[y] + k}\eta 
    & \defin \eta[\var \mapsto \eta (\var[y]) + k]\\
    \semnr{\var := \var[y] - k}\eta 
    & \defin \eta[\var \mapsto \eta (\var[y]) - k]\\
    \semnr{\com_1 \ndet \com_2} \eta
    & \defin \semnr{\com_1} \eta \sqcup \semnr{\com_2} \eta\\
    \semnr{\com_1 \seq \com_2} \eta
    & \defin \semnr{\com_2} (\semnr{\com_1} \eta)\\
    \semnr{\kleene{\com}} \eta
    & \defin \textstyle \bigsqcup_{i \in \nat} \semnr{\com}^i (\eta)\\
    \semnr{\fix{\com}} \eta
    & \defin  \lfp(\lambda \mu. (\eta \sqcup \semnr{\com} \mu))
  \end{align*}
\end{definition}