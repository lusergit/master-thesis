\subsection{Properties}\label{sub:nonrelprop}

The non-relational collecting semantics is similar to the interval
semantics we defined in Section~\ref{sec:intervals}, in the sense that
both of them do not model relation between variables. Differing from
interval analysis however, non-relational collecting semantics grains
back additivity, which we lost with interval semantics. With
additivity we can infer that \(\fix\com\) and \(\kleene\com\) have the
same semantics (with Proposition~\ref{prop:samenr}).

\medskip

\noindent
Let's denote as \(\semnr{\cdot}\) the abstract semantics over
\(\bCnr\), i.e., \(\semnr{\cdot} = \semi[\bCnr]{\cdot}\).

\begin{lemma}[Additivity]
  Let \(\eta, \theta \in \bCnr\), \(\com\in\imp\) then
  \begin{equation*}
    \semnr{\com}\left(\eta \ovdot\cup \theta\right) = \left(\semnr{\com}\eta\right) \ovdot\cup \left(\semnr{\com}\theta\right)
  \end{equation*}
\end{lemma}

\begin{proof}
  We can work by induction on \(\com\):

  \medskip
  
  \noindent
  \textbf{Case}  (\(\var \in S\)).
  % 
  Then
  \begin{align*}
    \semnr{\var \in S}(\eta \ovdot\cup \theta) & = (\eta \ovdot\cup \theta)[\var \mapsto (\eta \ovdot\cup \theta)\var \cap S] \\
                                               & = (\eta \ovdot\cup \theta)[\var \mapsto (\eta\var \cap S) \cup (\theta\var \cap S)] \\
                                               & = (\eta[\var \mapsto (\eta\var \cap S)]) \ovdot\cup (\theta[\var \mapsto \theta\var \cap S]) \\
                                               & = \semnr{\var \in S}\eta \ovdot\cup \semnr{\var \in S}\theta
  \end{align*}
  
  % \textbf{Case}  \(\com \equiv \tru\). Then observe that
  % \(\semnr{\tru}(\eta \ovdot\cup \theta) = \eta \ovdot\cup \theta =
  % \semnr{\tru}\eta \ovdot\cup \semnr{\tru}\theta\)
  % \textbf{Case}  \(\com \equiv \tru\). Then observe that
  % \(\semnr{\ff}(\eta \ovdot\cup \theta) = \bot = \semnr{\ff}\eta \ovdot\cup
  % \semnr{\ff}\theta\)
  \medskip

  \noindent
  \textbf{Case}  (\(\var := k\)).
  % 
  Then
  \begin{align*}
    \semnr{\var := k}(\eta \ovdot\cup \theta) & = (\eta \ovdot\cup \theta)[\var \mapsto \{k\}] \\
                                              & = (\eta[\var \mapsto \{k\}]) \ovdot\cup (\theta[\var \mapsto \{k\}]) \\
                                              & = \semnr{\var := k}\eta \ovdot\cup \semnr{\var := k}\theta
  \end{align*}

  \medskip

  \noindent
  \textbf{Case}  (\(\var := \var[y] + k\)).
  % 
  Then
  \begin{align*}
    \semnr{\var := \var[y] + k}(\eta \ovdot\cup \theta) & = (\eta \ovdot\cup \theta)[\var \mapsto \var[y] + k] \\
                                                        & = (\eta[\var \mapsto \var[y] + k]) \ovdot\cup (\theta \mapsto \var[y] + k) \\
                                                        & \semnr{\var := \var[y] + k}\eta \ovdot\cup \semnr{\var := \var[y] + k}\theta
  \end{align*}

  % \medskip

  % \noindent
  % \textbf{Case} (\(\var := \var[y] - k\)).
  % % 
  % Is analogous to the latter case.

  \medskip 

  \noindent
  \textbf{Case}  (\(\com \equiv \com_1 + \com_2\)).
  % 
  Then
  \begin{align*}
    \semnr{\com_1 + \com_2}(\eta \ovdot\cup \sigma) & = \semnr{\com_1}(\eta \ovdot\cup \sigma) \ovdot\cup \semnr{\com_2}(\eta \ovdot\cup \sigma) & \text{by definition}\\
                                                    & = \semnr{\com_1}\eta \ovdot\cup \semnr{\cup_1}\theta \ovdot\cup \semnr{\com_2}\eta \ovdot\cup \semnr{\cup_2}\theta & \text{by inductive hypothesis} \\
                                                    & = \semnr{\com_1 + \com_2}\eta \ovdot\cup \semnr{\com_1 + \com_2}\theta
  \end{align*}

  \medskip

  \noindent
  \textbf{Case}  (\(\com \equiv \com_1 ; \com_2\)).
  Then
  \begin{align*}
    \semnr{\com_1 ; \com_2}(\eta \ovdot\cup \sigma) & = \semnr{\com_2}(\semnr{\com_1}(\eta \ovdot\cup \theta)) \\
                                                    & = \semnr{\com_2}\left(\semnr{\com_1}\eta \ovdot\cup \semnr{\cup_1}\theta\right) & \text{by inductive hypothesis} \\
                                                    & = \semnr{\com_2}(\semnr{\com_1}\eta) \ovdot\cup \semnr{\com_2}(\semnr{\com_1}\theta) & \text{by inductive hypothesis}
  \end{align*}

  \medskip

  \noindent
  \textbf{Case}  (\(\com \equiv \com^*\)).
  % 
  Then
  \begin{equation*}
    \semnr{\com^*}(\eta\ovdot\cup\theta) = \textstyle \ovdot\bigcup_{i\in\n} \semnr{\com}^i(\eta \ovdot\cup \theta) \\
  \end{equation*}
  What we have to show now is that \(\forall i \in \n\)
  \(\semnr{\com}^i(\eta \ovdot\cup \theta) = \semnr{\com}^i \eta \ovdot\cup
  \semnr{\com}^i\theta\). We can show this by induction on \(i\):
  \begin{itemize}
  \item  \(i=0\). Then
    \begin{equation*}
      \semnr{\com}^0(\eta\ovdot\cup\theta) = \eta\ovdot\cup\theta = \semnr{\com}^0\eta \ovdot\cup \semnr{\com}^0\theta
    \end{equation*}
    and the statement holds.
  \item  \(i \implies i + 1\). Notice that
    \begin{align*}
      \semnr{\com}^{i+1}(\eta \ovdot\cup \theta) & = \semnr{\com}\left(\semnr{\com}^i(\eta \ovdot\cup \theta)\right) \\
                                                 & = \semnr{\com}(\semnr{\com}^i\eta \ovdot\cup \semnr{\com}^i\theta) & \text{by inductive hypothesis} \\
                                                 & = \semnr{\com}^{i+1}\eta \ovdot\cup \semnr{\com}^{i+1}\theta & \text{by additivity}
    \end{align*}
  \end{itemize}
  Therefore
  \begin{align*}
    \semnr{\com^*}(\eta\ovdot\cup\theta) & = \textstyle \ovdot\bigcup_{i\in\n} \semnr{\com}^i(\eta \ovdot\cup \theta) \\
                                         & = \textstyle \ovdot\bigcup_{i\in\n}\semnr{\com}^i(\eta \ovdot\cup \theta) \\
                                         & = \textstyle \ovdot\bigcup_{i\in\n}\semnr{\com}^i\eta \ovdot\cup \semnr{\com}^i\theta \\
                                         & = \left(\textstyle \ovdot\bigcup_{i\in\n} \semnr{\com}^i\eta\right) \ovdot\cup \left(\textstyle \ovdot\bigcup_{i\in\n} \semnr{\com}^i\theta\right) \\
                                         & = \semnr{\com^*} \eta \ovdot\cup \semnr{\com^*}\theta
  \end{align*}
\end{proof}

\begin{proposition}\label{prop:samenr}
  \(\fix\com\) and \(\kleene\com\) semantics coincide:
  \begin{equation*}
    \semnr{\fix\com} = \semnr{\kleene\com}
  \end{equation*}
\end{proposition}

\begin{proof}
  Let
  \(f = \lambda \mu . (\eta \ovdot\cup \semnr{\com}\mu)\)
  \begin{align*}
    \semnr{\fix{\com}}\eta & = \lfp(f) \\
                           & = \textstyle\ovdot\bigcup_{i\in\n}\{f^n\bot \mid n \in \n\} & \text{by fixpoint Theorem~\ref{th:fixpoint}} \\
                           & = \textstyle\ovdot\bigcup_{i\in\n}\semnr{\com}^i\eta \\
                           & = \semnr{\com^*}\eta
  \end{align*}
\end{proof}
